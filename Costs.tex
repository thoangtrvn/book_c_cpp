\chapter{Costs}


\section{using virtual function}
\label{sec:overhead-of-virtual-function}


Using inheritance, we can avoid the overhead of virtual functions


\section{instantiating}
\label{sec:overhead-of-instantiating}

\subsection{object in C++ vs. struct in C}
\label{sec:overhead-of-instantiating-object_in_C++-vs_struct_in_C}

In C++ the cost of instantiating an object is the same as instantiating a struct
in C.

All an object is, is a block of memory big enough to store the v-table (if it
has one - Sect.\ref{sec:V-table}) and all the data attributes. Methods consume
no further memory after the v-table has been instantiated.

\section{orders of declaration}
\label{sec:overhead-of-order-of-declaration}


\subsection{of data members}
\label{sec:overhead-of-order-of-declaration-data-members}

C++ 03 standard \verb!9/12!:
\begin{verbatim}
Nonstatic data members of a (non-union) class declared without an intervening
access-specifier are allocated so that later members have higher addresses 
within a class object. 
The order of allocation of nonstatic data members separated 
by an access-specifier is unspecified (11.1). 
Implementation alignment requirements might cause 
two adjacent members not to be allocated immediately after each other; 
so might requirements for space for managing virtual functions (10.3) 
and virtual base classes (10.1).
\end{verbatim}
\url{http://stackoverflow.com/questions/3824213/does-order-of-members-of-objects-of-a-class-have-any-impact-on-performance}

\subsection{of objects}
\label{sec:overhead-of-order-of-declaration-objects-in-a-function}


C++ guarantees that the order of objects in memory is the same as the order of
declaration, unless an access qualifier intervenes.

Objects which are directly adjacent are more likely to be on the same cacheline,
so one memory access will fetch them both (or flush both from the cache). Cache
effectiveness may also be improved as the proportion of useful data inside it
may be higher. Simply put, spatial locality in your code translates to spatial
locality for performance.


Order may affect the amount of padding.  Unnecessary padding may increase the
total size of the structure, leading to higher memory traffic.
\url{http://stackoverflow.com/questions/3824213/does-order-of-members-of-objects-of-a-class-have-any-impact-on-performance}


