\chapter{Data types for character and string}
\label{chap:string}

\section{Character}
\label{sec:character}

An encoding scheme (Sect.\ref{sec:encoding-scheme}) define how to map a symbol
(that represents printed characters (e.g. letters or digits)) to a code point
which is a binary number either stored in RAM or in the permanent storage.
Depending where it is stored, we call it {\it internal representation} (if
stored in RAM memory) and {\it external representation} (if stored in persistent
disk as file) - Sect.\ref{sec:internal-representation}).


% The internal representation refers to the representation that
% store in the RAM (memory) to represent the character/string.
% The external representation is how a given integer value from RAM is
% interpreted as a a value in decimal or hexadecimal when saving to file/writing
% to permanent storage (terminal or hard disk).



\subsection{MSE}
\label{sec:MSE}

ISO C90 has no support (i.e. utility functions) for multi-byte characters.
Multibyte support extension (MSE) was the first amendment added to ISO C
standard, i.e. ISO C Amendment 1 (C Integrity) which later formed C94/C95
(Sect.\ref{sec:C95}).

The MSE consists of a set of library functions that provide a relatively
complete and consistent set of functions for application programming using
multibyte and wide-characters (Sect.\ref{sec:character_wide}). In MSE model,
extended characters can be represented using different internal representations

\begin{enumerate}
  \item multibyte character encodings: Extended Unix Code (EUC) based on
  ISO-2022 standard that can encode character sets using sequences of 7-bit
  codes of maximum 94 or 8836 ($94^2$) characters, or 830584 ($94^3$)
  characters, UTF-8 (support all the glyphs of the EUC codes and more)
  
  Three multibyte encoding schemes
  \begin{enumerate}
    \item restartable multibyte encodings: the multibyte data stream can be
    encoded faithfully, i.e. giving the exact result regardless of where you
    start from the data stream. In summary: each byte should stand for one
    character only, e.g. UTF-8
    
    \item stateful multibyte encodings: the concept of {\it shift sequences} can
    effectively change the data stream to a different state, i.e., changing the
    meaning of the byte sequences according to the current shift state.
    
    \item generalized multibyte encodings: we can detect the locale associated
    with the given file, from that we can determine which routines to process
    the file. 
  \end{enumerate}
  
  \item wide-character encodings: \verb!wchar_t! (Sect.\ref{sec:character_wide})

  \item generalized multibyte encodings
\end{enumerate}


\subsection{locale}
\label{sec:locale}

Read Chap.\ref{chap:locale}.


\subsection{C character}
\label{sec:C_character}
\label{sec:char}

\begin{enumerate}
  \item narrow-character (\verb!char! type):

From the beginning, \verb!char! a narrow-character type is indeed an integral
type of 1-byte (Sect.\ref{sec:character-narrow}).   
  
  \item wide-character 
  \begin{itemize}
  \item \verb!wchar_t! type: Sect.\ref{sec:wchar_t}

  ISO C90 (stddef.h) introduced \verb!wchar_t! wide-character type with a bigger
  size, yet its size is operating system dependent (which can be 1-byte, 2-byte
  or 4-byte).

  \item \verb!wint_t! type: \verb!wint_t! is capable of storing any
  value of \verb!wchar_t!.
  \end{itemize}
The reason to use \verb!wint_t! instead of \verb!wchar_t! is that it can store
the result of evaluating the \verb!WEOF! macro. So \verb!wint_t! is to
\verb!wchar_t! just like \verb!int! to \verb!char!. Many functions dealing with
\verb!wchar_t! return a value of type \verb!wint_t!.
  
\end{enumerate}
Amendment 1 (or C94/C95) added more header files with utilities to handle
wide-character strings.


\subsection{* narrow-character}
\label{sec:character-narrow}

C language first developed to support representing ANSI characters, which
requires 7-bit (narrow-character set) - Sect.\ref{sec:ANSI}. To provide internal
representation for ANSI characters, C provided one-byte data type called
\verb!char!, with the highest bit set to 0. Thus, if the highest bit in an 8-bit
byte is set to 1, then it is an ASCII extension character. There is always a 1:1
mapping between a byte value and an ANSI character.

\verb!char! is the basic data type to represent a single character in C
\begin{verbatim}
char 			// 8-bit signed (-128..127)
unsigned char   // 8-bit unsigned  (0..225)
\end{verbatim}

{\bf HEADER FILE}: \verb!char! type is defined  in one of the header files
\verb!stdio.h, time.h! \verb!stdlib.h, string.h! (in C). To use a pointer to the
character, we can use \verb!char*!; which also means a pointer to the character
string (Sect.\ref{sec:string}).

\begin{Verbatim}
#include <string.h>

int main()
{
  char capital = 'A';

  printf ("The letter is %c.\n",capital);
  printf ("The string is %s.\n",&capital);
  return 0;

  int c;

  while (EOF != (c = getc(stdin)))
  putc(c, stdout);

}
\end{Verbatim}

We will focus on how to read string data from file in UTF-8 encoding scheme. As
the character can be multiple byte, we cannot use \verb!char!. Thus, the using
of \verb!char! should be avoided for a single UTF-8 character, and use
\verb!wchar_t! (Sect.\ref{sec:wchar_t}).
However, at the string level, we can either use \verb!wchar_t []! or
\verb!char*! to represent an UTF-8 string (Sect.\ref{sec:C_string}).


\subsection{* wide characters (wchar\_t, wint\_t)}
\label{sec:character_wide}

\textcolor{red}{Using wide-character set is NOT recommended as it's not
portable, i.e. not fixed size}. 

\subsection{-- wchar\_t}
\label{sec:wchar_t}

{\bf HEADER FILE}: 
The type
\verb!wchar_t! is defined in different header files: \verb!<stddef.h>! (ISO
C90), \verb!<stdlib.h>!, \verb!wchar.h! (in C95); or \verb!<cstring>! and
\verb!<cwchar>! (in C++).

NOTE: In ISO C90, however the existing character/string-oriented functions in
C90 have not been designed to handle this data, until C94/C95 with the Amendment
1 (multibyte support extension) (Sect.\ref{sec:MSE}).


IMPORTANT: The size of \verb!wchar_t! is not well defined
\footnote{\url{http://www.gnu.org/software/libc/manual/html_node/Extended-Char-Intro.html}}
\begin{itemize}
  \item ISO C90 standard didn't specify the size of \verb!wchar_t!.

  \item  In embedded system with C, \verb!wchar_t! is the same as \verb!char!
  (1-byte).
  
  \item GNU C library: \verb!wchar_t! is always 32 bits wide and, therefore,
  capable of representing all UCS-4 values, covering all  ISO 10646 encoding
  schemes
  
  \item Some UNIXs define \verb!wchar_t! as 16-bit, which follows
  reading string in UTF-16.

  \item On Windows, \verb!wchar_t! is 16-bit.
  
\end{itemize}
The only requirement was that it is capable of storing all elements of the basic
character set based on the locale setting (Sect.\ref{sec:locale}). 

% the encoding used for \verb!wchar_t! is unspecified.
% Sect.\ref{sec:FILE_with-lock-wide-character} discusses how to display/save a
% character/string encoded in wide-character set to console or to file.


In most implementations of C compilers, \verb!wchar_t! is a typedef of an integral
type which makes it impossible to be used in template, function overload. 
This is not a problem when using C++ header file where \verb!wchar_t! is a
distinct fundamental (built-in) type.

\begin{mdframed}
\verb!wchar_t! was designed for internationalization, i.e. it can represent
codes for any locale's char encoding and use fixed char length. The original
goal of using \verb!wchar_t! is to make sure the string's code-unit
(Sect.\ref{sec:string}) has the one-to-one mapping with the text's character,
i.e. all the characters have the same size. However, Unicode breaks that
assumption with variable char-length \verb!char*!. Thus, it is recommended to
use \verb!char*! for UTF-8. 

\end{mdframed}

\subsection{--- wint\_t}
\label{sec:wint_t}

{\bf HEADER FILE}: \verb!wint_t! type was introduced since Amendement 1
to ISO C90 (defined in \verb!wchar.h!).
\begin{itemize}
   \item \verb!wint_t! is equal or bigger in size than \verb!wchar_t!
   
   To get a \verb!wchar_t! value from \verb!wint_t! value, we don't want to (as
the value can be cut-off). To get a \verb!wint_t! value from \verb!wchar_t! value, a
simple assignment is enough (just like from \verb!char! to \verb!int!).
\begin{verbatim}
wchar_t x1;
wint_t  x2;

x2 = x1 ; //safe
\end{verbatim}

  \item it is mainly used for data holding parameter of returned value of a
  function representing a single wide character

\begin{verbatim}
wint_t c;

while ((c = wgetc (fp)) != WEOF)
\end{verbatim}
\end{itemize}
% is less used; as it is only used to keep
% the returned value of functions operating on \verb!wchar_t!, which can exceed the ranges of available values in \verb!wchar_t!, e.g. WEOF
% (wide-end-of-file). 

Some macros defined for the range is found in \verb!<stdint.h>! and
\verb!__STDC_LIMIT_MACROS!
\begin{verbatim}
introduced since Amendment 1 to ISO C90.
                   
   TYPE          macro
wint_t          WCHAR_MIN
                WINT_MIN         (signed: -32767
                                  unsigned: 0                     
                WCHAR_MAX    
                WINT_MAX         (signed:32767
                                  unsigned: 65535)
          
                WEOF             (represent a value different from any member of
                                 the extended character set) 
\end{verbatim}

Example:
\begin{Verbatim}
#include <stdio.h>
#include <wchar.h>


    wint_t c;
    while((c=getwc(stdin))!=WEOF)
      {
    printf("%lc",c);
    
   while (WEOF != (c = getwc(stdin)))
   putwc(c, stdout);
    
\end{Verbatim}


\subsection{literal character}
\label{sec:literal-character}

To represent a literal string encoded in wide-character set, we use \verb!L! in
the front, example

\begin{verbatim}
''   : empty char
L''  : epty wchar_t

'hello'  : char-string
L'hello' : wchar_t-string
\end{verbatim}
I/O operations with a character/string in wide-character set is given in
Sect.\ref{sec:FILE_with-lock-wide-character} and
Sect.\ref{sec:C-char_utilities}. 


\subsection{C++ character}
\label{sec:C++_character}

\subsection{* C-backward compatible}

For C backward compatiblility, C++ also  support  
\begin{itemize}
  \item \verb!char! in the header file \verb!<cchar>! or \verb!<cstring>!, and 

  \item \verb!wchar_t! in the header file \verb!<cwchar>!.
\end{itemize}

\subsection{ C++03: char}
\label{sec:char-C++03}

The definition of the type char has been modified to explicitly express that
it's at least the size needed to store an eight-bit coding of UTF-8, and large
enough to contain any member of the compiler's basic execution character set. It
was formerly defined as only the latter in the C++ standard itself, then relying
on the C standard to guarantee at least 8 bits.

\subsection{* C++0x, C++11: char16\_t and char32\_t}
\label{sec:char16_t}
\label{sec:char32_t}
% Sect.\ref{sec:character_sets} describes the number of bytes for each characters
% at different Unicode schemes, e.g. UTF-8, UTF-16, and UTF-32.

To allow using wide-character, C++11 has added a number of new data types:
\verb!char16_t! is for 16-bit encoding scheme and \verb!char32_t!
is for 32-bit encoding scheme.

Even though \verb!char8_t! is not defined, we can add
\begin{verbatim}
typedef unsigned char char8_t; 
typedef std::basic_string<unsigned char> u8string; 
\end{verbatim}
The reason that \verb!char8_t! is not defined as it is not possible to treat
UTF-8 and narrow-encoded string interchangably in portable code. So, \verb!char!
is still used for 8-bit character.

The problem with \verb!char16_t! and \verb!char32_t! is that they are not fully
supported, not even in the standard C++ library
(Sect.\ref{sec:standard-C++-library}): for example, there are no streams
supporting these types directly and it more work than just instantiating the
stream for these types.


\subsection{Escape character}
\label{sec:escape_char}

Characters consisting of a backslash (\verb!\!) followed by a letter or by
combination of digits are called {\it escape characters}. The list of reference is here:
\url{http://msdn.microsoft.com/en-us/library/h21280bw(v=vs.80).aspx}

\subsection{Utilities}
\label{sec:C-char_utilities}

\subsection{-- char}
Utilities for \verb!char, char*! reside in : stdio.h, stdlib.h, string.h and
time.h. Example: with c of type \verb!integer! whose value either EOF or
representable as an \verb!unsigned char!. 
\begin{enumerate}
  \item \verb!tolower(c)!: to lowercase
  \item \verb!toupper(c)!: to uppercase

  \item \verb!isalnum(c)!: check \ldots is a alphanumeric ?
  \item \verb!isalpha(c)!: is alphabetic ?
  
  \item \verb!iscntrl(c)!: check \ldots a control character (which is NOT
  printable). They are 0x7f (DEL), and those between 0x00 (NUL) to 0x1f (US).
  
  \item \verb!isspace(c)!: check \ldots. white-space (i.e. blank, tab, newline,
  vertical tab, line-feed, carriage return)
  
  \item \verb!isgraph(c)!: check \ldots has a graphical representation, i.e.
  isprintable except white space \verb!' '!.
  \item \verb!isprint(c)!: check if a character is printable (i.e. graphical
  presentation and white-space \verb!' '!)
  
  \item \verb!ispunct(c)!: check \ldots is a punctuation characters, i.e.
  graphic characters.
  \item \verb!islower(c)!: is lowercase letter ?
  \item \verb!isupper(c)!: is uppercase letter ?
  \item \verb!isdigit(c)!: is a decimal digit ?
  \item \verb!isxdigit(c)!: is  hexadecimal digit ?
  
  \item \verb!isblank(c)! (C++11) : is blank ?
\end{enumerate}

Example:
\begin{lstlisting}
auto loc = std::locale("");

char s[] = "hello";
for (char &c : s) {
  c = toupper(c, loc);
}
\end{lstlisting}


\subsection{-- wchar\_t, wint\_t}

Utilities for \verb!wchar_t, wchar_t*, wint_t! reside in: \verb!wchar.h!,
\verb!cctype.h!. Using header file \verb!<ctype.h>! (in C) or \verb!<cctype>!
(in C++).  To convert between multibyte characters to wide-characters and vice versa, we
use \verb!wchar.h! and \verb!stdlib.h!

The utilities have \verb!w! as the prefix, e.g. printf(char) becomes
\verb!wprintf(wchar_t)!.

Example:
\begin{lstlisting}
auto loc = std::locale("");

wchar_t s[] = L"hello";
for (wchar_t &c : s) {
  c = toupper(c, loc);
}
\end{lstlisting}

Unfortunately this doesn't really work all the time, e.g. 
in German letter $\beta$ where the uppercase version is actually the two
characters SS instead of a single character.
So internationalized text handling is intrinsically harder than ASCII and cannot
really be simplified in the way the designers of \verb!wchar_t! intended.

\url{http://www.cplusplus.com/reference/cctype/}

\subsection{in Windows: CHAR, TCHAR, \_TCHAR, \_TSCHAR, \_TUCHAR, \_TXCHAR}

In Windows, \verb!char! is ``multibyte'' (each glyph is composed of one or more
\verb!char!s, while \verb!wchar_t! is considered ``widechar'' (each glyph is
composed of one or two \verb!wchar_t!). However, we typically don't use these
name directly. Instead, Windows define new names for them. 


\begin{verbatim}
CHAR      8-bit signed character type

TCHAR     8-bit or 16-bit signed ... (depending on whether 
          _UNICODE macro is defined or not)
          
_TCHAR
_TSCHAR
_TUCHAR
_TXCHAR
_T()     macro

\end{verbatim}
The data type in Windows C language is in upper case.

\verb!_TCHAR! is based on \verb!_UNICODE! defined macro, and is
used by C-runtime applications; while \verb!TCHAR! is based on \verb!UNICODE!
defined macro. As both \verb!UNICODE! and \verb!_UNICODE! are often used/omitted
together, \verb!_TCHAR! and \verb!TCHAR! are interchangable, except. 
\begin{itemize}
  \item  TCHAR is defined in
header file \verb!<winnt.h>! (or defined in \verb!<tchar.h>! when compile the
code with \verb!/Ze! option) to use in Unicode-aware application. You get winnt.h when you
include \verb!<windows.h>!
 
% It is defined as
% \verb!wchar_t! during Unicode build (i.e. \verb!UNICODE! macro defined).

  \item \verb!_TCHAR! is defined conditionally in header file \verb!<tchar.h>!
  only.
\begin{verbatim}
#ifdef _UNICODE
  #define _TCHAR wchar_t
#else
  #define _TCHAR char
#endif
\end{verbatim}

NOTE: The string utilities APIs starting with \verb!_tcs! operate on
\verb!_TCHAR! units (not bytes).
\end{itemize}

\verb!<tchar.h>! defines
\begin{verbatim}
_TCHAR
_TSCHAR    //signed TCHAR
_TUCHAR    //unsigned TCHAR
\end{verbatim}
If you programs in Windows, you are working with WinAPI
(Sect.\ref{sec:WinAPI}). 

\subsection{in .NET: Char}

In Managed C++ or C++/CLI, we can use data from .NET
\begin{verbatim}
#using <mscorlib.dll>
using namespace System;

Char     structure, represent a character as UTF-16 code unit 
         (since .NET 1.1)          
\end{verbatim}
\url{https://msdn.microsoft.com/en-us/library/system.char(v=vs.71).aspx}



\section{String in C}
\label{sec:string}
\label{sec:C_string}

The term {\it string} refers to an ordered sequence of characters whose type
is one of the given type described in Sect.\ref{sec:C_character} and
Sect.\ref{sec:C++_character}. As the atomic element in a string is a character
(Sect.\ref{sec:character}). So, a good understanding how the computer store a
character symbol in the memory (internal representation) or in the disk
(external representation) is important - Sect.\ref{sec:internal-representation}.

Depending on the number of bytes to store one character, we have
narrow-character string (Sect.\ref{sec:narrow-character-string}) and
wide-character string (Sect.\ref{sec:wide-character-string}).

For operations related to
input/output from console, read Sect.\ref{sec:IO_console}.

\subsection{ narrow-character string (char *)}
\label{sec:narrow-character-string}

In C language standard, a string is actually an array of narrow character, i.e.
\verb!char*! or \verb!char[buf_len]!, that ends with a NULL character (or a 0
value).
In C, to add utilities for string support, we use the header file
\verb!<string.h>! which comes with many utility functions

Example: Date and time strings: Chap.\ref{chap:date-time}
\begin{verbatim}
  static const char wday_name[][4] = {
    "Sun", "Mon", "Tue", "Wed", "Thu", "Fri", "Sat"
  };
  static const char mon_name[][4] = {
    "Jan", "Feb", "Mar", "Apr", "May", "Jun",
    "Jul", "Aug", "Sep", "Oct", "Nov", "Dec"
  };
\end{verbatim}

Using \verb!char*! is recommended for an object in memory to contain string data
(as most existing C string library will work with UTF-8 data and they assume a
string is NULL-terminated). 
\textcolor{red}{NOTE}: We can have \verb!const char*!, a pointer to a string
data whose content cannot be modified, e.g. literal string
(Sect.\ref{sec:literal-string}).


\begin{verbatim}
#include <string.h>

int strlen = 10;
char[strlen] mystring;  // fixed-sized array (not fixed-length string)

char* yourstring;
malloc(yourstring(sizeof(char)*(strlen+1));

const char* str3 = "The content here"; /* the content of str3 is fixed */
\end{verbatim}


%Notable exceptions are \verb!strchar()!.


{\bf IMPORTANT}: \verb!char* id;! here \verb!id! can be used as a pointer
pointing to the byte containing the single character, or a pointer pointing to the
character string.
\begin{verbatim}
char x;
char[10] str;
char*  ptr;
malloc(ptr, sizeof(char)* (10+1));

char* y;

y = &x; // okay
y = &str; // okay
y = ptr; // okay
\end{verbatim}

{\bf IMPORTANT}: The array size is not the same as the string length returned by
\verb!strlen()! function which returns number of characters before the NULL
character. Regardless of the size of the array, the NULL character must be right
after the last meaningful character in the array. Thus, the length of the string
is the total number of meaningful characters, counting from the beginning, upto
and not including the NULL character.


{\bf IMPORTANT}: To be resuable at different places in your code,
\textcolor{red}{it is recommended to define new type for a string that can tell the information
about the string-length}
\begin{verbatim}
typedef char String80[81];
typedef char String20[21];
\end{verbatim}
NOTE: The two numbers differ by 1 for the NULL character to be stored, yet is
not considered as part of the string.


\subsection{-- char* vs. char[]}

\begin{verbatim}
char *s = "Hello world";
\end{verbatim}
places Hello world in the read-only parts of the memory and making s a pointer
to that. So
\begin{verbatim}
s[0] = 'J'  ; //illegal
\end{verbatim}

\begin{verbatim}
char s[] = "Hello world";

char s[256] = "Hello world";
\end{verbatim}
places Hello world in the read-only parts of the memory and then copy that
content to a newly allocated memory on the stack and making s a pointer to
that memory on stack. So
\begin{verbatim}
s[0] = 'J'  ; //legal
\end{verbatim}


\subsection{-- char const* vs. char* const vs. const char*}
\label{sec:const_char}

NOTE: \verb!const! take affect on the keyword on the left-side, if present,
otherwise it takes the effect on the keyword on the right-side. So
\verb!char const *! and \verb!const char*! are the same. 
\begin{verbatim}
char *const p1="John";
char const* p2="Mary";
const char* p3="Tom";

char* const* p4 ="Part";
const char* const p5 = "Tim"; 
\end{verbatim}

\begin{enumerate}
  \item p1 is a constant pointer, i.e. p1 is not allowed to point to a different memory
address. However, the content at the memory location where p1 is pointing to is
allowed to change. At the  moment, the value is 'John'
\footnote{\url{http://www.cplusplus.com/forum/beginner/10602/}}.
\begin{verbatim}
*p1 = 'C'; // allowed
p1 = "Margarth"; // not allowed 
\end{verbatim}

  \item p2 is a mutable pointer to a memory region containg a constant string
  'Mary'.
  It means that p2 is allowed to change the address it points to. However, we're anot allowed to change the value
that p2 is initially pointing to which is the memory region having 'Mary' as
the content.
\begin{verbatim}
*p2 = 'C'; // not allowed.

p2 = "Margareth";  // allowed
*p2 = 'C'; // now it is allowed.
\end{verbatim}

  \item p3 is a mutable pointer to a memory region containing a constant string
  (like p2 above)

  \item p4 is a mutable pointer to pointing to \verb!char*const! memory, i.e. 
  the memory whose content is the fixed address to another memory region
  containg a narrow-character string.
  
It means that p4 can be changed, but (*p4) cannot be
changed, but (**p4) is allowed to change the value.

 \item p5 is a constant pointer pointing to a memory region containing a
 constant string.

\end{enumerate}

Example:
``{\it const int* vs. int* const: 

The beginner will simply not use const.

The intermediate will write const int* and, giving themselves a false sense of
security, may accidentally overwrite the pointer (for instance, the common
mistake of = instead of ==) 

The expert will know that const int* is a mutable pointer to a constant value;
int* const is a constant pointer to a mutable int (to get a constant pointer to
a constant value, use const int* const!)}''  



\subsection{ wide-character string}
\label{sec:wide-character-string}
\label{sec:wchar*}

Similar to narrow-character string (Sect.\ref{sec:narrow-character-string}), we
can use pointers or array of wide-character elements to define wide-character
string.
\begin{verbatim}
wchar_t *  str;

const wchar_t *   str;

int buf_len = 20;
wchar_t[buf_len] str;
\end{verbatim}

However, as the number of bytes to store one character is
more than one, thus to work with \verb!wchar_t*! string, a different family of
functions need to be used, which start with \verb!w!, e.g. wprintf(), wcslen().
To get the length of the wide-character string, we use
\verb!wcslen()!. The string is still NULL-terminated.

IMPORTANT: Working with \verb!wchar_t*! string requires converting to a proper
locale setting (Sect.\ref{sec:locale}). To enable UTF-8 output to disk, we can
do
\begin{verbatim}
char* locale;
locale = setlocale(LC_CTYPE, "en_ca.UTF-8")
\end{verbatim}

IMPORTANT: To read and print a wide-character string (e.g. the data is
stored in the file in UTF-8 format), we use \verb!%ls! format. 

\begin{verbatim}
#include 
#include 
#include <wchar.h>
#include 

int main() {

    wchar_t wsKey[] = L"apple\n";
    wchar_t wsInput[80];
  
    wchar_t string[100];

    setlocale(LC_ALL, "");

    printf ("Enter a string: ");
    scanf("%ls",string);

    printf("String Entered: %ls: length: %dn", string, wcslen(string));

    return 0;
}
\end{verbatim}


\subsection{ string manipulation}
\label{sec:string-manipulation-C}

\subsection{* string copy}

\textcolor{red}{\bf String copy}: In C, string are pointers to array, so to do
string copy, we cannot use '=' operator. We need to call either
\verb!strcpy()! (or its safer \verb!strncpy()!) or \verb!strdup()!.

\begin{verbatim}
strcpy(ptr2, ptr1) is equivalent to while(*ptr2++ = *ptr1++)
\end{verbatim}
which copy string to a pre-allocated memory on either heap or stack, depending
on how we declare ptr2. If the string is saved in heap
memory, it can be used in other functions. \verb!strdup()! only copy string to
memory on the heap.

Both \verb!strcpy()! and \verb!strncpy()! are C-standard. However,
\verb!strdup()! is not a ANSI C; but a POSIX.1-2001. So, some compiler
discourage use, e.g. MSVC use \verb!_strdup!. Also, it hides the memory
allocation (on the heap). What it  realy does is
\begin{verbatim}
char *strdup (const char *s) {
//More effective implementation: avoiding multiple return
    char *d = malloc (strlen (s) + 1);   // Allocate memory
    if (d != NULL)
        strcpy (d,s);                    // Copy string if okay
    return d;                            // Return new memory
}
char *strdup (const char *s) {
    char *d = malloc (strlen (s) + 1);   // Space for length plus nul
    if (d == NULL)
        return NULL;                     // No memory
    strcpy (d,s);                        // Copy the characters
    return d;                            // Return the new string
}
\end{verbatim}

Example:
\begin{verbatim}
char[20] oldstring = 'This is a string';

//this we don't need to free
char[20] newstring;
newstring = strdup(oldstring);

//this we need to free explicitly
char* newstring;
newstring = strdup(oldstring);
...
free(newstring);
\end{verbatim}
If we use assignment, \verb!str1=str2!, the pointer \verb!str1! will point to
the same place as \verb!str2! and then the current address that \verb!str1!
pointing to become unaccessible (MEMORY LEAK). 

To allocate memory on the stack, and you don't need to free memory, we can use
\verb!strdupa()! in the GNU C library. However, using \verb!strdupa()! is
considered dangerous unless your string to copy is very small. An portable
version, but longer code than using \verb!strdup! is
\begin{verbatim}
char *new = malloc(strlen(old)+1); return new ? strcpy(new, old) : 0;

//faster than using strcpy() as \0 char doesn't need to be searched again
size_t len = strlen(old) + 1; 
char *new = malloc(len); return new ? memcpy(new,old, len) : 0;
\end{verbatim}

% To make sure the string is not modified inside the function, we pass 
% \verb!const char*! as data type.
% \begin{verbatim}
% void do_something(const char *string) {
% }
% \end{verbatim}

\subsection{* concatenate: 2 strings}

\textcolor{red}{\bf To concatenate two strings}, we use \verb!strcat()!
(IMPORTANT: the two strings must not overlap in memory)
\begin{verbatim}
char * strcat ( char * destination, const char * source );
\end{verbatim}
which append the source string to the destination string. The terminating null
character in the destination is replaced by the first character in the source
string; also, the null-character is added to the end of the new string, pointed
by \verb!destination!. It also return a pointer to the ``new" destination
string. 


{\small \begin{verbatim} 
#include <stdio.h>
#include <string.h>

int main() {
  char str1[50] = "Hello ";
  char str2[] = "World";

//Option 1:
  strcat(str1, str2);
  printf("str1: %s\n", str1);

//Option 2:
  char *result = malloc((strlen(str1) + strlen(str2) + 1));
  strcpy(result, str1);
  strcat(result, str2);

  return 0;
}
\end{verbatim}}

To have more control or to avoid the situation that \verb!source! may not be
NULL-terminated, then we can explicitly specify the number of characters to copy
\begin{verbatim}
char * strncat ( char *destination, const char *source, size_t num);
\end{verbatim}
with \verb!num!=maximum number of characters to copy.

\subsection{* concatenate: >2 strings with sprintf(), snprintf(), asprintf(),
string\_format, (C++11) snprintf}
\label{sec:sprintf}
\label{sec:snprintf}
\label{sec:asprintf}

C++11:  write to a \verb!char*!, by using \verb!std::snprintf()!, and then
convert to a \verb!std::string!. First,find the desired length of the char
array; and allocate the memory to a \verb!std::unique_prt! smart pointer
\begin{lstlisting}
#include <memory>
#include <iostream>
#include <string>
#include <cstdio>

using namespace std; //Don't if you're in a header-file

template<typename ... Args>
string string_format( const std::string& format, Args ... args )
{
    size_t size = snprintf( nullptr, 0, format.c_str(), args ... ) + 1; // Extra space for '\0'
    unique_ptr<char[]> buf( new char[ size ] ); 
    snprintf( buf.get(), size, format.c_str(), args ... );
    return string( buf.get(), buf.get() + size - 1 ); // We don't want the '\0' inside
}
\end{lstlisting}

\textcolor{red}{\bf To concatenate more than two strings}: we can use
\verb!sprintf! (or better with \verb!snprintf! or \verb!asprintf!)
\begin{verbatim}
char buffer[101];
// C89: may experience the issue of buffer overrrun
//    if buffer size if not big enough to handle input
//    if the input is read-in from user
sprintf(buffer, "%s%s%s%s", "this", " is", " my", user_input_string);

// To fit into the size allocated, any thing out of the size limit
// are truncated, e.g. if size > 5, then we know truncation occurs
char *x = (char *) malloc(5 * sizeof(char));
int size = snprintf(x, 5, "%s%s%s", "12", "34", "56"); // writes "12345"

// we can also reallocate until the buffer size is big enough
char *x = (char *) malloc(BUF_LEN * sizeof(char));
int size = snprintf(x, 5, "%s%s%s", "12", "34", "56");
if(size > BUF_LEN) {
    realloc(&x,size * sizeof(char));
    snprintf(x, 5, "%s%s%s", "12", "34", "56");
}

// GNU extension: the above step can be done at once
// we don't need to pre-allocate the memory and specify the size 
// of the string. 
// IMPORTANT: the string then need to be explicitly freed
//      when no longer used. 
char *joined;

asprintf(&joined, "%s%s%s%s", "this", " is", " my", " story");
\end{verbatim}
NOTE: asprintf() and vasprintf() are GNU extensions of sprintf() and vsprintf().
The size of the buffer is determined by the system. However, we MUST check to
make sure the memory allocation actually succeeded. 
 
 

\textcolor{red}{\bf Using strn* function family}, rather than str* counterparts
is recommended, to make C program stable and secure. The reason is that
\verb!strcmp! compare the two strings the entire string down to the end until
it see the NULL characters which in some cases, you pass a string not
NULL-terminated, then it continues until it reaches the first non-accessible
memory and crashes the program; while \verb!strncmp! only compares the first
$n$ characters of the two strings e.g. to compare two string
\begin{verbatim}
#include <string.h>
int strcmp(const char *s1, const char *s2);
int strncmp(const char *s1, const char *s2, size_t n);

int result = strncmp(string1, string2, compareLimit);
\end{verbatim}
If the two strings are the same, return zero (0). Be careful here, it's zero,
not one. At first pair of mismatch characters, if this character in string1 less
than that in string2, then return negative value; otherwise, return
greater-than zero value. In C++, we can use \verb!std::string! and the
overloaded == operator.

\begin{verbatim}
char str1[] ="something";
char str2[] = "another thing";
/* In this case we know strings are null terminated. Pretend we don't. */
str1[sizeof(str1)-1] = '\0';
str2[sizeof(str2)-1] = '\0';
/* Now the following is safe. */
if (strcmp(str1, str2)) { /* do something */ } else { /* do something else */ }
\end{verbatim}

\textcolor{red}{Using string stream} (C++11): Sect.\ref{sec:stringstream}
\begin{lstlisting}
// char buf[50];
// sprintf(buf,"Pair (%d,%d) = %f\n",i,j,(double)i/j);

std::ostringstream oss1, oss2;
oss2 << "Pair (" << i << "," << j << ") = " << std::fixed << std::setprecision(1) << (double(i)/j) << std::endl;

std::string buf = oss2.str();
\end{lstlisting}

\textcolor{red}{Write a simple function} (C++11): using \verb!vsnprintf()! and
\verb!fmt!. NOTE: \verb!va_start! with a reference argument has problems on
MSVC (fails silently, and returns pointers to random memory); and a workaround
is using \verb!std::string fmt! instead of \verb!std::string& fmt! (or write a
wrapper object).
\begin{lstlisting}
#include <stdarg.h>  // For va_start, etc.

std::string string_format(const std::string fmt, ...) {
    int size = ((int)fmt.size()) * 2 + 50;   // Use a rubric appropriate for your code
    std::string str;
    va_list ap;
    while (1) {     // Maximum two passes on a POSIX system...
        str.resize(size);
        va_start(ap, fmt);
        int n = vsnprintf((char *)str.data(), size, fmt.c_str(), ap);
        va_end(ap);
        if (n > -1 && n < size) {  // Everything worked
            str.resize(n);
            return str;
        }
        if (n > -1)  // Needed size returned
            size = n + 1;   // For null char
        else
            size *= 2;      // Guess at a larger size (OS specific)
    }
    return str;
}
\end{lstlisting}

or using \verb!fmt_str! is safer and faster [but not work on some systems;and
faster depends entirely on the pre-allocation step being correct; otherwise
\verb!strcpy! renders it slower]. 
\url{https://stackoverflow.com/questions/2342162/stdstring-formatting-like-sprintf}
\begin{lstlisting}
#include <stdarg.h>  // For va_start, etc.
#include <memory>    // For std::unique_ptr

std::string string_format(const std::string fmt_str, ...) {
    int final_n, n = ((int)fmt_str.size()) * 2; /* Reserve two times as much as the length of the fmt_str */
    std::unique_ptr<char[]> formatted;
    va_list ap;
    while(1) {
        formatted.reset(new char[n]); /* Wrap the plain char array into the unique_ptr */
        strcpy(&formatted[0], fmt_str.c_str());
        va_start(ap, fmt_str);
        final_n = vsnprintf(&formatted[0], n, fmt_str.c_str(), ap);
        va_end(ap);
        if (final_n < 0 || final_n >= n)
            n += abs(final_n - n + 1);
        else
            break;
    }
    return std::string(formatted.get());
}
\end{lstlisting}

\subsection{* remove trailing blank characters}

\textcolor{red}{\bf To remove trailing blank characters}, 
\begin{enumerate}
  \item In .NET studio 2008: we use \verb!String.Trim()!
  \item We define \verb!trim()! function.
  \begin{verbatim}
  #include <ctype.h> // isspace()
  char *ltrim(char *s)
{
    while(isspace(*s)) s++;
    return s;
}

char *rtrim(char *s)
{
    char* back = s + strlen(s);
    while(isspace(*--back));
    *(back+1) = '\0';
    return s;
}

char *trim(char *s)
{
    return rtrim(ltrim(s)); 
}
\end{verbatim}
The function \verb!isspace()! consider all this are space
\begin{verbatim}
' ' (0x20)	space (SPC)

'\t' (0x09)	horizontal tab (TAB)

'\n' (0x0a)	newline (LF)

'\v' (0x0b)	vertical tab (VT)

'\f' (0x0c)	feed (FF)

'\r' (0x0d)	carriage return (CR)
\end{verbatim}
\end{enumerate}

\subsection{* search substring}

\textcolor{red}{\bf To search for substring}: To search for first occurence of a
substring
\begin{verbatim}
const char *strstr(const char *s1, const char *s2)
\end{verbatim}
which returns a pointer to the start of the word in \verb!s1! where \verb!s2! is
found (if not, return NULL).
\begin{verbatim}
if(strstr(sent, word) != NULL) {
    /* ... */
}
\end{verbatim}


\subsection{* split the string into tokens: single character delimiter}

\textcolor{red}{\bf To split string into tokens based on a single character
delimiter}:

{\bf OPTION}:
\begin{enumerate}
  \item \verb!strtok()! (4.3BSD, SVr4, POSIX.1-2001, C89, C99) is not
  thread-safe (as it uses a statis buffer while parsing), i.e.
  cannot work on two strings at the same time. 
  
  \item \verb!strtok_r()! (POSIX.1-2001) is thread-safe, but not C standard
  
  \item \verb!strtok_s! (Windows extension to \verb!strtok_r!) is thread-safe
  and have the same prototype with \verb!strtok_r()!.

NOTE: Check Sect.\ref{sec:macro-detect-OS}
\begin{verbatim}
#if defined(_WIN32) || defined(_WIN64)
/* We are on Windows */
  # define strtok_r strtok_s
#endif
\end{verbatim}
This would probably break if compiling on cygwin (as \verb!__CYGWIN__! and
\verb!__unix__!) which reports itself as windows but has posix interfaces like
\verb!strtok_r! already defined. Using something like \verb!#ifndef HAVE_STRTOK_R! (and detecting it in the build scripts) would be better.

\end{enumerate}
\url{http://linux.die.net/man/3/strtok_r}  

\begin{mdframed}
The function name means 'string tokenizer', i.e. split a string into tokens.
  Both \verb!strtok()! and \verb!strtok_r()! modify the original string (first
  argument), so making sure a copy of string is used if you want to preserve the
  original content.   It also means they cannot be used on constant strings.
  
%   Both modify original string (i.e. the first argument), i.e. make sure to have
%   a copy if you don't want the original string to be lost

\end{mdframed}

\begin{verbatim}
char * strtok ( char * str, const char * delimiters);
char * strtok_r ( char * str, const char * delimiters, char** p1_savedptr);
\end{verbatim}
\textcolor{red}{Here the delimiters can be a NULL-delimited string, but
individual characters will be used as the splitter}. 
Using these functions goes through 2 stages (for a given string): first call +
subsequence calls. The subsequent calls should be put inside a loop, and the
first call is before the loop.

% We use \verb!strtok()! or
% \verb!strtok_r()! and a loop.
% \verb!strtok_r()! also returns the p1 pointer as the third argument.

EXPLAIN: Internally, it uses a static pointer (here we call it p1) pointing to
the position for the search. In the beginning this position is
the first character of the original string str.
\begin{itemize}
  
  \item [FIRST RUN] If \verb!str! is not NULL, the static pointer internal to
  the function p1 is assigned to the first character of \verb!str!.
  Then it uses p1 to keep track of the next character to scan, until one of the
  character in the delimiter string is found or the \verb!'\0'! NULL character
  is found. It means p1 keeps changing the location during the search.
  
  If a character in the delimiter string is found, it is replaced by the
  \verb!'\0'! NULL character, and the pointer pointing to p1 is returned, then
  p1 will point to the next character, after the found delimiter.
  
\begin{verbatim}
#include <string.h>

p1_savedptr_mirror = strtok (char *str, 
                           const char* delimiters);

p1_savedptr_mirror = strtok_r (str, delimiters, 
                            char** p1_savedptr);
\end{verbatim}  

  \item [NEXT RUN] (in the loop): It continues to scan the string, starting from
  position pointed by p1, and do the similar job.
   
   For both functions, in subsequent calls after the first call, we must pass
   NULL as the first argument. This is the signal that the function uses to
   detect that it shoulds continues scanning the static buffer where a previous
   successful call to the function ended. 
\begin{verbatim}
char * strtok   (NULL, delimiters);
char * strtok_r (NULL, delimiters, char** p1_savedptr);
\end{verbatim}

% If a NULL-terminated string is passed, it is considered as working on 
%    if it should pass a new string to the static buffer or
%    continue working on the data on the static buffer

\end{itemize}

\begin{mdframed}
EXPLAIN: The first time, give s1, and in subsequent calls to
\verb!strtok_r()!, instead of passing s1, we only pass NULL. 
\begin{verbatim}
  char *hello = "Hello World, Let me live.";
  char *rest; // to point to the rest of the string after token extraction.
  char *token; // to point to the actual token returned.
  char *ptr = hello; // make q point to start of hello.
  // loop till strtok_r returns NULL.
  token = strtok_r(ptr, " ,", &rest)
  while(token != NULL) {
       printf("%s\n", token); // print the token returned.
       token = strtok_r(NULL, " ,", &rest)    
  }
\end{verbatim}

EXPLAIN:
When using \verb!strtok()!, it doesn't need the
third argument.
\begin{verbatim}
  token = strtok(ptr, " ,")
  while(token != NULL) {
       printf("%s\n", token); // print the token returned.
       token = strtok(NULL, " ,")    
  }
\end{verbatim}
\url{http://www.cplusplus.com/reference/cstring/strtok/}

\end{mdframed}

\begin{verbatim}
#include <stdio.h>
#include <string.h>

int main(void) {
  char *hello = "Hello World, Let me live.";
  char *rest; // to point to the rest of the string after token extraction.
  char *token; // to point to the actual token returned.
  char *ptr = hello; // make q point to start of hello.
  // loop till strtok_r returns NULL.
  while(token = strtok_r(ptr, " ,", &rest)) {
       printf("%s\n", token); // print the token returned.
       ptr = rest; // rest contains the left over part..assign it to ptr...and start tokenizing again.    
  }
  int i = 0;
  while(i < 5) {
    printf("%s\n", tokens[i++]);
  }
}
\end{verbatim}

{\bf IMPORTANT}: The two functions above actually modify the original string s1. 
So, do not pass string literals to them, as they are read-only, e.g. this is
wrong
\begin{verbatim}
char *token;
char **rest;

//wrong
token = strtok_r("Xin chao, cac ban!", " ,", &rest);
\end{verbatim}
How the original string is modified? Once the delimiter is found, the
corresponding delimiter character in the original string is replaced by the
'\\0' character, and then the function return the pointer to the beginning of
the string, and the current string pointing to the next character
\footnote{\url{http://www.cplusplus.com/faq/sequences/strings/strtok/}}.

IMPORTANT: Only one string can be tokenized at a time (i.e. the function is not
reentrant). It means that tokenize one string completely, then go to the next
string. Make a copy of the immutable string (e.g. \verb!const! string or string
literal) before working on it.

% \verb!strtok_r()! provides the third argument to keep the progress, and thus
% there is a second way, we pass s3.

% IMPORTANT: Here, it returns pointers to the original data. So, remember to make
% a copy in case you want to manipulate the data, to avoid modifying the original
% data.

\subsection{* split the string into tokens: string delimiter}

REMEMBER: \verb!strstr! can be used to find the location of a substring inside a
string
\begin{verbatim}
char str[] = "this is abc a big abc input string abc to split up";
char *pos = strstr(str, "abc");
\end{verbatim}
\url{http://stackoverflow.com/questions/29788983/split-char-string-with-multi-character-delimiter-in-c}

Here, we use the two-phase approach: first run and a loop for subsequent runs.

User-defined function  (not thread-safe)
\begin{lstlisting}
char *multi_tok(char *input, char *delimiter) {
    static char *string;
    if (input != NULL)
        string = input;

    if (string == NULL)
        return string;

    char *end = strstr(string, delimiter);
    if (end == NULL) {
        char *temp = string;
        string = NULL;
        return temp;
    }

    char *temp = string;

    *end = '\0';
    string = end + strlen(delimiter);
    return temp;
}
\end{lstlisting}


User-defined function (thread-safe)
\begin{lstlisting}
typedef char *multi_tok_t;

char *multi_tok(char *input, multi_tok_t *string, char *delimiter) {
    if (input != NULL)
        *string = input;

    if (*string == NULL)
        return *string;

    char *end = strstr(*string, delimiter);
    if (end == NULL) {
        char *temp = *string;
        *string = NULL;
        return temp;
    }

    char *temp = *string;

    *end = '\0';
    *string = end + strlen(delimiter);
    return temp;
}
\end{lstlisting}

Example:
\begin{lstlisting}
int main() {
    char input [] = "this is abc a big abc input string abc to split up";
    char[] delimiter = "abc";

    char *token = multi_tok(input, &s, delimiter);

    while (token != NULL) {
        printf("%s\n", token);
        token = multi_tok(NULL, &s, "abc");
    }
}
\end{lstlisting}


\subsection{* compare 2 strings}

Use \verb!<string.h>! (in C) or \verb!<cstring>! (in C++) on  C-string
\verb!char*!
\begin{enumerate}
  \item \verb!strcmp(s1, s2)! : compare two strings (binary comparison byte to
  byte, not character to character), the result of the comparison depends on the
  current locale-specific rules. To make sure it follows the locale you want,
  use \verb!strcoll()!.
  
  \item \verb!strcoll()!: compare two strings, the rule follows
  \verb!LC_COLLATE! category of the selected C-locale.
  
  \item \verb!strncmp()!: 
\end{enumerate}

Use \verb!<wchar.h>! (in C) or \verb!<cwchar>! (in C++) on wide-character
\verb!wchar_t*!: provide the same functionality like those for \verb!char*!  
except the prefix \verb!str! is replaced by \verb!wcs!, e.g. strcmp() becomes
wcscmp().
\begin{enumerate}
  \item \verb!wcscmp(wcs1, wcs2)!: 
\end{enumerate}

\subsection{* work on character and a string}

Use \verb!<string.h>! (in C) or \verb!<cstring>! (in C++) on  C-string
\verb!char*!
\begin{enumerate}
  \item \verb!strrchr()!: last occurence of a character c in string str
\end{enumerate}

Use \verb!<wchar.h>! (in C) or \verb!<cwchar>! (in C++) on wide-character
\verb!wchar_t*!: provide the same functionality like those for \verb!char*!  
except the prefix \verb!str! is replaced by \verb!wcs!, e.g. strcmp() becomes
wcscmp().

\subsection{* use a string (buffer) to transfer data}

If you have to exchange data between different machines using different encoding
scheme. To make life easier, you are recommended to use one of the following
libraries
\begin{enumerate}
  
  \item conversion between character coding: \verb!iconv()! (part of the C
  library \verb!<iconv.h>!)
  \url{http://www.gnu.org/savannah-checkouts/gnu/libiconv/documentation/libiconv-1.13/iconv_open.3.html_node}
  or the library \verb!libiconv! \url{http://www.gnu.org/software/libiconv/}
  
  
  \item  UTF-8 string: We can also use GLIB, QT when dealing with UTF-8 string with non-ASCII
characters: \url{https://developer.gnome.org/glib/}
   
   \item complicated text manipulation: An ICU library written by IBM
   (Sect.\ref{sec:ICU}): 
   \url{http://site.icu-project.org/}
   
\end{enumerate}

\url{http://www.evanjones.ca/unicode-in-c.html}

\subsection{* convert string and number}
\label{sec:convert-C-string-to-number}

To convert a string to/from a number, e.g. ``124'' to 124, 
or 124 to ``124''
\begin{enumerate}
  \item C-like function \verb!atoi()! which cannot report error upon conversion
  
  \item \verb!strtol()! which can report error
  
  \item C++11: <string> header-file
\begin{verbatim}
stoi(str, int) : string to signed integer
stol(str, long) 
stoll(str, long long)

 
stoul(str, unsigned long) : string to unsigned integer
stoull(str, unsigned long long)

stof(str, float)     : strong to float
stod(str, double)
stold(str, long double)

to_string(int)     : wchar_t
to_wstring(int)    : char*
\end{verbatim}

\end{enumerate}

\subsection{*: Convert multibyte string from/to wide-char string}


Convert from multibyte string to wide-character string

We can use \verb!mbstowcs()! function (MultiByte String TO Wide-Character
String), which returns the number of wide-characters written to \verb!dest!. The
multibyte string is pointed by \verb!src!. 
\begin{Verbatim}
 /* C language */
size_t mbstowcs (wchar_t* dest, const char* src, size_t max);

 /* C++ */
#include <cwchar>

std::size_t mbsrtowcs( wchar_t* dst,
                       const char** src,
                       std::size_t len,
                       std::mbstate_t* ps );
\end{Verbatim}
The convertion process stop until the maximum number of wide-characters to be
converted is \verb!max!, or a NULL-character is reached in \verb!src!. 


\subsection{* length of a C string}

In an ASCII NULL-terminated string (\verb!char*!), the function
\verb!strlen(str)! returns the number of 'meaningful' bytes in the string, not
including the terminating NULL byte. As each character is also 1 byte, then the
return value also means the number of character in the ASCII string.
\begin{verbatim}
#include <string.h>

size_t strlen(const char *s);
\end{verbatim}
\textcolor{red}{The IMPORTANT point is it returns the number of bytes, not
number of characters.} So, in a multibyte string (UTF-8), it also returns the
number of bytes.

In a wide-character string, i.e. all characters have the same size, then to
return the string length, we use \verb!wcslen()! (not counting the NULL-wide
character).
\begin{verbatim}
#include <wchar.h>

size_t wcslen (const wchar_t* wcs);
\end{verbatim}


Return the number of characters of a multibyte string (examining at most
\verb!max! bytes)
\begin{Verbatim}
int mblen (const char* pmb, size_t max);
\end{Verbatim}

\subsection{* loop through very characters in a C string}

C-style string (NULL-terminated)
\begin{verbatim}
char* str = "This is my string";
for(char* it = str; *it; ++it) {
    do_things_with(*it);
}

for (int i = 0; str[i] != '\0'; i++){
    do_things_with(str[i]);
}
for (int i = 0; i < strlen(str); i++){
    do_things_with(str[i]);
}

\end{verbatim}

\subsection{format a C string}
\label{sec:format-C-string}

We can use the approach in Sect.\ref{sec:sprintf}

\subsection{NULL character (string terminator)}
\label{sec:null-character}

The NULL macro is defined in \verb!<stddef.h>! or \verb!stdio.h! header files.

In C, a string is an array of character (Sect.\ref{sec:character_sets}) with
0-terminated (i.e. NULL-terminated), i.e. it can receive any of the following
values: \verb!0! or \verb!\0! (ASCII string, which is a single byte of value
\verb!0x00!); or \verb!U+0000! (UTF-8 string, which is encoded as a 2-byte
sequence \verb!0xC0! and \verb!0x80!).

String literal
\begin{verbatim}
const char *str = u8"This is a UTF-8 string.";
\end{verbatim}

{\bf \%c and \%s }: In \verb!sprintf! and \verb!printf!, to specify the format
for a single character, we use \verb!%c!, as \verb!%c! expect the argument
of type \verb!char!. \verb!%s! expect an argument of type \verb!char*!, and
print until it reaches a \verb!0! (or \verb!\0!, the same thing). In other
words, \verb!%s! expects a zero-terminated string of characters.

\subsection{* check NULL-terminated string?}

\begin{mdframed}
In Windows, avoid using the following APIs to check for NULL-terminated string
\footnote{\url{http://blogs.msdn.com/b/oldnewthing/archive/2006/09/27/773741.aspx}}.
Using IsBadXXXPtr can induce random crashes in other parts of the program which
make it harders to debug.

\begin{verbatim}
IsBadStringPtr
IsBadReadPtr
IsBadWritePtr
\end{verbatim}
\end{mdframed}

\subsection{String I/O}

\begin{Verbatim}
#include <stdio.h>
#include <wchar.h>

  wchar_t wsInput[256];
  wprintf (L"Enter a sentence: ");
  fgetws ( wsInput, 256, stdin );  /* includes newline characters */
  wprintf (L"You entered %u characters.\n",wcslen(wsInput));
  return 0;
\end{Verbatim}


\subsection{Working with Unicode}
\label{sec:string_Unicode}

C language has not been designed to work with data from file in Unicode format.
Neither \verb!char! nor \verb!wchar_t! is directly tied to Unicode.
\verb!wchar_t! size just guarantee that it can store the largest character set.


Glib library provides \verb!Glib::ustring! class which has much the same
interface as \verb!std:string! yet it contains Unicode characters encoded in UTF-8.
\begin{verbatim}
#include <glibmm/ustring.h>
\end{verbatim}
\url{https://developer.gnome.org/glibmm/unstable/classGlib_1_1ustring.html}

In Windows: 
\begin{enumerate}
  \item MFC and C-runtime support for Unicode: (1) define symbol \verb!_UNICODE!
  before compiling the code, (2) In Visual Studio, project's property page, in
  {\bf output} page of the Linker folder, set Entry Point symbol to
  \verb!wWinMainCRTStartup!, (3) use Unicode-aware string manipulation APIs with
  prefix \verb!_tcs! (e.g. \verb!_tcscpy()!)
  
  \item  \verb!BSTR! is called counted Unicode type, i.e. an interface type
used for communicating with APIs written in Visual Basic, COM, Active X,
etc.

  \item \verb!UNICODE_STRING! : a kernel data type (used as an interface to
  underlying low-level kernel APIs)
  
  \item \verb!PUNICODE_STRING! : a pointer to \verb!UNICODE_STRING! structure
  
  \item \verb!PCUNICODE_STRING! : a pointer to \verb!const CUNICODE_STRING! 
\end{enumerate}


\section{String in C++}

To represent an object with string content, we can use C-style string, i.e.
an array of characters (Sect.\ref{sec:C-style-string}). 
To make it work with Object-Oriented programming (OOP), C++ provides a string
class (Sect.\ref{sec:C++_string}).


\subsection{ C-style string}
\label{sec:C-style-string}

We can use the string as defined in C (Sect.\ref{sec:C_string}) in C++ code,
given that a proper header file is included: 
\begin{itemize}
  \item \verb!<string.h>!
  
  \item \verb!<cstring>!: 

{\bf NOTE}: \verb!cstring! is just a wrapper, it includes \verb!<string.h>!
within itself.
\footnote{\url{http://www.cplusplus.com/reference/cstring/}}

  \item \verb!<iostream!> : this file includes \verb!<cstring>! within itself
\end{itemize}
Before C++11, \verb!std::string! is not guaranteed to provide a
contiguous, writable buffer. Since C++11 this has changed so that std::string
does provide you with a contiguous buffer.

The reason to keep using \verb!char*! in C++ is 
\begin{verbatim}
- backward compatibility
- potential performance gain 
  For example, if performance is absolutely critical, a small C-array on the
  stack may be a better solution than std::string. 
- writing a program where you need absolute control over memory
allocation/deallocation

- NOTE: (Before C++11) std::string isn't guaranteed to provide you with a
contiguous, writable buffer *, so you can't directly write from a file into an std::string
However, in the event you need to do this, std::vector would still probably be
preferable to using a raw C-array.
\end{verbatim}

{\bf NOTE}: assignment is changing reference, so two pointers pointing to the
same memory location (Sect.\ref{sec:array-assignment}. You need to use
explicit copy function to do the real string copy (Sect.\ref{sec:array-copy}).

\begin{verbatim}
#include <cstring> 

/* or */

#include <iostream>
\end{verbatim}
with type \verb!char! in C++98. 

Since C95, the wide-character is supported \verb!wchar_t!, and size of
\verb!char! is guaranteed to store 8-bit coding of UTF-8, even though no support
for Unicode yet. C++11 supports Unicode encoding (UTF-8, UTF-16, and UTF-32),
with 2 new types: \verb!char16_t! and \verb!char32_t!
(Sect.\ref{sec:character_sets}).


\subsection{C++ strings: std::string, std::wstring}
\label{sec:C++_string}

\subsection{-- std::string (std::basic\_string)}
\label{sec:std::string}
\label{sec:std::basic_string}

IMPORTANT: \verb!std::string! class is the typedef for the
\verb!std::basic_string!
\footnote{\url{http://en.cppreference.com/w/cpp/string/basic_string}} templated
on \verb!char!. It only supports fixed-length character encoding scheme, i.e.
not supporting UTF-8. If you need to support variable-length encodings you could
try either 
\begin{enumerate}
  \item  ICU4C library (Sect.\ref{sec:ICU}).
  \item QString class (Sect.\ref{sec:QString}) from QT library.
\end{enumerate}

\begin{Verbatim}
template <class T, class traits = 
          class Alloc = allocator<char> >
class std::basic_string<T> { ...};

typedef std::string std::basic_string<char>
\end{Verbatim}


\verb!std::string! is the C++ version for \verb!char*! data type in C
(Sect.\ref{sec:narrow-character-string}), and it makes your code more readable
and safer than using \verb!char*!. 

%This is used in object-oriented programming (OOP).  

The header file is
\begin{itemize} 
  \item \verb!<string>!
  
  \item \verb!<iostream>!: which indeeds include \verb!<string>!.
  
% NOTE: C++ standard allow other headers to include other standard header files.
% So, if we include <iostream>, it also have <string>.
However, it's better to explicitly include <string> if we do string processing.
  
\end{itemize} 

\begin{mdframed}

The internal implementation is still using \verb!char*!.
It means the  object containing the string content once defined is fixed-size
(NOTE: this is the size of the object, not the length of the string it
contains), with an internal pointer pointing to \verb!char*!, and another member
\verb!nchars! holding the number of non-NULL characters of the string
\begin{verbatim}
class string
{
    char *buffer;
    size_t nchars;

  public:
    // interface
};
\end{verbatim}

Pre C++-11, the internal representation of \verb!std::string! is not guaranteed
to be contiguous (though many STL implementation use a contiguous memory
approach). So, when you issue this command to copy, it may fails
\begin{verbatim}
std::string s;
s.resize( strLength );  
// strLength is a size_t with the length of a C string in it. 

memcpy( &s[0], str, strLength );
\end{verbatim}

From C++11, the new standard forces the internal data representation has to be
contiguous. To enforce C++11 standard, check the compiler options.
\url{http://stackoverflow.com/questions/1986966/does-s0-point-to-contiguous-characters-in-a-stdstring}
C++11 standard says [21.4.1.5]
\begin{verbatim}
The char-like objects in a basic_string object shall be stored contiguously.

That is, for any basic_string object s, the identity &*(s.begin() + n) ==
&*s.begin() + n shall hold for all values of n such that 0 <= n < s.size().
\end{verbatim}

\end{mdframed}

The reason to switch to using \verb!std::string! is 
\begin{verbatim}
- more intuitive to use (assignment is a duplicate of memory)
- better searching, more utilities function (copy string, append to string...)
- reduced risk of segmentation fault (the internal char* array always not NULL)
- no need to worry about allocation/reallocation/deallocation memory
   the class manage these tasks automatically
\end{verbatim}

Example:
\begin{verbatim}
std::string str = "hello";
str::string str2 = str + " hi!";
\end{verbatim}
which is better than
\begin{verbatim}
const char * str = "hello";
char str2[1024];
strcpy (str2, str1);
strcat(str2, " hi!");
\end{verbatim}


% 
% To use C++ new style of string which defines \verb!std::string! class and
% related functions and operators, we use
\begin{verbatim}
#include <string>

std::string mystring("Some content");
\end{verbatim}


\subsection{-- std::wstring}
\label{sec:std::wstring}

\verb!std:wstring! is the C++ version of \verb!wchar_t*! as in C language
(Sect.\ref{sec:wchar*}). Similar to \verb!std::string!, wstring is a
\verb!std::basic_string! templated on \verb!wchar_t!
\begin{Verbatim}
typedef std::wstring std::basic_string<wchar_t>
\end{Verbatim}
\url{http://en.cppreference.com/w/cpp/string/basic_string}

\subsection{-- QString (unicode-aware)}
\label{sec:QString}

QString class is part of QT library (QTL).
When it comes to strings, QString offers much more complete functionality
compared to \verb!std::basic_string! and it is completely unicode aware.


\subsection{-- C++11: u16string, u32string}
\label{sec:u16string}
\label{sec:u32string}

C++11 standard defines 2 new character data types: \verb!char16_t, char32_t!
(Sect.\ref{sec:char16_t}), along with that the header file \verb!<string>! also
define two new classes
\begin{verbatim}
typedef std::u16string std::basic_string<char16_t>

typedef std::u32string std::basic_string<char32_t>
\end{verbatim}


\subsection{ string manipulation}
\label{sec:string-manipulation-C++}

\textcolor{red}{\bf Tokenize a string}:
You can convert from std::string to \verb!char*! and apply the methods given in
Sect.\ref{sec:string-manipulation-C}
\begin{lstlisting}
char *dup = strdup(str.c_str());
token = strtok(dup, " ");
free(dup);
\end{lstlisting}

{\bf OPTION}:
You can use \verb!BOOST::tokenizer! class which is powerful as you can define
your tokenizer function 
\url{http://www.boost.org/doc/libs/1_52_0/libs/tokenizer/tokenizerfunction.htm}
\begin{lstlisting}
struct my_tokenizer_func
{
    template<typename It>
    bool operator()(It& next, It end, std::string & tok)
    {
        if (next == end)
            return false;
        char const * del = ">=";
        auto pos = std::search(next, end, del, del + 2);
        tok.assign(next, pos);
        next = pos;
        if (next != end)
            std::advance(next, 2);
        return true;
    }

    void reset() {}
};

int main()
{
    std::string to_be_parsed = "1) one>=2) two>=3) three>=4) four";
    for (auto i : boost::tokenizer<my_tokenizer_func>(to_be_parsed))
        std::cout << i << '\n';
}
\end{lstlisting}


{\bf OPTION} (single character delimiter): You can simply use
\verb!std::istringstream! (or \verb!stringstream!) wrapping the original string, and go through the stream
to extract the content from the current position to the character before the delimiter
\begin{lstlisting}
#include <iostream>
#include <string>
#include <sstream>

std::string myText("some-text-to-tokenize");
std::istringstream iss(myText);
std::string token;
while (std::getline(iss, token, '-'))
{
    std::cout << token << std::endl;
}
\end{lstlisting}
NOTE: You can use \verb!stringstream! as well, but it is better to indicate that
the strinng stream is for input only with \verb!istringstream!
(Sect.\ref{sec:stringstream})
\url{http://en.cppreference.com/w/cpp/string/basic_string/getline}

{\bf OPTIOIN} (a string delimiter): Use 
\verb!.find()! and \verb!.substr()! to get a token
\begin{lstlisting}
std::string s = "scott>=tiger";
std::string delimiter = ">=";

/* single token only */
std::string token = s.substr(0, s.find(delimiter)); // token is "scott"

/* multiple tokens */
size_t pos = 0;
std::string token;
while ((pos = s.find(delimiter)) != std::string::npos) {
    token = s.substr(0, pos);
    std::cout << token << std::endl;
    /* erase the original string from starting position to the position
       at the end of the delimiter string
    */
    s.erase(0, pos + delimiter.length());
}
std::cout << s << std::endl;

/* NOTE:
 find(const string& str, size_t pos = 0) - return the position
          of the first occurrence, or return 'npos' value to indicate nothing
          found
 
 substr(size_t pos = 0, size_t n = npos) - return
*/
\end{lstlisting}
Sect.\ref{sec:npos} discuss \verb!std::string::npos! value.
\url{http://stackoverflow.com/questions/14265581/parse-split-a-string-in-c-using-string-delimiter-standard-c}
{\bf OPTION} (a string as delimiter): Write your own function, which returns all
the tokens in a vector container
\begin{lstlisting}
void split(const string& str, const string& delim, vector<string>& parts) {
  size_t start, end = 0;
  while (end < str.size()) {
    start = end;
    while (start < str.size() && (delim.find(str[start]) != string::npos)) {
      start++;  // skip initial whitespace
    }
    end = start;
    while (end < str.size() && (delim.find(str[end]) == string::npos)) {
      end++; // skip to end of word
    }
    if (end-start != 0) {  // just ignore zero-length strings.
      parts.push_back(string(str, start, end-start));
    }
  }
}
\end{lstlisting}
or (just the order is different)
\begin{lstlisting}
void Tokenize(const string& str,
                      vector<string>& tokens,
                      const string& delimiters = " ")
{
    // Skip delimiters at beginning.
    string::size_type lastPos = str.find_first_not_of(delimiters, 0);
    // Find first "non-delimiter".
    string::size_type pos     = str.find_first_of(delimiters, lastPos);

    while (string::npos != pos || string::npos != lastPos)
    {
        // Found a token, add it to the vector.
        tokens.push_back(str.substr(lastPos, pos - lastPos));
        // Skip delimiters.  Note the "not_of"
        lastPos = str.find_first_not_of(delimiters, pos);
        // Find next "non-delimiter"
        pos = str.find_first_of(delimiters, lastPos);
    }
}


/* USE IT */
   vector<string> tokens;

    string str("Split me up! Word1 Word2 Word3.");

    Tokenize(str, tokens);

    copy(tokens.begin(), tokens.end(), ostream_iterator<string>(cout, ", "));
\end{lstlisting}
\url{http://oopweb.com/CPP/Documents/CPPHOWTO/Volume/C++Programming-HOWTO-7.html}

\subsection{* concatenate: 2 strings}

\begin{verbatim}
std::string a("Hello ");
std::string b("World");

//Option 1
std::string c = a + b;

//Option 2
a.append(b);

//Option 3
a += b;
\end{verbatim}

\subsection{* concatenate: > 2 string}

\textcolor{red}{\bf To concatenate two or more string}, in C++
\begin{verbatim}
//Multiple strings
//they all must be std::string to use more than 2
s += string("Hello world, ") + string("nice to see you, ") + string("or not.");

// use string-stream
#include <sstream>
#include <string>

std::stringstream ss;
ss << "Hello, world, " << myInt << niceToSeeYouString;
std::string s = ss.str();

//or
std::string s = static_cast<std::ostringstream&>(std::ostringstream().seekp(0)
<< "HelloWorld" << myInt << niceToSeeYouString).str();
\end{verbatim}

NOTE: When using stringstream, to reuse the stream, we should use
(Sect.\ref{sec:stringstream-reset})

\subsection{* check character in string}

\textcolor{red}{To check for the existence of a character in a string}. We can
use this to search for string with whitespaces only
\begin{verbatim}
char a = ' ';

std::string str;

if (str.find_first_not_of(a) != std::string::npos) 
{
  // a is in str
}
\end{verbatim}

\subsection{* create an array of strings}

\textcolor{red}{\bf To create an array of string}, in C++
\begin{enumerate}
  \item Using array
\begin{verbatim}
std::string strArr[] = {"Monday", "Tuesday", "Wednesday", "Thursday", "Friday",
"Saturday", "Sunday"};
\end{verbatim}
  \item Using STL vector (RECOMMENDED)
\begin{verbatim}
std::vector<std::string> strVec;
strVec.push_back("Monday");
strVec.push_back("Tuesday");
...
\end{verbatim}

In C++11, we can do like array for STL vector
\begin{verbatim}
std::vector<std::string> strVec = {"Monday", "Tuesday", "Wednesday", "Thursday",
"Friday", "Saturday", "Sunday"};
\end{verbatim}

\end{enumerate}


\subsection{* replace a substring in a string with a different content}

\textcolor{red}{\bf To replace a substring with something else}:
\begin{verbatim}
string replaceInString(std::string str, std::string)
\end{verbatim}

\subsection{* remove trailing blank characters}

\textcolor{red}{\bf To remove trailing blank characters}, in C++ [NOTE: This is
a special case of the function above where the substring is white-space]
\begin{enumerate}
  \item Boost library has implemented
  \verb!trim()!\footnote{\url{http://www.boost.org/doc/libs/1_38_0/doc/html/string_algo/usage.html}}
  which can trim anything, not only spaces, e.g. \verb!trim_left_if()!.
\begin{verbatim}
 #include <boost/algorithm/string.hpp>
using namespace std;
using namespace boost;
  
string str1=" hello world! ";
string str2=trim_left_copy(str1);   // str2 == "hello world!     "
string str3=trim_right_copy(str2);  // str3 == "     hello world!"
trim(str1);                         // str1 == "hello world!"
// Use trim_right() if only trailing whitespace is to be removed.

string phone="00423333444";
// remove leading 0 from the phone number
trim_left_if(phone,is_any_of("0")); // phone == "423333444"

// trim on both end
trim(str);

// remove white-space in between
// however it's slower than 'remove_if' approach
erase_all(str, " ");        
\end{verbatim}
  
  \item Right-trim
  \begin{verbatim}
std::string s;
s.erase(s.find_last_not_of(" \n\r\t")+1);
  \end{verbatim}
  
  \item Right-trim, Left-trim and Trim-from-both-ends. The definition of white
  spaces depend on locale being used, e.g. locale (VS2005, en) means tabs,
  spaces, carriage returns, newlines, vertical tabs and form feeds are trimmed
  \begin{verbatim}
#include <algorithm> 
#include <functional> 
#include <cctype>
#include <locale>

// trim from start
static inline std::string &ltrim(std::string &s) {
   s.erase(s.begin(), std::find_if(s.begin(), s.end(), 
      std::not1(std::ptr_fun<int, int>(std::isspace))));
   return s;
}

// trim from end
static inline std::string &rtrim(std::string &s) {
   s.erase(std::find_if(s.rbegin(), s.rend(), 
      std::not1(std::ptr_fun<int, int>(std::isspace))).base(), s.end());
   return s;
}

// trim from both ends
static inline std::string &trim(std::string &s) {
   return ltrim(rtrim(s));
}  
  \end{verbatim}
IMPORTANT: We need to put the definition of \verb!static inline! function in the
header files. Otherwise, you will get the warning
\begin{verbatim}
warning: inline function ... used but never defined
\end{verbatim}
Also, to use \verb!find_if()! we need to include <algorithm>. Otherwise, you'll
get an error
\begin{verbatim}
error: 'find_if' is not a member of 'std'
\end{verbatim}
  
  \item Left-trim and Right-trim: but only works for spaces and tabs, i.e. not
  work for newline, line feed, carriage return.
\begin{verbatim}
// trim trailing spaces
size_t endpos = str.find_last_not_of(" \t");
if( string::npos != endpos )
{
    str = str.substr( 0, endpos+1 );
}

// trim leading spaces
size_t startpos = str.find_first_not_of(" \t");
if( string::npos != startpos )
{
    str = str.substr( startpos );
}
\end{verbatim}
  \item Trim both ends using stringstream. However, it cann't
  deal with string with internal white-spaces
\begin{verbatim}
std::stringstream trimmer;
trimmer << str;
str.clear();
trimmer >> str;
\end{verbatim}

  \item Remove white spaces in between (spaces is defined by in the function
  \verb!std::isspace!)
\begin{verbatim}
static std::string &remove_spaces(std::string &str)
{
#if defined(__CXX_EXPERIMENTAL_CXX0X__) || (__cplusplus >= 201103L)
  str.erase(std::unique(str.begin(), str.end(), [](char a, char b){return a==' '
  && b==' ';}), str.end());
#else
  stringstream line( str );
  stringstream newline;
  while( line >> str ) { // read the words, ignoring whitespace
        newline << " " << str;
  }
  str = newline.str();
  ltrim(str);
}
#endif
}
\end{verbatim}
 
  \item Remove all white-spaces
\begin{verbatim}
//However this doesn't compile on standardc-conforming implementations
// due to locale-taking overloads of std::isspace
  str.erase(remove_if(str.begin(), str.end(), std::isspace), str.end());
  //the solution is to use ::isspace  
  str.erase(remove_if(str.begin(), str.end(), ::isspace), str.end());

// or
std::string::iterator end_pos = std::remove(str.begin(), str.end(), ' ')
str.erase(end_pos, str.end())
\end{verbatim}
  
  \item 
\begin{verbatim}
string choppa(const string &t, const string &ws)
{
    string str = t;
    size_t found;
    found = str.find_last_not_of(ws);
    if (found != string::npos)
    	str.erase(found+1);
    else
    	str.clear();            // str is all whitespace

    return str;
}
\end{verbatim}
\end{enumerate}
References:\url{http://stackoverflow.com/questions/216823/whats-the-best-way-to-trim-stdstring}

\subsection{* split a string into an array of tokens}

\textcolor{red}{\bf To split a string into an array}:
\begin{enumerate}
  \item Write a funciton that return a vector of string
  
Example: delimiter is any character from the string 'delimiters'

\begin{lstlisting}
void StringUtils::Tokenize(const string& str,
                           vector<string>& tokens,
                           const string& delimiters)

{/* It uses any character in the string 'delimiters'-argument as the delimiter
	*/
    // Skip delimiters at beginning.
    string::size_type lastPos = str.find_first_not_of(delimiters, 0);
    // Find first "non-delimiter".
    string::size_type pos = str.find_first_of(delimiters, lastPos);

    tokens.clear(); /* make sure the vector is empty */
    while (string::npos != pos || string::npos != lastPos) {
        // Found a token, add it to the vector.
        tokens.push_back(str.substr(lastPos, pos - lastPos));
        // Skip delimiters.  Note the "not_of"
        lastPos = str.find_first_not_of(delimiters, pos);
        // Find next "non-delimiter"
        pos = str.find_first_of(delimiters, lastPos);
    }
}
\end{lstlisting}

  \item Using istringstream

NOTE: Only work with delimiter as space

\begin{verbatim}
#include <sstream> //std::istringstream
#include <iterator> //std::istream_iterator
#include <vector> //std::vector

while (std::getline(input, line))
{
  std::istringstream iss(line);
  std::istream_iterator<std::string> begin(iss), end;
  
  //space as delimiter for iss
  // but here output to std::cout
  copy(istream_iterator<string>(iss),
       istream_iterator<string>(),
       ostream_iterator<string>(cout, "\n"));
       
  //space as delimiter for iss
  // but put to vector 'tokens'
   vector<string> tokens;
   copy(istream_iterator<string>(iss),
     istream_iterator<string>(),
     back_inserter(tokens));

  //space as delimiter for iss
  // quick way to put to vector 'tokens'
  std::vector<std::string> tokens (begin, end);  
}
\end{verbatim}

  \item Using boost

Example: delimiter is a space character (space, tab)
\begin{verbatim}
vector<string> v;
 
 //the splitter is space
boost::split(v, line, ::isspace);
\end{verbatim}

Example: delimiter is any character from the given string
\begin{verbatim}
#include <boost/algorithm/string.hpp>
std::string inputString("One!Two,Three:Four");
std::string delimiters("|,:");
std::vector<std::string> parts;
boost::split(parts, inputString, boost::is_any_of(delimiters));
\end{verbatim}
\end{enumerate}

\subsection{* compare 2 strings}

\textcolor{red}{\bf Compare 2 strings}: 
Comparing string in C++ is easy using the overloaded == operator, and copy
string in C++ can be done with \verb!=! operator.

To test an empty string, we can use
\begin{lstlisting}
if (mystr == "") {}

if (mystr.empty()) {}
\end{lstlisting}
The latter method is preferred as it is more general, which can be applied to
not only \verb!std::string!, but also \verb!std::wstring!
(Sect.\ref{sec:std::wstring}). 

\textcolor{red}{IMPORTANT}: \verb!""! is interepreted as \verb!char!, which
cannot be compared with \verb!wchar_t!.
\url{http://stackoverflow.com/questions/483337/c-is-string-empty-always-equivalent-to-string}


\subsection{* convert string to number}


\begin{enumerate}
  \item We can map C++ string to C string, and use function in Sect.\ref{sec:convert-C-string-to-number}
\begin{verbatim}
const char* s = cppstring.c_str();
\end{verbatim}
  
  \item Use fastreams
  
\begin{lstlisting}
ifastream<basic_formatters, string_reader> myString(&string);
int value;
myString >> value;
\end{lstlisting}

   \item Use istringstream (slower)
   
 \begin{lstlisting}
string str = "123";
int numb;
istringstream ( str ) >> numb;

// or	 
 
#include <sstream>
#include <string>
using namespace std;

string myStream = "45";
istringstream buffer(myString);
int value;
buffer >> value;   // value = 45 
 \end{lstlisting}
 
   \item Use \verb!Boost::lexical_cast!
 
\begin{lstlisting}
#include <boost/lexical_cast.hpp>

try {
    int x = boost::lexical_cast<int>( "123" );
} catch( boost::bad_lexical_cast const& ) {
    std::cout << "Error: input string was not valid" << std::endl;
}
\end{lstlisting}
\end{enumerate}


\subsection{* convert a std::string to char* or std::vector<char>}

If you don't want to modify the content, then 
\begin{verbatim}
char const* ca = str.c_str();
\end{verbatim}

If you want to make a new string, dynamically allocated
\begin{verbatim}
char* ca = new char[str.size()+1];
std::copy(str.begin(), str.end(), ca);
ca[str.size()] = '\0';
\end{verbatim}
and make sure to \verb!delete[]! later.

If you want to make a new string, statitically allocated
\begin{verbatim}
size_t const MAX = 80; // maximum number of chars
char ca[MAX] = {};
std::copy(str.begin(), (str.size() >= MAX ? str.begin() + MAX : str.end()), ca);
\end{verbatim}
and no need for memory free.

This is the right answer
\begin{verbatim}
std::string str = "some string" ;
char *cstr = &str[0u];
\end{verbatim}
HOWEVER< it may fails, as pre-C++11, there is no guarantee that the internal
representation for the sequence of characters is contiguous.

You can use \verb!std::vector<char>! which guarantees contiguous memory
\begin{verbatim}
std::string str = "string";
std::vector<char> chars(str.c_str(), str.c_str() + str.size() + 1u);
\end{verbatim}
or
\begin{verbatim}
vector<char> v(str.begin(), str.end());
char* ca = &v[0]; // pointer to start of vector
\end{verbatim}

\url{http://stackoverflow.com/questions/7352099/stdstring-to-char}

\subsection{copy a string}


Suppose you have an existing a string, and you want to copy the content from a
given position, say 2, to the end of the string (to indicate all the way to the
end, we use \verb!std::string::npos! value)
\begin{lstlisting}
string s = 'text';
// functions that take string subsets as arguments 
    // use npos as the "all the way to the end" indicator
    std::string s2(s, 2, std::string::npos);
    std::cout << s2 << '\n';
 
\end{lstlisting}

\subsection{format a C++ string}
\label{sec:format-C++-string}

We have a number of options
\begin{enumerate}
  \item fastformat - Sect.\ref{sec:fastformat}: type-safe, fastest, thread-safe,
  
  
  \item output the data, with appropriate format, to string stream, and finally
  convert to string (Sect.\ref{sec:stringstream})
  
  \item Boost::format: very powerful - Sect.\ref{sec:Boost::format()}
  
  \item std::snprintf (C++11) - Sect.\ref{sec:std::snprintf}  
\end{enumerate}

\subsection{* fastformat library}
\label{sec:fastformat}

\url{http://www.fastformat.org/index.html}

\subsection{* stringstream: ostringstream, istringstream}
\label{sec:stringstream}
\label{sec:ostringstream}
\label{sec:istringstream}


For string appending, instead of using \verb!std::string::append()! method, we
can use \verb!<<! operator (for output data to string stream), or \verb!>>!
operator (for reading from a string stream), and then finally convert to a
string using \verb!std::stringstream::str()! method.

\begin{lstlisting}
std::ostringstream os;
os << "Content-Type: " << contentType << ";charset=" << charset << "\r\n";
std::string header = os.str();

std::string header("Content-Type: ");
header.append(contentType);
header.append(";charset=");
header.append(charset);
header.append("\r\n");
\end{lstlisting}
Of course, using string stream takes less code.
The question is which one is faster.


We can have \verb!std::stringstream! or 
\verb!std::istringstream! or \verb!std::ostringstream! 

A \verb!stringstream! is somewhat larger, and might have slightly lower
performance -- multiple inheritance can require an adjustment to the vtable
pointer. The main difference of using istringstream or ostringstream is (at
least in theory) better expressing your intent, and preventing you from
accidentally using \verb!>>! where you intended \verb!<<! (or vice versa). OTOH,
the difference is sufficiently small that especially for quick bits of
demonstration code.
\url{http://stackoverflow.com/questions/3292107/whats-the-difference-between-istringstream-ostringstream-and-stringstream-w}



The Standard library design for streams supports much more than \verb!snprintf!
does. The design is meant to be extensible, and includes \verb!protected!
\verb!virtual! methods that are called by the publicly exposed methods. This
allows you to derive from one of the stream classes, with the assurance that if
you overload the protected method you will get the behavior you want. I believe
that a compiler could avoid the overhead of the \verb!virtual! function call,
but I'm not aware of any compilers that do.

Additionally, stream operations often use growable buffers internally; which
implies relatively slow memory allocations.





The function \verb!read()! reads 5 bytes from the stream (or until ends of
stream). The function \verb!get()! reads 4 bytes from the stream (or until it
hits the delimiter (which is \verb!\n! by default) or end of stream); it reads 4
instead of 5 as it reserves the last character for NULL-terminated string.
\begin{verbatim}
#include <iostream>

char a[5];

std::stringstream ss;

ss.read(a,5);

ss.get(a,5);
\end{verbatim}

\subsection{** reset a stringstream}
\label{sec:stringstream-reset}

There are a few options

\begin{verbatim}
ss.str("");             // more intuitive

ss.str(std::string());  // more efficient (though may be the same as above in
                        // modern compiler
       
m=std::stringstream();

std::stringstream().swap(m); // swap m with a default constructed stringstream


//RECOMMEND: add the following line as well if the first 2 options are used
ss.clear()  ; //just clear the error state of the stream, e.g. if a file stream
    // has the error state set to eofbit (end-of-file), then calling clear()
    // will set the error state back to goodbit (no error).

\end{verbatim}


\subsection{* Boost::format()}
\label{sec:Boost::format()}

A  format object is constructed from a format-string, and is then given
arguments through repeated calls to operator \%. Each of those arguments are
then converted to strings, who are in turn combined into one string, according
to the format-string.

\begin{verbatim}
cout << boost::format("writing %1%,  x=%2% : %3%-th try") % "toto" % 40.23 % 50; 
     // prints "writing toto,  x=40.230 : 50-th try"
\end{verbatim}

\subsection{* std::snprintf (C++11)}
\label{sec:std::snprintf}

GOAL: Write to a char* by using std::snprintf and then convert that to a
std::string.

\begin{lstlisting}
#include <memory>
#include <iostream>
#include <string>
#include <cstdio>

using namespace std; //Don't if you're in a header-file

template<typename ... Args>
string string_format( const std::string& format, Args ... args )
{
    size_t size = snprintf( nullptr, 0, format.c_str(), args ... ) + 1; // Extra space for '\0'
    unique_ptr<char[]> buf( new char[ size ] ); 
    snprintf( buf.get(), size, format.c_str(), args ... );
    return string( buf.get(), buf.get() + size - 1 ); // We don't want the '\0' inside
}
\end{lstlisting}
\url{http://stackoverflow.com/questions/2342162/stdstring-formatting-like-sprintf}

\subsection{Loop through very characters in the C++ string}

C++11 support range-based \verb!for! loop
\begin{verbatim}
// C++11
std::string str = "Thi is the string";
for(char& c : str) {
    do_things_with(c);
}
//NOTE: read-only purpose, we use either 
// for (char c : str)
// for (char const & c : str)

for (auto it = s.begin(), end = s.end(); it != end; ++it) {
	do_things_with(*it);
}

std::for_each(s.begin(), s.end(), [](char & c) {
	do_things_with(c);
}
\end{verbatim}

C++98
\begin{verbatim}
// Use string iterator
std::string str = ???;
for(std::string::iterator it = str.begin(); it != str.end(); ++it) {
    do_things_with(*it);
}
//NOTE: read-only purpose, we use 'std::string::const_iterator'
// instead of std::string::iterator

// Use old-fashioned for loop
for(std::string::size_type i = 0; i < str.size(); ++i) {
    do_things_with(str[i]);
}
for (int i = 0; i < str.size(); i++){
    do_things_with(str[i]);
}
\end{verbatim}

References:
\begin{enumerate}
  \item
  \url{http://stackoverflow.com/questions/9438209/c-for-every-character-in-string}
\end{enumerate}


\subsection{Intrinsic C vs. C++ strings}

Different types have been developed for character in a string
(Sect.\ref{sec:character}), both in C and C++. However, it is suggested to use
UTF-8 and then \verb!char*! is good enough. Don't use \verb!wchar_t! or any
C++11 newer type \verb!char16_t, char32_t!.

In general, there are three important things:
\begin{enumerate}
  \item an array of character:
\begin{Verbatim}
char my_str[10];

/* or */

char[10] my_str; 
\end{Verbatim}

   \item a C-string (Sect.\ref{sec:C_string}): the string must be terminated by
   a NULL character (Sect.\ref{sec:null-character}. The library to deal with
   string in this form is from \verb!<cstring>! (using in C++) or
   \verb!string.h! (in C)
\begin{Verbatim}
char * my_str;
int strlen = 10;
 
  //NOTE: potential bug
allocate(my_str, size_of(char)* (strlen));

  // SHOULD add 1 for the NULL-character
allocate(my_str, size_of(char)* (strlen+1));
\end{Verbatim}

   \item a C++-string object \verb!std::string! whose internal representation we
   don't need to know. The header file is \verb!<string>!
   (Sect.\ref{sec:C++_string})
\end{enumerate}
To convert between C-style and C++-style string, we read
Sect.\ref{sec:string_convert-C-C++}. 

A good comparison \url{http://cs.stmarys.ca/~porter/csc/ref/c_cpp_strings.html}
\begin{verbatim}
C-strings  (#include <cstring>)         C++ strings  (#include <string>)
===============================         ================================
!!!!!!!!!!!!!!!!!!!!!!!!!!!!!!!         !!!!!!!!!!!!!!!!!!!!!!!!!!!!!!!!

Declaring a C-string variable           Declaring a C++ string object
-----------------------------           -----------------------------
char str[10];                           string str;

Initializing a C-string variable        Initializing a C++ string object
--------------------------------        --------------------------------
char str1[11] = "Call home!";           string str1("Call home!");
char str2[] = "Send money!";            string str2 = "Send money!";
char str3[] = {'O', 'K', '\0'};         string str3("OK");
Last line above has same effect as:
char str3[] = "OK";
                                        string str4(10, 'x');
                                        
Assigning to a C-string variable        Assigning to a C++ string object
--------------------------------        --------------------------------
Can't do it, i.e., can't do this:       string str;
char str[10];                           str = "Hello";
str = "Hello!";                         str = otherString;

Concatenating two C-strings             Concatenating two C++ string objects
---------------------------             ------------------------------------
strcat(str1, str2);                     str1 += str2;
strcpy(str, strcat(str1, str2));        str = str1 + str2;

Copying a C-string variable             Copying a C++ string object
---------------------------             ---------------------------
char str[20];                           string str;
strcpy(str, "Hello!");                  str = "Hello";
strcpy(str, otherString);               str = otherString;

Accessing a single character            Accessing a single character
----------------------------            ----------------------------
str[index]                              str[index]
                                        str.at(index)
                                        str(index, count)
                                        
Comparing two C-strings                 Comparing two C++ string objects
-----------------------                 --------------------------------
if (strcmp(str1, str2) < 0)             if (str1 < str2)
    cout << "str1 comes 1st.";              cout << "str1 comes 1st.";
if (strcmp(str1, str2) == 0)            if (str1 == str2)
    cout << "Equal strings.";               cout << "Equal strings.";
if (strcmp(str1, str2) > 0)             if (str1 > str2)
    cout << "str2 comes 1st.";              cout << "str2 comes 1st.";
    
Finding the length of a C-string        Finding the length of a C++ string object
--------------------------------        -----------------------------------------
strlen(str)                             str.length()

Output of a C-string variable           Output of a C++ string object
-----------------------------           -----------------------------
cout << str;                            cout << str;
cout << setw(width) << str;             cout << setw(width) << str;
\end{verbatim}

\subsection{* length of a C++ string}

We can use either one of the two methods \verb!.size()! or \verb!.length()!.

\begin{verbatim}
std::string mystr("Hello!");

mystr.size(); 
mystr.length();
\end{verbatim}

\subsection{* length of a string in Windows}

Windows define the DIM() macro to return the exact number of characters,
regardless of the number of bytes for each character
\begin{verbatim}
#define DIM(x) ( sizeof((x)) / sizeof((x)[0]) )
\end{verbatim}

Example:
\begin{verbatim}
class Whatever { ... };
Whatever data[] = {
   { ... },
    ...
   { ... },
};

for(int i = 0; i < DIM(data); i++) // scan the table looking for a match 

\end{verbatim}

\section{Literal string (string literal)}
\label{sec:string_literal}
\label{sec:literal-string}

% In the computer, to define a constant string (known as literal strings) which can
% be enclosed by either a single or double quotes, depending on the programming

There are some cases that you define a fixed-content string in the program
(known at compile-time), or you read the data from file, and you want it to be
read-only (known at run-time). 
\begin{itemize}
  \item known at compile-time: you define literal string (or string literal)
  
A literal string is any string enclosed by either a single or double quotes,
depending on the programming language. In C/C++, it uses double quotes for
literal string and single quote for literal character.
A literal character is a special case of literal string, and is described in
Sect.\ref{sec:literal-character}.
  
  \item known at run-time: you pass data to \verb!const char*! variable.
\end{itemize}

Depending on how you define, the data is stored in the {\bf text segment} in the
program's object code (for literal string), which is read-only, or in the {\bf
data segment} which is a modifiable region but marked with not-modifiable flag
for \verb!const char*! data.

\subsection{raw form vs. cooked form}

Literals can be extended in both raw and cooked forms, with the exception of
string literals, which can be processed only in cooked form. 

\subsection{pointer assignment vs. string initialization}

String literals have static storage duration, and thus exist in memory for the
life of the program.  A string literal is a bit of text between single quotes or
double quotes. In C/C++, the enclosing delimiters are double quotes, e.g.
``Hello, world!'', while for a character, it uses single quotes
\begin{Verbatim}
cout << "Hello, world!" << endl;

cout << 'x' << endl;  // a single-character string
\end{Verbatim}
The data type for this string is \verb!char*!. There are different ways to
define a regular literal string: (1) pointer assignment, (2) string
initialization (using [] operator).

\begin{verbatim}
 // pointer assignment
char *szStr = "1234";

 // string initialization
char szStr[] = "Hello, world!";
\end{verbatim}


\begin{mdframed}
The important difference between using pointer assignment vs. string
initialization. 

{\it string initialization}: If we use an array, then the data is put into the
{\bf data segment}; then \verb!str! points to a location in the data segment,
with the content is the copy of the string 'Hello', and a string literal 'Hello'
is put into text segment memory. We are allowed to modify the content of the
string \verb!szStr!. The string is called a {\bf mutable string}, i.e. the
string is modifiable.

{\it pointer assignment}: If we use pointer assignment, we have the pointer pointing
to the literal string; then remember that you can access the
content using the pointer, but are not allowed to change the content of the
string. The string is called \textcolor{red}{immutable string}. If your code try
to modify it, the result is undefined behavior and you can get segmentation fault.
\begin{verbatim}
char *szStr = "Hello, C++!";
szStr[2] = 'A';      // Results undefined (Seg.error)
\end{verbatim}

To allow the compiler to detect operations that try to modify an immutable
string, we should use \verb!const char*!, then the compiler will tell you during
compilation time if you try to modify it \footnote{\url{https://www.securecoding.cert.org/confluence/display/seccode/STR05-C.+Use+pointers+to+const+when+referring+to+string+literals}}.
From C++0x standard, whenever you use pointer assignment, the compiler must
issue an error message and as you to use \verb!const char*!
\begin{verbatim}
const char * bird = "wren";  //requirement in C++11
\end{verbatim}
\end{mdframed}



NOTE: String literals placed side-by-side to initialize a character array are
concatenated during compilation, e.g.
\begin{verbatim}
char szStr[] = "Hello,"  " world!"; // yield a single string
\end{verbatim}

A {\it multiline regular string literal} can be created using either
\begin{verbatim}
// indentation doesn't matter
const char* text = "Four score and seven years "
        "ago, our forefathers brought forth "
        "upon this continent a new nation.";
        
// indentation here affect the string content
const char *text2 = "Here, on the other hand,  \
span several lines, \
quoting each line's \
This works,	but you can't indent.";

// we can use the trick
// However, to add comma (,) to the string
// we need to enclose within paranthesis or quotes.
// and compiler warning is created as well
#define MULTI_LINE_STRING(a) #a
const char *text = MULTI_LINE_STRING(
  Using this trick(,) you don't need to use quotes.
  Though newlines and     multiple     white   spaces
  will be replaced by a single whitespace.
);

// consume everything between the parantheses
// consecutive whitespaces are replaced by a single space
#define MULTILINE(...) #__VA_ARGS__
\end{verbatim}
but in this case we don't have the pointer pointing to the string literal.


NOTE: As double quotes is used to mark the beginning or the end of the string,
to add double quotation mark in the content of the literal string, we need to
use \verb!\"!. Using a single quotation mark in a string is okay, without using
backslash. To add a single backslash character in the literal string, use
\verb!\\!.
\begin{verbatim}
char *szStr = "1234\"\\ hello";
\end{verbatim}
szStr is the pointer pointing to the memory address in the text segmen.

% yields the (single) string "Hello, world!", i.e. similar to
% \begin{verbatim}
% char szStr[] = "Hello, world!";
% \end{verbatim}


\begin{mdframed}
It's important to know that using escape character (Sect.\ref{sec:escape_char})
in string literal should be with careness.
\begin{verbatim}
char szStr1[] = "\01" "23";
char szStr2[] = "\0123";
\end{verbatim}
which can gives different representation in memory.
\begin{verbatim}
"\01" "23" --> \01 | 2 | 3 | \0 | (1 block = 1 byte)
"\0123"    --> \012| 3 | \0
\end{verbatim}
\end{mdframed}

\subsection{C++03: regular string literal}
\label{sec:regular-string-literal}

C++03 offers 2 kinds of string literal 
\begin{itemize}
  \item contains within double quote:
  
  The result is a null-terminated array of type \verb!const char!

  \item contains within \verb!L""!
  
  The result is a null-terminated array of type \verb!const wchar_t!
  (Sect.\ref{sec:wchar_t})
  
\end{itemize}
None of them offers support for any unicode encoding in C++03, only there is a
change in the definition of \verb!char! type in C++03 (Sect.\ref{sec:char}).

You can use an un-named literal string by passing directly the literal string to
a function
\begin{verbatim}
my_func("hello world");

void my_func(const char* str)
{
  // working with 'str' without modifying it
}
\end{verbatim}

You can also define a name, i.e. a variable, whose content is a literal string
by using the type \verb!const char[]!
\begin{verbatim}
const char[] str = "hello world";
\end{verbatim}
When we define a regular string literal, we actually define a pointer pointing
to a memory.

So, when you write two overload function
\begin{verbatim}

void doSomethingWith(const void* i) { cout << "void*" << endl; };
void doSomethingWith(double d)        { cout << "double" << endl; };
void doSomethingWith(const string& s) { cout << "string" << endl; };

// and pass a string literal
doSomethingWith("foo");
\end{verbatim}
\verb!const void*! is the only type in our example than can point directly to
the \verb!const char[]!, so that one is picked. 
\url{http://blog.knatten.org/2012/05/11/a-string-literal-is-not-a-string/}

\subsection{-- Literal string wide-character (C++03)}

Since C++03, it offers 2 kinds of string literal:
\begin{enumerate}
  \item ``\ldots'' produce a string literal of type \verb!const char!.
  \item L''\ldots'' produce a null-terminated array of type \verb!const wchar_t!
\end{enumerate}
However, neither of them work wells with UTF-8, UTF-16 or any kinds of
Unicode. See C++11 supports.

\begin{verbatim}
L'x'

L"this is a string"
\end{verbatim}

\subsubsection{Literal string (Unicode, C++11)}

C++11 defines new integral data type to handle Unicode characters, that you need
to prefix the literal string with \verb!u8! or \verb!u! or \verb!U!
(Sect.\ref{sec:C++_character})
\begin{verbatim}
const char[] = u8"I'm a UTF-8 string."
const char16_t[] = u"This is a UTF-16 string."
const char32_t[] = U"This is a UTF-32 string."
\end{verbatim}


References:
\begin{enumerate}
  \item \url{http://en.cppreference.com/w/cpp/language/string_literal}
\end{enumerate}

\subsection{C++11: regular string literal}
\label{sec:regular-string-literal-C++11}

Since C++11, it supports 3 Unicode encodings: UTF-8, UTF-16, and UTF-32.
\begin{verbatim}
u8"I'm a UTF-8 string."  // whose type is 'const char[]'
u"This is a UTF-16 string."  //   type is 'const char16_t[]'
U"This is a UTF-32 string."  //   type is 'const char32_t[]'
\end{verbatim}
which uses two new types \verb!char16_t! and \verb!char32_t!
(Sect.\ref{sec:char16_t}).

NOTE: We can also insert Unicode codepoint directly into the literal string,
e.g. \verb!\u2018! with the number after \verb!\u! is HEXadecimal number
(16-bit) and \verb!\U! (for 32-bit code points).
%\subsection{-- Regular string literal}

\subsection{-- Raw string literal (C++11)}
\label{sec:raw_string_literal-C++11}


C++0x has planned to add raw string literal
(Sect.\ref{sec:raw-string-literal}) to the language standard
\footnote{\url{http://www.open-std.org/jtc1/sc22/wg21/docs/papers/2006/n2053.html}}.
It becomes official in C++11.

NOTE: In general case, the begin is marked with \verb!''delimiter(! and end with
\verb!)delimiter''!; with \verb!delimiter! is any string upto 16 characters
(including an empty string) that we use as delimiter.
In an empty delimiter, everything between \verb!''(! and \verb!)''!
is treated as part of the string, including the backslash \verb!\! character. 

\begin{verbatim}
R"(The String Data \ Stuff " )"
R"delimiter(The String Data \ Stuff " )delimiter"
\end{verbatim}
Here, everything between ``( and )'' is part of the string, i.e. we don't need
to be use backslash for \verb!\\! and \verb!``!. Also, we're flexible to use
``splitter( and )splitter'', rather than ``( and )''. In the second example, the delimiter is
the word '{\it splitter}'. It can be any string up to 16 characters in length,
including the empty string, with no space, no control character and no
\verb!'\\'! character. This allows us to put ( and ) inside the string.

\subsection{in Windows}

In Windows, to make your code Unicode-awareness (compile whether Unicode is
enabled or not), you should use TCHAR as the data type, which is mapped to the proper C-lang
intrinsic data type, and TEXT() macro (or \verb!T()! or \verb!_T()!) to wrap the
literal string.
\begin{verbatim}
TCHAR *automessage = TEXT("This message can be either ASCII or UNICODE!");
       // using TEXT() is the same as using T() or _T()
#ifdef UNICODE
  #define TEXT(t) L##t
  #define _T(t) L##t
  #define T(t) L##t
#else
  #define TEXT(t) t
  #define _T(t) t
  #define T(t) t
#endif       
\end{verbatim}

\begin{mdframed}
{\bf Microsoft specific}: the maximum length for a string literal is 2,048
bytes (both for type \verb!char[]! and \verb!wchar_t!). Type \verb!wchar_t! means 2-byte for
each character, including the terminal character '\verb!\0!'; and we need to add
'L' preceding the literal string.
\begin{verbatim}
wchar_t wszStr[] = L"1a1g";
\end{verbatim}
\end{mdframed}


\textcolor{red}{CString} data type:
\begin{verbatim}
#include <atlstr.h>

CString s = "This is a test";     // 8-bit only
CStringA s = "This is a test";    // 8-bit characters will work in Unicode app*
CString s = L"This is a test";    // Unicode only
CStringW s = L"This is a test";   // Unicode characters will work in 8-bit app*
CString s = _T("This is a test"); // Unicode-aware
CString s("This is a test");      // 8-bit only (ANSI)

CStringA s("This is a test");     // 8-bit characters will work in Unicode app*
CStringW s(L"This is a test");    // Unicode characters will work in an 8-bit app*
CString s(_T("This is a test"));  // Unicode-aware


   // This implicitly call MultiByteToWideChar operation of
   // CString constructor to convert, at run-time, the narrow-character string
   // to 16-bit 
CString s = "This is a test"; 

// NOTE:
CString gray("Gray");
CString cat("Cat");
CString graycat = gray + cat;  // ANSI string

CString gray(_T("Gray"));
CString cat(_T("Cat"));
CString graycat = gray + cat; // Unicode string
\end{verbatim}


% Code to works on ANSI compilation
% \begin{verbatim}
% \end{verbatim}

% Code that is Unicode-aware, using \verb!_T()! macro
% \begin{verbatim}
% \end{verbatim}

% NOTE: 
% \begin{verbatim}
%   // non-Unicode
% #define _T(x) x // non-Unicode version
% 
%   // Unicode
% #define _T(x) L##x // Unicode version    
% \end{verbatim}



\section{String in Windows C++}

\subsection{in Windows}
\label{sec:string_Windows}


Instead of using C-lang intrinsic data types, Visual C++ in Windows defines a
number of data types to use (CString, basic\_string, System.String) and a number
of macros that can make your code Unicode-awareness. In Windows, with string
processing, there are three concepts: ANSI string, MultiByte-character string
(MBCS) and Unicode string. By default, ANSI string is used.
\begin{itemize}
  \item CString : MFC applications only
  \item CAtlString, CAtlStringA, and CAtlStringW: MFC appliction, with or
  without C-runtime support.
    
  \item CStringT: use in CRT applications.
  \item System.String: use managed C++/CLI applications.
  \item  
\end{itemize}
To add more features to the string class, you can create a subclass of
\verb!CStringT!. 

To use MBCS, enable the macro \verb!MBCS! when compile the code.

To enable Unicode support in an MFC application, it requires
the program to be compiled with \verb!UNICODE! macro defined before including
\verb!<windows.h>! header file.
\begin{verbatim}
#define UNICODE
#include <windows.h>
\end{verbatim}

To enable Unicode support in a C-runtime application, you uses a different macro
\verb!_UNICODE! (with the underscore). This, if defined, determine how the
string manipulation APIs starting with \verb!_tcs! (in \verb!<tchar.h>!) are
mapped to normal versions or wide-character versions. Typically, you are
expected to define both, and include the two header files
\begin{verbatim}
#define UNICODE
#define _UNICODE
#include <windows.h>
#include <tchar.h>
\end{verbatim}

As two macros are expected to be used at the same time, if you write a code
that use Unicode, you are suggested to put these at the beginning
\begin{verbatim}
#ifdef UNICODE
  #ifndef _UNICODE
    #define _UNICODE
  #endif
#endif
#ifdef _UNICODE
  #ifndef UNICODE
    #define UNICODE
  #endif
#endif
\end{verbatim}


\subsubsection{CString, CAtlString (MFC, native projects)}
\label{sec:CString}

\verb!CString! is a useful data type being used in MFC applications, making it
easier to do string manipulation than C-lang intrinsic data types. If you write
an MFC application, you should always use \verb!CString! type which you need to
include the header-file \verb!<atlstr.h>!. CAtlString is another option to use
in an MFC application, with or without CRT support.

The macro \verb!UNICODE! (or \verb!_UNICODE!) maps the data type to either
\verb!CStringW! class (16-bit) or \verb!CStringA! class (8-bit). IMPORTANT:
Using CStringA and CStringB explicitly is available only in VS.Net version, not
in Visual Studio 6.
\begin{itemize}
  \item CStringA class wrap \verb!char! type, thus support ANSI string or MBCS
  string. 
  
  \item CStringW class wrap \verb!wchar_t! type, thus support Unicode string.
\end{itemize}
The stored character data for a CString object is kept in a CStringData object.
It can generate the data using a NULL-terminated C-style string; yet it doesn't
retain the NULL character in the stored character data. The length information
is stored instead.

The only time you need to do conversion is when your code interface (e.g. call)
with the APIs that requires other string data types. A CString object contains 3
values: a pointer to a buffer, a count of the valid characters in the buffer,
and a buffer length. By default, you know nothing about the size of the buffer
length as it is handled internally, so you can do string manipulation easily.
Instead, it is less efficient




String concatenation is very straight forward with CString
\begin{verbatim}
CString gray(_T("Gray"));
CString cat(_T("Cat"));
CString graycat = gray + cat;
\end{verbatim}

IMPORTANT: string concatenation operator + is defined as an overloaded operator
on two objects: (1) at least one is CString, (2) maximum one is LPCSTR.
\begin{verbatim}
  // OKAY
CString graycat = CString("Gray") + CString("Cat");
CString graycat = CString("Gray") + "Cat";
CString graycat = CString(_T("Gray")) + _T("Cat");


//ERRORs (not allow two LPCSTR objects with operator +)
CString graycat = "Gray" + "Cat";
CString graycat("Gray" + "Cat");
\end{verbatim}

\subsubsection{CStringT (CRT projects)}

\verb!CStringT! is used in C run-time (CRT) application. CStringT is an alias of
\verb!CString! \footnote{\url{http://www.flounder.com/cstring.htm}}, yet it's
better as the character type can be one of the 3 options: \verb!char!,
\verb!wchar_t! or \verb!TCHAR!
\footnote{\url{http://msdn.microsoft.com/en-us/library/5bzxfsea.aspx}}.

\subsubsection{System.String (C++/CLI projects)}
\label{sec:System.String}

\verb!System::String! is used in managed-code (C++/CLI) projects
(Sect.\ref{sec:managed_C++}).

\subsubsection{\_bstr\_t}
\label{sec:_bstr_t}

\verb!_bstr_t! type
 
 \url{http://msdn.microsoft.com/en-us/library/ms235631.aspx}
 
\subsubsection{CComBSTR}
\label{sec:CComBSTR}

\subsubsection{basic\_string}
\label{sec:basic_string}

\verb!basic_string! 

\subsubsection{TCHAR*, LPCSTR (WinAPI)}
\label{sec:TCHAR}

WinAPI also define several new types: PSTR, LPCSTR, LPTSTR, LPCTSTR, LPWTFISALL
Most kernel APIs want LPCTSTR parameters.  Because the (LPCTSTR) operator is
defined for CString (Sect.\ref{sec:CString}), the compiler will automatically
invoke the conversion.

\begin{verbatim}
#define  LPSTR          char*
#define  LPCSTR         const char*

#define  LPWSTR         wchar_t*
#define  LPWCSTR        const wchar_t*

#define  LPTSTR         TCHAR*
#define  LPCTSTR        const TCHAR* 
\end{verbatim}
NOTE: LP = long pointer, C = constant, W = wide, T = tchar, STR = string.

It's very common that programs on Windows are developed using WinAPI
(Sect.\ref{sec:WinAPI}) which defines several macros to make it easier handling
narrow character and wide-character string. To switch between using \verb!char*!
or \verb!wchar_t*! strings, WinAPI use \verb!UNICODE! and/or \verb!_UNICODE!
macro (usually both). This is how to use in a Windows program
\footnote{\url{http://www.cplusplus.com/articles/2w6AC542/}}
\begin{verbatim}
#define UNICODE   // must be before the windows.h
#include <windows.h>
\end{verbatim}
or
\begin{verbatim}
#define UNICODE
#define _UNICODE
#include <windows.h>
#include <tchar.h>
\end{verbatim}

Some header files automatically define both at the same time if once is used 
\footnote{\url{http://en.wikibooks.org/wiki/Windows_Programming/Unicode}}
\begin{verbatim}
#ifdef UNICODE
  #ifndef _UNICODE
    #define _UNICODE
  #endif
#endif

#ifdef _UNICODE
  #ifndef UNICODE
    #define UNICODE
  #endif
#endif 
\end{verbatim}

The reason we need to use both \verb!UNICODE! and \verb!_UNICODE! as
\verb!UNICODE! is used by Windows headers, and \verb!_UNICODE! is used by
C-runtime/MFC headers
\url{http://blogs.msdn.com/b/oldnewthing/archive/2004/02/12/71851.aspx}
\begin{verbatim}
The plain versions without the underscore affect the character set the Windows
header files treat as default. So if you define UNICODE, then GetWindowText will
map to GetWindowTextW instead of GetWindowTextA, for example. Similarly, the
TEXT macro will map to L"..." instead of "...".   

The versions with the underscore affect the character set the C runtime header
files treat as default. So if you define _UNICODE, then _tcslen will map to
wcslen instead of strlen, for example. Similarly, the _TEXT macro will map to
L"..." instead of "...".   
\end{verbatim}

To make the code that can be build for ANSI or for Unicode, we use \verb!TCHAR!
data type, i.e. if we use \verb!#define UNICODE!, it
will be mapped to  \verb!wchar_t!; otherwise it maps to \verb!char!. However,
there is some issues that cannot be resolved using TCHAR, especially with string
literal
\begin{verbatim}
const char*    a = "Foo";
const wchar_t* b = L"Bar";  // <-- note the L.  That makes it wide. 
\end{verbatim}
To get around this problem, WinAPI provides some macros \verb!_T()! and
\verb!T()! and \verb!TEXT()! that all do the same thing. This codes can be
compiled in either ANSI or Unicode properly
\begin{Verbatim}
const TCHAR*   d = _T("foo");  // works in both Unicode and ANSI builds
\end{Verbatim}
This is how the macro is defined
\begin{Verbatim}
#ifdef UNICODE
  #define TEXT(t) L##t
  #define _T(t) L##t
  #define T(t) L##t
#else
  #define TEXT(t) t
  #define _T(t) t
  #define T(t) t
#endif
\end{Verbatim}

Again, how to use a single function that accepts different argument type
(\verb!char*! or \verb!wchar_t*!) depending on ANSI or Unicode use ? Untill
C++11, C programming language doesn't have molymorphic ability, i.e. no support
for function overloading. For example, we need two versions for a function to
delete a file
\begin{verbatim}
#ifdef UNICODE
#define DeleteFile DeleteFileW
#else
#define DeleteFile DeleteFileA
#endif

\end{verbatim}
To make it easier, WinAPI defines macro \verb!DeleteFile()! which maps to either
one of the below functions
\begin{verbatim}
DeleteFile   <-  Takes a TCHAR string (LPCTSTR)
DeleteFileA  <-  Takes a char string (LPCSTR), ANSI version
DeleteFileW  <-  Takes a wchar_t string (LPCWSTR), Unicode version

DeleteFile( _T("myfile.txt") );   
\end{verbatim}


WinAPI also defines macros for struct that have character string member, e.g. OPENFILENAME
structure. 
\begin{verbatim}
OPENFILENAME  <-  has TCHAR strings
OPENFILENAMEA <-  has char strings
OPENFILENAMEW <-  has wchar_t strings
\end{verbatim}

IN SUMMARY:
\begin{enumerate}
  \item Use TCHAR for characters and \verb!TCHAR*! for C string.
  
  \item Use \verb!std::basic_string<TCHAR>! instead of \verb!std::string! for
  C++ string.
  
\begin{verbatim}
typedef std::basic_string<TCHAR> tstring; 
\end{verbatim}  

  \item Put all string literal in macros \verb!_T(), T()! or \verb!TEXT()!
  whenever working with WinAPI. Don't use with other libs, e.g. standard lib is
  not Unicode friendly.
  
  \item Replace standard lib functions (e.g. strcpy, strcar, \ldots) with
  \verb!tstring! member functions, e.g. 
\begin{verbatim}
_tcscpy, _tcscat, ...
\end{verbatim}
  
  \item Avoid using C or C++ style type cast for
  string (Sect.\ref{sec:type-cast})
  
  \item When reading from file or writing text data to file, DON'T use
  \verb!TCHAR!, make sure you use the type that you know the size.  If you write
  data out, you should use Unicode encoding. When you have to use explicitly
  \verb!char! or \verb!wchar_t!, and you want to copy to \verb!TCHAR!, you need
  to do one by one character, i.e. write your own copy function to do that.
  
\begin{Verbatim}
// this function copies a char C string to a TCHAR C string:
void ustrcpy(TCHAR* dst, const char* src)
{
  while(*src)
  {
   *dst = *src;
   ++dst;
   ++src;
  }
  *dst = *src;
}

/* or you can write template */
template <typename T, typename TT>
void ustrcpy( T* dst, const TT* src )
{
  //.. same as above
}
\end{Verbatim}
  
\end{enumerate}

\subsection{UNICODE\_STRING}


This is a Window special data type for kernel APIs.
\begin{verbatim}
typedef struct _UNICODE_STRING {
    USHORT Length;
    USHORT MaximumLength;
    PWSTR Buffer;
} UNICODE_STRING, *PUNICODE_STRING;
\end{verbatim}



\section{String in .NET}

In Managed C++ or C++/CLI, we can use data from .NET, i.e. convert \verb!System::String! to \verb!wchar_t*! or \verb!char*!

This always returns a wide Unicode string pointer because CLR strings are internally Unicode. 
\url{https://msdn.microsoft.com/en-us/library/d1ae6tz5.aspx}

\begin{verbatim}
// compile with: /clr

#include < vcclr.h >

using namespace System;
String ^str = "Hello";

pin_ptr<const wchar_t> wch = PtrToStringChars(str);
  printf_s("%S\n", wch);
  
  
 // Conversion to char* :
   // Can just convert wchar_t* to char* using one of the 
   // conversion functions such as: 
   // WideCharToMultiByte()
   // wcstombs_s()
   // ... etc
size_t  sizeInBytes = ((str->Length + 1) * 2);
char    *ch = (char *)malloc(sizeInBytes);
 
  size_t convertedChars = 0;
  
   err = wcstombs_s(&convertedChars, 
                    ch, sizeInBytes,
                    wch, sizeInBytes);     
\end{verbatim}

\section{Covert string between C++ and C-style}
\label{sec:string_convert-C-C++}

\subsection{* const char* from std::string}

% In previous C++ standard, the implementation of \verb!.c_str()! member
% function is allowed to return anything, 
% including a \verb!char*! to the internal buffer of a C++-string, we use
% \verb!std::

In C++98, \verb!.c_str()! returns to the \verb!const char*! to the internal
buffer data of the std::string object. Even though we can use \verb!char*! to
get access (pointing) to the first character in the buffer data, any
non-\verb!const! method applied to \verb!s2! will 
\begin{verbatim}
using namespace std;

string s = "Hello world";  /* locate on stack */
  /* s2 point to the location on stack */
char* s2 = s.c_str(); /* okay, but be careful 
                         that s2 will be invalidated once a
                         non-const method is used on 's' or
                         's' going out-of-scope
                      */

const char* s3 = s.c_str(); /* recommended */

  /* location on heap */
string* s = new string("Hello, mr. heap...");
\end{verbatim}

From C++11, both \verb!.c_str()! and the newer member function
\verb!.data()! (C++11) can get directly access to the internal data object

\begin{verbatim}
/* C++98: const char* c_str() const; */

/* C++11: const char* c_str() const noexcept; */

std::string cppstr = "hello";


char* str = cppstr.c_str(); /* we cannot modify str */

/* never do this
char* pointer=(char*)str.c_str();
*/
\end{verbatim}
This returns the pointer pointing directly to the internal array currently used
by {\it std::string} object to store the characters. Do not try to modify the
content using \verb!char*! as it is expected to be a const data.

\textcolor{red}{\bf IMPORTANT}: any modification to the string directly to
\verb!std::string! can invalid the returned \verb!char*! pointer, as the internal memory may be
reallocated to another position, and \verb!char*! pointer does not know about
that to point to the new location.
\url{http://stackoverflow.com/questions/17402980/what-does-c-str-method-from-string-class-returns}

So, we should make a copy if we want to do something without affecting
the {\it std::string} object. The member function is safe even for empty string.
\begin{verbatim}
//const char* c_str() const;

#include <cstring>
#include <string>

int main() {
std::string myfile("Data.txt");

if (filetest(myfile.c_str(), S_IFREG) != 0) {
  printf("File %s do not exist", myfile.c_str());
}

std::string str ("Please split this sentence into tokens");

char * cstr = new char [str.length()+1];
// copy to a new location
std::strcpy (cstr, str.c_str());
\end{verbatim}

To pass \verb!std::string! to a function that needs \verb!char*! (a writable
copy), we do either
\begin{enumerate}
  \item Non-standard C/C++ function, a Posix function \verb!strdup()!:
  \begin{verbatim}
  char* writable=strdup(str.c_str());
  \end{verbatim}
  
  \item or
\begin{verbatim}
std::string str;
char * writable = new char[str.size() + 1];
std::copy(str.begin(), str.end(), writable);
writable[str.size()] = '\0'; // don't forget the terminating 0

// don't forget to free the string after finished using it
delete[] writable;
\end{verbatim}
\end{enumerate}
However, the approach is not exception safe, as any thing that stop the program
between 'new' call and 'delete' can cause memory leak, as nothing free the
memory automatically. 

%\subsection{* const char* from std::string}

To pass \verb!std::string! to a function that needs \verb!const char*!, we do
\begin{verbatim}
std::string str;
const char * c = str.c_str();
\end{verbatim}

\subsection{* std::string from const char*}

To get C++-string of a C string equivalent, we should use \verb!const char*!,
NOT \verb!char *!, as string literal cannot be modified
(Sect.\ref{sec:string_literal})

\begin{verbatim}
const char* str="Hello, world!";
std::string mycpp_str(str);

//or concatenate an existing C++-string
std::string mycpp_str = cpp_str_old + std::string(str);
\end{verbatim}
 

\subsection{*: char* (TCHAR*) vs. CString}

Example: one string
\begin{verbatim}
char * p = "This is a test";

char * p = _T("This is a test");

LPTSTR p = _T("This is a test");

TCHAR * p = _T("Gray");
\end{verbatim}

now we convert to CString
\begin{verbatim}
CString s = p;
CString s(p);

p = _T("Cat");
s += p;

  // or using Format() method of CString
CString s;
s.Format(_T("%s is equal to %s, valid data"), parm1, parm2);  
\end{verbatim}

\url{http://www.flounder.com/cstring.htm}


Example: a CString object can be casted to an object of LPCTSTR type (const
char*)
\begin{verbatim}
CString s("GrayCat");
   // type cast
LPCTSTR p =  s;
\end{verbatim}


Example: get access to the internal pointer of CString object using
\verb!GetBuffer()! method, and then \verb!ReleaseBuffer()! method.
\verb!GetBuffer()! method can accept zero or one parameter.
\begin{itemize}
  \item one parameter (a number): return a pointer to the buffer, and it
  guarantees the size of the buffer is the same as that number (in characters)
  [NOTE: not bytes, as CString is Unicode-aware]
\begin{verbatim}
LPTSTR p = s.GetBuffer(1024);
\end{verbatim}  

  \item no parameter: just return the pointer to the buffer (you need to make
  sure not to modifying the size of the buffer). The pointer is actually
  pointing to a read-only memory region. If you really modify the memory
  pointed by the pointer, you will get error.
  
\end{itemize}
\begin{verbatim}
CString s(_T("File.ext"));
  
LPTSTR p = s.GetBuffer();  // once you do this
      // DO NOT use any CString method, until you call ReleaseBuffer()
      // to maintain the integrity of the data
      
LPTSTR dot = strchr(p, '.'); // OK, should have used s.Find...
if(p != NULL)
    *p = _T('\0');
    
    
s.ReleaseBuffer();  // what is does: recompute the length
\end{verbatim}

Example: BAD CODING
\begin{verbatim}
CString s(...);
LPTSTR p = s.GetBuffer();
//... lots of things happen via the pointer p
int n = s.GetLength(); // BAD!!!!! PROBABLY WILL GIVE WRONG ANSWER!!!
s.TrimRight();         // BAD!!!!! NO GUARANTEE IT WILL WORK!!!!
s.ReleaseBuffer();     // Things are now OK
int m = s.GetLength(); // This is guaranteed to be correct
s.TrimRight();         // Will work correctly
\end{verbatim}

\subsection{* exception-safe approach}

The approaches above are not exception safe, as any thing that stop the
program between 'new' call and 'delete' can cause memory leak, as nothing free the
memory automatically. 

There are two options to provide exception safe:
\begin{enumerate}
  \item \verb!std::vector!: which manage the memory for you
  \begin{verbatim}
std::string str;
std::vector<char> writable(str.size() + 1);
std::copy(str.begin(), str.end(), writable.begin());

// get the char* using &writable[0] or &*writable.begin()
  \end{verbatim}
  
  \item \verb!boost::scoped_array!: which need Boost library, and automatically
  delete memory upon going out of scope
  \begin{verbatim}
std::string str;
boost::scoped_array<char> writable(new char[str.size() + 1]);
std::copy(str.begin(), str.end(), writable.get());
writable[str.size()] = '\0'; // don't forget the terminating 0

// get the char* using writable.get()

// memory is automatically freed if the smart pointer goes 
// out of scope
  \end{verbatim}
\end{enumerate}


\section{<string.h>, <cstring>, and <string>}
\label{sec:header-file-comparison-string}

In C, you uses \verb!<string.h>! header file.


To use C++ string, we use \verb!<string>! header file.


To use C-string in C++, we can choose between \verb!<string.h>! or
\verb!<cstring>!. However, you need to knows the difference between them.
\begin{itemize}
  \item In <string.h>: some C types are defined as typedef
  (Sect.\ref{sec:typedef}), e.g. \verb!wchar_t! is a typedef.
  
  Using this in C++ prevents things like template specialization from working on
  those types. 
 
\begin{lstlisting}
typedef int Foo;
template<typename T> struct Bar {};
template<> struct Bar<int> {};
template<> struct Bar<Foo> {};
\end{lstlisting}
here \verb!Foo! and \verb!int! are considered the same because of typedef, so
you get errors
\begin{verbatim}
main.cpp:7:19: error: redefinition of 'Bar<int>'
template<> struct Bar<Foo> {};
                  ^~~~~~~~
main.cpp:5:19: note: previous definition is here
template<> struct Bar<int> {};
                  ^
1 error generated.
\end{verbatim}
  
  \item In <cstring>:
  Using \verb!<cstring>! is better which makes those types defined as
  typedef in <string.h> into real types.
  
  C++ also allows overloading, and <cstring> specifies some overloades to C
  functions to allow a function to return a pointer to non-const data, if the
  input argument is a pointer to a non-const data; wherease C functions in
  <string.h> takes and returns only pointers to const. 
  
\end{itemize}


\verb!<string.h>! places all the identifiers in the global namespace, while
\verb!<cstring>! put them inside the \verb!std! namespace.
So, when you use 
\begin{verbatim}
#include <string.h>

memcpy(...)
\end{verbatim}
but
\begin{verbatim}
#include <cstring>

std::memcpy(...)
\end{verbatim}
or 
\begin{verbatim}
#include <cstring>

using namespace std;
memcpy(...)
\end{verbatim}

NOTICE: These files can be implicitly included, once the code include another
standard header file, e.g. iostream.

In C++11, in the standard (20.9.14.6 and 7):
the content of \verb!<cstring>! is the same as the standard C library header
\verb!<string.h>! with the exception that the function \verb!memchr()! is
replaced by two versions (one accepts const data and the other is not)
\begin{verbatim}
const void* memchr(const void* s, int c, size_t n);
void* memchr( void* s, int c, size_t n);
\end{verbatim}
(both have the same behavior as the original destination)

\url{http://stackoverflow.com/questions/8380805/difference-between-string-h-and-cstring}

\section{Convert string from/to number}

The faster way is C-style, and the safer way is C++-style. For C++ string, we
just convert to C-style string \verb!std::string().c_str()! member function,
then use the following APIs.

\subsection{from integer or floating-point}

\begin{framed}
On Windows CE derived platforms, there are no iostreams by default. The way to
go there is preferaby with the \verb!_itoa<>! family, usually \verb!_itow<>!
(since most string stuff are Unicode there anyway).
\end{framed}

C-style: use \verb!itoa()! function. However, it's not a part of the
language standard, and thus is not portable
\begin{verbatim}
#include <stdlib.h>

	
\end{verbatim}
C-style: Use \verb!sprintf()! or \verb!snprintf()! to convert from number to
string.
\begin{verbatim}
char str[12];
int num = 3;
sprintf(str, "%d", num); // str now contains "3"

//to avoid buffer overflow
snprintf(str, sizeof(str), "%d", num);
\end{verbatim}

Write your own \verb!itoa()! function
\begin{verbatim}
#include <string>
string itoa(int a)
{
    string ss="";   //create empty string
    while(a)
    {
        int x=a%10;
        a/=10;
        char i='0';
        i=i+x;
        ss=i+ss;      //append new character at the front of the string!
    }
    return ss;
}
\end{verbatim}

C++-style: The third option is to use \verb!std::stringstream!.
\textcolor{red}{IMPORTANT: If you uses parallelism with multiple threads, there
may be potential locking at various locale information during formatting. To
avoid this, use sprintf() instead.}
\begin{verbatim}
#include <sstream>
   #include <string>
   template<typename T>
   T StringToNumber(const std::string& numberAsString)
   {
      T valor;

      std::stringstream stream(numberAsString);
      stream >> valor;
      if (stream.fail()) {
         std::runtime_error e(numberAsString);
         throw e;
      }
      return valor;
   }
   
  int main() {
    double number= StringToNumber<double>("0.6");
  }
\end{verbatim}
or
\begin{verbatim}
int myint = 10;

std::stringstream str;
str << myint;

std::string result;
   str >> result;
//or
   result = str.str();   
\end{verbatim}


Another option is to use \verb!Boost.Lexical_Cast!. However,
\verb!boost::lexical_cast! throw exception so you should prepare to deal with it
when you pass invalid value
\begin{verbatim}
#include <boost/lexical_cast.hpp>
#include <iostream>
#include <string>

int main() {
    std::string str("0.6");
    double dub = boost::lexical_cast<double>(str);
}
\end{verbatim}
or
\begin{verbatim}
template<typename to, typename from>
to lexical_cast(from const &x) {
  std::stringstream os;
  to ret;


  os << x;
  std::string x(os.str());
  
  os >> ret;

  return ret;  
}
\end{verbatim}

Boost.Format which can does a lot more thing than simple conversion
\footnote{\url{http://www.boost.org/doc/libs/1_36_0/libs/format/doc/format.html}}
\begin{verbatim}
int i = 10;
std::string result;

result = str(boost::format("%1%", i));

fastformat::fmt(result, "{0}", i);
fastformat::write(result, i);
\end{verbatim}


C++11 has \verb!std::to_string()!
\footnote{\url{http://en.cppreference.com/w/cpp/string/basic_string/to_string}}
\begin{verbatim}
std::string str = std::to_string(50);
\end{verbatim}

C++11: The fourth option is to use STL with \verb!std::stod! and
\verb!std::to_string!. 
\begin{verbatim}
include <string>
include <cstring>
include <cstdlib>

std::string  s  = "0.6"
std::wstring ws = "0.7"
double d  = std::stod(s);
double dw = std::stod(ws);
\end{verbatim}



\subsection{to integer}


\begin{verbatim}
#include <stdlib.h>     /* atoi */

// to 4-byte integer
int atoi (const char * str);

// to long integer
long int atol ( const char * str );
  // parsing the string and interpreting it 
  // as a number being represented of a specified 'base'
  // 'endptr' will point to the next character in the string 'str'
  // after the number
long int strtol (const char* str, char** endptr, int base);

// to unsigned long integer
unsigned long int strtoul (const char* str, char** endptr, int base);

\end{verbatim}

In C++11, we can use \verb!std::to_string()!. Prior to that, we can use
stringstream
\begin{verbatim}
#include <string>
#include <sstream>

#if defined(__CXX_EXPERIMENTAL_CXX0X__) || __cplusplus >= 201103L
	std::string s = std::to_string(5);
#else
	int i = 5;
	std::string s;
	std::stringstream out;
	out << i;
	s = out.str();
#endif
\end{verbatim}


Example:
\begin{verbatim}
 char szNumbers[] = "2001 60c0c0 -1101110100110100100000 0x6fffff";
  char * pEnd;
  long int li1, li2, li3, li4;
  li1 = strtol (szNumbers,&pEnd,10);
  li2 = strtol (pEnd,&pEnd,16);
  li3 = strtol (pEnd,&pEnd,2);
  li4 = strtol (pEnd,NULL,0);
\end{verbatim}

\subsection{to floating-point number}

If the conversion fails, then it return 0.0. If the correct value is out of
range of representable values, a positive/negative \verb!HUGE_VAL! is returned,
and \verb!errno! is set to \verb!ERANGE! [NOTE: \verb!errno! is the macro that
keep the last error number, of of the possible value is ERANGE].
\begin{verbatim}
#include <stdio.h>      /* printf, NULL */
#include <stdlib.h>     /* strtod */

// str must contains only numeric value
double atof (const char* str);

// if 'str' has number and then alphabet characters
// then use this to extract the first part containing the number
double strtod (const char* str, char** endptr);
\end{verbatim}
In the second function, it extract the number, and put \verb!endptr! pointing to
the next non-numeric character in the string (very useful when parsing a string
with multiple values, separated by white-spaces etc.). This function is
very useful.

After the white-spaces are discarded from the front of the string, the string of
characters must be
\begin{enumerate}
  \item C90 (C++98): 
\begin{verbatim}
[+/-] digits [.] digits [{e/E} [+/-] digits]
\end{verbatim}
  \item C99/C11 (C++11): the string can has more than one form
\begin{verbatim}
// form 1
[+/-] digits [.] digits [{e/E} [+/-] digits]

// form 2
[0x/0X] hexa-digits [.] [{p/P} [+/-] hexa-digits]

// form 3
INF/INFINITY

// form 4
NAN or NAN{sequence-characters}
\end{verbatim}
with \verb!{sequence-character}! can be alphanumeric or the under score
(\verb!_!).
\end{enumerate}

NOTE: []=optional, \verb!{e/E}! represents the exponential part; \verb!{p/P}!
represents the power of 2 exponent. We don't have \verb!{! or \verb!}! in the
string.

Example:
\begin{verbatim}
char szOrbits[] = "365.24 29.53";
  char* pEnd;
  double d1, d2;
  d1 = strtod (szOrbits, &pEnd);
  d2 = strtod (pEnd, NULL);
  
\end{verbatim}


\subsection{in Windows}

Using \verb!CString! class data type, we can use \verb!Format! method easily. We
don't have to worry about the buffer size

\begin{verbatim}
CString s;
  // decimal number to string
  // using _T() for Unicode-aware
s.Format(_T("The total is %d"), total);

  // string to integer
  // use with Unicode-aware code
  // the macros _tcstoul() and _ttoi()
CString hex = _T("FAB");
CString decimal = _T("4011");
ASSERT(_tcstoul(hex, 0, 16) == _ttoi(decimal));
\end{verbatim}

One radix to another, e.g. hex digits to integer: The answer is strtoul,
wcstoul, or better still, \verb!_tcstoul!.
\begin{verbatim}

ULONG strtoul(LPCSTR ptr, LPSTR * endptr, int base)
ULONG wcstoul(LPCWSTR ptr, LPWSTR * endptr, int base)
ULONG _tcstoul(LPCTSTR ptr, LPTSTR * endptr, int base)

These functions expect an input string of the form

[whitespace] [{+ | -}] [0 [{ x | X }]] [digits]
\end{verbatim}
The \verb!whitespace! can be space or tab (and is ignored during the
conversion). The \verb!base! value can be any integer from 2 to 36, or 0. If
base=0, then the special rule is used
\begin{verbatim}
If the first digit is 0 and the character which follows it is not 'x' or 'X',
then the number is interpreted as if base were specified as 8.  

If the first digit is 0 and the character which follows is 'x' or 'X', then the
'0x' or '0X' is ignored as input and the remainder of the number is interpreted
as if base had been 16.  Otherwise, it is interpreted as if base were 10.    
\end{verbatim}


The macro \verb!_ttoi()! becomes \verb!_atoi()! (ANSI code) or
\verb!_wtoi()! (Unicode-code). Use \verb!_tcstoul()! for unsigned-conversion to any radix
(e.g. 2,8, 10 or 16), or \verb!_tstoul()! for signed-conversion to any radix.

IMPORTANT: string to double (this is a real pain before Visual Studio 2005)
\begin{verbatim}
  // Before VS6
  // use T2A() macro to convert to narrow-character string first
USES_CONVERSION
CString s = _T("123.45");
double d = atof(T2A(s));   
 
 // From Visual Studio 2005 (VS6), a wide-character version has been added
double wtof(const wchar_t * string);
  // and the Unicode-aware versio is _ttof()
CString s = _T("123.45");
#ifdef _MSC_VER < 1300
  USES_CONVERSION
  double d = atof(T2A(s));
#else
double d = _ttof(s);
#endif

\end{verbatim}

\section{Special values in std::string}

\subsection{npos}
\label{sec:npos}

\verb!std::string::npos! is a special value equal to the maximum value
representable by \verb!size_type!.
It is generally used to indicate the end of the string.
Example: when a function is expecting an index of a character inside the string,
and if this \verb!npos! value is returned, it means no character is found.

\begin{lstlisting}
  // string search functions return npos if nothing is found
    std::string s = "test";
    if(s.find('a') == std::string::npos)
        std::cout << "no 'a' in 'test'\n";
\end{lstlisting}
\url{http://en.cppreference.com/w/cpp/string/basic_string/npos}