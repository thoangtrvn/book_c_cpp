\chapter{Intel MKL}
\label{chap:Intel_MKL}

Intel MKL is designed to run on multiple processors and O/Ss. To work with
different compilers and libraries, it provides different interfaces
(Sect.\ref{sec:LinkLine-Advisor}).

The current version (11.1) supports a number of different packages
\begin{enumerate}
  \item BLAS: provide basic linear algebra subprograms (consider as the {\it
  de factor } APIs for linear algebra libraries). There are three levels: Level
  1 (vector operations), Level 2 (matrix-vector), Level 3 (matrix-matrix).
  
This is the building block for many other libraries: LAPACK, LINPACK, MatLab,
NumPy, R, Mathematica.
  
NOTE: \verb![D]GEMM! is the general matrix multiplication, D=double, and is
the target for optimization due to its widely use in solving problems. 

Other implementations optimized for different hardwares: ATLAS (portable
self-optimizing BLAS), GotoBLAS. To run on GPU, Nvidia provided CUDA BLAS

Other implementations to balance between speed and its ease of use: Armadillo
(using template class), uBLAS (part of Boost library, using advanced C++
features) \url{http://en.wikipedia.org/wiki/General_Matrix_Multiply}




  \item LAPACK: linear algebra package 

NOTE: LINKACK is designed to run on vector computers with shared memory. LAPACK
does the same thing, yet is designed to exploit caches. LAPACK needs to use BLAS

  \item sparse BLAS: 
  \item Extended EigenSolver
  \item PARDISO (Intel MKL): based on PARDISO 2006, a thread-safe,
  high-performance, robust and memory efficient, easy to use for solving large
  sparse matrix (symmatrix and unsymmetric linear systems of equations) on {\it
  shared-memory} and {\it distributed-memory} systems.
  
NOTE: PARDISO has many featuers not available here.
\footnote{\url{http://www.pardiso-project.org/}}
  
  \item FFT: fast-fourier-transform (optimized for Intel AVX-512, improved
  performance when length is not power of 2 on Intel MIC)
  
  \item VML:
  \item VSL RNG: random number generators (MRG32K3A, and MT2203 BRNGs)
\end{enumerate}
\url{https://software.intel.com/en-us/articles/intel-math-kernel-library-documentation}

\section{Link-Line Advisor}
\label{sec:LinkLine-Advisor}


\url{https://software.intel.com/en-us/articles/intel-mkl-link-line-advisor/}


\section[SDL]{Single-Dynamics Library (Intel MKL 10.3+)}
\label{sec:SDL}

When a code is compiled into library, it can be statically link to the program
that use it, which makes the program bigger; or it can be compiled as shared
library, and can be loaded, when needed, by one or more programs that
dynamically link to the library. If we have multiple programs, which all link to
the same library, then dynamic loading will help reduced the
memory.\footnote{\url{http://stackoverflow.com/questions/311882/what-do-statically-linked-and-dynamically-linked-mean}}

In high performance computing, it is simpler and preferred to statically link.
When one node has mainly one process running at one time, the advantage of
dynamic loading (i.e. less memory use) is not available. Secondly, scientific
codes tend not so sophisticated in terms of language features or program design,
i.e. rarely needs to use plugin modules. Third, dynamic libraries make the
program less portable across operating systems, which typically requires some
tools to detect the missing libraries for automatic installation of the dynamic
libraries, which can sometimes create more problems than they
solve.\footnote{\url{http://scicomp.stackexchange.com/questions/1673/what-does-static-dynamic-and-single-dynamic-linking-mean}}
\textcolor{red}{On some HPD systems, static linking is REQUIRED, e.g. IBM
BlueGene/L, as the reduced O/S doesn't support dynamic linking}. 

\subsection{Problem 1}

The size of a primitive integer type varies from system to system. There
different {\bf interface layers} to choose: LP64, LLP64, ILP64.
\begin{Verbatim}
Type	ILP64	LP64	LLP64
int	64	32	32
long	64	64	32
pointer	64	64	64
long long	64	64	64
\end{Verbatim}
Windows 64-bit uses LLP64, while Solaris and Unix use LP64 platfoms. 


\textcolor{red}{In Intel MKL 9.1 and older, there is only one interface
available: LP64.} After that, Intel MKL use ILP64 for 64-bit integer type, and
LP64 for indexing 32-bit integer type, which are compiled into separate libraries
\begin{verbatim}

    libmkl_intel_lp64.a or libmkl_intel_ilp64.a for static linking
    libmkl_intel_lp64.so or libmkl_intel_ilp64.so for dynamic linking

\end{verbatim}
ILP64 interface supports (1) large data arrays ($> 2^{31}-1$ elements), (2)
enable compiling the Fortran code with \verb!-i8! compiler option. To simplify
this, i.e. linking only a single library which can be mapped to LP64 or ILP64
easily, Intel provided Single-Dynamics Library.

\subsection{Problem 2}


\subsection{Why SDL?}

\textcolor{red}{Intel MKL is suggested to use dynamic linking, support different
compilers and interfaces, different OpenMP* implementations (serial and multiple
threads), wide range of processors}, which is important to resolve the problem
above.
\begin{enumerate}
  \item Interface layer
  \item Threading layer
  \item Computational layer
\end{enumerate}
Traditionally, depending on which option for each layer being used, we need to
use a different compiling option. To know which library to link to, typically
we check in the order (1) choose the interface library, (2) the threading
library for that interface layer, and (3) which computational library

{\bf Single Dynamics Library} is an Intel-specific term which help reduce the
problem mentioned above when using dynamic libraries. To simplify the linking
process, Intel MKL packages the dynamic libraries into a single meta-library
\verb!libmkl_rt.so! which need to be linked when compiling your program

\begin{verbatim}
icc <yourcode.c> -lmkl_rt
\end{verbatim}

\url{http://scc.ustc.edu.cn/zlsc/chinagrid/intel/mkl/mkl_userguide/GUID-87821148-338B-4022-8C90-F24C866F2878.htm}


\section{Issues and workaround solution}

\subsection{11.1}

{\bf Update 3}:
\begin{enumerate}
  \item crash on 64-bit AMD Jaguar processor (Sect.\ref{sec:AMD_Jaguar}): try to
  reduce optimization from AVX to SSE4.2 by setting
\begin{verbatim}
MKL_CBWR=SSE4_2
\end{verbatim}

 
\end{enumerate}