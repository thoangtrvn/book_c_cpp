%%
%% GenerateData.tex
%% Login : <hoang-trong@hoang-trong-laptop>
%% Started on  Thu Oct  1 12:07:44 2009 Hoang-Trong Minh Tuan
%% $Id$
%% 
%% Copyright (C) 2009 Hoang-Trong Minh Tuan
%%

\chapter{Generate Data}
\label{chap:generate-data}

The Mersenne Twister algorithm (Sect.\ref{sec:PRNG_Mersenne-Tiwstter}) is
available in C++ language since C++11 standard.

In early version of C++, the header file \verb!cstdlib! (C++) or \verb!stdlib.h! provides function for PRNGs
\begin{itemize}
  \item \verb!RAND_MAX! - a macro that extends to an integer constant
  
  \item \verb!std::rand()! - a function returning a PRN in the interval \verb!(0, RAND_MAX)!
  
  \item \verb!std:srand()! - pass a seed to the PRNG used by \verb!std:rand()!
\end{itemize}
The algorithms underlying even this minimal support have typically been unspecified, and
hence their use has historically been nonportable. 
\url{http://www.open-std.org/jtc1/sc22/wg21/docs/papers/2013/n3551.pdf}

\section{C++11: <random>}
\label{sec:C++11_random}


C++11 provides a new header file for PRNG 
\begin{lstlisting}
#include <random>
\end{lstlisting}
In C++11, there is a clear difference between the {\bf engine} and the {\bf distribution}.
\begin{itemize}
  \item the {\bf engine}'s role is to return unpredictable (random) bits, ensuring the probability (or likelihood) of obtaining
  a 0 bit is always the same as that of obtaining a 1 bit.
  
The purpose of the engine is to serve as a source of randomness. The sequence would be in uniform distribution.  
  
  \item the {\bf distribution}'s role is to convert the bits in uniformly distributed sequence into the number of proper distribution.
\end{itemize}
The terms engine and distribution are applied not only to types meeting certain requirements specified in the C++11
standard, but also to objects of such types.

\subsection{Create an engine}

The engine controls the sequence of PRNG.
The two engines initialized of the same state always return the same two sequence of PRNGs.
Such replication can be very useful while debugging a program or reproducing research, for example.

To escape replication, the engine's seed must vary from run to run
\begin{itemize}
  \item traditional approach: use system clock, \verb!time(0)!, or access to system-specific resource, \verb!/dev/urandom!
  
  \item C++11 approach: use \verb!std::random_device! as the seed
\end{itemize}

There are two ways to initialize an engine: implicit vs. explicit (which we can
also provide a seed). The type of the seed must be the same as (or convertible
to) the type of the values produced by the engine.


\begin{verbatim}
#include <random>

std::default_random_engine e1; // implicitly default-initialized
static std::default_random_engine e{};  // explicit default-initialized

std::default_random_engine e3{13607};

std::random_device rdev{};
std::default_random_engine e{rdev()};
\end{verbatim}

An engine's state can be reset any time after it has been initialized
\begin{lstlisting}
void f( std::default_random_engine & e4, std::random_device & rd )
{
  e4.seed(13607); // e4 state now as if initialized by 13607

  e4.seed(); // now as if default-initialized

  e4.seed(rd()); // now as if initialized via the device-obtained seed
}
\end{lstlisting}

List of all engines:
\begin{itemize}
  \item light-weight use:   \verb!std::default_random_engine!
  \item knowledge use: 
\begin{itemize}
  
  \item \verb!std::mt19937!, \verb!std::mt19947_64!: mersenne twister
  
  \item \verb!minstd_rand0, minstd_rand!: linear congruential engines
  
  \item \verb!ranlux24_base, ranlux48_base!: subtract with carry engine
  
   \item \verb!ranlux24, ranlux48!: discard block engine
   \item \verb!knuth_b! : shuffle order engine
\end{itemize}
  \item for expert (researcher): the following templates can be configured via template parameters 

\begin{verbatim}
linear_congruential_engine 
discard_block_engine
mersenne_twister_engine 
independent_bits_engine
subtract_with_carry_engine 
shuffle_order_engine

\end{verbatim}
\end{itemize}

\subsection{Create a distribution}

{\bf Uniform distribution}:
\begin{lstlisting}
int lbound = 1;
int ubound = 6;
static std::uniform_int_distribution<int> d{lbound, ubound};
\end{lstlisting}

\url{http://www.open-std.org/jtc1/sc22/wg21/docs/papers/2013/n3551.pdf}

\subsection{Return a PRN based on the created distribution and engine}

We pass the engine to the distribution.

\begin{lstlisting}
#include <random>
int roll_a_fair_die( )
{
  static std::default_random_engine e{};
  static std::uniform_int_distribution<int> d{1, 6};
  return d(e);
}
\end{lstlisting}




\section{File with given sizes}

Create file at a given size, containing only zeros
\begin{verbatim}
dd if=/dev/zero bs=1024 count=<number-of-kilobytes> > myfile
\end{verbatim}

%%% Local Variables: 
%%% mode: latex
%%% TeX-master: "fortran_manual"
%%% End: 
