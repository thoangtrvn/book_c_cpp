\chapter{C/C++ in Linux: glibc, libstdc++/libc++}
\label{chap:CCpp_Linux}

Chap.\ref{chap:CCpp_Windows} will discuss APIs' implementation of C/C++ standard
specification for Windows O/S.
In this chapter, we will discuss API's implementaiton of C/C++ standard
specification for Linux-based O/S and how to write portable code across platform
(Sect.\ref{sec:CCpp_portable-code}).

The Linux kernel is written in C language and the access to the underlying CPU
platform is exposed to users through a set of C-language-based APIs implemented
in a library, which can be either
\begin{enumerate}
  \item POSIX C library (Sect.\ref{sec:POSIX-C-library})
  
  \item C standard library (Sect.\ref{sec:C-standard-library})
\end{enumerate}

%{\bf glibc}.

\section{How the Linux O/S runs a program (ld-linux.so.2)}
\label{sec:how-linux-run-a-program}

The GNU dynamic loader (e.g. \verb!/lib/ld-linux.so.2!) is one of the main
components of the user space in a Linux-based system.

The main task of the dynamic loader is to handle the interaction between the
program and the system's shared libraries by relocating unresolved symbols.


Your executable program is compiled with a given version of GCC
(Sect.\ref{sec:GCC}) which indeed utilizes the APIs as provided by the standard
C or C++ library, i.e. uses a given version of \verb!libstdc.so! or
\verb!libstdc++.so! (Sect.\ref{sec:libstdc++}), this \verb!libstdc++.so! was
compiled and is depending upon a particular version of another library, e.g.
\verb!libgcc_s.o! for example. So, always there is a chain of version
dependency.
  
  
\textcolor{red}{Now, what is actually your executable program?}
\begin{itemize}
  
  \item Under Linux, and on most modern Unix(es), an executable program is just
  the same kind of file of a shared object or shared library
  
  And exactly as shared library, they can also export symbols (Sect.\ref{sec:symbol-shared-library}).
  
  \item 
\end{itemize}

% \begin{itemize}
%   \item 
%   \item the extension for a given compiler is written in C++, which may takes
%   advantage of newer C++ features so it is compiled by a newer version GCC or in
%   other case compiled by an older version of GCC, which in either cases depends
%   on a different version B of \verb!libstdc++.so!
% \end{itemize}


Issues when you try to run your program
\begin{enumerate}
  \item no available dynamically linked library, e.g. \verb!libstdc++.so!, for the program to use
  
\begin{verbatim}
$> ./myapp   

./myapp: error while loading shared libraries: libstdc++.so.6: cannot open shared object file: No such file or directory
\end{verbatim}
SOLUTION: Copy \verb!libstdc++.so.6! 


  \item not the same version of the libstdc++ used to compile the program
\begin{verbatim}
$> ./myapp  

./myapp: /usr/lib64/libstdc++.so.6: version `GLIBCXX_3.4.15` not found ( required by myapp )
\end{verbatim}
EXPLAIN:  As gcc evolves, so do the contents of libstdc++ library, i.e. 

SOLUTION: use the right version, e.g. \verb!libstdc++-4.5.1.so!
\begin{verbatim}
libstdc++.so.6       ---> libstdc++-4.5.1.so
\end{verbatim}

We may want to specify the path of the shared library via \verb!LD_LIBRARY_PATH!
\begin{verbatim}
$>LD_LIBRARY_PATH=libs/ ./myapp   

Hello!
\end{verbatim}

We can check the library dependencies, i.e. one library file uses which other
library files
\begin{verbatim}
$>LD_LIBRARY_PATH=libs/ ldd ./myapp  

linux-vdso.so.1 => (0x00007fffbe9f2000) 
libstdc++.so.6 => libs/libstdc++.so.6 (0x00007fcb13b85000) 
libc.so.6 => /lib/x86_64-linux-gnu/libc.so.6 (0x00007fcb1379b000) 
libm.so.6 => /lib/x86_64-linux-gnu/libm.so.6 (0x00007fcb13496000) 
/lib64/ld-linux-x86-64.so.2 (0x00007fcb13e8b000) 
libgcc_s.so.1 => libs/libgcc_s.so.1 (0x00007fcb13280000)
\end{verbatim}

We can check the ELF
\begin{verbatim}
$> readelf -d myapp
[...]
0x00000001 (NEEDED)                     Shared library: [libtest.so]
0x00000001 (NEEDED)                     Shared library: [libc.so.6]
\end{verbatim}

  \item to eliminate the use of \verb!LD_LIBRARY_PATH! every time you run the program:
  (1) put that setting in the \verb!~/.bashrc! file, (2) compile your program and specify the user-defined path to look for the 
  shared library via \verb!rpath! option
  
\begin{verbatim} 
 // only look for shared library in the default system search path 
 //   (defined in ld.so and ldconfig)
g++ -o myapp myapp.cpp

 // use rpath option
 // with the special token $ORIGIN which causes ld.so to locate
 //    the shared libraries relative to the location of the binary 'myapp'
g++ -Wl,-rpath=\$ORIGIN//lib myapp.cpp -o myapp

 // more flexible
g++ -Wl,-rpath=\$ORIGIN//lib --enable-new-dtags myapp.cpp -o myapp
\end{verbatim}
Sect.\ref{sec:linker} explains how to hard-code a path to a shared library.

We can check the \verb!.dynamic ELF! section
\begin{verbatim}
$> readelf -d main
[...]
0x00000001 (NEEDED)                     Shared library: [libtest.so]
0x00000001 (NEEDED)                     Shared library: [libc.so.6]
0x0000000f (RPATH)                      Library rpath: [$ORIGIN]
\end{verbatim}
The \$ORIGIN variable within the \verb!DT_RPATH! value refers to the current execution directory of the main program.

The general drawback using \verb!DT_RPATH! is that you cannot overwrite this
embedded path setting with \verb!LD_LIBRARY_PATH!. This limits the flexibity of
using shared libraries. There are two options: (1) remove the shared library
from the given path, (2) use \verb!-enable-new-dtags! compiling option which set
the value in \verb!DT_RUNPATH!.
In the later option, the dynamic loader searches the \verb!LD_LIBRARY_PATH!
before the embedded path in the executable.
\url{http://blog.lxgcc.net/?tag=dt_runpath}

  \item we have the right version of \verb!libstdc++.so.6!, but this library depends on another 
  library whose version is not correct
\begin{verbatim}
$> ./myapp

myapp: /lib/x86_64-linux-gnu/libc.so.6: version `GLIBC_2.14' not found (required by libs/libstdc++.so.6)
\end{verbatim}  
EXPLAIN: \verb!libstdc++.so.6! requires version 2.14 of \verb!libc.so.6!, but the system cannot find this version.

SOLUTION: we may do the similar approach, i.e. copy the right version of
\verb!libc.so.6!.


   \item But chances are you will broke your system if you change the
   \verb!libc.so.6!
versions as other programs, e.g. the Linux kernel, may depends on the existing
version of libc.so.6.

\begin{verbatim}
$> ./myapp  

FATAL: kernel too old  
Segmentation fault
\end{verbatim}
SOLUTION: No solution as replacing the Linux kernel is not a good option.
  
\end{enumerate}
\url{http://www.crankuptheamps.com/blog/posts/2014/03/04/Break-The-Chains-of-Version-Dependency/}

\subsection{libc.so.6}

32-bit version
\begin{verbatim}
/lib/i386-linux-gnu/libc.so.6
\end{verbatim}

64-bit version
\begin{verbatim}
/lib/x86_64-linux-gnu/libc.so.6
\end{verbatim}

This 
\begin{verbatim}
/usr/lib/x86_64-linux-gnu/libc.so
\end{verbatim}
 is not a library, but a linker script file, which refers to the above symlinks.
 
 
 First, regardless of libc version installed, the linker will always search for
 libc.so, because the compiler driver will always pass to the linker the -lc
 options. The name libc stays the same and denotes to most recent version of the
 library. The symlinks libc.so.6 are named after the soname of the library,
 which, more or less corresponds to the ABI version of the library. The
 executables, linked against libc.so in fact contain runtime dependencies on
 libc.so.6.

The file is created when you compile glibc 
\begin{verbatim}
#>cat /usr/lib64/libc.so
 
/* GNU ld script
   Use the shared library, but some functions are only in
   the static library, so try that secondarily.  */
OUTPUT_FORMAT(elf64-x86-64)
GROUP ( /lib64/libc.so.6 /usr/lib64/libc_nonshared.a  AS_NEEDED ( /lib64/ld-linux-x86-64.so.2 ) )
\end{verbatim}



\section{IMPORTANT: Understanding GNU C library (glibc) and its variants
(Linux libc, eglibc)}
\label{sec:POSIX-C-library}
\label{sec:library-implement-standard-C-language}

\verb!libc! refers to a specific implementation of the standard library
functionality described in ISO C and POSIX standards, optionally with common
extensions, intended for use on a Linux-based systems.

Whereas the kernel (Linux) governs access to hardware, memory, filesystems, and
the privileges for accessing these resources, the C library is responsible for
providing the actual C function interfaces userspace applications see, and for
constructing higher-level buffered stdio, memory allocation management, thread
creation and synchronization operations, and so on using the lower-level
interfaces the kernel provides, as well as for implementing pure library
routines of the C language like strstr, snprintf, strtol, exp, sqrt, etc.

A libc-type library is the core library that sits between user space and the
kernel. The first implementation of such library was named {\bf glibc}
(Sect.\ref{sec:glibc}) \verb!glibc! was first written mainly by Roland McGrath
from 1980s.

Then, a fork of glibc version, used in early Linux kernel, is \verb!libc!
(Linux libc) library.

\subsection{glibc (version 1, 2)}
\label{sec:glibc}
\label{sec:GLIBC}

{\bf glibc} is the first implementation of the standard specification for C
language (Sect.\ref{sec:library-implement-standard-C-language}).
Nowadays, Glibc is used in majority of distribution: Fedora-like, RHEL,
Debian-like, SuSE, Gentoo, Archlinux.

By 1988, glibc was supposed to have all features based on the standardization of
ANSI C language (Sect.\ref{sec:ANSI-C}). By 1992, features for ANSI C-1989 and
POSIX.1-1990 were also added. So, glibc is also known as POSIX C library.
glibc is known as GNU C library as it was first used by the GNU operating
systems and is managed by FSF (Free Software Foundation).

NOTE: {\bf glibc} is just one implementation of the C language specification
(Sect.\ref{sec:libc}). 

Nowadays, glibc version 2.x is not a single \verb!*.so! file - there are many.
It includes \verb!libc! (libc.so.6), and other essential libraries (\verb!libm!
for math, \verb!libpthread! for p-thread). Because of that, glibc is considered
as the C standard library (Sect.\ref{sec:C-standard-library}) and is being used
for Linux-like O/S or GNU/FreeBSD O/S. This library is also used by the Linux
kernel so is considered as a critical component of the Linux O/S.

Since 1995, Ulrich Drepper made his first contribution to the glibc project and
later become the main contributor. 

Since 2001, the development of glibc has been overseen by a committee, with
Drepper as the lead contributor.

Since 2012, the steering committee voted to disband itself and remove Drepper in
favor of a community-driven development process due to  the long standing
controversies around Drepper's leading style and external contribution
acceptance.

SUMMARY: glibc implements (more details see
Sect.\ref{sec:glibc-2.0})
\begin{enumerate}
  \item C library described in ANSI,c99,c11 standards. It includes macros,
  symbols, function implementations etc.(printf(),malloc() etc)
  
  \item POSIX C standard library. The "userland" glue of system calls.
  (open(),read() etc. 
  
  Actually glibc does not "implement" system calls. kernel does it. But glibc
  provides the user land interface to the services provided by kernel so that
  user application can use a system call just like a ordinary function.
  
  \item  some nonstandard but usefull stuffs.
\end{enumerate}

To tell the compiler to use a particular version of Standard C, we use \verb!-std=!
\begin{verbatim}
-std=c11
-std=gnu99
-std=c99
\end{verbatim}


\subsection{-- glibc 2.x (libc6)}
\label{sec:glibc-2.0}
\label{sec:glibc-2.x}

In Jan 1997, {\bf glibc 2.0} (or known as libc6 in Linux kernel -
Sect.\ref{sec:libc}) was released with (1) better internationlization, (2)
multilingual function, (3) IPv6 capability, (4) 64-bit data access, (5)
multithreaded support, (6) code more portable, (7) future version compatibility,
(8) more complete POSIX standards compliance than Linux libc
(Sect.\ref{sec:libc}).

Different versions of glibc 2.x
\begin{enumerate}

  \item glibc 2.3.4 (Dec. 2004): the standard for operating systems (O/S)
  following LSB-3.0 standard (Sect.\ref{sec:LSB_C-lang})
  
  \item glibc 2.4 (Mar 2006): with partial inotify support
  
  It is the standard for operating systems (O/S)
  following LSB-4.0 standard (Sect.\ref{sec:LSB_3.1}).
   

  \item glibc 2.5: full inotify support
  
\begin{verbatim}
* New Linux interfaces: splice, tee, sync_file_range, vmsplice (glibc 2.5)
* RFC 3542: Advanced socket interface for IPv6 (glibc 2.5)
* Support for the ELF .gnu.hash section (glibc 2.5)
* Real-time priority inheritance mutexes and priority protected mutexes
	(glibc 2.5)
\end{verbatim}
  
  \item glibc 2.6

\begin{verbatim}
* New Linux interfaces: epoll_pwait, sched_getcpu (glibc 2.6)
* New generic interfaces: strerror_l (glibc 2.6) 
\end{verbatim}

  \item glibc 2.7 (Oct 2007): used by Ubuntu 8.04 (Debian 5)

\begin{verbatim}
* New Linux interfaces: mkostemp, mkostemp64 (glibc 2.7)
* New Linux interfaces: signalfd, eventfd, eventfd_read, eventfd_write
	(glibc 2.7)
\end{verbatim}


  \item glibc 2.9 (Nov 2008)
    
  \item glibc 2.11 (Oct 2009): used by Ubuntu 10.04 [NOTE: Debian 6 used eglibc]
  
  \item glibc 2.15 (Mar 2012): used by Ubuntu 12.04 and 12.10 
  
  \item glibc 2.16 (Jun 2012): support ISO C11, x32 API, and SystemTap
  
  \item glibc 2.17 (Dec 2012): support 64-bit ARM, and used by Ubuntu
  13.04, RHEL 7.
  
  \item glibc 2.18 (Aug 2013): improve C++11 support, etc.
  
  \item glibc 2.19 (Feb 2014): ???
  \url{http://en.wikipedia.org/wiki/GNU_C_Library}
\end{enumerate}

Check which version of glibc is being supported in the current libc.so.6 library
\begin{verbatim}
$>strings /lib64/libc.so.6 | grep 'LIBC'

GLIBC_2.2.5
GLIBC_2.2.6
GLIBC_2.3
GLIBC_2.3.2
GLIBC_2.3.3
GLIBC_2.3.4
GLIBC_2.4
GLIBC_2.5
GLIBC_2.6
GLIBC_2.7
GLIBC_2.8
GLIBC_2.9
GLIBC_2.10
GLIBC_2.11
GLIBC_2.12
GLIBC_PRIVATE
LIBC_FATAL_STDERR_
\end{verbatim}


\subsection{-- version (GLIBCXX, GLIBCPP)}
\label{sec:glibc-version}
\label{sec:GLIBCXX}
\label{sec:GLIBCPP}

\url{https://lwn.net/Articles/492624/}

The most static part of GLIBC is the portion that implements standards.
Standards support is important because it allows people and code to move between
different architectures and platforms. The "new-ish" standards support that the
GLIBC community is working on now is the C11 support, which he guesses will be
available in GLIBC 2.16 or 2.17.

One of the more interesting features in C11 is the C-level atomic operations
(Sect.\ref{sec:C11_atomic_op}).

\begin{enumerate}
  \item  Glibc in 3.0.0 to 3.0.4, they are not versioned.
  
  
  \item Glibc in 3.1.0 to 3.3.3: they use \verb!GLIBCPP!

{\tiny
\begin{verbatim}
GCC 3.1.0: GLIBCPP_3.1, CXXABI_1

GCC 3.1.1: GLIBCPP_3.1, CXXABI_1

GCC 3.2.0: GLIBCPP_3.2, CXXABI_1.2

GCC 3.2.1: GLIBCPP_3.2.1, CXXABI_1.2

GCC 3.2.2: GLIBCPP_3.2.2, CXXABI_1.2

GCC 3.2.3: GLIBCPP_3.2.2, CXXABI_1.2

GCC 3.3.0: GLIBCPP_3.2.2, CXXABI_1.2.1

GCC 3.3.1: GLIBCPP_3.2.3, CXXABI_1.2.1

GCC 3.3.2: GLIBCPP_3.2.3, CXXABI_1.2.1

GCC 3.3.3: GLIBCPP_3.2.3, CXXABI_1.2.1

GCC 3.4.0: GLIBCXX_3.4, CXXABI_1.3

GCC 3.4.1: GLIBCXX_3.4.1, CXXABI_1.3
\end{verbatim}
}
   \item Glibc 3.3.4+:
   
{\tiny 
\begin{verbatim}
GCC 3.4.2: GLIBCXX_3.4.2

GCC 3.4.3: GLIBCXX_3.4.3

GCC 4.0.0: GLIBCXX_3.4.4, CXXABI_1.3.1

GCC 4.0.1: GLIBCXX_3.4.5

GCC 4.0.2: GLIBCXX_3.4.6

GCC 4.0.3: GLIBCXX_3.4.7

GCC 4.1.1: GLIBCXX_3.4.8

GCC 4.2.0: GLIBCXX_3.4.9

GCC 4.3.0: GLIBCXX_3.4.10, CXXABI_1.3.2

GCC 4.4.0: GLIBCXX_3.4.11, CXXABI_1.3.3

GCC 4.4.1: GLIBCXX_3.4.12, CXXABI_1.3.3

GCC 4.4.2: GLIBCXX_3.4.13, CXXABI_1.3.3

GCC 4.5.0: GLIBCXX_3.4.14, CXXABI_1.3.4

GCC 4.6.0: GLIBCXX_3.4.15, CXXABI_1.3.5

GCC 4.6.1: GLIBCXX_3.4.16, CXXABI_1.3.5

GCC 4.7.0: GLIBCXX_3.4.17, CXXABI_1.3.6

GCC 4.8.0: GLIBCXX_3.4.18, CXXABI_1.3.7

GCC 4.8.3: GLIBCXX_3.4.19, CXXABI_1.3.7

GCC 4.9.0: GLIBCXX_3.4.20, CXXABI_1.3.8

GCC 5.1.0: GLIBCXX_3.4.21, CXXABI_1.3.9

GCC 6.1.0: GLIBCXX_3.4.22, CXXABI_1.3.10

GCC 7.1.0: GLIBCXX_3.4.23, CXXABI_1.3.11

GCC 7.2.0: GLIBCXX_3.4.24, CXXABI_1.3.11

GCC 8.0.0: GLIBCXX_3.4.25, CXXABI_1.3.11
\end{verbatim}
}
\url{https://gcc.gnu.org/onlinedocs/libstdc++/manual/abi.html}
\end{enumerate}

To know which glibc library is being used
\begin{verbatim}
%> /sbin/ldconfig -p | grep stdc++

libstdc++.so.6 (libc6) => /usr/lib/libstdc++.so.6
\end{verbatim}

In each library, every API is associated with a given 
\verb!GLIBCXX_*!.  So, \verb!GLIBCXX_*! don't apply to the entire library, but
to each symbol (symbol versioning), e.g.
\begin{verbatim}
std::char_traits<wchar_t>::eq@@GLIBCXX_3.4.5 
std::ios_base::Init::~Init()@@GLIBCXX_3.4 
\end{verbatim}
 
Then, using that path, to know which GLIBC versions the library supports
\begin{verbatim}
%> strings /usr/lib/glibstdc++.so.6 | grep GLIBC

GLIBCXX_3.4
GLIBCXX_3.4.1
GLIBCXX_3.4.2
GLIBC_2.2.5
GLIBC_2.3
GLIBC_2.4
GLIBC_2.3.4
GLIBC_2.3.2
GLIBCXX_FORCE_NEW
GLIBCXX_DEBUG_MESSAGE_LENGTH
\end{verbatim}

NOTE: The fact that your program needs \verb!GLIBCXX_3.4.9! probably means that
it has been linked against a symbol that has been introduced/has changed
semantics on \verb!GLIBCXX_3.4.9!.

NOTE: In earlier version of glibc, \verb!GLIBCPP! was used instead
\begin{verbatim}
// libdatestamp.cxx
#include <cstdio>

int main(int argc, char* argv[]){
#ifdef __GLIBCPP__
    std::printf("GLIBCPP: %d\n",__GLIBCPP__);
#endif
#ifdef __GLIBCXX__
    std::printf("GLIBCXX: %d\n",__GLIBCXX__);
#endif
   return 0;
}
\end{verbatim}

\subsection{-- update + troubleshooting}

To update glibc
\begin{verbatim}
sudo apt-get install libstdc++6

sudo ldconfig
\end{verbatim}
or you may need to add the repository path using this PPA 
\begin{verbatim}
apt-get install python-software-properties  !! to use add-apt-repository
sudo add-apt-repository ppa:ubuntu-toolchain-r/test 
sudo apt-get update
sudo apt-get upgrade
sudo apt-get dist-upgrade
\end{verbatim} 

When you add a new PPA, you may get the error
\begin{verbatim}
W: GPG error: http://ppa.launchpad.net lucid Release: The following signatures
couldn't be verified because the public key is not available: NO_PUBKEY 1E9377A2BA9EF27F
\end{verbatim}
This is a new feature of \verb!apt-get! to guarantee the authenticity of
servers. SOLUTION: notice the number (e.g. 1E9377A2BA9EF27F) and
run\footnote{\url{http://en.kioskea.net/faq/809-debian-apt-get-no-pubkey-gpg-error}}
\begin{verbatim}
gpg --keyserver pgpkeys.mit.edu --recv-key  1E9377A2BA9EF27F     
gpg -a --export 1E9377A2BA9EF27F | sudo apt-key add -
\end{verbatim}
You may get the error
\begin{verbatim}
gpg: requesting key BA9EF27F from hkp server pgpkeys.mit.edu
\end{verbatim}
This is due to proxy server setting. An alternate solution is using
\begin{verbatim}
apt-key adv --keyserver hkp://p80.pool.sks-keyservers.net:80
         --recv-keys 1E9377A2BA9EF27F
\end{verbatim}
\url{http://askubuntu.com/questions/53146/how-do-i-get-add-apt-repository-to-work-through-a-proxy}

If you still cannot find the newer verstion, run
\begin{verbatim}
aptitude
\end{verbatim}
search for libstdc++, and select the dependency.

\url{http://stackoverflow.com/questions/10354636/how-do-you-find-what-version-of-libstdc-library-is-installed-on-your-linux-mac}

\subsection{-- glibc.i586, glibc.i686, glibc.x86\_64}

Glibc is a relatively special case, where the compiler optimizations
used to compile it can actually have a significant effect across the
system, even if the other items are not compiled with the same set of
optimizations.


\begin{verbatim}
On 32-bit x86
===
Pick One:
* i586 (good)
* i686 (better, but only if your hardware supports i686)

On 64-bit x86_64
===
Pick one:
* x86_64 (only option)

On a Multilib system (32bit and 64bit libs installed side by side)
===
* Pick One from the 32-bit x86 list (i586 or i686)
* Pick One from the 64-bit x86_64 list (x86_64)
\end{verbatim}

To get \verb!glibc.i686! (on a 64-bit system), run
\begin{verbatim}
// Any commands: RedHat, CentOS
yum install /lib/ld-linux.so.2

sudo yum install glibc.i686

sudo yum install glibc.i386


// Debian (Ubuntu)
sudo apt-get install ia32-libs
\end{verbatim}

\subsection{-- glibXext: glibC with X components}
\label{sec:glibXext}

If you're intending on running an GUI application using 32-bit libs, install
some X components:

NOTE: If you don't use \verb!sudo!, run \verb!su -! to acquire superuser
authority first
\begin{verbatim}
su -c 'yum install glibc.i686 libXext.i586'
\end{verbatim}

\subsection{Linux libc (libc4, libc5)}
\label{sec:libc}
\label{sec:Linux-libc}

In early 1990s, a folk of \verb!glibc! version 1 (Sect.\ref{sec:glibc}) written
for Linux kernel was released under the name {\bf Linux libc}. Linux libc was
released from version 2 to 5, which existed in  various versions (libc5, libc5).

With the mature of \verb!glibc! version 2, the development for Linux libc was
then stopped and Linux kernel switched to using glibc 2.0
(Sect.\ref{sec:glibc-2.0}). The last Linux libc is {\it libc.so.5}. It means
that {\bf libc.so.6} in Linux O/S indeeds refers to glibc 2.x.

Example:
\begin{verbatim}
libc.so.6 ---> glibc.so.2.4
\end{verbatim}

\begin{itemize}
  \item libc.so.6
  
  \item libc.so.6.1: used by DEC Alpha 64-bit and Intel IA64 architectures.
\end{itemize}

\subsection{eglibc (embedded system Linux-kernel, Ubuntu < 15.04)}
\label{sec:eglibc}

NOTE: {\bf eglibc} is a variant of GNU glibc optimized for use in embeded
devices, and maintain source- and binary-compatible with standard glibc. In
2009, Debian project decided to move from standard glibc to eglibc. Then from
2014, Debian project switched back to standard glibc; as eglibc is no longer
being developed.  

Ubuntu (until 14.04) is still using eglibc 2.19. Ubuntu won't convert to glibc
until Ubuntu 15.04. Check with
\begin{verbatim}
ldd --version
\end{verbatim}

\subsection{uClib (embedded-system Linux kernel)}
\label{sec:uClibc}

In embedded systems, we use uClib.

Later uClibc versions added the two typedefs (\verb!typedef long __kernel_long_t!; and
\verb!typedef unsigned long __kernel_ulong_t;!) to
\verb!libc/sysdeps/linux/i386/bits/kernel_types.h! (actually to all
\verb!libc/sysdeps/linux/*/bits/kernel_types.h!), as they were added to by kernels 3.4
and later.


\url{http://stackoverflow.com/questions/18598995/building-uclibc-linux-3-10-2-debian-jessie-x86-64-fails-due-to-missing-types}

\subsection{newlib (embedded-system Linux kernel)}
\label{sec:newlib}

Newlib is a cross-platform C Standard Library (developed by RedHat) for
embedded systems.

\subsection{bionic (Android Linux kernel)}
\label{sec:bionic}

A derivation of BSD's standard C library code targetting Android O/S
for Linux kernel is called Bionic libc. Bionic libc is much smaller than
GNU C library (glibc) and somewhat smaller than uClibc. It is designed
for CPU at relatively low clock frequencies.

\subsection{musl}
\label{sec:musl}

musl is a new general-purpose implementation of the C library. It is
lightweight, fast, simple, free, and aims to be correct in the sense of
standards-conformance and safety.
\url{http://www.etalabs.net/compare_libcs.html}

\verb!musl! requires Linux kernel 2.6 or later.
\verb!musl! library has been compiled on (these supported) cpu architecture:
i386, \verb!x86_64!, arm, mips, microblaze, or powerpc

As of 2015, Linux distributions that use musl as the standard C library include
Alpine Linux, Dragora 3, OpenWRT, Sabotage, Morpheus Linux and
optionally prebuilt for Void Linux.


To build \verb!musl!, you will also need a C99 compiler with support for
gcc-style \verb!__asm__! statements and assembly source files, and weak symbol
support in the linker. gcc 3.3 or later (with the GNU assembler and linker) and
clang 3.2 or later are known to work. Users have also had success building musl
with PCC and Firm/cparser.

\url{https://www.musl-libc.org/faq.html}

\url{https://wiki.musl-libc.org/functional-differences-from-glibc.html}

\section{Compile glibc}


\subsection{Install multiple versions of glibc}
\label{sec:glibc-install}

Each distro of a Linux-based O/S comes with a shared library of a particular
version of glibc. 

Even though it is common to have many versions of a library, the story is quite
different with glibc. glibc consists of many pieces (200+ shared libraries)
which all must match for the O/S to boot and run normally.
Thus, installing multiple versions of glibc is not trivial.

To compile:
\begin{enumerate}
  \item download and extract the glibc with right version
  
  \item run
\begin{verbatim}
mkdir glibc
mv glibc-2.16.0.tar.gz ./glibc
cd glibc
tar -xvf glibc-2.16.0.tar.gz
mkdir build
cd build
../glibc-2.16.0/configure --prefix=$HOME/bin --libdir=$HOME/bin/lib64
\end{verbatim}
\end{enumerate}
For cross-compilation, read
\url{http://www.linuxfromscratch.org/clfs/view/1.0.0/mips64/cross-tools/glibc-64.html}


One of the pieces is \verb!ld-linux.so.2! (Sect.\ref{sec:ld-linux.so.2}), and it
must match libc.so.6, or you'll see the errors you are seeing.
\begin{verbatim}
/myapp: /lib/i686/libc.so.6: version `GLIBC_2.3' not found (required by ./myapp)
./myapp: /lib/i686/libpthread.so.0: version `GLIBC_2.3.2' not found (required by ./myapp)

\end{verbatim}

So compile them and put these files in one folder, called newglibc
\begin{verbatim}
libpthread.so.0
libm.so.6
libc.so.6
ld-2.3.3.so
ld-linux.so.2 -> ld-2.3.3.so
\end{verbatim}


\subsection{Cross-compile glibc}

Three important ./configure flags:

\begin{verbatim}
--build= 
       The system performing the build. 
       Looks like yours is x86_64-pc-linux-gnu.
--host= 
       The system on which the generated objects will be used. 
       You want to set  this to i386-pc-linux-gnu.
       
--target= 
      If you are building a compiler, the system for which the built compiler
    will generate objects.
\end{verbatim}


So, when you build a glibc library, it's important that you need to know (1)
which kernel version the glibc is running on (which can be specified using
\verb!--enable-kernel=2.4.3! it means glibc will cut out any code to support
kernel interfaces older than whatever value you specified, e.g. for Linux kernel
2.4.3, passing to \verb!./configure!. NOTE: we can skip this if we compile for
the current kernel on the current machine), (2) what gcc version you want to
compile the glibc library, (3) the target machine (i.e. 32-bit or 64-bit,
x86/x86-64 or arm)

\begin{verbatim}
./glibc-2.6
./build
\end{verbatim}
We use \verb!--host=!, \verb!--build=! options.


Suppose target machine is 32-bit and host machine is 64-bit
\begin{verbatim}
mkdir build
cd build
../glibc-2.6/configure --prefix=$HOME/glibc32-2.6 \
     CC="gcc -m32" CXX="g++ -m32" \
     CFLAGS="-O2 -march=i686" \
     CXXFLAGS="-O2 -march=i686" \
     i686-linux-gnu
\end{verbatim}

or
\begin{verbatim}
../glibc-2.6/configure --prefix=$HOME/glibc32-2.6 \
     --host=i686-linux-gnu \
     --build=i686-linux-gnu \
     CC="gcc -m32" CXX="g++ -m32" \
     CFLAGS="-O2 -march=i686" \
     CXXFLAGS="-O2 -march=i686"
\end{verbatim}
To perform a cross-compile, you must specify both --build= and --host=. When you
only specify --host=, it will still attempt to build a native (\verb!x86_64!)
glibc.


% \section{LSB in C}
% \label{sec:LSB_C-lang}
% 
% LSB is Linux Standard Base, and ISO standard for GNU/Linux. The purpose is to
% keep hundreds of Linux distros compatible. Thus, LSB encompasses several
% standards, including POSIX. For more information, read the book Sys-Admin
% (Chapter 1). Here, we focus on LSB in C language.


% \section{LSB in C++}



\section{Write portable code}
\label{sec:CCpp_portable-code}

The choice of APIs to use depending upon the 
library file that implements the C standard library
(Sect.\ref{sec:C-standard-library}). In Linux, it is glibc, but in another O/S,
it can be different.

To write portable code,   it is important to be able to detect 
\begin{enumerate}
  \item  the underlying O/S
(Sect.\ref{sec:macro-detect-OS}) or 

  \item CPU architecture
(Sect.\ref{sec:macro-detect-CPU}).

  \item the type of compiler being used to compile the code
(Sect.\ref{sec:macro-detect-compiler}). Some compiler may provide advanced
features than others.

  \item the compiler verion (Sect.\ref{sec:macro-detect-compiler-version}).
  
Newer C/C++ language may provides new constructs and/or functions to help
writing code shorter or run faster.
    
\end{enumerate}

\url{http://nadeausoftware.com/articles/2012/01/c_c_tip_how_use_compiler_predefined_macros_detect_operating_system}

\subsection{Macro to detect O/S}
\label{sec:macro-detect-OS}

Macros are often used to detect the O/S and to choose the correct code to compile at compile-time.
ANSI C defines a number of macros, and Microsoft C++ implements several more.

\begin{itemize}
  \item \verb!_WIN32! (Windows 32-bit) and \verb!_WIN64! (Windows 64-bit): they are both non-POSIX 

\begin{lstlisting}
#if defined(_WIN64)
    /* Microsoft Windows (64-bit) */
#elif defined(_WIN32)
    /* Microsoft Windows (32-bit) */
#endif
\end{lstlisting}
  
  NOTE: There is no \verb!WIN32! macro. If it is being used elsewhere, either the code is wrong, or
  the macro has been explicitly defined by the programmer elsewhere.
  
  \item \verb!_AIX! (IBM AIX):
  
\begin{lstlisting}
#if defined(_AIX)
	/* IBM AIX. ------------------------------------------------- */

#endif
\end{lstlisting}

  \item Apple O/S:
\begin{lstlisting}
#if defined(__APPLE__) && defined(__MACH__)
    /* Apple OSX and iOS (Darwin) */
#include <TargetConditionals.h>
#if TARGET_IPHONE_SIMULATOR == 1
    /* iOS in Xcode simulator */
#elif TARGET_OS_IPHONE == 1
    /* iOS on iPhone, iPad, etc. */    
#elif TARGET_OS_MAC == 1
    /* OS X */
#endif
#endif
\end{lstlisting}

  \item Linux-based OS:
\begin{lstlisting}
#if defined(__linux__)
    /* Linux  */
#endif
\end{lstlisting}

 
  \item Unix-style OS:
\begin{lstlisting}
#if !defined(_WIN32) && (defined(__unix__) || defined(__unix) || (defined(__APPLE__) && defined(__MACH__)))
    /* UNIX-style OS. ------------------------------------------- */

#endif
\end{lstlisting}

In a Unix-style OS, you can then check if it is POSIX-compliant
\begin{lstlisting}
#include <unistd.h>
#if defined(_POSIX_VERSION)
    /* POSIX compliant */
#endif
\end{lstlisting}

or a BSD-derived system
\begin{lstlisting}
#if defined(__unix__) || (defined(__APPLE__) && defined(__MACH__))
#include <sys/param.h>
#if defined(BSD)
    /* BSD (DragonFly BSD, FreeBSD, OpenBSD, NetBSD). ----------- */

#endif
#endif
\end{lstlisting}

  \item Cygwin POSIX under Windows:
\begin{lstlisting}
#if defined(__CYGWIN__) && !defined(_WIN32)
    /* Cygwin POSIX under Microsoft Windows. */
#endif
\end{lstlisting}

Cygwin 1.7.22 is the first 64-bit release. Before that, Cygwin POSIX libraries
are 32-bit-only, so 64-bit POSIX applications cannot be built. So, the macro
\verb!__CYGWIN64__! is not defined until Cygwin 1.7.22.

\end{itemize}

The full list
\begin{lstlisting}
_AIX, __APPLE__, __CYGWIN32__, __CYGWIN__, __DragonFly__, __FreeBSD__,
__gnu_linux, hpux, __hpux, linux, __linux, __linux__, __MACH__, __MINGW32__,
__MINGW64__, __NetBSD__, __OpenBSD__, _POSIX_IPV6, _POSIX_MAPPED_FILES,
_POSIX_SEMAPHORES, _POSIX_THREADS, _POSIX_VERSION, sun, __sun, __SunOS, __sun__,
__SVR4, __svr4__, TARGET_IPHONE_SIMULATOR, TARGET_OS_EMBEDDED, TARGET_OS_IPHONE,
TARGET_OS_MAC, UNIX, unix, __unix, __unix__, WIN32, _WIN32, __WIN32, __WIN32__,
WIN64, _WIN64, __WIN64, __WIN64__, WINNT, __WINNT, __WINNT__
\end{lstlisting}


\subsection{-- BSD 4.4}

To distinguish between 4.4 BSD-derived systems and the rest of the world, you
should use the following code.
\begin{verbatim}
#include <sys/param.h>
#if (defined(BSD) && BSD >= 199306)
/* BSD-specific code goes here */
#else
/* non-BSD-specific code goes here */
#endif

\end{verbatim}


\begin{verbatim}
#define __USE_BSD 1
#define __unix__ 1
#define __linux 1
#define __unix 1
#define __linux__ 1
#define _POSIX_SOURCE 1
#define __STDC_HOSTED__ 1
#define __STDC_IEC_559__ 1
#define __gnu_linux__ 1
#define __USE_SVID 1
#define __USE_XOPEN2K 1
#define __USE_POSIX199506 1
#define _G_USING_THUNKS 1
#define __USE_XOPEN2K8 1
#define _BSD_SOURCE 1
#define unix 1
#define linux 1
#define __USE_POSIX 1
#define __USE_POSIX199309 1
#define __SSP__ 1
#define _SVID_SOURCE 1
#define _G_HAVE_SYS_CDEFS 1
#define __USE_POSIX_IMPLICITLY 1
\end{verbatim} 
\url{http://stackoverflow.com/questions/4575098/documentation-concerning-platform-specific-macros-in-linux-posix}


\subsection{-- Windows: \_WIN32}

\verb!_WIN32!

\subsection{-- POSIX-compliant platforms}

Test the following macros
\begin{verbatim}
Cygwin      __CYGWIN__
DragonFly   __DragonFly__
FreeBSD     __FreeBSD__
Haiku       __HAIKU__
Interix     __INTERIX
IRIX        __sgi (TODO: get a definite source for this)
Linux       linux, __linux, __linux__
Mac OS X    __APPLE__
MirBSD      __MirBSD__ (__OpenBSD__ is also defined)
Minix3      __minix
NetBSD      __NetBSD__
OpenBSD     __OpenBSD__
Solaris     sun, __sun
\end{verbatim}


\subsection{-- XOPEN or POSIX}
\label{sec:_XOPEN_SOURCE}
\label{sec:_POSIX_C_SOURCE}

In early versions of UNIX-based O/S, the APIs may be implemented based on XPG
(i.e. X/Open Portability Guide - Sect.\ref{sec:XPG}) or POSIX
(Sect.\ref{sec:POSIX}).

When 
\begin{verbatim}
  /* source code */
#define _XOPEN_SOURCE <some number>

  # compile
cc -D_XOPEN_SOURCE=<some number>
\end{verbatim}
it tells your compiler to include definitions for some extra functions that are
defined in the X/Open and POSIX standards. This will give you some extra
functionality that exists on most recent UNIX/BSD/Linux systems, but probably
doesn't exist on other systems such as Windows.

\begin{verbatim}
500 - X/Open 5, incorporating POSIX 1995
600 - X/Open 6, incorporating POSIX 2004
700 - X/Open 7, incorporating POSIX 2008
\end{verbatim}
\url{http://man7.org/linux/man-pages/man7/feature_test_macros.7.html}

Check \verb!usr/include/features.h! file
\begin{verbatim}
#XOPEN_SOURCE	Includes POSIX and XPG things.  Set to 500 if
#			Single Unix conformance is wanted.
#_XOPEN_SOURCE_EXTENDED XPG things and X/Open Unix extensions.
_XOPEN_SOURCE	Includes POSIX and XPG things.
           = 500	Single Unix conformance
           = 600	X/Open 2000 conformance
           
_XOPEN_SOURCE_EXTENDED (together with _XOPEN_SOURCE)
               XPG things and X/Open Unix extensions.

 _LARGEFILE_SOURCE	Some more functions for correct standard I/O.
_LARGEFILE64_SOURCE	Additional functionality from LFS for large files.
_FILE_OFFSET_BITS=N	Select default filesystem interface.
    __USE_LARGEFILE64	Define LFS things with separate names.
    __USE_FILE_OFFSET64	Define 64bit interface as default.


_POSIX_C_SOURCE   If ==1, like _POSIX_SOURCE; if >=2 add IEEE Std 1003.2;
 			if >=199309L, add IEEE Std 1003.1b-1993;
 			if >=199506L, add IEEE Std 1003.1c-1995

__USE_POSIX199506	Define IEEE Std 1003.1, .1b, .1c and .1i things.
__USE_XOPEN		Define XPG things.
__USE_XOPEN_EXTENDED	Define X/Open Unix things.
__USE_XOPEN2K	Define X/Open 2000 things.
__USE_UNIX98		Define Single Unix V2 things.

#if defined _BSD_SOURCE || defined _SVID_SOURCE
# define __USE_MISC     1   
           /* which means Define things common to BSD and SystemV Unix. */ 
#endif
\end{verbatim}

\verb!__USE_MISC!:
\url{http://stackoverflow.com/questions/10231885/what-is-use-misc-macro-used-for}

Full list: 
\url{http://www.netbsd.org/docs/pkgsrc/fixes.html#fixes.build.cpp}

% NOTE: XPG = X/Open Portability Guide (Sect.\ref{sec:XPG}) - the specification;
% with XPG3 (1988) was aimed to converge to POSIX.
 
\url{http://sourceware.org/ml/libc-alpha/2000-02/msg00005.html}


\subsection{Macro to detect CPU architecture}
\label{sec:macro-detect-CPU}

Some macros can receive more than two possible values, and the meaning of the
value depends on the compiling option

In Windows:
\begin{itemize}
  
  \item x86 CPU: \verb!_M_IX86! (can be 600 (default: Blend if compile with /GB,
  or (Pentium Pro, II, III) if compile with /G6), 500 (Pentium), 300 (8-386),
  400 (80486))
  
  
  \item x64 CPU: \verb!_M_X64!
  
  
  \item Intel IA64 CPU: \verb!_M_IA64!
\end{itemize}
\url{https://msdn.microsoft.com/en-us/library/b0084kay(VS.80).aspx}


\subsection{-- GNU macros: i386, MIPS, SPARC}
\label{sec:macro-GNU-to-detect-CPU}

\begin{verbatim}
i386        i386, __i386, __i386__
MIPS        __mips
SPARC       sparc, __sparc
\end{verbatim}

{\bf Alpha CPUs}
\begin{verbatim}

\end{verbatim}

{\bf PowerPC CPUs}
\begin{verbatim}
#if defined(__powerpc__) || defined(__ppc__) || defined(__PPC__)
	/* POWER ---------------------------------------------------- */

#if defined(__powerpc64__) || defined(__ppc64__) || defined(__PPC64__) || \
	defined(__64BIT__) || defined(_LP64) || defined(__LP64__)
	/* POWER 64-bit --------------------------------------------- */

#else
	/* POWER 32-bit --------------------------------------------- */

#endif
#endif
\end{verbatim}
\url{http://nadeausoftware.com/articles/2012/02/c_c_tip_how_detect_processor_type_using_compiler_predefined_macros}

\subsection{Macro to detect compiler type}
\label{sec:macro-detect-compiler}

Print out the list of all macros
\begin{verbatim}
gcc -E -dM - < /dev/null |less
\end{verbatim}

\begin{itemize}
  \item \verb!__GNUC__! : GNU C compiler
  
\begin{lstlisting}
#ifdef __GNUC__
// do my gcc specific stuff
#else
// ... handle this for other compilers
#endif
\end{lstlisting}
\end{itemize}

The full list
\begin{lstlisting}
__clang__, 
__GNUC__, __GNUG__, 
__HP_aCC, __HP_cc, 
__IBMCPP__, __IBMC__, 
__ICC, __INTEL_COMPILER, 
_MSC_VER, 
__PGI, 
__SUNPRO_C, __SUNPRO_CC
\end{lstlisting}



\subsection{-- compiler version: GCC, MIPSpro, SUNPro, SunPro C++}
\label{sec:macro-detect-compiler-version}

In Windows and use Visual Studio: we check \verb!_MSC_VER! macro which reports

\begin{verbatim}
GCC         __GNUC__ (major version), __GNUC_MINOR__
MIPSpro     _COMPILER_VERSION (0x741 for MIPSpro 7.41)
SunPro      __SUNPRO_C (0x570 for Sun C 5.7)
SunPro C++  __SUNPRO_CC (0x580 for Sun C++ 5.8)
\end{verbatim}

The full list
\begin{itemize}
  \item Clang - Sect.\ref{sec:clang}
  
  \item GNU C Compiler (gcc)- Sect.\ref{sec:GCC}
  
  \item Intel C Compiler (icc) - Sect.\ref{sec:icc}
  
  \item PGI C compiler (pgcc) - Sect.\ref{sec:pgcc} 
  
  \item XL C compiler (xlcc) - Sect.\ref{sec:xlcc}
\end{itemize}

\begin{lstlisting}
  // Clang (LLVM-based compiler)
__clang_major__, __clang_minor__, __clang_patchlevel__, __clang_version__,
__GNUC_MINOR__, __GNUC_PATCHLEVEL__, __GNUC__, __GNUG__, 
__INTEL_COMPILER_BUILD_DATE, 
_MSC_BUILD, _MSC_FULL_VER, _MSC_VER, 
__PGIC_MINOR__, __PGIC_PATCHLEVEL__, __PGIC__, 
__VERSION__, __xlC_ver__, __xlC__, __xlc__
\end{lstlisting}
%__HP_aCC, __HP_cc, __IBMCPP__, __IBMC__, __ICC, __INTEL_COMPILER,
%__SUNPRO_C, __SUNPRO_CC,

\url{http://stackoverflow.com/questions/2989810/which-cross-platform-preprocessor-defines-win32-or-win32-or-win32}

