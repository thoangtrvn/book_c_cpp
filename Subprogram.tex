\chapter{Function and Procedures}

\section{Function prototype}

In header file, you only declare function prototype. YOu don't have to use the
argument name there, but it's recommend to make the code easy to read
\begin{verbatim}
#.h
void do_something(int, int)

# the same as
void do_something(int a, int b)
\end{verbatim}

\section{Optional parameters}

NOTE: After the first optional argument, all the remaining should be optional.
So, always put the optional argument the last.
\begin{verbatim}
int myfunction( int optional = 3);
\end{verbatim}

RECOMMEND: Your arguments must be placed in order, sorted by how often you will
pass values to them.
\begin{verbatim}
int myfunction( int mandatory, int optional1 = 3, int optional2 = 5 );
\end{verbatim}

If we want to use function prototype, i.e. put the prototype in the header file,
and the implementation in the code, then only put the optional argument with
default value only in the header file, and declare as a normal argument in the
source (.cpp) file
\begin{verbatim}
#.h
void myfunc(char *blah, int mode = NULL);

#.cpp
void myfunc(char *blah, int mode)
{
if(mode == NULL)
do_something();
else
do_something_else();
}
\end{verbatim}
\footnote{\url{http://www.velocityreviews.com/forums/t268385-function-prototype-with-optional-parameters.html}}

\section{Overloaded functions}
\label{sec:overloaded-function}
 
A second approach: a simple thing is to use overloaded function, i.e. define two
different functions with the same name, but different number of arguments. Example: we
want the default value of 'c' is 2.
\begin{verbatim}
foo (int a)  
foo (char b)  
foo (float c , int d)
\end{verbatim} 


 \subsection{C}

However, overloaded function is not supported in C until C1X
(Sect.\ref{sec:overloaded-function_C1X}). However, we can achieve this using
different strategies

It can use
\begin{enumerate}  
  \item One bad way is to use two arguments: one is \verb!void! pointer, and the second
one is typeid, to indicate the first argument's type. 
   
  \item variadic function: use \verb!va_args! package (header file
  \verb!<stdarg.h>!). It is \verb!printf! style functions, i.e. type as
  argument (Sect.\ref{sec:variadic_function}).
  
\begin{verbatim}
#include <stdarg.h>
#include <stdio.h>

void va_overload2(int p1, int p2);
  
  /* use this only when p2 == 7 */
void va_overload3(int p1, int p2, int p3);

static void va_overload(int p1, int p2, ...)
{
	if (p2 == 7)
	{
		va_list v;
		va_start(v, p2);
		
		int p3 = va_arg(v, int);
		va_end(v);
		va_overload3(p1, p2, p3);
		
		return;
	}

	va_overload2(p1, p2);
}
\end{verbatim}

  
  \item tagging with the type: use a struct as argument, this struct contain a
  typeid, and a union.
\begin{verbatim}
typedef struct {
    int type;
    union {
    	int a; 
    	float b; 
    	char c;
    } my_union;
} my_struct;

#define T_INT   1
#define T_FLOAT 2
#define T_CHAR  3

void set_overload (my_struct *whatever) 
{
    switch (whatever->type) 
    {
    	case T_INT:
    		whatever->my_union.a = 1;
    		break;
    	case T_FLOAT:
    		whatever->my_union.b = 2.0;
    		break;
    	case T_CHAR:
    		whatever->my_union.c = '3';
    }
}

void printf_overload (my_struct *whatever) {
    switch (whatever->type) 
    {
    	case T_INT:
    		printf("%d\n", whatever->my_union.a);
    		break;
    	case T_FLOAT:
    		printf("%f\n", whatever->my_union.b);
    		break;
    	case T_CHAR:
    		printf("%c\n", whatever->my_union.c);
    		break;
    }

}

int main (int argc, char* argv[])
{
    my_struct s;

    s.type=T_INT;
    set_overload(&s);
    printf_overload(&s);
}
\end{verbatim}
  
  \item \verb!opengl! style functions, i.e. type in function name. 
\begin{verbatim}
set_int(int*)

set_float(float*)

\end{verbatim}

\url{http://www.glprogramming.com/red/chapter01.html\#name3}
  
  \item Use C++ compiler for C code.
  
  \item Use the function to determine if two types are the same. This function
  ignores top level qualifiers (const, volatile, \ldots). It means \verb!int!
  and \verb!const int! are considered the same. 
\begin{verbatim}
int __builtin_types_compatible_p (type1, type2);
  // 1 => the two unqualified versions are the same
  // 0 => otherwise (not-compatible)

NOTE:
int      ==    const int
int[]    ==    int[5]

long int <>    char*
short*   <>    short**
enum     ==    enum  
\end{verbatim}   

Example: 
\begin{verbatim}
#define foo(x)                                                  \
  ({                                                            \
    typeof (x) tmp;                                             \
    if (__builtin_types_compatible_p (typeof (x), long double)) \
      tmp = foo_long_double (tmp);                              \
    else if (__builtin_types_compatible_p (typeof (x), double)) \
      tmp = foo_double (tmp);                                   \
    else if (__builtin_types_compatible_p (typeof (x), float))  \
      tmp = foo_float (tmp);                                    \
    else                                                        \
      abort ();                                                 \
    tmp;                                                        \
  })

#define foo(a) \
((__builtin_types_compatible_p(int, a)?foo(a):(__builtin_types_compatible_p(float, a)?foo(a):)
\end{verbatim}  


\end{enumerate}

\subsection{C1x}
\label{sec:overloaded-function_C1X}

C1X support type generic macro (type generic expression).

\begin{verbatim}
#define cbrt(X) _Generic((X), long double: cbrtl, \
                              default: cbrt, \
                              float: cbrtf)(X)
\end{verbatim}




\subsection{C++}
 
C++ supports overloaded functions using ``name mangling''
(Sect.\ref{sec:name_mangling})
 
\begin{verbatim}
int function(int a, int b, int c)
{
    return a*b*c;
}

int function(int a, int b)
{
    return a*b*2;
}
\end{verbatim}

Actually, the second approach is recommend in many cases,
e.g.\footnote{\url{http://stackoverflow.com/questions/703453/optional-function-parameters-use-default-arguments-null-or-overload-the-funct}}
\begin{verbatim}

\end{verbatim}

\section{Restrictions on function declaration}


You CAN'T have
\begin{enumerate}
  \item a function return a function: foo()()
  \item a function return an array: foo()[]
  \item an array hold function: arr[]()
\end{enumerate}

You CAN have
\begin{enumerate}
  \item a function returning a pointer to a function: int (*foo())()

\verb!foo! is a function (no argument) and returns a pointer to a function
that (1) accepts no argument, (2) returns an integer.  
  
  \item a function returning a pointer to an array: int (*foo())[]
  
\verb!foo! is a function (no argument) and returns a pointer to a an array
of type 'int'.  
  
  \item an array holding pointers to functions: int (*arr[])()

\verb!arr! is an array, each element is a pointer to a function that
(1) accepts no argument, (2) returns an integer.  
    
  \item an array holding arrays : int arr[][];
  
  
\end{enumerate}

\section{Pass by reference vs. Pass by pointer vs. Pass by value}
\label{sec:pass-by-reference}
\label{sec:pass-by-pointer}
\label{sec:pass-by-value}



With the exception of array and functions, \textcolor{red}{C always pass
argument 'by value'}, i.e. a copy of the data is pass to the function. So, in C,
the only way to modify a data, at an existing location, inside the function that
take the effect even after the function complete is to pass its address to the
function, i.e. pass by pointer and then use the \verb!&! operator when calling.
Passing an array is indeed passing a pointer
(Sect.\ref{sec:function_pass-array}). 

\begin{mdframed}
When passing by value, parameters are passed to registers (for speed) where
possible. So, it's not correct to say that parameters are always put on the
stacks from right to left, which is an over simplification. A struct, no matter
how simple it is, e.g. a single member, is passed to the stack.

Passing a single \verb!int i! to a function can be put in the register. Passing
a struct \verb!struct s_tag {int i;} s! always be put on the stack. 
\end{mdframed}

In C++, we can pass by reference.

Example: pass by pointer vs. pass by reference

\begin{minipage}[t]{0.5\textwidth}
{\small \begin{verbatim}
//pass by reference
void swapnum(int &i, int &j) {
  int temp = i;
  i = j;
  j = temp;
}

int a = 10;
int b = 20;

swapnum(a, b);
\end{verbatim}}
\end{minipage}
\begin{minipage}[t]{0.5\textwidth}
{\small \begin{verbatim}
//pass by pointer
void swapnum(int *i, int *j) {
  int temp = *i;
  *i = *j;
  *j = temp;
}

int a = 10;
int b = 20;

swapnum(&a, &b);
\end{verbatim}}
\end{minipage}


If we don't want the callee to modify the data, passing by value is acceptable
for 'simple' types like primitives (int, long). For objects that are expensive
to copy, we should use \verb!const! references.

In C/C++, we can pass an argument in three forms
\begin{verbatim}
void f1(std::string const& s); // Pass by reference-to-const
void f2(std::string const* sptr); // Pass by pointer-to-const
void f3(std::string s); // Pass by value
\end{verbatim}
So, when we call
\begin{verbatim}
std::string str('helloworld!');

f1(str); 
f2(&str); 
f2(NULL);
\end{verbatim}

IMPORTANT: A pointer can point to nothing, i.e. pointing to NULL. By a reference
is always associated with an object. So, a pointer parameter can receive a NULL
parameter, but a reference parameter CANNOT. If you define a code using pass by
reference, it's pushing an implicit NULL-check at the caller before switching to
control to the callee. In overall, use reference when you can, and use pointer if you
have to
\footnote{\url{http://www.parashift.com/c++-faq-lite/refs-vs-ptrs.html}}. It
means you should use pass-by-pointer if there's ever a chance that you could 
pass 'no object' where it can brings some meaning, e.g. .

NEVER use references as outputs, use pointers.


\subsection{TIPS: Passing pointer (* or **) with 'const' non-modified purpose}

Just like assigning between pointer types, passing pointers requires the same
2 rules
\begin{enumerate}
  \item both operands are pointers to qualified or unqualified versions of
  compatible types
  
  \item the type pointed to by the left has all the qualifiers of the type
  pointed to by the right
\end{enumerate}

Consider the example
\begin{verbatim}
 // C and C++ 
void foo (const int & p)
{
  int a = p+1;
  p = 2; // error
}

  // C++ only
void foo (const int * p)
{
  int a = *p+1;
  *p = 2 ;; // error
}
\end{verbatim}

We can pass
\begin{verbatim}
int i;

 // C and C++
foo(i);

  // C++ only
foo(&i);
\end{verbatim}
NOTICE: \verb!int*! and \verb!const int *! are equivalent, as both are pointers
to a qualified and unqualified version of \verb!int!, and the left side
(\verb!const int! has all the qualifiers of \verb!int!)
\begin{verbatim}

int * = a pointer pointing to a type 'int'

const int * = a pointer pointing to a type 'const int' 
\end{verbatim}
\verb!const! is the qualifier for the pointer type.


However, this is an error
\begin{verbatim}
main (int argc, char ** argv) {

   foo (argv);
}

void foo(const char ** p){} 
\end{verbatim}
The reason is that 
\begin{verbatim}
char ** = an array, each element is of type 'char*' (a string)
  // a pointer pointing to 'char*' 
 
const char** = an array, each element is of type 'const char*'
  // a pointer pointing to 'const char*' 
\end{verbatim}
The \verb!const! qualifier for the pointer type \verb!char*! should be written
as
\begin{verbatim}
char * const
\end{verbatim}
It means \verb!const char*! is not compatible with \verb!char*!.

\url{https://groups.google.com/forum/#!topic/comp.lang.c/4vK-MYAHKEs}

\section{Passing an array (i.e. a pointer)}
\label{sec:function_pass-array}

In C, passing an array to a function is actually passing a pointer
(Sect.\ref{sec:pointer_array}). When passing to a function, an array decays
immediately into a pointer; thus, an array is never actually passed to a
function. Example:
in both cases below, x is a pointer, not an array, even if it is declared as
\verb!x[]!
Example:
\begin{verbatim}
void foo(char x[]);
  // is the same as
void foo(char* x);
\end{verbatim}
Inside the function foo(), \verb!sizeof(x)! is the same as \verb!sizeof(char*)!
which is the size of the memory to hold a pointer (Sect.\ref{sec:type_info}).

The conversion from array-like declarator into pointers hold only within
functional formal parameter declaration, nowhere else (Question 6.4
\footnote{\url{http://c-faq.com/~scs/cgi-bin/faqcat.cgi?sec=aryptr}}). So arrays
are passed by reference, even though the rest of C language uses pass by value.

C++ does not allow to pass an entire array as an argument to a function.
However, you can pass a pointer to an array by specifying the array's name
without an index. If you don't wan to use C-style array (as we describe in this
session), then you can use STL container, e.g. vector. 

\subsection{array with minimum length}

When passing an array, if the keyword \verb!static! appears in the array type
derivation, then the value of the actual argument shall provide access to the
first element of an array with at least as many elements as specified by the
size expression (in this case is 100).
\begin{verbatim}
void someFunction(char someArray[static 100])
{
    // do something cool here
}
\end{verbatim}
Simply, it means the function expects the size to be at least 100 elements.

\subsection{passing an array that cannot be changed}

\begin{Verbatim}
void someFunction(char someArray[const])
{
    // do something cool here
}
\end{Verbatim}
We can not do \verb!someArray=someOtherArray!.

\subsection{1D-array}

An array name is also the pointer to the first position in the array, so remeber
that 
\begin{verbatim}

    t[i] = *(t+i)

\end{verbatim}

\begin{enumerate}
  \item Use pure pointer
\begin{verbatim}
void myFunction(int *param)
{  }
\end{verbatim}

  \item Use a sized array (array of fixed size)
\begin{verbatim}
void myFunction(int param[10])
{}
\end{verbatim}

  \item Use an unsized array
\begin{verbatim}
void myFunction(int param[], int sizearray)
{ }
\end{verbatim}
NOTE: We'd better pass the size of the array as well.

\end{enumerate}

\subsection{2D-array}

Using 2D or 3D array is not recommend in C/C++ as they are not continuous in
memory, you'd better use 1D array. 
\begin{verbatim}
void myFunction(double * arr, int R, int C)
{
    arr[x * C + y] = 5;
    etc...
}
\end{verbatim}


If you really need to use 2D array, then
\begin{enumerate}
  \item You pass along the array size information
\begin{Verbatim}
double maximum(int n, int m, double a[n][m]);
double maximum(int n, int m, double a[*][*]);
double maximum(int n, int m, double a[ ][*]);
double maximum(int n, int m, double a[ ][m]);
\end{Verbatim}

or we use \verb!restrict! and \verb!static!
\begin{Verbatim}
void f(double (* restrict a)[5]);
void f(double a[restrict][5]);
void f(double a[restrict 3][5]);
void f(double a[restrict static 3][5]);
\end{Verbatim}
We use \verb!restrict! (C99) to indicate strict-aliasing, as two dimensional
arrays is indeed an array of pointers (row-major in C), each pointer pointing to
a 1D array. The last one means that the actual argument must be a non-NULL
pointer pointing to the first of at least 3 arrays of 5 doubles.

  \item Use pointer to pointer
\begin{verbatim}
void someFunction(int** someArray)
{
  someArray[y][x] = example;
}

void someFunction(int **array, size_t rows, size_t cols)
\end{verbatim}

   \item An array containing pointer
\begin{verbatim}
void passFunc(int *a[10]) //array containing pointers
{  }

void passFunc(int (*array)[10], size_t rows)
{  }
\end{verbatim}

  \item A 'real' 2D array, with the size of one dimension MUST be given
\begin{verbatim}
void someFunc(int someArray[][3])
{
}
\end{verbatim}
 
   \item Use template to allow dynamic size in the second dimension
\begin{verbatim}
template <size_t size_y>
void func(double arr[][size_y])
{
    printf("%X\n", arr);
}
\end{verbatim}
   \item Or even this is better
\begin{verbatim}
template<int R, int C>
void myFunction(double (&myArray)[R][C])
{
    myArray[x][y] = 5;
    etc...
}
\end{verbatim}

\end{enumerate}

\subsection{3D-array}

NOT recommend at all. Use STL vector instead.

\section{Passing a string as argument to a function}

As a general rule of const correctness, when passing to/from functions, if
you're not changing the string data, use \verb!const char*!.
It's important to create a local string (using either \verb!strcpy()!,
\verb!strncpy()!, or \verb!strdup()!) and work with this local string, so that
it doesn't modify the global data.

Example:
\begin{verbatim}
void do_something(const char *string)
{
   char* line;
   line = strdup(string);
   // work with 'line'
   ...
}
\end{verbatim}
See the discussion on how to copy a string in Sect.\ref{sec:C_string}.

\verb!char*! should only be used if you're modifying it.
Also, make sure to allocate the memory before passing the pointer and also pass
the size of the data object.

For C++ string we use
\begin{verbatim}
void do_something(const std::string &string)
{ //no copy and make sure it is not modified

}
\end{verbatim}
We can also define implicit value (optional argument)
\begin{verbatim}
//in the header
void do_something(const std::string &str = std::string("default-value"));
\end{verbatim}


\section{Return by reference or const reference}

In terms of speed, both are the same, as there is no data copy occur, only the
reference is returned.

Return by reference, i.e. you can modify the returned value
\begin{lstlisting}
class A {

  public:
    // non-const member function
    int& getImem() { return imem;};
  private:
    int imem;
}
\end{lstlisting}
NOTE: \verb!imem! is a short form of \verb!this->imem!. 
In a non-const member,  \verb!this! is of type \verb!A* const!, i.e. the pointer
is fixed, but not the instance it points to.

Return by const reference, i.e. you cannot modify the returned value
\begin{lstlisting}
class A {

  public:
    int& getImem() const { return imem;};
  private:
    int imem;
}
\end{lstlisting}
In a non-const member,  \verb!this! is of type \verb!const A* const!, i.e. the
pointer is const, and also the instance it points to is const.


\section{Return an array or a pointer}

There is no way to return an array in C. Instead, it returns the pointer to the
first element in the array. Example: 
\begin{verbatim}
char * foo(...)
{

}
\end{verbatim}

POTENTIAL BUGS: declare an automatic array, which is local to the function and
return with the pointer to that array
\begin{verbatim}
char * foo()
{
   char buffer[120];
   
   /* do something */
   
   return buffer;
}
\end{verbatim}
Automatic variables are allocated on the stack. Once the flow of control leaves
the scope of the function, the automatic variables declared in that function are
not guaranteed to exist, as the stack is available for reuse, and will
certainly be overwritten by the next function to be called. Depending the
location of the automatic variables declared in the stack, and the number of
variables declared in the next function call, the next function call may or may
not overwrite the automatic variables. Thus, the data for the automatic
variable might be overwritten immediately (at the next function call), or later,
leading to a hard-to-find bug problem.

There are different strategies to overcome that automatic array issue:
\begin{enumerate}
  \item  (char* only) Return a pointer to the string literal
\begin{verbatim}
char * foo() 
{
  return "Some strings here";
}
\end{verbatim} 
This is the simplest solution, and the literal string is stored in a read-only
memory (Sect.\ref{sec:read-only_memory}). However, the returned string cannot be
changed.

  \item use globally declared array
\begin{verbatim}
#define BUF_SIZE 120
char my_global_array[BUF_SIZE];

char * foo()
{
  my_global_array[i] = 'a';  //update the string
  return my_global_array;
}
\end{verbatim}
However, it is not thread-safety and any functions can overwrite the data.

  \item Use a static array
\begin{verbatim}
char * foo()
{
  static char buffer[20];
  
  return buffer;
}
\end{verbatim}
The static data is stored in the {\bf static memory space}.  However, the data
can be modified in the next call to this function. Also, the buffer size is
fixed, and cannot be changed.

  \item Explicit allocate the memory inside the function, and let the caller
  handle the free
\begin{verbatim}
char * foo()
{
   char * buffer;
   buffer = malloc(BUF_SIZE);
   
   return buffer;
}
\end{verbatim}
There is a potential of memory leak (if the caller doesn't free the memory), or
it can crash if the memory is freed while it is being used somewhere else.

   \item [BEST] This is the best solution: let the caller allocate the memory,
   and handle the memory free. Then, we need to pass the buffer size as well
\begin{verbatim}
void foo(char* buffer, int size) 
{

  // do something with buffer, within the 'size' limit
}


void main()
{
buffer = malloc(BUF_SIZE);

foo(buffer, BUF_SIZE);

free(buffer);
}
\end{verbatim}
\end{enumerate}

So, it's important to dynamically allocated the
memory inside the array.
\begin{verbatim}
int * foo()
{
   int * a;
   malloc(a, 8 * sizeof(int));
   
   /* do something with a */
   return a;
}
\end{verbatim}
IMPORTANT: In the caller, remember to free the returned array, as
\verb!malloc()! is hidden in the callee. 

However, it's better to pass the array as an argument. The advantage is that the
caller can allocate the memory before passing it to the function.
\begin{verbatim}
/* Convert string of integers into int array. */
void splitString( const char string[], int result[], int n) {
    //....
}
\end{verbatim}
rather than
\begin{verbatim}
int * splitString( char string[], int n )
{
    int newArray[n];

    // CODE

    return ( newArray );
}
\end{verbatim}
\url{http://stackoverflow.com/questions/5378768/returning-arrays-pointers-from-a-function}


\section{Nested function}

Nested function is not allowed in standard C nor standard C++. So, depending on
the compiler, you may or may not be able to use this. In GCC, nested function is
available as a language extension in C. To turn-off this extension in GCC,
compile the code with \verb!-ansi! flag (set standard to c89, and thus not able to use
features incompatible with ISO C90). In Fortran 90, we use \verb!CONTAIN!
keyword (one level of nested function).

\begin{framed}
In C++, we can use lambdas as inner functions.

\begin{verbatim}
double some_function( double x, double y)
{
   auto inner_function = [&]() { return x * x; }

   double z;
   z = inner_function ();
   return z + y;
}
\end{verbatim}
\end{framed}

\begin{verbatim}
double stuff(double a, double b)
{
  struct parameters
  {
    double a, b;
  };

  double f(double x, void * params)
  {
    struct parameters p = (struct parameters *) params;
    double a = p->a, b = b->b;
    return some_expression_involving(a,b,x);
  }
  struct parameters par = {a,b};

  return integrate(&f, &par);     // return added!
}
\end{verbatim}

\url{http://stackoverflow.com/questions/957592/functions-inside-functions-in-c}

\section{Function pointer}
\label{sec:function_pointer}

Example: function pointer vs. normal function
\begin{lstlisting}
int f(), *f(), (*f)();
\end{lstlisting}
declare a function returning an integer, a function returning a pointer to an
integer, a pointer to a function returning an integer; 


The type of a function pointer is just like the function declaration, but with
"(*)" in place of the function name. So a pointer to:
\begin{verbatim}
int foo( int )
\end{verbatim}
would be
\begin{verbatim}
int (*)( int )
\end{verbatim}
If we want to put a name instance, i.e. a variable, we put the name inside (*)
\begin{verbatim}
int (*foo_ptr)( int )
\end{verbatim}
which declares a variable called \verb!foo_ptr! that points to a function of this type.


Example: function pointer vs. normal function
\begin{lstlisting}
int f(), *f(), (*f)();
\end{lstlisting}
declare a function returning an integer, a function returning a pointer to an
integer, a pointer to a function returning an integer.

\subsection{a pointer to a function}

The syntax for declaring a function pointer might seem messy at first, but
the key is to read inside-out. Based on the precedence rule in C and C++, the
parentheses () has the highest precedence of all.
\begin{verbatim}
  // f is a function accepting an 'int' param
  //     and returning a pointer of type 'int' 
int *f(int)

  // f is a function pointer pointing to a function that accepts a single
  //  argument of type 'int' and return a value of type 'int' 
int (*f)(int) 
\end{verbatim}

NOTE: \textcolor{red}{How to identify a function pointer?} A function pointer is
preceded by an asterisk (*) and then enclosed by a parenthese ().

\begin{verbatim}
void (*foo)(int);  /* read inside-out: so (*foo) is a function
                      or 'foo' as a pointer to a function that
                      accept a single input of type 'int' and return 'void'
                   */
                   
void *(*foo)(int *); /* read inside-out (*foo) is a function 
                      so 'foo' as a pointer to a function that
                      accept a single input of type 'int' and return 'void*'
                   */                    
void (*signal(int sig, void (*func)(int)))(int); 
                    /* read inside-out (*func) is a function 
                      so 'func' as a pointer to a function that
                      accept a single input of type 'int' and return 'void'
                      
                             then (*signal) is a function
                    so 'signal' is a pointer to a function that
               	    accepts 2 inputs, an 'int' and a pointer to a function like
               	                      (*func),
               	    and returns a pointer to a function taking an 'int' param
               	                                        and returning void
               	     */                    
\end{verbatim}

Explain:
\begin{verbatim}
       signal                                     -- signal
       signal(                          )         -- is a function
       signal(    sig,                  )         -- with a parameter named sig
       signal(int sig,                  )         --   of type int
       signal(int sig,        func      )         -- and a parameter named func
       signal(int sig,      (*func)     )         --   which is a pointer
       signal(int sig,      (*func)(   ))         --   to a function
       signal(int sig,      (*func)(int))         --     taking an int parameter
       signal(int sig, void (*func)(int))         --     and returning void
      *signal(int sig, void (*func)(int))         -- returning a pointer
     (*signal(int sig, void (*func)(int)))(   )   -- to a function
     (*signal(int sig, void (*func)(int)))(int)   --   taking an int parameter
void (*signal(int sig, void (*func)(int)))(int);  --   and returning void
\end{verbatim}
\url{http://stackoverflow.com/questions/9500848/how-do-i-read-this-complex-declaration-in-c}


The pointer to a function can be used as a regular variable, though it's not a
traditional variable. This allows the use of {\it callback function} in Windows,
i.e. a function that runs in response to a given event
(Sect.\ref{sec:callback}).

\subsection{typedef a function pointer}
\label{sec:typedef-function-pointer}

As we have a pointer to a data type, and we have a pointer to a function, and a
function that can return a function pointer, can we make a function pointer as a
data type?

The answer is YES, using \verb!typedef! A function pointer is a user-defined
data type.


Example: equivalent statements
\begin{verbatim}
extern SignalHandler alt_signal(void);

extern void (*alt_signal(void))(int signum);
\end{verbatim}
with
\begin{verbatim}
typedef void (*SignalHandler)(int signum);
\end{verbatim}

% 
% The hard part is the type is a pointer to functions, where, unlike other data
% type, the functions can have a return value or parameters. As a function pointer
% store the address of a function, the syntax thus looks odd with the keyword is
% surrounded by other things

{\bf Example}: \verb!t_somefunc! is the name of the new data type
\begin{verbatim}
typedef int (*t_somefunc)(int,int);
\end{verbatim}
It means \verb!t_somefunc! is the keyword that represent a pointer to a function
which accepts 2 parameters of type \verb!int! and returns the value of type
\verb!int!.

So we can use
\begin{verbatim}
int product(int u, int v) {
  return u*v;
}

t_somefunc afunc = &product; /*  a pointer to a specific function of the same
                                 signature
                             */
 ...
int x2 = (*afunc)(123, 456); // call product() to calculate 123*456
\end{verbatim}

{\bf TIPS}: breaking down to small chunks is better
\begin{verbatim}
typedef int (*FUNC_TYPE_1)(void);
typedef double (*FUNC_TYPE_2)(void);
typedef FUNC_TYPE_1 (*FUNC_TYPE_3)(FUNC_TYPE_2);
\end{verbatim}
than (not we keep the innermost name which is \verb!FUNC_TYPE_3!)
\begin{verbatim}
typedef  int(*(*FUNC_TYPE_3)(double (*)(void)))(void);
\end{verbatim}

{\bf TIPS}: using the website cdecl to give you the explanation
\url{http://cdecl.org} which makes use of the \verb!cdecl! utility
\url{http://linuxcommand.org/man_pages/cdecl1.html}
% which is a function that accepts 2 arguments: one type \verb!int!, and one is a
% pointer to a function that ( accepts one argument type \verb!int! and returns
% nothing), and returns a pointer to a function that (takes one argument of type
% \verb!int! and returns nothing).
% We can use typedef to simplify the above declaration
% \begin{verbatim}
% typedef void (*SignalHandler)(int signum);
% 
% extern  SignalHandler signal(int signum, SignalHandler handler);
% \end{verbatim}

% Example: the \verb!signal()! function is a system-call that accepts the first
% argument as the SIGNAl type and the second argument as the pointer to the
% function to be evoked to handle the signal. ANSI C standard for this system call
% \begin{verbatim}
% void (*signal(int sig, void (*func)(int))) (int);
% \end{verbatim}
% By using \verb!typedef!, we make the decleration simpler
% \begin{verbatim}
% typedef void (*ptr_to_func)(int);
% 
% ptr_to_func signal(int, ptr_to_func);
% \end{verbatim}



\subsection{Array of function pointer}
\label{sec:function-pointer-array}

Arrays follow the normal C syntax of putting the brackets near the variable's identifier
\begin{verbatim}
int (*foo_ptr_array[2])( int )
\end{verbatim}
declares a variable called \verb!foo_ptr_array! which is an array of 2 function pointers.


The syntax can get pretty messy, so it's often easier to make a typedef to the
function pointer and then declare an array of those instead
\begin{verbatim}
typedef int (*foo_ptr_t)( int );
foo_ptr_t foo_ptr_array[2];


//a pointer to a function with 'n' arguments, and return type 'R'
typedef R (*fptr)(A1, A2... An);
\end{verbatim}



Dynamically allocated an array of function pointers
\begin{verbatim}
int (**a1)( int ) = calloc( 2, sizeof( int (*)( int ) ) );
foo_ptr_t * a2 = calloc( 2, sizeof( foo_ptr_t ) );

fptr* arr = calloc(num_of_elements,sizeof(fptr));

//for some type T
T (*afp[N])(); 

T (**pfp)() = calloc(num_elements, sizeof *pfp);

//or

T (**pfp)() = malloc(num_elements * sizeof *pfp);
\end{verbatim}
then we use
\begin{verbatim}
T x = (*pfp[i])();

//or

T x = pfp[i](); // pfp[i] is implicitly dereferenced
\end{verbatim}

A different way:  you can declare a pointer to an array of pointers to functions, and then allocate that as follows:
\begin{verbatim}
T (*(*pafp)[N])() = malloc(sizeof *pafp);

//here,  you would have to deference the array pointer when making the call:
x = (*(*pafp)[i])();
\end{verbatim}

\url{https://stackoverflow.com/questions/5488608/how-define-an-array-of-function-pointers-in-c/5488718}

\section{Applications of function pointer}

A function pointer is used in 
\begin{enumerate}
  \item enable passing the function name as a parameter to another function  
  
  \item callback - Sect.\ref{sec:callback}
  
  \item signal - Sect.\ref{sec:signal}
\end{enumerate}

\subsection{modify function's body}


When programming, you want to reduce the number of APIs to develop, by using the
same API for related operations, e.g. you have to write two different function
for a sorting procedure
\begin{verbatim}
result = sort_ascend(input)
result = sort_descend(input)
\end{verbatim}

Typically, when you design a function whose processing can be customized, you
use a parameter whose values can be used as a flag for choosing what method is
used
\begin{verbatim}
result = sort(input, 'a') /* ascending */

result = sort(input, 'd') /* descending */
\end{verbatim}

However, there are situations that you want a function whose processsing
to be more flexible, i.e. user can pass a pointer to a function whose does the
exact processing the want
\begin{verbatim}

 /* sort(int * arr[], (int*)(int*)) */
result = sort(input, my_own_processing)

int * sort_ascending(int* input)
{


}

int * sort_descending(int* input)
{


}

int * sort_ascending_and_keep_positive_values_only(int* input)
{


}

\end{verbatim}

\subsection{signal() function in C}
\label{sec:signal}

Read Sect.\ref{sec:signal()} for more information about signals.
\begin{verbatim}
extern void (*signal(int, void(*func)(int)))(int);
\end{verbatim}
\verb!signal! associates a signal handler function \verb!func! with a signal
\verb!sig!, and returns the pointer to the old signal handler function.

% {\bf Example}: the signal() function from C-standard
% \begin{verbatim}
% extern void (*signal(int, void(*)(int)))(int);
% \end{verbatim}
So, by defining a new data type
\begin{verbatim}
typedef void (*SignalHandler)(int signum);
\end{verbatim}
which is a type as a pointer to a function accepting an 'int' parameter and
return 'void'. So, a simpler form for a signal() function is
\begin{verbatim}
extern  SignalHandler signal(int signum, SignalHandler handler);
\end{verbatim}

Example: the struct \verb!Handlers! keeps a pair of (signal-ID,
its-associated-handler)
\begin{lstlisting}
static void alarm_catcher(int signum)
{
    fprintf(stderr, "%s() called (%d)\n", __func__, signum);
}

static void signal_catcher(int signum)
{
    fprintf(stderr, "%s() called (%d) - exiting\n", __func__, signum);
    exit(1);
}

static struct Handlers
{
    int              signum;
    SignalHandler    handler;
} handler[] =
{
    { SIGALRM,   alarm_catcher  },
    { SIGINT,    signal_catcher },
    { SIGQUIT,   signal_catcher },
};

int main(void)
{
    size_t num_handlers = sizeof(handler) / sizeof(handler[0]);
    size_t i;

    for (i = 0; i < num_handlers; i++)
    {
        SignalHandler old_handler = signal(handler[i].signum, SIG_IGN);
        if (old_handler != SIG_IGN)
            old_handler = signal(handler[i].signum, handler[i].handler);
        assert(old_handler == SIG_IGN);
    }

    ...continue with ordinary processing...

    return(EXIT_SUCCESS);
}
\end{lstlisting}

\url{http://stackoverflow.com/questions/1591361/understanding-typedefs-for-function-pointers-in-c-examples-hints-and-tips-ple}


\subsection{callback}
\label{sec:callback}

Callbacks in C are usually implemented using function pointers and an associated
data pointer. When an event happens, your function is called with your data and some
event-specific data.

You pass your function \verb!on_event()! and data pointers to a framework
function \verb!watch_events()!.

\begin{itemize}
  \item define the type representing events

An event need (1) user-data, (2)   
\begin{verbatim}
struct event_cb {
    event_cb_t cb;
    void *data;
};
\end{verbatim}

  \item first, define the type of functions being used as callback
\begin{verbatim}
typedef void (*event_cb_t)(const struct event *evt, void *userdata);
\end{verbatim}


  \item define the register function that is used to register a particular
  callback, to a particular event
\begin{verbatim}
int event_cb_register(event_cb_t cb, void *userdata);
\end{verbatim}

   \item register a callback
 
 \begin{verbatim}
 static void my_event_cb(const struct event *evt, void *data)
{
    /* do stuff and things with the event */
}

...
   event_cb_register(my_event_cb, &my_custom_data);
...
 \end{verbatim} 
\end{itemize}

Example:
\begin{verbatim}
void populate_array(int *array, size_t arraySize, int (*getNextValue)(void))
{
    for (size_t i=0; i<arraySize; i++)
        array[i] = getNextValue();
}

int getNextRandomValue(void)
{
    return rand();
}

int main(void)
{
    int myarray[10];
    populate_array(myarray, 10, getNextRandomValue);
    ...
}
\end{verbatim}

\section{How to end a program}
\label{sec:C_end-program}

During the execution of a function, class's method, there is a chance that an
error occur that require the process associated with that program to stop.

It is advised that you should try to unwind as close to \verb!main()! as
possible before terminating.


\subsection{Failure/Success portable}

When a program/subroutine exit, it passes an {\it exit status} to the parent
process, which is a value between 0 to 255. The most common convention is 
\begin{itemize}
  \item 0 = success
  \item 1 = failure
  \item 1 = mismatch
  \item 2 = inability to compare
  \item 128-up = for special purpose, e.g. 128=failure to execute another
  program in a subprocess.
\end{itemize}
However, all of the above conventions are not portable.

C/C++ defines some macros that can be used as the exit code. 
\begin{itemize}
  \item  \verb!EXIT_FAILURE! is the only portable way to indicate failure in
  C/C++ program
  
   \item \verb!EXIT_SUCCESS! or 0 to indicate a success exit  (zero is
   guaranteed to signal successfull completion). 
\end{itemize}
by passing it either to \verb!return! or \verb!exit()!
(Sect.\ref{sec:return_exit}).

To use either \verb!EXIT_FAILURE! or \verb!EXIT_SUCCESS! we need to include
\begin{verbatim}
#include <stdlib.h>  // C or
#include <cstdlib>  // C++

exit(EXIT_SUCCESS); 
return EXIT_SUCCESS;
\end{verbatim}

Since C99 standard and all version of C++, by reaching the end of a main()
routine, it does an implicit running of \verb!return 0!.  

\subsection{exit(), return}
\label{sec_return_exit}

When \verb!return! is called in main(), the destructor for the locally scoped
objects will be called. When \verb!exit()! is called in main(), until C++03,
\textcolor{red}{no} destructors for the locally scoped objects will be called.
Since C++11, objects associated with the current thread with thread storage
duration are destroyed. 
\begin{verbatim}
// C
void exit (int status);

// C++
[[noreturn]] void exit (int status);
\end{verbatim}
NOTE: 
\begin{itemize}
  \item exit(0) means 'successful' which is portable
  
Better to use \verb!exit(EXIT_SUCCESS)! from <cstdlib>.

  \item exit(1) means 'unsucessful' but it is non-portable. 

Better to use \verb!exit(EXIT_FAILURE)! from <cstdlib>.
\end{itemize}
NOTE: C99 only defines the meaning of exit(0), and those two macros with
<cstdlib>.
\footnote{\url{http://stackoverflow.com/questions/9944785/what-is-the-difference-between-exit0-and-exit1-in-c}}
\url{http://en.cppreference.com/w/cpp/utility/program/EXIT_status}

NOTE: \verb!static! objects (Sect.\ref{sec:static_C}) always be cleaned up either
using \verb!return! or \verb!exit! 

\subsection{abnormal termination}
\label{sec:abnormal_termination}

We use \verb!abort()! to cause abnormal termination, i.e. without cleaning up
the allocated data. When \verb!abort()! is used, no objects (i.e. global
objects, local objects, static objects) will have their destructors called. 

\subsection{Register some functions to be called on exit()}

We register one or more functions to be called at \verb!exit()!
(Sect.\ref{sec:normal_termination}) using \verb!std::atexit()!.

\subsection{normal termination}
\label{sec:normal_termination}

\begin{verbatim}
#include <cstdlib>

// Until C++11, 
void exit( int exit_code );

// Since C++11, there is no return
void exit( int exit_code );
\end{verbatim}
with the return value \verb!exit_code! which indicates a successful termination
(if 0 or \verb!EXIT_SUCCESS!) and an unsuccessful termination (if
\verb!EXIT_FAILURE!). \verb!exit()! is called to cause normal program
termination. What it does is
\begin{enumerate}
  \item All C streams (open with functions in <cstdio>) are flushed and closed.
  \item All files created with \verb!std::tmpfile! are removed.
  \item Destructors of objects (static storage duration) are called (C++). Since
  C++11, destructors of objects with thread local storage duration are also destroyed. 
  
  However, destructors of those with automatic storage duration not called. HOW TO RESOLVE THIS???
  
  \item Functions register with \verb!std::atexit()! are called. If an exception
  occurs with any of these functions, \verb!std::terminate()! is called.
  \item Finally, control is returned to the host environment, with one exit code
  which can be either \verb!EXIT_SUCCESS! or \verb!EXIT_FAILURE!.
\end{enumerate}

Since C++11, a new function \verb!quick_exit()! is supported to do quick normal
termination, without completely cleaning up the code.
