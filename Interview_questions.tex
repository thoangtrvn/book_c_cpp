\chapter{Interview Questions}


\section{Encapsulation vs. Polymorphism}
\label{sec:encapsulation}
\label{sec:polymorphism}

Encapsulation is a mechanism of a programming language to limit the access from
one class to another. In C++, there are three level of encapsulation: private,
public and protected:
\begin{enumerate}
  \item protected: only the current class and its subclass can have access to
  its field data and methods.
  \item public: the field data or method can be access from outside  
  \item private: only the current class can access to its field data or method
\end{enumerate}


Polymorphism refers to the programming language's ability to process objects
differently depending on their data type/class it is pointing to. It ties
closely with the concept {\it inheritance}. Polymorphism is NOT the same as
method overloading (i.e. methods same name but different output interface) or
method overriding (i.e. rewrite the content).


\section{Compile-time polymorphism vs. Run-time polymorphism}


\subsection{runtime polymorphism (dynamic polymorphism)}
\label{sec:polymorphism-runtime}
\label{sec:runtime-polymorphism}

{\bf It's about subclassing (and method overriding)}: It's being used in
pointer, i.e. reference of class X can hold object of class X or an object of
any sub classes of class X.
\begin{verbatim}
X obj = new Y(); // class Y:: X {}

X obj = new X();
\end{verbatim}

If the derived class Y has override the method defined in X, so at runtime, the
compiler need to know whether \verb!obj! is pointing to X or Y to call the right
method. This is known as runtime polymorphism.

\subsection{compile-time polymorphism (static polymorphism)}
\label{sec:polymorphism-static}
\label{sec:compile-time-polymorphism}

{\bf It's about method overloading} (C++, Java)
A class can have more than one methods with same name but with different number
of arguments or different types of arguments or both. 
It may or may not have same return type. IMPORTANT: We cannot overload just by
return type, i.e. at least one parameter must be different. This behavior is
same in Java and C++.

When the compiler scanning through the code for compiling, the compiler is able
to figure out the exact method call at compile-time that's the reason it is
known as compile time polymorphism.

In C++, templating is a case of method overloading
(Sect.\ref{sec:template-C++98}), i.e. different types can be used in a function
to represent the same concept, despite being completely unrelated.
\begin{lstlisting}
template <typename T>
T add(const T& lhs, const T& rhs) { return lhs + rhs; }
\end{lstlisting}
Here, there is no need for subclassing, except the type T just has to define
the + operator. 

\begin{mdframed}
Templates as in C++ do not exist in Java. The best approximation is generics.
\url{http://stackoverflow.com/questions/2159338/what-is-the-java-equivalent-of-cs-templates}
\end{mdframed}

\url{http://stackoverflow.com/questions/2128838/compile-time-polymorphism-and-runtime-polymorphism}

\section{Code execute only once}

Suppose that you have a function that get call multiple times, yet there is once
piece of code in that function that you want to be run only once.

\subsection{pthread\_once\_t (Linux only)}

Use \verb!pthread_once_t!, and make the function/method that you want to call
once a \verb!static! function/method.

\begin{lstlisting}
#include <pthread.h>

//1. define a variable as static 
static pthread_once_t semaphore = PTHREAD_ONCE_INIT;

class FooBar()
{
 public:
   static void test(); //must be static and void
   int somefunction();
}

//2. map that variable to the function to be called once
void FooBar::test()
{ // code to run once
}
int FooBar::somefunction()
{
   pthread_once( & semaphore, FooBar::test() );
   
   // do others
   ...
}
\end{lstlisting}
\url{http://stackoverflow.com/questions/6911126/is-it-possible-to-define-the-static-member-function-of-a-class-in-cpp-file-inst}

\url{http://apiexamples.com/c/pthread/pthread_once.html#L0}
\url{https://www-01.ibm.com/support/knowledgecenter/ssw_aix_71/com.ibm.aix.basetrf1/PTHREAD_ONCE_INIT.htm}


\subsection{boost::call\_once, boost::once\_flag (portable, thread-safe)}
\label{sec:boost::call_once}

We use \verb!boost::call_once!.
This has become part of C++11 (Sect.\ref{sec:std::call_once-C++11})


\begin{lstlisting}
    static boost::once_flag flag = BOOST_ONCE_INIT;
    boost::call_once([]{callee();}, flag);  
\end{lstlisting}
The first line can be inside or outside a function, since it is static.

\subsection{Define a static class}

Suppose the function you want to call once is \verb!callee()!, to be called by 
the function \verb!caller()!. The treat is define a static class object; with
the constructor calls \verb!callee()!.

IMPORTANT: This is not thread-safe in C++03, thread-safe in C++0x. 

\begin{verbatim}
void caller()
{
    static class Once { public: Once(){callee();}} Once_;
}
\end{verbatim}
\url{http://stackoverflow.com/questions/4173384/how-to-make-sure-a-function-is-only-called-once}

\subsection{std::call\_once, std::once\_flag (C++11)}
\label{sec:std::call_once-C++11}


\begin{verbatim}
template< class Callable, class... Args >
void call_once( std::once_flag& flag, Callable&& f, Args&&... args );
\end{verbatim}
The Callable $f$ is called only once, even if called from several threads.

We need to pass an object of \verb!std::once_flag! along with the function to be
called $f$ to \verb!call_once()!. Any call to $f$ with the same
\verb!std::once_flag! object ensures $f$ only run 1 time.

\begin{lstlisting}
#include <thread>
#include <mutex>

std::once_flag flag1, flag2;

void may_throw_function(bool do_throw)
{
}

void do_once(bool do_throw)
{
  try {
  // do_throw is the argument of may_throw_function
    std::call_once(flag2, may_throw_function, do_throw);
  }
  catch (...) {
  }
}

\end{lstlisting}
\url{http://en.cppreference.com/w/cpp/thread/call_once}

\subsection{constructor attribute GCC}

constructor function \verb!__attribute__! of GCC

\url{http://stackoverflow.com/questions/8412630/how-to-execute-a-piece-of-code-only-once}

\section{Write a function that dynamically allocate a 2D array and return it
back to the caller}

\begin{lstlisting}
int ** pdata; // a 2D array [X][Y]

void createData(pdata, int &X, int &Y); 

// free memory
for(int i = 0; i < X; ++i) {
    delete [] ary[i];
}
delete [] ary;

\end{lstlisting}

NOTE: We can pass by reference
\begin{verbatim}
void createData(int** &pdata, int &x, int &y)
{
 // do some calculation to find 'x' and 'y'
 
 // allocate memory
 // Option 1:
 pdata = new int*[x];
 for(int i = 0; i < X; ++i) {
    pdata[i] = new int[Y];
 }
 // Option 2:
 pdata = (int**) malloc(sizeof(int*) * (X));
}

\end{verbatim}