\chapter{Vala programming language}
\label{chap:vala-language}


\section{vala programming language: valac compiler}
\label{sec:valac}
\label{sec:vala-language}

Vala is a new object-oriented programming language, that translate vala source
code into C source and header files. It uses GObject type system to create classes and
interfaces (Sect.\ref{sec:GObject}).

Vala aims to bring modern programming language features to GNOME developers
without imposing any additional runtime requirements and without using a
different ABI compared to applications and libraries written in C.

Ubuntu 14.04 only has valac 0.22; if you needs newer version, follow the steps
\begin{verbatim}
sudo add-apt-repository ppa:ricotz/testing 
sudo apt-get update
sudo apt-get install valac
\end{verbatim}
\url{https://askubuntu.com/questions/612342/how-to-install-the-vala-environemt}

Vala is syntactically similar to \verb!C#! and includes several features such
as:
anonymous functions, signals, properties, Ggenerics, assisted memory management,
exception handling, type inference, and foreach statements 


\section{GObject}
\label{sec:GObject}

GObject is used by vala language (Sect.\ref{sec:vala-language})

If we need GObject introspection, we need to install
\begin{verbatim}
sudo apt-get install gobject-introspection  -y
  // the below package also include
  // gobject-introspection1.0-dev 
  //  handling GObject introspection data (development files)
sudo apt-get install libgirepository1.0-dev -y 
\end{verbatim}