\chapter{BSD Socket}
\label{chap:Socket}

The basic abstraction of a construct for inter-process communication is a {\bf
socket}. A socket object represents a single (bidirectional) connection between
two processes. NOTE: The processes can be on the same machines, or on different
machines. Depending on the implementation, the support can be one or both.

\begin{enumerate}

  \item  {\bf Unix domain sockets} do not use the network. 
  \label{sec:socket-Unix-domain}
  
  Their api makes it appear to be (mostly) the same to the developer as a
  network socket but all the communication is done through the kernel and the
  sockets are limited to talking to processes on the machine upon which they are
  running.
  
  \item {\bf BSD socket (Berkeley socket)}:  are what we know as network sockets
  on POSIX platforms today.
  
  In the past there were different lines of Unix development (e.g. Berkeley or
  BSD, System V or sysV, etc.) Berkeley sockets essentially won in the
  marketplace and are effectively synonymous with Unix sockets today.

  Sockets uses the BSD interface that was developed by BBN for the TCP/IP
  protocol suite under DARPA contract on 4.1aBSD and released in 4.2BSD. BSD
  Sockets provides a set of primary API functions that are typically
   implemented as system calls. The BSD Sockets interface is non-standard,
   operated differently from the POSIX interface in subtle ways, and is now
   deprecated in favour of the POSIX/SUS standard Sockets interface.
  
  \item  {\bf POSIX socket}: POSIX Sockets. Sockets were standardized by X/Open,
   later the OpenGroup, and IEEE in the POSIX standardization process. They appear in
  XNS 5.2 [XNS99], SUSv1 [SUS95], SUSv2 [SUS98] and SUSv3 [SUS03]. POSIX/SUS
  Sockets is now the common application environment for accessing networking,
  deprecating the XTI for TCP/IP networking applications.

  Strictly speaking there isn't a TCP socket. There are network sockets that can
  communicate using the TCP protocol. It's just a linguist shorthand to refer to
  them as a tcp socket to distinguish them from a socket using another protocol
  e.g. UDP, a routing protocol or whatnot  
  
  An "internet" socket is a mostly a meaningless distinction. It's a socket
  using a network protocol. That eliminates unix domain sockets,but most network
  protocols can be used to communicate on a LAN or the internet, which is just
  collection of networks. (Though do note there are protocols specific to local
  networks as well as those that manage collections of networks.)   
  
  \item {\bf winsock}: 
  
  \url{http://tangentsoft.net/wskfaq/articles/bsd-compatibility.html}
  
  winsocks supports overlapped I/O (with callbacks etc.) through functions like
   WSARecv (and other similar), which can make porting to bsd-sockets harder
   
\end{enumerate}
\url{https://stackoverflow.com/questions/22897972/unix-vs-bsd-vs-tcp-vs-internet-sockets}

In BSD socket, i.e. a socket implementation that support processes on different
machines, there are 3 types of sockets: stream socket, datagram socket, and raw
socket. Each type of socket uses a given communication protocol, e.g. TCP or
UDP.


%Different libraries that implement sockets have been developed: \ldots

\section{Introduction}

\subsection{TCP}

\subsection{UDP}

\subsection{SCTP}

Stream Control Transmission Protocol (SCTP) [RFC2960] is an end-to-end transport
protocol. SCTP supports
\begin{enumerate}
  \item multistream: multiple, independent logical streams of messages within an
  SCTP association. Data within a single stream is delivered in order. Data
  between different streams have no order constraint. 
\end{enumerate}

\section{Ethernet vs. Infiniband}

Gigabit Ethernet can deliver 10 Gbps.

Infiniband can deliver 40 Gbps, with RDMA feature to enable direct memory access
from one machine to another. The protocol to transfer data can be IPoIB. 

At booting up, if a host channel adapter (HCA) is detected, Ubuntu loads these
modules \verb!ib_mthca! and \verb!mlx4_*!. 
\begin{itemize}
  \item \verb!ib_mthca! : supports older Mellanox InfiniHost III HCAs
  \item \verb!mlx4_*! (\verb!mlx4_core, mlx4_ib!) : provide support ConnectX and
  newer Mellanox HCAs. In Ubuntu, the later module \verb!mlx4_core! is not
  loaded by default. 
  
  \item \verb!ib_ipoib! : the last module required for an IPoIB node.
  
  \item \verb!ib_umad! : not required to use IPoIB (but is required to use
  OpenSM), but is useful to load, which enables us to use scripts like
  \verb!ibstat, ibhosts! from infiniband-diags package.
  
  \item a subnet manager: which is required for any Infiniband fabric. An option
  is to use OpenSM (cross-platform open-source subnet manager) which requires
  \verb!ib_umad!. 
\end{itemize}
\url{http://www.servethehome.com/configure-ipoib-mellanox-hcas-ubuntu-12041-lts/}



\section{stream socket (uses TCP)}


\section{datagram socket (uses UDP)}


\section{raw socket}



