\chapter{Locale - Internationalization - Globalization}
\label{chap:locale}

Locale refers to the way the application behaves acoording to the user's
language environment. This include character encoding, formatting of
dates/times, currency and large numbers.

Support for international locale started since C89 (Sect.\ref{sec:C89_locale}).
A (character) language environment defines
\begin{enumerate}
  \item native language of the country-specific
  \item character set and coded character set : how to map a character of a
  character set into a code point in a set of code points. A code point is a
  hexadecimal value (binary value if stored in computer). 

  \item collating and ordering: define the relative orders of characters (used
  for sorting)
  
  \item character classification: define the type of characters in which groups
  (alphabetic, numeric)
  
  \item character case conversion: mapping between lower case and upper case
  
  \item data, time format: names of the weekdays, order of months, etc.
  \item format of numbers and monetary quantities: numeric grouping,
  decimal-point character, and monetary symbols.
  
  \item format of affirmative and negative system responses: 
\end{enumerate}
A \verb!locale! tells us how the language-specific data (given above)
is processed, data I/O (printed and displayed). 

Before you can compile your program, the compile must be able to read the source
code. So, it's important to tell the compiler which encoding
scheme of the source file (Sect.\ref{sec:locale_source-files})

% What it tells is It mainly deals with character and string processing.

You can set the locale in the code so that it knows how to interpret data based
on that locale. A locale define the code sets, date and time formatting conventions,
monetary conventions, decimal formatting convention, etc. Each can be set by
modifying a given environment variable, which can be
queried using \verb!locale! command.
\begin{verbatim}
LANG=en_US.UTF-8
LANGUAGE=
LC_CTYPE="en_US.UTF-8"
LC_NUMERIC="en_US.UTF-8"
LC_TIME="en_US.UTF-8"
LC_COLLATE="en_US.UTF-8"
LC_MONETARY="en_US.UTF-8"
LC_MESSAGES="en_US.UTF-8"
LC_PAPER="en_US.UTF-8"
LC_NAME="en_US.UTF-8"
LC_ADDRESS="en_US.UTF-8"
LC_TELEPHONE="en_US.UTF-8"
LC_MEASUREMENT="en_US.UTF-8"
LC_IDENTIFICATION="en_US.UTF-8"
LC_ALL=
\end{verbatim}
At the machine level, the locale name can be set in the \verb!LANG! environment
variable. If set inside the program, it overwrite the setting in the environment
variable. 

C or POSIX locale is the set of locale setting that is required for all
POSIX-compliant systems/softwares. Single UNIX Specification (SUS) version 3
defined \verb!C locale! (Sect.\ref{sec:LSB_C-lang}).  
\url{http://www-01.ibm.com/support/knowledgecenter/ssw_aix_53/com.ibm.aix.nls/doc/nlsgdrf/locale.htm}
``C'' and ``POSIS'' are the same. 
On GNU systems, only \verb!LC_ALL=C! or \verb!LC_ALL=POSIX! or
\verb!LC_MESSAGE=C|POSIX! override \verb!$LANGUAGE!
setting.\footnote{\url{http://unix.stackexchange.com/questions/87745/what-does-lc-all-c-do}} 

A C program (automatically) inherits the locale information from the environment
variables when it starts up. However, it doesn't control how the locale used by
the library function. Thus, to ensure all the functions (including in the
library) use the locale specified by the environment, we run \verb!locale()!
function with \verb!LC_ALL! as the first argument and an empty string in the
second argument.

\begin{verbatim}
#include <locale.h>

int main()
{
  char* locale;
  locale = setlocale(LC_ALL, "C");  /* set using C-locale */
  locale = setlocale(LC_ALL, "");  /* set based on environment variables */
  
  return 0;
}
\end{verbatim}
\url{http://www.gnu.org/software/libc/manual/html_node/Setting-the-Locale.html}


See how different locales' behavior?
\begin{verbatim}
$ LC_TIME=fr_FR.UTF-8 date
jeudi 22 aot 2013, 10:41:30 (UTC+0100)

$ LC_MONETARY=fr_FR.UTF-8 locale currency_symbol


$ LC_ALL=es_ES man
Que pgina de manual desea?

$ LC_ALL=C man
What manual page do you want?

$ LC_ALL=en_US sort <<< $'a\nb\nA\nB'
a
A
b
B

$ LC_ALL=C sort <<< $'a\nb\nA\nB'
A
B
a
b
\end{verbatim}


Example: converting ASCII to UPPERCASE
\begin{verbatim}
auto loc = std::locale("");

char s[] = "hello";
for (char &c : s) {
  c = toupper(c, loc);
}
\end{verbatim}

Example: To be internationalize, we use \verb!wchar_t! instead.
\begin{verbatim}
auto loc = std::locale("");

wchar_t s[] = L"hello";
for (wchar_t &c : s) {
  c = toupper(c, loc);
}
\end{verbatim}
