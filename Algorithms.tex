\chapter{Algorithms}


\section{Very large computation (e.g. n!)}

Suppose factorial problem n!, when using \verb!long long! data type, it exceeds
the size at 20!
\begin{verbatim}

121645100408832000      <-  18!
2432902008176640000     <-  19!
-4249290049419214848    <-  20!
\end{verbatim}


This is based on the idea of carried over of multiplication
\begin{verbatim}
   12
 x 13
 -------
   36  = 3x12
  12   = 1x12
 --------
  158   
\end{verbatim}

Solution, either use a string of character, or an array in which each element
keep a single digit. NOTICE that as we now the less-significant number first, so
they will be stored as the first element in the array. So '123' is stored as 
a = [3,2,1]. This is important for I/O purpose that we need to go through from
a[m-1] to a[0].

\begin{verbatim}
int a[200]; //array will have the capacity to store 200 digits.
            // we actually can use 'byte' to save memory

int m;  //keep the maximum index being used, i.e. a[0] ... a[m-1]
int temp; //keep the carry value
\end{verbatim}

\begin{verbatim}
a[0] = 1;
m = 1;
temp = 0;

for (int i = 0; i <= n; i++)
{
  for (int j = 0;j < m; j++)
  {
     int x = a[j] * i + temp; //digit-by-number product
     a[j] = x%10; //contain the less-significant digit
     temp = x/10; //contain the carry value
  }
  while (temp > 0)
  {
     a[m] = temp%10;
     temp = temp/10;
     m++;
  }
} 
\end{verbatim}

Now to print,
\begin{verbatim}
for (int i = m-1;0; i--)
  print(%d, a[i]);
  
\end{verbatim}
\section{Recursion}

A function that call itself is called a recursive function
\begin{verbatim}
def recsum(x):
 if x == 1:
  return x
 else:
  return x + recsum(x - 1)
\end{verbatim}
The function doesn't returns, as it need to call a function and that function
need to be completed before a sum with 'x' can be done and return the value.

Each functional call creates a stack frame in the call stack
(Sect.\ref{sec:heap_stack}). The problem with this is that if the number of
recursive call is too large, \textcolor{red}{stack overflow} error may occur.

A solution is to use {\it tail-recursive} function. The recursive call is the
last thing that happens so the functional call can be replaced with a jump to
the function's entry point. \textcolor{red}{Most current gcc compilers do
tail-call optimization.} Here, the result is passed as an argument to the
function, so that the result is constantly updated as each function call
\begin{verbatim}
def tailrecsum(x, running_total=0):
  if x == 0:
    return running_total
  else:
    return tailrecsum(x - 1, running_total + x)
\end{verbatim}
The function returns, and call another function. So, we don't have the problem
of stack overflow.

In C++: the n! = 1.2.3\ldots. n
\begin{verbatim}
int factorial(int n, int acc=1) {
  if (n <= 1) return acc;
  else        return factorial(n-1, n*acc);
}
\end{verbatim}
However, this actually calculate n.(n-1).(n-2)\ldots1. It means that it requires
the operator, i.e. multiplication, is both associative and commutative. This is
luckily correct, in this case, but again, not true for a general operator. A
better solution is
\begin{verbatim}
int factorial(int n, int k = 1, int acc=1) {
  if (n <= 1) return acc;
  else        return factorial(n-1, k+1, k*acc);
}
\end{verbatim}
\url{http://stackoverflow.com/questions/15537432/convert-recursion-to-tail-recursion}

What if the problem is double-recursion, like the Fibonacci problem
\begin{verbatim}
F(n) = F(n-1) + F(n-2)
\end{verbatim}
with
\begin{verbatim}
F1 = 1, F2 = 1
or 
F1 = 0, F2 = 1
\end{verbatim}

Using \verb!f_a! as F(n-1) and \verb!f_b! as F(n-2)
\begin{verbatim}
int fibonacci(int n, int f_a =1, int f_b = 1)
  if (n == 0) return f_a;
  else   return fibonacci(n-1, f_a+f_b, f_a)
}
\end{verbatim}

CONCLUSION: By keep adding enough parameters to the tail-recursive function, it
can solve for a recursive sequence where each element can be expressed by some
formula of the previous $k$ elements.

\section{std::transform}
\label{sec:std::transform}

We can apply unitary operation or binary operation.
However, std::transform does not guarantee in-order application.
To apply a function to a sequence in-order or to apply a function that modifies
the elements of a sequence, use \verb!std::for_each!
(Sect.\ref{sec:std::for_each})

{\bf unitary operation}
\begin{verbatim}
#include <algorithm>

//assume 's' is some iterable object (e.g. array, vector)
std::transform( s.begin(), s.end(), unitary_operator)

// with unitary_operator can be
// 1. lambda
 std::string s("hello");
    std::transform(s.begin(), s.end(), s.begin(),
                   [](unsigned char c) { return std::toupper(c); });
    std::cout << s;
\end{verbatim}
Sect.\ref{sec:}

{\bf binary operator}
\begin{lstlisting}
//assume 'si' is some iterable object (e.g. array, vector)
std::transform( s.begin(), s.end(), s2.begin(), result.begin(), binary_operator)
\end{lstlisting}
\url{http://stackoverflow.com/questions/13728430/element-wise-multiplication-of-two-vectors-in-c}

\section{std::for\_each}
\label{sec:std::for_each}

\section{Reduction}

\subsection{Sum}

\begin{verbatim}
vector<int> V{1,2,3};  

sum = 0;  // 0 - value of 3rd param
for (auto x : V)  sum += x;
\end{verbatim}

Another way is using 
\begin{enumerate}
  \item std::accumulate
\end{enumerate}

\subsection{-- std::accumulate}
\label{sec:std::accumulate}

This is a reduction operation
\begin{verbatim}
vector<int> V{1,2,3};  
// NOTE: 0 is the initial value for the reduction operation
// sum = 0 + V[0] + V[1] + ...
int sum = accumulate(V.begin(), V.end(), 0);
// sum == 6
\end{verbatim}
It is important to choose the right matrix.

There is also optional 4th parameter, which allow to replace addition with any
other operation.

\begin{verbatim}
vector<int> V{1,2,3};
int product = accumulate(V.begin(), V.end(), 1, multiplies<int>());
\end{verbatim}

\url{http://stackoverflow.com/questions/12633950/understanding-stdaccumulate}