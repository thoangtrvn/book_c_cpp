\chapter[Intel Thread Building Block]{Intel Thread Building Block (multi-core)
library}
\label{chap:TBB}

Threading Building Blocks (TBB) is a C++ template library developed by Intel for
writing software programs that take advantage of multi-core processors.

TBB is better than the following lower-level libraries
\begin{enumerate}
  \item  use of native threading packages such as POSIX threads, 
  
  \item Windows  threads, or 
  
  \item the portable Boost Threads
\end{enumerate}
as such libraries require individual threads of execution are created,
synchronized, and terminated manually.

A TBB program creates, synchronizes and destroys graphs of dependent tasks
according to algorithms, i.e. high-level parallel programming paradigms (a.k.a.
Algorithmic Skeletons - Sect.\ref{sec:Algorithmic_Skeletons}).
\textcolor{red}{Tasks are then executed respecting graph dependencies.}

\section{Introduction}

Intel TBB was under GPL license until August 2016. As of September 2016 Intel
TBB is relicensed under Apache v2.0 license.

As of September 2016 Intel TBB is dual-licensed, with a commercial (COM)
license as part of the suite products (Intel Parallel Studio, Intel System
Studio), and as open source under Apache v2.0 license.

If we run
\begin{verbatim}
make -j10
\end{verbatim}
it creates
\begin{verbatim}
./build/macos_intel64_gcc_cc4.2.1_os10.8.2_release

./build/macos_intel64_gcc_cc4.2.1_os10.8.2_debug
\end{verbatim}
in such folder, there should be tbbvars.sh and tbbvars.csh, where we can source


\url{https://github.com/01org/tbb}



\section{Algorithmic Skeletons}
\label{sec:Algorithmic_Skeletons}

Algorithmic skeletons take advantage of common programming patterns to hide the
complexity of parallel and distributed applications. Starting from a basic set
of patterns (skeletons), more complex patterns can be built by combining the
basic ones. Algorithmic skeletons were first introduced by Cole in 1989


{\it The most outstanding feature of algorithmic skeletons}, which
differentiates them from other high-level parallel programming models, is that
orchestration and synchronization of the parallel activities is implicitly
defined by the skeleton patterns. \textcolor{red}{Programmers do not have to
specify the synchronizations between the application's sequential parts.}

\url{https://en.wikipedia.org/wiki/Algorithmic_skeleton}

{\bf Non-object oriented algorithmic skeleton frameworks}

\begin{enumerate}
  \item Eden library: Haskell language
\end{enumerate}


{\bf Object oriented algorithmic skeleton frameworks}
\begin{enumerate}
  \item FastFlow library (2009-): C++ used for CUDA/OpenCL/C++11
  
   Initially developed to target multi-core platforms, and then extended to
   heterogeneous platforms composed of clusters of shared-memory platforms,
   possibly equippped with computing accelerators (CUDA, Xeon Phi, Tilera
   TILE64)
  
  FastFlow is typically comparable to (and is some cases slightly faster than)
  state-of-the-art parallel programming frameworks such as Intel TBB, OpenMP, Cilk, 
  
  {\it Skeleton set}: 
  Pipeline, Farm, ParallelFor, ParallelForReduce, MapReduce, StencilReduce,
  PoolEvolution, MacroDataFlow
  
  \item Marrow library (2013-): C++   
\end{enumerate}