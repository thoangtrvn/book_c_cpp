\chapter{API Design}

\section{Boolean vs. Enum parameter}

80\% of the time a Boolean argument is passed in as a constant, and its
intention is to turn a piece of behavior on or off. 

Example: 
\begin{verbatim}
FileStream f  = File.Open ("foo.txt", true, false);
\end{verbatim}
This call gives you no context whatsoever to understand the meaning behind true
and false. 

A better approach:
\begin{verbatim}
FileStream f  = File.Open ("foo.txt", CasingOptions.CaseSenstative,
                           FileMode.Open);
\end{verbatim}


In .NET: As far as the runtime is considered an enum is treated just like its
underlying type which is in most cases Int32 (the same in memory size Booleans
take up). Generally speaking, enums don't cost noticeably more than Booleans.  
Today there is a very small overhead (on the order of 300 bytes for each enum
type referenced).

\subsection{Use Boolean}

guideline for boolean parameters are to only use them if the method is a
statement. ie. 
\begin{verbatim}
myObject.Enable( true ); 
\end{verbatim}

\subsection{Use Enum}

 Using Boolean parameter is not recommended.
\begin{verbatim}
do_sorting(sortAscending=True)
\end{verbatim}

RECOMMENDED:
\begin{verbatim}

enum SortOrder = {ASCENDING, DESCENDING}

do_sorting(sortOrder = SortOrder.ASCENDING)
\end{verbatim}

If you have to use, try to add comments, i.e.
always name boolean params (or other unclear params of any kind) with inline
comments, e.g.
\begin{verbatim}
 // C/C++ language
box.pack(child, /* expand = */ false, /* fill = */ true);
\end{verbatim}

References:
\begin{enumerate}
  \item \url{http://blogs.msdn.com/b/oldnewthing/archive/2006/08/28/728349.aspx}
  
  \item \url{http://blogs.msdn.com/b/brada/archive/2004/01/12/57922.aspx}
  
  \item \url{http://blog.ometer.com/2011/01/20/boolean-parameters-are-wrong/}
\end{enumerate}


