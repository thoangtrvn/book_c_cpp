\chapter{Object-Oriented C++}

\section{namespace}
\label{sec:namespace_C++}

In C++, to avoid name clashes, we can put the data type in different namespace,
i.e. define the scope. The namespace that cover everything is \verb!std! namespace.

\begin{verbatim}
#include <iostream>

using namespace std;

class Rectangle {
    int width, height;
  public:
    void set_values (int,int);
    int area() {return width*height;}
};

void Rectangle::set_values (int x, int y) {
  width = x;
  height = y;
}
\end{verbatim}


\section{Class}
\label{sec:class-C++}

A class in C++ is an extended concept of {\it struct} in C. A class can have
data members (Sect.\ref{sec:data-member-C++-class}), and member function
(Sect.\ref{sec:member-function-C++-class}) to operate on these data members.
A special type of member function for a class is constructor
(Sect.\ref{sec:constructor-C++-class}) and destructor
(Sect.\ref{sec:OO_destructor}).

An object is an instance of that class. A class is actually a user-defined data
type
\begin{verbatim}
class Rectangle {
    int width, height;
  public:
    void set_values (int,int);
    int area (void);
} rect;   // we define the object directly

\end{verbatim}

\subsection{Friend of a class}

A class A declares that class B of a functon foo() is a friend of it, then class
B can get access to \verb!private! and \verb!protected! members of class A.

To declare a friend, class A use \verb!friend! keyword

\subsection{* Friend function of a class}


\subsection{* Friend class of a class}

\begin{lstlisting}
class Rectangle  {
private:	
  int width, height;
public:
  friend Rectangle duplicate (const Rectangle&);

  Rectangle() {}
  Rectangle (int x, int y) : width(x), height(y) {}
  int area() {return width * height;}
}

// here the function can get direct access to 
//  the class Rectangle data member
Rectangle duplicate (const Rectangle& param)
{
  Rectangle res;
  res.width = param.width*2;
  res.height = param.height*2;
  return res;
}

\end{lstlisting}
Typical use cases of friend functions are operations that are conducted between
two different classes accessing private or protected members of both. 

\url{http://www.cplusplus.com/doc/tutorial/inheritance/}

\subsection{std::cout friendly class}

Suppose you want to be able to
\begin{verbatim}
AClass objA;
std::cout << objA;
\end{verbatim}

then
\begin{lstlisting}
// .H file
#include <iostream>
class AClass 
{

  public: 
  // define a method specifically for std::cout purpose
  std::string getString();
}
extern std::ostream& operator<<(std::ostream& os, AClass & di);


// .C file
std::ostream& operator<<(std::ostream& os, AClass& di)
{
   os<<di.getType()<<"["<<di.getString()<<"]";
   return os;
}

\end{lstlisting}

\subsection{Nested class}
\label{sec:nested-class}

They're exactly like regular classes... except nested.
Use them when a class's internal implementation is so complex that it can most
easily be modeled by several smaller classes.

If you feel that class B is so deeply connected to class A, but the objects of A
and B are not necessarily related, then you might want the class B to be only
accessible via scoping the A class (it would be referred to as A::Class).

Using nested class is useful to hide implementation details
\begin{lstlisting}
class List
{
    public:
        List(): head(nullptr), tail(nullptr) {}
    private:
        class Node
        {
              public:
                  int   data;
                  Node* next;
                  Node* prev;
        };
    private:
        Node*     head;
        Node*     tail;
};
\end{lstlisting}
\url{https://stackoverflow.com/questions/4571355/why-would-one-use-nested-classes-in-c}

Bill Oneal suggested NOT to use nested class as
\begin{verbatim}
1. because it's simply not necessary -- you can do the same with externally defined classes, 
which reduces the scope of any given variable, which is a good thing. 

2. because you can do everything nested classes can do with typedef. 

3. because they add an additional level of indentation in an environment
 where avoiding long lines is already difficult 
 
4. because you are declaring two conceptually separate objects in a single class declaration, etc. 
\end{verbatim}


\subsection{Empty class: tag dispatch}
\label{sec:empty-class-C++}
\label{sec:tag-dispatch-C++}

An {\bf empty class} is a class that have no members, neither methods nor data or typedef.
Example:
\begin{verbatim}
class Empty {};
\end{verbatim}

Below are not empty classes: 
\begin{itemize}
  \item  Abstract base classes of class hierarchies often carry only declarations of
virtual functions. 

  \item Classes used in template metaprogramming may contain only
typedefs or static members that are used at compile time.
 
\end{itemize}

Using empty class has a number of good use cases.
One of them is \textcolor{red}{tag dispatch.}

As empry class is a strongly typed language. If there are two empty classes,
they are different types. They don’t have anything to do with each other.
Objects of those types can not be converted to each other. There is nothing we
can do with them but construct them and let them get destroyed.
We can pass them to functions. That is where tag dispatch comes in.

 It usually is used in generic programming, i.e. in templates. Imagine two or
 more alternatives of a function that take the same list of parameters. They
 need to have the same name but different implementations
\begin{verbatim}
struct Variant1Tag {};
struct Variant2Tag {};

void functionWithVariants(int i, double d, std::string str, Variant1Tag) {
  // ...
}

void functionWithVariants(int i, double d, std::string str, Variant2Tag) {
  // ...
}

functionWithVariants(42, 3.14, "less obvious values next time", Variant1Tag{});

\end{verbatim}

\subsection{-- use case: iterator tag in STL}

Each tag is an empty type and corresponds to one of the five (until C++20) or six (since C++20).

\url{https://en.cppreference.com/w/cpp/iterator/iterator_tags}

Iterator category tags carry information that can be used to select the most
efficient algorithms for the specific requirement set that is implied by the
category.

For every iterator type It, a 
\begin{verbatim}
typedef std::iterator_traits<It>::iterator_category 
\end{verbatim}
is available, which is an alias to
one of these that is chosen by the given template type.

\begin{lstlisting}
struct input_iterator_tag { };
struct output_iterator_tag { };
struct forward_iterator_tag : public input_iterator_tag { };
struct bidirectional_iterator_tag : public forward_iterator_tag { };
struct random_access_iterator_tag : public bidirectional_iterator_tag { };


//(since C++20)
struct contiguous_iterator_tag: public random_access_iterator_tag { };
\end{lstlisting}


Example:
\begin{verbatim}
#include <iostream>
#include <vector>
#include <list>
#include <iterator>
 
template< class BDIter >
void alg(BDIter, BDIter, std::bidirectional_iterator_tag)
{
    std::cout << "alg() called for bidirectional iterator\n";
}
 
template <class RAIter>
void alg(RAIter, RAIter, std::random_access_iterator_tag)
{
    std::cout << "alg() called for random-access iterator\n";
}
 
template< class Iter >
void alg(Iter first, Iter last)
{
    alg(first, last,
        typename std::iterator_traits<Iter>::iterator_category());
}
 
int main()
{
    std::vector<int> v;
    alg(v.begin(), v.end());
 
    std::list<int> l;
    alg(l.begin(), l.end());
 
//    std::istreambuf_iterator<char> i1(std::cin), i2;
//    alg(i1, i2); // compile error: no matching function for call
}

//RESULT: vector implements 'random-access'; while list implements 'bidirectional'
alg() called for random-access iterator
alg() called for bidirectional iterator
\end{verbatim}


\subsection{-- use case: to distinguish two different versions of a constructor}

Tag dispatch can be used to distinguish between the iterator version and the
integral type version of the constructor

Example: we leave out the allocator for simplicity
\begin{verbatim}
template <class T>
class vector {
public:
  //constructs the vector with N copies of that value.
  vector(size_type N, T const& value = T());
  
  //takes a pair of iterators to a range of values that shall be copied into the newly constructed vector
  template <class It>
  vector(It first, It last);
};
\end{verbatim}

\begin{lstlisting}



template <class It>
vector<T>::vector(It first, It last) {
  typedef get_iterator_tag_for<It>::type tag_type;
  construct(first, last, tag_type{});
}

template <class It>;
vector<T>::construct(It first, It last, std::input_iterator_tag) {
  // construct iterator style
}
\end{lstlisting}

\section{Constructor and Destructor}
\label{sec:constructor-C++-class}

A constructor is a special non-static member function of a class
(Sect.\ref{sec:class-C++}).

Any C++ compiler automatically add default constructor
(Sect.\ref{sec:default-constructor}), copy constructor
(Sect.\ref{sec:copy-constructor}), copy assignment operator (=) and a destructor
(Sect.\ref{sec:OO_destructor}), only if they are not defined in a class.





\subsection{constructor}
\label{sec:OO_constructor}

Constructors lacks something that the rest of the functions and methods have in
C++: {\bf a name}. A constructor is a member function
(Sect.\ref{sec:member-function-C++-class}) whose job is to construct the class.

The only limitation of a constructor is that it must have the same name of the
class. Because of that, to define different ways to construct the class, we
typically need to overload with different interfaces.q

\begin{verbatim}
class myClass 
{
    myClass();
    myClass(myClass&);
    myClass(int, int);
 private:
    int a,b;
}
\end{verbatim}

This is good, but sometimes not enough. {\it Indeed, sometimes we need several
ways to construct a class with the \textcolor{red}{same types of parameters.}}
One way to go about this and to keep code expressive is to use tag dispatching
(Sect.\ref{sec:tag-dispatch-C++}).

The “tag” in tag dispatching refers to a type that has no behaviour and no data, i.e. empty class/struct
\begin{verbatim}
struct MyTag {};
\end{verbatim}

\subsection{* use enum to select behavior}


Enums and tag dispatching are two ways to introduce several behaviours in the
same interface in C++. With them, we can pass arguments that determine a facet
of how we want a function to behave.

\begin{verbatim}
enum class BehaviourType
{
    thisWay,
    thatWay
};
 
void f(int argument, BehaviourType behaviourType);
{
    // check the value of behaviourType and behave accordingly
    if (myBehaviour == BehaviourType::thisWay)
{
    f(value, BehaveThisWay());
}
else if (myBehaviour == BehaviourType::thatWay)
{
    f(value, BehaveThatWay());
}

}
\end{verbatim}


Essentially, tags get dispatched at compile-time while enums values can be read at runtime.


\begin{lstlisting}
enum TableEventEnum {
    NO_VALUE = 0,
    ATTRACT = 1,
    OPEN = 2,
    CLOSED = 3
};

class E_TableEvent : public E_EnumerationBase<enum TableEventEnum>
{
public:
    E_TableEvent();
};
\end{lstlisting}


\begin{lstlisting}
template<typename T> class E_EnumerationBase : public SimpleElement
{
public:
    E_EnumerationBase();
    virtual bool setValue(QString choice);
    virtual T getState();

protected:
    T state;
    QHash<QString, T> dictionary;
};

template<typename T> E_EnumerationBase<T>::E_EnumerationBase() {
    state = 0;
}

template<typename T> bool E_EnumerationBase<T>::setValue(QString choice) {
    T temp;
    temp = dictionary.value(choice, 0);
    if (temp == 0) {
        return false;
    }

    value = choice;
    state = temp;
    return true;
}

template<typename T> T E_EnumerationBase<T>::getState() {
    return state;
}
\end{lstlisting}

\subsection{* use tag dispatch to select behavior}

Example: \textcolor{red}{Define class option 1} and two ways to use it
\begin{verbatim}
class MyClass
{
public:
    struct constructThisWay{};
    struct constructThatWay{};
 
    explicit MyClass(constructThisWay);
    explicit MyClass(constructThatWay);
 
    // ...
};


// USE 1:
// NOTICE the double parantheses as
// the object is instantiated directly at call site like the tag
MyClass x((MyClass::constructThisWay()));

// NOTE: if we have more than one parameters to the constructor, and the
// this extra parameter passed to the constructor 
//    wasn’t instantiated directly at call site like the tag is
// ... then we can call it regularly
int num=4;
MyClass x(num, MyClass::constructThisWay());
\end{verbatim}
or
\begin{lstlisting}
// USE 2:
// Another way: is to use uniform initialization, with braces {}:
MyClass x(MyClass::constructThisWay{});
\end{lstlisting}
Sect.\ref{sec:uniform-initialization-C++11} introduces uniform initialization (C++11).

Example: \textcolor{red}{Define class option 2} and use it

\begin{lstlisting}
// Another way is to define static data member
class MyClass
{
public:
    static struct ConstructThisWay{} constructThisWay;
    static struct ConstructThatWay{} constructThatWay;
 
    explicit MyClass(ConstructThisWay);
    explicit MyClass(ConstructThatWay);
}
//and use it
MyClass x(MyClass::constructThatWay);
\end{lstlisting}
This leads to more code in the class definition. It’s a trade-off, as no more
vexing parse nor braces, since the argument is no longer a type.

Finally, whichever way you decide to go with, nothing prevents you from having a
real default constructor that takes no parameters, on the top of all that:
\begin{verbatim}
class MyClass
{
public:
    static struct ConstructThisWay{} constructThisWay;
    static struct ConstructThatWay{} constructThatWay;
 
    MyClass();
    explicit MyClass(ConstructThisWay);
    explicit MyClass(ConstructThatWay);
};
\end{verbatim}


\subsection{* uniform initialization (C++11)}
\label{sec:uniform-initialization-C++11}

Uniform initialization is a feature in C++ 11 that allows the usage of a
consistent syntax to initialize variables and objects ranging from primitive
type to aggregates.

It introduces brace-initialization that uses braces ({}) to enclose initializer
values.
\begin{verbatim}
type var_name{arg1, arg2, ....arg n}
\end{verbatim}


Example:
\begin{lstlisting}
// uninitialized built-in type
int i;    

// initialized built-in type
int j=10; 

// initialized built-in type
int k(10);

// Aggregate initialization
int a[]={1, 2, 3, 4} 

// default constructor
X x1; 

// Parametrized constructor
X x2(1); 

// Parametrized constructor with single argument
X x3=3; 

// copy-constructor
X x4=x3; 
\end{lstlisting}

Example:
\begin{lstlisting}
int i{};     // uninitialized built-in type

int j{10}; // initialized built-in type

int a[]{1, 2, 3, 4} // Aggregate initialization

X x1{}; // default constructor

X x2{}; // Parametrized constructor;

X x4{x3}; // copy-constructor

// declaring a dynamic array 
    // and initializing using braces 
    int* pi = new int[5]{ 1, 2, 3, 4, 5 }; 
    
    
 class A { 
    int arr[3]; 
  
    public: 
        // initializing array using uniform initialization 
        A(int x, int y, int z) 
            : arr{ x, y, z } {}; 
}


\end{lstlisting}



\subsection{* rule for evoking constructor from base class}

\subsection{Types of constructors}

\begin{enumerate}
  \item default constructor: the one with no argument
  - Sect.\ref{sec:default-constructor}
  
  \item Parameterized Constructors: 
  
  To create a parameterized constructor, simply add parameters to it the way you
  would to any other function. When you define the constructor’s body, use the
  parameters to initialize the object.
  
  \item Copy constructor:  initializes an object using another object of the same class - Sect.\ref{sec:copy-constructor}
  
\end{enumerate}

Apart from that, you can make any of the above constructors either
\verb!private!, \verb!protected! or \verb!public! or \verb!explicit!. IMPORTANT:
There is no \verb!virtual! constructor in C++
(Sect.\ref{sec:virtual-constructor}).


\subsection{* default constructor: zero-argument constructor}
\label{sec:default-constructor}

A default constructor is the constructor without any argument
\begin{verbatim}
class myClass
{
	myClass();
}
\end{verbatim}

To enforce the compiler to create the default ones, even if an alternate
versions have been implemented by the programmer, we can use the C++11
\verb!default! keyword (Sect.\ref{sec:C++11_final-override-default-delete}).


\subsection{* protected constructor}
\label{sec:protected_constructor}

A \verb!protected! constructor has the same rule as a protected member
(Sect.\ref{sec:C++03_protected-functions_static-functions}).
A protected constructor prevents the user from instantiating, i.e. creating an
object of such class, using that constructor.


We use a \verb!protected! constructor if
\begin{enumerate}
  \item  the class has (at least one) pure virtual function (Sect.\ref{sec:pure_virtual_function}), which makes the class
  becoming an abstract class
  
  \item the class is supposed to be an abstract class by design (Sect.\ref{sec:abstract_class_C++})
  
  As you want there is no way to instantiate an object of that class but allowing it to be
  subclassed
\begin{Verbatim}
class Base
{
protected:
    Base();
};

class Child : protected Base
{
public:
    Child() : Base();
};
\end{Verbatim}

However any class that inherits from it can call the protected constructor
unless a friend modifier is used.


  \item to allow a \verb!static! member function to instance its own class. This
  is widely used in Java.
\end{enumerate}
  
  
\subsection{* virtual constructor???}
\label{sec:virtual-constructor}

\textcolor{red}{There is no virtual constructor in C++}.
As explained by Bjarne Stroustrups
\begin{verbatim}
A virtual call is a mechanism to get work done given partial information. In
particular, "virtual" allows us to call a function knowing only any interfaces
and not the exact type of the object.

To create an object you need complete information. In particular, you need to
know the exact type of what you want to create. Consequently, a "call to a
constructor" cannot be virtual.
\end{verbatim}
\url{http://stackoverflow.com/questions/733360/why-do-we-not-have-a-virtual-constructor-in-c}

However, it has virtual destructor (Sect.\ref{sec:virtual-destructor}).

\subsection{* 'explicit' constructor (with 1 parameter)}
\label{sec:explicit-keyword}

Since C++11, \verb!explicit! can be applied to more than just constructors. 

SCENARIO: DoBar() is defined to accept a Foo object, but when being used, it is
passed with an \verb!int!.  Consider the example: when you're passing an
argument (e.g. 42) to the function (e.g. DoBar) which accepts an argument of a
different type (e.g. Foo class).


By default, in C++, it allows the compiler to make one implicit conversion to
resolve 1 parameter to a function if it detects the single-paramater constructor
of the desired class accepts the given argument (e.g. the Foo::Foo(int)).
There exists a constructor for Foo that takes an int so this constructor can be
used to convert the parameter to the correct type.
{\it The compiler is allowed to do this once for each parameter.}


\begin{lstlisting}
class Foo
{
public:
  // single parameter constructor, can be used as an implicit conversion
  Foo (int foo) : m_foo (foo) 
  {
  }

  int GetFoo () { return m_foo; }

private:
  int m_foo;
};

void DoBar (Foo foo)
{
  int i = foo.GetFoo ();
}

int main ()
{
  DoBar (42); // which the compiler turns into DoBar(Foo(42));
}
\end{lstlisting}

\textcolor{red}{The compiler can use constructors callable with {\bf a single
parameter} to convert from one type to another in order to get the right type
for a parameter.}

Prefixing the \verb!explicit! keyword to the constructor prevents the compiler
from using that constructor for implicit conversion.
EXPLAIN: If we do
\begin{verbatim}
class Foo
{
  explicit Foo (int foo) : m_foo (foo) 
  {
  }
}
\end{verbatim}
It is now necessary to call for conversion explicitly with
DoBar (Foo (42)). As a result, the code is less error prone.

Since C++11, this \verb!explicit! is also applicable to a regular method in a class.

Example: 
\begin{verbatim}
class String {
public:
    String(int n); // allocate n bytes to the String object
    String(const char *p); // initializes object with char *p
};

// if we try
String mystring = 'x';
// it will implicitly converted to int 
// and then will call the String(int) constructor.  
// which is not what we want.

\end{verbatim}
To avoid such conditions, we define that constructor as \verb!explicit!
\begin{verbatim}
class String {
public:
    explicit String (int n); //allocate n bytes
    String(const char *p); // initialize sobject with string p
};
\end{verbatim}

\url{http://stackoverflow.com/questions/121162/what-does-the-explicit-keyword-in-c-mean}

\subsection{copy constructor}
\label{sec:copy-constructor}

\begin{lstlisting}
class A {
public:
    A() {
        cout << "A::A()" << endl;
    }
};

A b(3);

// The copy constructor
A a = b;

// The assignment constructor
A c;
c = a;
\end{lstlisting}
A copy constructor is a constructor that accepts an object of the same type, and
then duplicate it to a new object.
\begin{itemize}
  \item  A copy constructor is used to initialize a previously uninitialized object from some other object's data.
  
\begin{lstlisting}
class A
{
   A(const A& rhs) : data_(rhs.data_) {}
}
\end{lstlisting}

  \item An assignment operator is used to replace the data of a previously initialized object with some other object's data.
\begin{lstlisting}
A& operator=(const A& rhs) {data_ = rhs.data_; return *this;}
\end{lstlisting}

\end{itemize}

A copy constructor is a function that knows how to create an object of a class,
allocate space for its members, and copy their values from another object

\begin{verbatim}
class myClass
{
	myClass (const myClass&);
}
\end{verbatim}

\begin{verbatim}
class Point 
{ 
private: 
    int x, y; 
public: 
    Point(int x1, int y1) { x = x1; y = y1; } 
  
    // Copy constructor 
    Point(const Point &p2) {x = p2.x; y = p2.y; } 
  
    int getX()            {  return x; } 
    int getY()            {  return y; } 
}; 
\end{verbatim}

NOTE: To disable a constructor, we can use \verb!=delete! keyword
(Sect.\ref{sec:C++11-delete}).


\subsection{copy assignment (=)}



\subsection{move constructor (C++11)}
\label{sec:move-constructor-C++11}

\url{http://blog.smartbear.com/c-plus-plus/c11-tutorial-introducing-the-move-constructor-and-the-move-assignment-operator/}

\subsection{destructor}
\label{sec:OO_destructor}

A destructor in C++, or object-oriented programming, is a class' member function
(Sect.\ref{sec:member-function-C++-class}) with the same name as the class,
and is prefixed with a tilde (\verb!~!); and should contains the code to free
the memory of the associated data members when the object is deallocated.

Example:
\begin{verbatim}
class Base 
{
    ~Base() { cout << "Clean Base";};
};

class Derived : public Base
{
    ~Derived()
    {
        // Do some important cleanup
        cout << "Clean Derived";
    }
}
\end{verbatim}

\subsection{* 'virtual' destructor}
\label{sec:OO_virtual-destructor}
\label{sec:virtual-destructor}

A virtual destructor is a destructor with \verb!virtual! keyword.
Virtual destructors are useful when you can delete an instance of a derived
class through a pointer to base class:

When using inheritance (Sect.\ref{sec:OO_inheritance}), if there is no
virtual destructor in the base class, the problem is we cannot delete the
derived object pointed to by the base object pointer.
\begin{verbatim}
Base *b = new Derived();
// use b
delete b; // Here's the problem! 
  // Only Base:~Base() is called but not Derived::~Derived()
\end{verbatim}
Virtual functions are used in order to invoke functions based on the type of
object pointed to by the pointer, and not the type of pointer itself.

So, the reason to define {\bf virtual destructor} is when they are meant to be
manipulated polymorphically. In other words, if the class is supposed to be
inheritted, then the base class destructor need to be \textcolor{red}{virtual}.
\begin{verbatim}
class Base 
{
    virtual ~Base() {}; // virtual destructor
    // some virtual methods
};

\end{verbatim}

The order of the destructor to be called is the child first, and then to the
parent, in that order. \textcolor{blue}{In other words, the derived class
destructor is called first}
\begin{verbatim}
delete b; // will print 
     // Cleanup Derived
     // Cleanup Base
\end{verbatim}

\begin{framed}
If we want to delete the derived-class object via the base-class pointer, then
we need to define the destructor of the base-class as \verb!virtual!.

IMPORTANT PARADIGM: If a class has one or more virtual function, then it should
have the virtual destructor.
\end{framed}

\section{Using initialization list in constructor}
\label{sec:initialization_list}

Instead of using assignment to initialize the data member
inside the body of a constructor (Sect.\ref{sec:OO_constructor})
\begin{verbatim}
class A
{
private:
  int _a, _b;
public:
  A (int a_var, int b_var)
  {
    _a = a_var;
    _b = b_var;
  }
}
\end{verbatim}

We can use initialization list (see after the colon (:)), and pass as argument
to the constructor

\begin{verbatim}
class A {
   private:
     int a, b;

   public:
     // here after the colon (:) is the initialization list
     // which can be : BaseClassConstructor(passed_value)
     //                data_member(passed_value)
     A(int a_var, int b_var):a(a_var), b(b_var) {}; 
 };
\end{verbatim}

\subsection{ * What data members can be initialized with initialization list?}

You can not initialize a \verb!static! data member with initialization list
(Sect.\ref{sec:static-data-member}).

Example:
\begin{verbatim}
class Fred {
public:
  Fred();
  // ...
private:
  int i_;
  static int j_;
};
\end{verbatim}

WRONG USAGE
\begin{verbatim}
Fred::Fred()
  : i_(10)  // Okay: you can (and should) initialize member data this way
  , j_(42)  // Error: you cannot initialize static member data like this
{
  // ...
}
\end{verbatim}

CORRECT USAGE:
\begin{verbatim}
// You must define static data members this way:
int Fred::j_ = 42;
\end{verbatim}
\url{https://isocpp.org/wiki/faq/ctors#explicit-define-static-data-mems}


\subsection{ * When to use initialization list}

IMPORTANT:
A class's data member variables are always initialized in the order in which
they are declared in the class.
They are not initialized in the order in which they are specified in the Member
Initalizer List. So, it 
is always a good practice to maintain the same order of members for member
initialization as the order in which they are declared in the class definition.

It's a common practice to use initialization list, rather than constructor.
Initialization lists can be used to initialize both POD (Plain Old Data) or
user-defined types. There is no difference between the two methods when using
POD. 

When using user-defined types, before the constructor is called, the constructor
for the parent class of class B (by default: no-argument constructor is called)
and any contained members of the parent class are invoked as well, which may
degrade the performance. By using initialization lists, only the default
constructor is called which is thus faster.

When we have to use initialization list?
\begin{enumerate}
  \item  The class has reference data member
  (Sect.\ref{sec:reference-data-member})

\textcolor{red}{reference}
\begin{verbatim}
struct bb {};

struct aa
{
    bb& rb;
    aa(bb& b ) : rb(b) {}
};

// usage:

bb b;
aa a(b);
\end{verbatim}
  
  \item The class has non-static \verb!const! data member
  
  \item The class is derived from a base class, and the base class has no
  default constructor

\textcolor{red}{require parameter, i.e. no default constructor}
\begin{verbatim}
struct bb {};

struct dd
{
    char c;
    dd(char x) : c(x) {}
};

struct aa : dd
{
    bb& rb;
    aa(bb& b ) : dd('a'), rb(b) {}
};
\end{verbatim}


    
  \item The class is derived from a base class, and you want to initialize base
  class's data members

  
  \item (not a MUST, but recommended), the constructor's parameter name is the
  same as that of data member
\end{enumerate}
\url{http://stackoverflow.com/questions/1711990/what-is-this-weird-colon-member-syntax-in-the-constructor}

Example:
\begin{lstlisting}
class MyClass
{
    public:
        //Reference member, has to be Initialized in Member Initializer List
        int &i;       
        int b;
        //Non static const member, must be Initialized in Member Initializer List
        const int k;  

    //Constructor's parameter name b is same as class data member 
    //Other way is to use this->b to refer to data member
    MyClass(int a, int b, int c):i(a),b(b),k(c)
    {
         //Without Member Initializer
         //this->b = b;
    }
};

class MyClass2:public MyClass
{
    public:
        int p;
        int q;
        MyClass2(int x,int y,int z,int l,int m):MyClass(x,y,z),p(l),q(m)
        {
        }

};
\end{lstlisting}

This is the case when the difference kicks in
\begin{verbatim}
class SleepyInt {
public:
  SleepyInt () { 
    std::this_thread::sleep_for(std::chrono::milliseconds( 10000 ));  
  }
  SleepyInt (int i) {}
  SleepyInt & operator = (int i) { return *this; }
};

class A {
   private:
     SleepyInt a, b;

   public:
     // by using initialization-list, only the proper constructor
     // is evoked
     is skipped A(int a_var, int b_var):a(a_var), b(b_var) {}; 
 };

class B {
   private:
     SleepyInt a, b;

   public:
     // takes at least 20s as by default the default constructor of the
     base-class is automatically evoked 
     B(int a_var, int b_var) : a(), b()
     //                      ^^^^^^^^^^ 
     {
        a = a_var;
        b = b_var;
     }
  };
\end{verbatim}


Example: 

\begin{verbatim}
struct aa
{
    int i;
    const int ci;

    aa() : i(0), ci(3) {} // works
};

struct aa
{
    int i;
    const int ci;       // constant member

    aa() : i(0) {} // will fail, constant member not initialized
};

struct aa
{
    int i;
    const int ci;

    aa() : i(0) { ci = 3;} // will fail, ci is constant
};
\end{verbatim}


\subsection{ * When NOT to use initialization list}

When the class has two different class constructors that need to initialize
the current (\verb!this!) object's data in different orders.
\begin{verbatim}
class Example1 {
 public:
  Example1(std::string decoded, std::string encoded)
      : decoded_(decoded),
        encoded_(encoded) {}
  
  explicit Example1(std::string encoded)
      : decoded_(),  // Can't use "decoded_(Decode())" since "encoded_" isn't initialised
        encoded_(encoded) {
    decoded_ = Decode();  // Assign here instead of initialising
  }
 private:
  std::string Decode();  // decodes class member "encoded_"
  std::string decoded_, encoded_;
};
\end{verbatim}
Sect.\ref{sec:explicit-}

{\small \begin{verbatim} 
Person::Person(int age) : _age(age) { }
\end{verbatim}}

To explicitly mention the class member, we use \verb!this! pointer.
However, it's often omitted unless we need to avoid ambiguity. NOTE:
The coding style: data member is often with an underscore \verb!_! in front of
the data member (Sect.\ref{sec:coding-style_data-member}).

{\small \begin{verbatim} 
Person::Person(int age) {
    _age = age; // this->_age = age;
}
\end{verbatim}}


A child class can call the parent constructor 

\begin{verbatim}
class Car 
{  
 public:
    Car();
    Car(double Price) ;
}

class Porche : public Car
{
public:
    Porche();
}

//this is how to implement
Porche::Porche():Car(price)
{
}
\end{verbatim}


\section{Methods (member functions)}
\label{sec:member-function-C++-class}

\subsection{(non-virtual) member functions}
\label{sec:non-virtual-member-function}

A public/private/protected regular member function of a class, without any
special keywords, is a non-virtual regular function.
A non-virtual method is a simple function with an implicit \verb!this! as its
first parameter. So, if a method has an argument with the same name as the data
member, we use \verb!this->! combined with the name to refer to the data
member of the object, e.g.
\begin{verbatim}
class A
{
public:
  void doSomething (int _myData)
  {
    this->_myData = _myData;
  }
private:
  int _myData;
}
\end{verbatim}
For naming convention, check Sect.\ref{sec:naming_style-C++}.

If class inheritance is used, it is recommended to provide virtual member
function to the base class (Sect.\ref{sec:virtual_member-function}). The below
example explains why.

EXAMPLE: when calling a non-virtual function that has implementation in both
parent class and derived class, the compiler does not know the function called
by a pointer is the one from which class (the parent one or the derived one);
and thus it always use the one in the base class.

\begin{verbatim}
class Base
{
protected:
 
public:
    const char* GetName() { return "Base"; }
};
 
class Derived: public Base
{
public:
    const char* GetName() { return "Derived"; }
};

int main()
{
    Derived cDerived;
    Base &rBase = cDerived;
    cout << "rBase is a " << rBase.GetName() << endl;
}
 
\end{verbatim}
REMEMBER: The base class and derived class are type-compatible
via the pointer (Sect.\ref{sec:OO_inheritance}).

The result is that it uses the GetName() of the base class, even though the
pointer is pointing to the object of derived class.
\begin{verbatim}
//the result is
rBase is a Base
\end{verbatim}

\begin{mdframed}

The question is if an object \verb!cDerived! of the derived-class is referenced
by a pointer \verb!rBase! from the base-class, how does the compiler knows that
when the method is called, i.e. \verb!rBase.GetName()! it should use the
implementation from the derived-class, instead of the base-class.
By default (non-virtual function), the implementation from the base-class is
used. 

\end{mdframed}


\subsection{singlet method ???}

%TODO: update this section: singlet method

\subsection{protected data/method: 'protected'}
\label{sec:OO_protected-data_method}

A \verb!protected! data/method can be accessed only by members of the current
class, the \verb!friend! class (Sect.\ref{sec:OO_friend-class}) and the friends
of the classes derived from the current class. This applies to all member
data/functions declared up to the next access specifier or the end of the class
A. It is used in emplementing {\bf encapsulation}
(Sect.\ref{sec:encapsulation}).

\subsection{private data/method: 'private'}

A \verb!private! field data or method of class A can only be accessed within
class A and \verb!friend! of class A (Sect.\ref{sec:OO_friend-class}). This
applies to all member functions declared up to the next access specifier or the
end of the class A.
It is used in emplementing {\bf encapsulation} (Sect.\ref{sec:encapsulation}).

Typically, class A provides two methods: getter and setters so that outside can
get or modify the internal data of a class A's object.

\subsection{static method: 'static'}
\label{sec:static-method}

IMPORTANT: A static method is a class method that has \verb!static! as the
keyword. Inside the static method, we should not use and/or modify any class's
non-static data members; but only \verb!static! data members
(Sect.\ref{sec:static-data-member}).
A \verb!static! member function can be called using object or class name.

A static method should (1) return something independent from class instance; (2)
interact with the \verb!static! data member, e.g.
if the \verb!static! data member (Sect.\ref{sec:OO_static-member}) is
accompanied with a \verb!private!, (even though the initialization is the same)
there is no way we can access to it directly using the class name. 

Example:
\begin{lstlisting}
// File.h
class Something
{
private:
    static int s_nValue;
    
public:
    // can define in header file
    static int GetValue() { return s_nValue; }
    
    // or in source file
    static void bool DoSomething(); 
};

// File.cxx
#include "File.h"
void Something::DoSomething() // note: no static keyword here
{
 // code
} 
 
 
int Something::s_nValue = 1; // initializer
 
 
// main.C
int main()
{
    std::cout << Something::GetValue() << std::endl;
} 
\end{lstlisting}

\textcolor{red}{\bf IMPORTANT}: A static member function is not the same as
static function, an extension to C language static function in C++
(Sect.\ref{sec:static-functions}).



\subsection{return by reference (T\&) or const reference (const T\&)}
\label{sec:return-by-reference}

NOTE: \verb!T! represents the datatype.

In terms of speed, both are the same, as there is no data copy occur, only the
reference is returned.

Return by reference, i.e. you can modify the returned value
\begin{lstlisting}
class A {

  public:
    // non-const member function
    int& getImem() { return imem;};
  private:
    int imem;
}
\end{lstlisting}
NOTE: \verb!imem! is a short form of \verb!this->imem!. 
In a non-const member,  \verb!this! is of type \verb!A* const!, i.e. the pointer
is fixed, but not the instance it points to.

Return by const reference, i.e. you cannot modify the returned value
\begin{lstlisting}
class A {

  public:
    //int& getImem() const { return imem;};
    int& getImem() const { return imem;};
  private:
    int imem;
}
\end{lstlisting}
In a non-const member,  \verb!this! is of type \verb!const A* const!, i.e. the
pointer is const, and also the instance it points to is const.




\subsection{const T\& function-name(arguments)}

Remeber the binding rule of using \verb!const! keyword
(Sect.\ref{sec:const_C++}). The \verb!const! (and \verb!volatile!) qualifier
binds to the left. This means that any time you see const, it is being applied
to the token to the left of it.
There is one exception, however; if there's nothing to the left of the const, it
binds to the right, instead. It's important to remember these rules.

Using \verb!const T&!, the return type is a reference to a \verb!const T!, i.e.
which is an unmodifiable object.

\begin{verbatim}
const T& data()  { return data_; }
^^^^^
\end{verbatim}
means it will return a const reference to T (here is \verb!data_! object), i.e.
The receiver has to be decalred as follows and cannot modify the data it gets.
\begin{verbatim}
Class c;
T& t = c.data()             // Not allowed.
const T& tc = c.func()      // OK.
\end{verbatim}


\subsection{T\& <function-name>(arguments) const}

\begin{verbatim}
       T& data() const { return data_; }
                ^^^^^
\end{verbatim}
means the function (inside this function) will not modify any member variables
of the class (unless the member is \verb!mutable!).

\begin{verbatim}
void Class::data() const {
   this->data_ = ...;  // is not allowed here since data() is const (unless 'data_' is mutable)
   this->anything = ... // Not allowed unless the thing is 'mutable'
}
\end{verbatim}


\subsection{const T\& <function-name>(arguments) const}

Using \verb!const! before and after the member function name. So, when we
combine \verb!const! to the front and end of the function, it means that we
cannot modify the data member inside the body, and the receiver cannot modify
the returned value as well.
\begin{verbatim}
const T& data() const  { return data_; }
^^^^^           ^^^^

\end{verbatim}

\section{Move assignment operator (C++11)}
\label{sec:move-assignment-operator-C++11}

\url{http://blog.smartbear.com/c-plus-plus/c11-tutorial-introducing-the-move-constructor-and-the-move-assignment-operator/}

\section{Data members}
\label{sec:data-member-C++-class}

\subsection{order of data members???}	

\textcolor{red}{A class's data member variables are always initialized in the
order in which they are declared in the class} (not the order data is passed to
the constructor - Sect.\ref{sec:constructor-C++-class}).
So, it is important to choose the right one to minimize memory usage, e.g. to
avoid padding for memory alignment; a similar case is in bitfield padding
(Sect.\ref{sec:bitfield-padding}).



% \subsection{static data member}
% 
% 
% A static data member is declared with \verb!static! specifier.
% When we declare a member of a class as static it means no matter how many
% objects of the class are created, there is only one copy of the static member.
% 
% All static data is initialized to zero when the first object is created, if no
% other initialization is present.
% 
% To define the value for a static data member, we have to define outside the
% class body
% \begin{lstlisting}
% class Box
% {
%    public:
%       static int objectCount;
%       
%       ...
% }
% 
% // Initialize static member of class Box
% int Box::objectCount = 0;
% 
% 
% \end{lstlisting}

\subsection{'static' data member}
\label{sec:OO_static-member}
\label{sec:static-data-member}

By default, each instantiated object has its own copy of all the data members
in a given class. 
% If the data member of class defined as \verb!static!, then
% only a single copy exist and shared for all objects of that class. 
When we declare a member of a class as \verb!static! it means no matter how many
objects of the class are created, there is only one copy of the static member.
If you want to define a static data member as \verb!private!, make sure you use
static member funciton as well (Sect.\ref{sec:static-method}).

All static data is initialized to zero when the first object is created, if no
other initialization is present.

\subsection{-- initialize option 1}

To initialize a \verb!static! data member, we need to initialize the
value for it outside the class body, preferably inside the source file
(NOT the header file), using
\footnote{\url{http://www.learncpp.com/cpp-tutorial/811-static-member-variables/}}
\begin{verbatim}
//.cpp file
<type> CLASS_NAME::static_variable = <value>;
\end{verbatim}

Example:
\begin{lstlisting}
// .H file
class Box
{
   public:
      static int objectCount;

      ...
}

// .C file
// Initialize static member of class Box
int Box::objectCount = 0;
\end{lstlisting}

Example: template class
\begin{lstlisting}
template< class Dummy >
struct BaseClass_statics
{
    static std::string bstring;
};

template< class Dummy >
std::string BaseClass_statics<Dummy>::bstring = ".";

class BaseClass
    : public BaseClass_statics<void>
{};
\end{lstlisting}

\subsection{-- initialize option 2}

An alternative is to use a function, as Dietmar suggested. Essentially that is a Meyers' singleton (google it).

\subsection{-- initialize option 3 (nested static structure) (C++11)}

\url{https://stackoverflow.com/questions/18860895/how-to-initialize-static-members-in-the-header}

To keep the definition of a static value with the declaration in C++11 a nested static structure can be used
\begin{lstlisting}
class BaseClass
{
public:
  static struct _Static {
     std::string bstring {"."};
  } global;
};

//the whole static structure is initialized
BaseClass::_Static BaseClass::global;

\end{lstlisting}

Then the value is accessed with
\begin{verbatim}
BaseClass::global.bstring;
\end{verbatim}


PITFALL: this solution still suffers from the problem of the order of
initialization of the static variables. When a static value is used to
initialize another static variable, the first may not be initialized, yet.

Here: the static variable headers will contain either { "" } or { ".h", ".hpp"
}, depending on the order of initialization created by the linker.

\begin{verbatim}
// file.h
class File {
public:
  static struct _Extensions {
    const std::string h{ ".h" };
    const std::string hpp{ ".hpp" };
    const std::string c{ ".c" };
    const std::string cpp{ ".cpp" };
  } extension;
};

// file.cpp
File::_Extensions File::extension;

// module.cpp
static std::set<std::string> headers{ File::extension.h, File::extension.hpp };
\end{verbatim}


\subsection{-- initialize option 4 (C++17)}

In C++17 you can use inline variables, which you can use even outside classes.
\begin{verbatim}

The inline specifier, when used in a decl-specifier-seq of a variable with
static storage duration (static class member or namespace-scope variable),
declares the variable to be an inline variable.

A static member variable (but not a namespace-scope variable) declared constexpr
is implicitly an inline variable.
\end{verbatim}

Example:
\begin{lstlisting}
class Someclass {
public:
    inline static int someVar = 1;
};

//or 
namespace SomeNamespace {
    inline static int someVar = 1;
}
\end{lstlisting}

\subsection{-- access data option 1}

\textcolor{red}{ACCESS static data member}: We can access the \verb!static! data
member using the object name or using the class name. NOTE: Using class name is
better, as it make clears the data member is shared.

Example:
\begin{Verbatim}
class Something
{
public:
    static int s_nValue;
};
 
int Something::s_nValue = 1;
 
int main()
{
    Something cFirst;
    cFirst.s_nValue = 2;
 
    Something cSecond;
    std::cout << cSecond.s_nValue;
 
    return 0;
}
\end{Verbatim}

\subsection{Using pointer for data member}

When to use the second scenario?

\begin{verbatim}
// Scenario 1
private:
    SomeClass instance_;
    
// Scenario 2
private:
   Someclass * instance_; // then use new in the constructor, and 
                          //  use delete in the destructor
\end{verbatim}

An advantage of using pointer for data member:
\begin{enumerate}
  \item you can do lazy initialization
  
  \verb!Boost::optional! is recommended whenever possible. It's safer than
  pointer because you can't do arithmetic on it
  
  \item a pointer can be deleted any time, i.e. you have more control about the
  live time and can recreate an object
  
  \item reduce compile time with forward declaration
  
  NOTE: If reducing compile time and header dependencies is important, use the
  pimpl idiom \url{http://c2.com/cgi/wiki?PimplIdiom}
  
  
\end{enumerate}

Disadvantage:
\begin{enumerate}
  \item Have to write your own copy constructor, assignment operator, to make
  sure a copy of the object is creatd properly.
  
  \item Make sure to delete the object in the destructor.
  
  \item  Finally, accessing non-pointer members is likely to be faster, because
  they are contiguous in memory. On the other hand, accessing pointer data
  member pointing to an object allocated on the heap is likely to cause a cache
  miss, making it slower.
  
\end{enumerate}
\url{http://stackoverflow.com/questions/1175646/c-when-should-i-use-a-pointer-member-in-a-class}

\subsection{Forward declaration for data member of an external type}


When a data meber is of type from an external class.
As long as you don't have any code in your header file, then we can use forward
declaration, without \verb!#include! the header file defined that
external class.
  
\begin{verbatim}
class ExtCamera;  // forward declaration to external class type in "ExtCamera.h"

class MyCamera {
public: 
  MyCamera() : m_pCamera(0) { }

  void init(const ExtCamera &cam);

private:
   ExtCamera  *m_pCamera;   // do not use it in inline code inside header!
   //or   ExtCamera  m_pCamera;   // do not use it in inline code inside header!
   
};
\end{verbatim}

\textcolor{red}{The advantage is to reduce the compile time and prevents side
effects by including too many other headers}.

\subsection{A reference data member}
\label{sec:reference-data-member}

IMPORTANT: A reference data member must be initialized immediately, at the time
object is created.  Example: 
\begin{verbatim}
class Actor
{
private:
   MaJR::Animation& activeAnimation;  // ERROR
	
public:
   Actor();
}
\end{verbatim}
you may get the error
\begin{verbatim}
error: uninitialized reference member
\end{verbatim}

A solution is to use initialization list in the constructor
(Sect.\ref{sec:initialization_list}) and it's important the passed
reference argument must be \verb!const!
\begin{lstlisting}
class    ClassB
{
    private:
        const int&    internalX;
    public:
        ClassB(const int& tempX);
}

ClassB::ClassB(const int& tempX):
     internalX(tempX)
{
}
\end{lstlisting}


It means that a reference MUST be initialized at the time the object is
created, i.e. in the constructor, assign the reference it to something. Then, we
cannot change the reference, even though we can change the value at which it
reference to. 

NOTE: If you need to dynamically change it, then you could use a pointer or,
probably better, a smart pointer. C++11 provides \verb!std::unique_ptr! and
\verb!std::shared_ptr! (Sect.\ref{sec:smart_pointer}). Boost also provide smart
pointer \footnote{\url{http://www.boost.org/doc/libs/1_49_0/libs/smart_ptr/smart_ptr.htm}}.

\begin{verbatim}
class B
{
private
  std::auto_ptr<int> _x;
  B(int x):_x(x) {};
}
\end{verbatim}

If you would like to conditionally initialize it, you can do something like
this:
\begin{verbatim}
MyObject& ref = (condition) ? MyObject([something]) : MyObject([something else]);
\end{verbatim}
\url{http://stackoverflow.com/questions/14882995/how-to-declare-a-reference-and-initialize-later}

\section{Class Inheritance}
\label{sec:OO_inheritance}

Inheritance is an important feature to achieve polymorphism in Object-Oriented
design (Sect.\ref{sec:polymorphism}).  One of the key features of class
inheritance is that a pointer to a derived class is type-compatible with a
pointer to its base class

Type-casting is the widely used technique when working with inheritance
(Sect.\ref{sec:type-cast-C++}) to detect the exact class type at runtime. 

\begin{lstlisting}


SomeClass obj = dynamic_cast<SomeClass>& obj_A;

if (obj == 0) {
  // then obj_A is not of type SomeClass 
}
else
{
  // then use 'obj' as an object of SomeClass type
}
\end{lstlisting}

It's important to know that when type-casting using upcasting or downcasting, we
should use {\it pointer} or {\it reference}; instead of using object, to avoid
the so-called {\it object-slicing} (Sect.\ref{sec:object-slicing}).


\begin{verbatim}
class IDriver
{
virtual void Init() = 0;
};

class USBDriver : public IDriver
{
virtual ~USBDriver(){}

void Init(){}
};
\end{verbatim}

Then to use it
\begin{verbatim}
Driver* driver = nullptr;
//runtime you decide that you need the USB not lets say another //RS232Driver

driver = new USBDriver();

delete driver;
\end{verbatim}

There are three types of inheritance: \verb!public! (default), \verb!protected!
and \verb!private!. 
\begin{enumerate}
  \item IS-A inheritance: \verb!public! - Sect.\ref{sec:inheritance_public}
  
  \item \verb!protected!: rarely useful, can be used to provide enhanced
  access of derived classes - Sect.\ref{sec:inheritance_protected}
  
  \item Implemented-in-terms-of: \verb!private! - Sect.\ref{sec:inheritance_private}
\end{enumerate}

\url{http://stackoverflow.com/questions/860339/difference-between-private-public-and-protected-inheritance-in-c}

\begin{verbatim}
class A 
{
public:
    int x;
protected:
    int y;
private:
    int z;
};

class B : public A  // this is the default, which means
// it's the same as
// class B : A 
{
    // x is public
    // y is protected
    // z is not accessible from B
};

class C : protected A
{
    // x is protected
    // y is protected
    // z is not accessible from C
};

class D : private A
{
    // x is private
    // y is private
    // z is not accessible from D
};
\end{verbatim}


\subsection{Base classes}

A class that is the parent of another one is considered as a base class (or
parent class). Thus, the following rules apply to how the child class can reuse
data member or method members of the base class
\begin{verbatim}
class Base {
    public:
        int publicMember;
    private:
        int privateMember;
    protected:
        int protectedMember;
};
\end{verbatim}

\begin{enumerate}
  \item  Everything that is aware of \verb!Base!, is also aware of publicMember
  of Base.
  \item Everything that is either a children of \verb!Base! or the children of
  these children, is aware of the protectedMember of Base.
  \item Only Base, but no one, can see privateMember of Base.
\end{enumerate}

\subsection{virtual base class}
\label{sec:virtual_base_class-C++}

Virtual base classes, used in \verb!virtual! inheritance, is a way of preventing
multiple "instances" of a given class appearing in an inheritance hierarchy when
using multiple inheritance.

Example:
\begin{verbatim}
class A { public: void Foo() {} };
class B : public A {};
class C : public A {};
class D : public B, public C {};
\end{verbatim}

An instance of D will be made up of B, {\it which includes A}, and C {\it which also
includes A}. So you have two "instances" (for want of a better expression) of A.
\begin{verbatim}
A   A
|   |
B   C
 \ /
  D
\end{verbatim}
This leads to you have the possibility of ambiguity
\begin{verbatim}

D d;
d.Foo(); // is this B's Foo() or C's Foo() ??

\end{verbatim}

Virtual inheritance is there to solve this problem. When you specify virtual
when inheriting your classes, you're telling the compiler that you only want a
single instance.
\begin{lstlisting}
class A { public: void Foo() {} };
class B : public virtual A {};
class C : public virtual A {};
class D : public B, public C {};


D d;
d.Foo(); // no longer ambiguous
\end{lstlisting}
as now you have
\begin{verbatim}
  A
 / \
B   C
 \ /
  D
\end{verbatim}

\url{https://stackoverflow.com/questions/21558/in-c-what-is-a-virtual-base-class}

\subsection{abstract class (data type)}
\label{sec:abstract_class_C++}

A base class that is abstract can not be an object but is an encapsulated
concept or idea that contains all of the information that is common between all
derived classes. It is a sub-object to an object that can be instantiated.
 

\textcolor{red}{Why an abstract class?}: One of the key features of class
inheritance is that the pointer to a derived-class is type-compatible with a
pointer to its base-class. An abstract class is served as the base-class {\it
per se} and that an object cannot be instantiated from the abstract class.
However, you can only have pointers or references to it, which enables us to
achieve the above goal.

Suppose Polygon is an abstract class, then
\begin{verbatim}
Polygon a; // ERROR
Polygon * b; // OK

Rectangle c;
Polygon& base=c; //OK
\end{verbatim}
Sect.\ref{sec:reference-data-member} describe how to use reference.

There are two ways to make a class becoming an abstract class:
\begin{enumerate}
  
  \item it contains at least one {\it pure virtual function})
  (Sect.\ref{sec:OO_pure-virtual-function}).

Some time we use the term {\bf pure abstract class} to indicate an abstract
class that only has pure virtual functions (and no data).
  
  \item it has a \verb!protected! constructor
  (Sect.\ref{sec:protected_constructor}), i.e. we cannot create an object from
  that class.

  NOTE:  The \verb!default! specifier CANNOT be used with an abstract class.
\end{enumerate}


Even though we cannot define an object of abstract class type, we can define a
pointer to that abstract class type. This pointer is often used to type-cast
(dereference) to an object of the derived-class of the abstract base class.

\begin{Verbatim}
class Polygon {
  protected:
    int width, height;
  public:
    void set_values (int a, int b)
      { width=a; height=b; }
    virtual int area (void) =0;
};

class Rectangle: public Polygon {
  public:
    int area (void)
      { return (width * height); }
};

class Triangle: public Polygon {
  public:
    int area (void)
      { return (width * height / 2); }
};

int main () {
  Rectangle rect;
  Triangle trgl;
  Polygon * ppoly1 = &rect;
  Polygon * ppoly2 = &trgl;
  ppoly1->set_values (4,5);
  ppoly2->set_values (4,5);
  cout << ppoly1->area() << '\n';
  cout << ppoly2->area() << '\n';
  return 0;
}	
\end{Verbatim}
See the recommended way to use smart pointer (Sect.\ref{sec:smart_pointer}).

\subsection{Public inheritance (default)}
\label{sec:inheritance_public}

The relationship between the derived class and the base-class is called {\bf
IS-a} (is-a) relationship. It means that an object of derived-class type is also
of base-class type. 

Example: \verb!public! keyword is used
\begin{Verbatim}
class Parent {
public:
	void sleep() {}
};

class Child: public Parent {
public:
	void gotoSchool(){}
};


int main( ) 
{ 
	Parent parent;
	Child child;

	// upcast - implicit type cast allowed
	Parent *pParent = &child; 

	// downcast - explicit type case required 
	Child *pChild =  (Child *) &parent;

	pParent -> sleep();
	pChild -> gotoSchool();
		
	return 0; 
}
\end{Verbatim}

\subsection{Private inheritance}
\label{sec:inheritance_private}

When \verb!private! is used for inheritance, it means that the \verb!public!
and \verb!protected! members of the base-class become the \verb!private! members
of the derived class.

\verb!private! inheritance is often used in combined with \verb!virtual! to
make a class uninheratable - Sect.\ref{sec:OO_uninheritable-class}). Other
usages:
\begin{enumerate}
  \item to expose some but not all the base-class's interface
\end{enumerate}

Example:
\begin{Verbatim}
class BaseClass {
public:
   // privMem accessible from member function
   int pubFunc() { return privMem; }
private:
   void privMem;
};

class DerivedClass2 : private BaseClass {
public:
   // pubFunc() accessible from derived class
   int usePublic() { return pubFunc(); }
};
\end{Verbatim}


\subsection{Protected inheritance}
\label{sec:inheritance_protected}


\subsection{Object slicing}
\label{sec:object-slicing}


When passing the object by reference, the manipulation is on the original
object, so there is no loss in information. 

When passing the object by value, the manipulation is on a new copy of the
original object. This is when the problem of 'object slicing' occur. The copy
constructor of the base-class is called, thus it only copy the data defined in
the base-class, but not in the derived-class. Then the copied object doesn't
have all the extra information from the derived-class. Also, the called to {\it
virtual} methods use the implementation from the base-class, not from the
derived-class. In other words, the data and the methods from the original object
is sliced. IN SUMMARY:
ALWAYS {\bf pass-by-reference} with type-cast (Sect.\ref{sec:type-cast-C++}).
When passing-by-reference, to avoid the object from being modified, we use
\verb!const! keyword

\begin{Verbatim}
class General {
public:
	General(const string& s) : name(s) {}
	virtual void eat() const {
		cout << "animal: " << name << " eat()" << endl;
	}
private:
	string name;
};

class Specific : public General {
private:
	string name;
	string habitat;
public:
	Bird(const string& sp, const string &s, const string &h)
		: Animal(sp), name(s), habitat(h) {};
	virtual void eat() const {
		cout << "bird: " << name << " eat() in " << habitat << endl;
	}
};

void WhatAreYouDoingValue(General a) {
	a.eat();
}

void WhatAreYouDoingReference(const Specific &a) {
	a.eat();
}
int main ()
{
	vector<General> gVec;
	Specific sp;
	/* sp object is copied as a base class and
	 put into gVec. So, its specialness is lost 
	 during the copying*/
	gVec.push_back(sp); 


	Animal animal("Animal");
	Bird bird("Eagle","Bald","US and Canada");

	cout << "pass-by-value" << endl;
	WhatAreYouDoingValue(animal);
	WhatAreYouDoingValue(bird);

	cout << "\npass-by-reference" << endl;
	WhatAreYouDoingReference(animal);
	WhatAreYouDoingReference(bird);
	
	return 0;
}
\end{Verbatim}
Output:
\begin{verbatim}
pass-by-value
animal: Animal eat()
animal: Eagle eat()

pass-by-reference
animal: Animal eat()
bird: Bald eat() in US and Canada
\end{verbatim}


\subsection{Multiple inheritance}
\label{sec:OO_multiple_inheritance}

A dreaded diamond scenario for multiple inheritance
\begin{verbatim}
This is
virtual inheritance   multiple inheritance
        A                 A      A
       / \                |      |
      B   C       -->     B     C
       \ /                 \   /
        D                    D
\end{verbatim}
Like this: A = top, B = left, C = right, B = bottom

\begin{verbatim}
class Top
{
public:
   int a;
};

class Left : public Top
{
public:
   int b;
};

class Right : public Top
{
public:
   int c;
};

class Bottom : public Left, public Right
{
public:
   int d;
};
\end{verbatim}

The memory layout in GCC will looks like
\begin{verbatim}
Top object
Top::a

Left object
Top::a
Left::b


Right object
Top::a
Right::c

Bottom object
Left::Top::a
Left::b
Right::Top::a
Right::c
Bottom::d
\end{verbatim}

Based on the memory layout, a Left object can be treated as it's a Top
object. This is okay
\begin{verbatim}
Left* left = new Left();
Top* top = left;
\end{verbatim}

Again, this is also okay, as a Bottom object can be treated as a Left object
\begin{verbatim}
Bottom* bottom = new Bottom();
Left* left = bottom;
\end{verbatim}


Let's see the problem with this multiple inheritance. We cannot treat a Bottom
object as a Right object as the memory layout of Bottom doesn't cover the memory
layout of an Right object, and are not compatible.
\begin{verbatim}
Bottom* bottom = new Bottom();
Right* right = bottom;
\end{verbatim}
		
To resolve the problem above, \verb!right! pointer need to be
adjusted and the memory location that \verb!right! pointer point to
is given as follow, along with \verb!bottom! pointer points to
\begin{verbatim}
Bottom object
Left::Top::a     <---   bottom pointer
Left::b
Right::Top::a    <--  right pointer
Right::c
Bottom::d
\end{verbatim}


Now, another problem arises, \verb!right! and \verb!bottom! doesn't point to the
same location. Another ambituity is that, the below \verb!top! pointer doesn't
know where it should point to, i.e. at \verb!bottom! pointer or \verb!right!
pointer ?
\begin{verbatim}
Top* top = bottom;

// error: `Top' is an ambiguous base of `Bottom'
\end{verbatim}

One possible solution is 
\begin{verbatim}
Top* topL = (Left*) bottom;
Top* topR = (Right*) bottom;
\end{verbatim}

A better solution is to use virtual inheritance
(Sect.\ref{sec:OO_virtual-inheritance}).

\subsection{Virtual base classes. Virtual inheritance}
\label{sec:OO_virtual-inheritance}

Virtual base classes is used to avoid ambiguity in multiple inheritance
(Sect.\ref{sec:OO_multiple_inheritance}). So, it should be used with virtual
inheritance.

Based on the example in the previous section, class Left and Right need both
virtual inheritance from class Top.
\begin{verbatim}
class Top
{
public:
   int a;
};

class Left : virtual public Top
{
public:
   int b;
};

class Right : virtual public Top
{
public:
   int c;
};

class Bottom : public Left, public Right
{
public:
   int d;
};
\end{verbatim}
So the new layout, from the programmer point of view
\begin{verbatim}
    A
   / \
  B   C
   \ /
    D
\end{verbatim}
so that only one instance of A inside D. From compiler's point of view.
\begin{verbatim}
Bottom
Left::Top::a
Left::b
Right::c
Bottom::d
\end{verbatim}
Here, again, we face the problem with Righ's pointer, as we cannot access a
'real' Right object in the same way as an upcasted Bottom oject.
\begin{verbatim}
Right* right = bottom;
\end{verbatim}


\url{http://www.learncpp.com/cpp-tutorial/118-virtual-base-classes/}

\url{http://www.phpcompiler.org/articles/virtualinheritance.html}



\section{Methods for class inheritance}

The below way of defining and implementing class's method is useful for class
inheritance (Sect.\ref{sec:OO_inheritance}).

\subsection{'virtual' member function}
\label{sec:virtual_member-function}

Suppose you have multiple classes that are related via inheritance
(Sect.\ref{sec:inheritance_private}). 
\begin{verbatim}
class A ()
{
  void methodX();
}

class B: public A 
{
  void methodX();
}
\end{verbatim}

Now, you define a pointer as of pointer type to base class; but you can use it
to point to an object of a derived class. Then, when you call a function, the compiler does not know
which implementation of the method (i.e. in which class) to use
\begin{verbatim}
A * pa; 

if condition:
   pa = new A();
else
   pa = new B();

pa->methodX();  // LOOK HERE [use method of A or B ???]
\end{verbatim}

Typically, the address of the function is known at compile-time, as relative to
the starting address of the object.
However, in this scenario, it means that the address of the function is not
known until run-time.

The challenge when using non-virtual methods
(Sect.\ref{sec:non-virtual-member-function}) in class inheritance is that if (1)
both the base class and derived class implements the same method, (2) the object
(from the derived class) is pointed by the pointer from the base class, then the
compiler does not know if the real object the pointer pointing to is of derived
class or base class and thus always use the base class's method.

From C++98, to help the compiler know what to do, we need to tell the compiler
that the method as \verb!virtual!. And the compiler uses a special data
structure, known as vtable (virtual table) of fixed size, defined inside the
object, and the value at one index of the table is the run-time known address of
the implementation of the virtual function to use.


A virtual member function is a function with \verb!virtual! keyword as prefix.
Then, this virtual member function can be redefined (i.e. reimplemented) in the
derived class [where it can have the virtual keyword or not; if it does not,
then the derived class of this class cannot re-implement the method].
\textcolor{red}{This is the single most important feature in Object-Oriented
design}, starting from C++98 (Sect.\ref{sec:C++98_virtual-functions}).

Importantly, you CAN'T change a non-virtual member function in the base
class, into a virtual member function in the derived class.

A related important concept is \textcolor{red}{pure virtual function}
(Sect.\ref{sec:pure_virtual_function}), it means the class, having at least one
pure virtual function is an abstract class.

\subsection{-- WARNING: virtual function override intended?}

When you have a virtual function in the parent class, and the derived class has
a method of the same name, but different signature, and also does not
implementing the parent's virtual function.

A member function of a class does not override (Sect.\ref{sec:override}) any
base class virtual member function. 
If a class function definition has the same name as the virtual function in the
base class, but not the same number or type of arguments, then this effectively
hides the virtual function in the base calsss.

Also, importantly, you CAN'T change a non-virtual member function in the base
class, into a virtual member function in the derived class.


\begin{lstlisting}
class DataItem
{
   virtual void std::string getString(Error* error=0);
}

class ArrayDataItem : public DataItem
{
 // does not override 'getString'; nor define a new one of the same name
}


class NumericArrayDataItem : public ArrayDataItem
{
   // a new (can be pure) method of the same name of a virtual function in the parent
   // you will get warning ' virtual function override intended?'
   virtual void std::string getString(std::vector<int> coords, Error* error=0) const = 0;
      
   // to fix that (i.e. to unhide the function in the parent class)
   // we need to tell the compiler to use 'setString(Error* error) from DataItem
   using DataItem::getString;
}
\end{lstlisting}


Example: a function in the drived class with the same name (regardless of that
name is a method or a data member) in the base class always hide the name in the
base class - hide a function often reflects a design error. If you don't want to
hide, first make sure the name is a virtual function, and the one in the derived
class has a matching function name and signature to the name in the base class.
\begin{lstlisting}
class InstanceFactoryDataItem : public DataItem
{
public:
   virtual InstanceFactoryDataItem& operator=( const InstanceFactoryDataItem& DI);
}


class ConstantTypeDataItem : public InstanceFactoryDataItem
{
public:

   //if you intend to hide the base method 'operator='
   // so that a derived class always use the derived method
   virtual ConstantTypeDataItem& operator=( const ConstantTypeDataItem& DI); 
}
\end{lstlisting}


Example:
\begin{lstlisting}
struct B
{
   int object;
   void func() {}
};

struct D: B
{
  void object() {};  // a funtion, and it hides the data member B::object
  int func;   // a data member, and it hides B::func()
}

int main() {
   D d; 
   d.object = 5 ; // ERROR
   d.func();      // ERROR

}
\end{lstlisting}


\subsection{ -- in base class}
\label{sec:virtual_member-function-in-base-class}

In the base class, the \verb!virtual! member function can be declared with or
without a body implementation. If a virtual member function without a body
implementation, it is called a {\it pure virtual member function}
(Sect.\ref{sec:OO_pure-virtual-function}), and any derived class MUST implement
that pure virtual method.

\begin{mdframed}

IMPORTANT: {\it virtual function} and {\it virtual base class} are two different
concepts, even though they use the same keyword \verb!virtual!. See virtual
destructor (Sect.\ref{sec:OO_virtual-destructor}).
\end{mdframed}

Because of that, the purpose of using \verb!virtual! function is (1) (in the
case of pure virtual function) tell the derived class to reimplement (augment or
override) the method (Sect.\ref{sec:virtual_member-function-derived-class}), (2)
allows the compiler to correctly call the method defined in both the base class
and derived class.

\textcolor{red}{Let's examine the following example to know why we need}
\verb!virtual! function.

Example: (based on the one in the previous section)
If the method function is defined as \verb!virtual!, then the compiler
can tell what class the pointer is referencing to, so that
it can execute the appropriate implementation. Using virtual
function, the compiler ensure that the replacement is always called, even if the
member function to derived-class object is pointed by base-class object.
\textcolor{red}{We will explain how the compiler handle this later.} 

Example: use \verb!virtual! function (from the base-class).
\begin{lstlisting}
class Base
{
protected:
 
public:
    virtual const char* GetName() { return "Base"; }
};
 
class Derived: public Base
{
public:
    virtual const char* GetName() { return "Derived"; }
};
\end{lstlisting}
\verb!virtual! tells the program to go look and see if there are any
more-derived versions of the function available, i.e. it checks for every
inherited class between \verb!Base! and \verb!Derived! and use the most-derived
version of the function.

Generally, a derived-class can modify the method defined in the base-class.
The new implementation can either fully replace the based-class's method
implementation (i.e. {\it override})
(Sect.\ref{sec:virtual_member-function-override}), or partially replace (i.e.
{\it augment}) (Sect.\ref{sec:virtual_member-function-augment}).

IMPORTANT: The function interface should be the same (i.e. same return type,
same input types). However, there are the case when the return type can beq
different (see below at RETURN TYPE). If we use different input types, it leads
to the so-called {\it overloading} (Sect.\ref{sec:name_mangling}). To avoid this
error that may occurs, C++11 add a new qualifier \verb!override!.

\textcolor{red}{\bf RETURN TYPE}:
\begin{itemize}
  \item In normal cases, the return type of a virtual function and its override
  must match.
  \item Under special case, it can be different, e.g. the return type is a
  pointer or a reference to a derived class, then the override functions can
  return a pointer or a reference to a derived class. This is known as {\bf
  covariant return types}.
\end{itemize}

\subsection{ -- in derived class}
\label{sec:virtual_member-function-derived-class}

When a derived class implements a virtual function defined in the base class
(Sect.\ref{sec:virtual_member-function-in-base-class}), \textcolor{red}{in the
derived class, we can use the } \verb!virtual! optionally for the virtual method
in the base class, so the below two approaches are the same in effect and
performance, yet the first approach is recommended for code cleaness
\begin{verbatim}
struct A {
    virtual void hello() = 0;
};

// Approach 1 (recommended: more cleaner)
struct B : public A {
    virtual void hello() { ... }
};

// Approach 2
struct B : public A {
    void hello() { ... }
};
\end{verbatim}
\url{http://stackoverflow.com/questions/4895294/c-virtual-keyword-for-functions-in-derived-classes-is-it-necessary}


\subsection{* override}
\label{sec:virtual_member-function-override}

Overrriding a method means we replace the implementation of the method in the
base class with a totally implementation of the same method in the derived
class. To do so, we just define the method (in both classes) as \verb!virtual!
and implement a new body.

\begin{lstlisting}
class Base
{
public:
    // This version of GetThis() returns a pointer to a Base class
    virtual Base* GetThis() { return this; }
};
 
class Derived: public Base
{
    // Normally override functions have to return objects of the same type as the base function
    // However, because Derived is derived from Base, it's okay to return Derived* instead of Base*
    virtual Derived* GetThis() { return this; }
};
\end{lstlisting}

\subsection{* augment}
\label{sec:virtual_member-function-augment}

Augmenting a method means we add something to the existing implementation of the
method in the base class to the implementation of the same method in the derived
class. To do so, we define the method in both classes as \verb!virtual!, and
inside the body of the method in derived class we call the body of the 
method in the base class, before (in both classes) and implement a new body.

NOTE: We often use this for \verb!void! method
\begin{lstlisting}
class Base
{
public:
    // 
    virtual void doSomething() { std:cout << "do something"; }
};
 
class Derived: public Base
{
    // 
    virtual void doSomething() { 
       Base::doSomething();
       // now the augmented part
       std::cout << "do something in derived class";
    }
};

\end{lstlisting}
\url{http://www.learncpp.com/cpp-tutorial/122-virtual-functions/}

\subsection{* guarantee virtual method: override (C++11)}
\label{sec:C++11_override}

A typical mistake when overriding a virtual function (Sect.\ref{sec:virtual_member-function})

Suppose you want to implement override of the \verb!virtual! function A().
However, the below code leads to overloading of A()
\begin{Verbatim}
class Base
{
    virtual void A(float=0.0);
    virtual void B() const;
};
 
class Derived: public Base
{
    virtual void A(int=0); // specifies parameter as int instead of float, treated as new function
    virtual void B(); // specifies function as non-const, treated as new function
};
\end{Verbatim}

To tell the compiler that we want an override method, so that it can help us to
detect the mistake, C++11 adds a new keyword \verb!override! that we add behind
the \verb!virtual! function
\begin{Verbatim}
class Base
{
    virtual void A(float=0.0);
    virtual void B() const;
    virtual void C();
    void D();
};
 
class Derived: public Base
{
    virtual void A(int=0) override; // compile error because Derived::A(int) does not override Base::A(float)
    virtual void B() override; // compile error because Derived::B() does not override Base::B() const
    virtual void C() override; // ok!  Derived::C() overrides Base::C()
    void D() override; // compile error because Derived::D() does not override Base::D()
};
\end{Verbatim}
The use of \verb!override! doesn't create any new function, it just enable
the compiler in helping to detect unexpected errors easier. \textcolor{red}{It
is highly recommended to use the keyword in new C++11 code}. NOTICE: If an
existing use \verb!override! as a variable, compiling the code with
C++11-enabled compiler can break the code.

\subsection{How g++ implement virtual functions?}
\label{sec:virtual-function_GCC}









\subsection{pure virtual function ( = 0)}
\label{sec:OO_pure-virtual-function}

A pure virtual function is
\begin{itemize}
  
  \item  a virtual function
(Sect.\ref{sec:virtual_member-function}), i.e. that is declared with
\verb!virtual! at the beginning,

  \item and has \verb!=0! at the end (an equal sign and a zero)
\end{itemize}

\begin{verbatim}
class Polygon
{
 public:
    virtual int area() = 0;
    
    virtual void attack() const = 0;
}
\end{verbatim}
A class with a pure virtual function is an abstract class
(Sect.\ref{sec:abstract_class_C++}).

NOTE: The = 0 is not related to const. Putting \verb!const! after a member
function indicates that the code inside it will not modify the containing object
(except in the case of mutable members).
This is useful because the compiler will report an error if you accidentally
modify the object when you didn't intend to.



IMPORTANT:
\begin{enumerate}
  
  \item (if it is defined inside a class which means that class is an abstract class) 
  
  a pure virtual function MUST be implemented in the derived-class to make this
  derived-class no longer abstract. If not, the derived class is still an
  abstract class (Sect.\ref{sec:abstract_class_C++}) 
  
  \item the base asbtract class can have NO implementation, or can still define
  an implementation. SO, \verb!= 0! means derived classes must provide an
  implementation, not that the base class can not provide an implementation.
  
  There is very little point in providing a definition, because it will never
  be called unless someone explicitly does. If the base class implements the pure
  virtual function, then a derived class can explicitly call the base class
  implementation (if access permissions allow it) by using a fully-scoped name
  (by calling BaseClassName::functionName()

\begin{verbatim}
// A.h
class A 
{
public:
  virtual void f() = 0;
}
// A.cpp
void A::f() {
    cout<<"Test"<<endl;
}

//B.cpp
class B : public A {

    virtual void f() {
        // class B doesn't have anything special to do for f()
        //  so we'll call A's

        // note that A's declaration of f() would have to be public 
        //  or protected to avoid a compile time problem

        A::f();
    }

};
\end{verbatim}  
\end{enumerate}


\begin{Verbatim}
// abstract class CPolygon
class Polygon {
  protected:
    int width, height;
  public:
    void set_values (int a, int b)
      { width=a; height=b; }
    virtual int area () =0;
};
\end{Verbatim}

\subsection{'friend' class/method}
\label{sec:OO_friend-class_method}

Having {\it friend} class/method is considered to violate the concept of
encapsulation, i.e. as one class can get access to the internals of another.
However, we can think of a friend class is  just an extend interface of it

\subsection{* friend class}
\label{sec:OO_friend-class}

If you want class A to be able to 'see' (i.e. use directly) the
\verb!protected! data from another class B, then declare class A as a friend of
class B. How? inside class B, declare that class A is a friend of it
\begin{verbatim}
class B {

   friend class A;
}
\end{verbatim}


Example:
\begin{verbatim}
class Node 
{
    private: 
    int data;
    int key;
    // ...

    friend class BinaryTree; // class BinaryTree can now access data directly
};

class BinaryTree
{
    private:
    Node *root;

    int find(int key);
};
int BinaryTree::find(int key)
{
    // check root for NULL...
    if(root->key == key)
    {
        // no need to go through an accessor function
        return root->data;
    }
    // perform rest of find
\end{verbatim}

Example: VERY IMPORTANT (A forward declaration of a class creates an incomplete
type, the class definition creates a complete type)
\begin{verbatim}
class A;  // we use forward declaration 
class B {
  int x;
public:
  B(int __x) : x(__x) {}
  friend bool A::method(B*) const;   // error: undefined member cannot be friend
//friend class A;                    // fine:  class A is declared and can be used
};

class A {
  int y;
pu8blic:
  A(int __y) : y(__y) {} bool method(B*b) const { return y == b->x; }
};
\end{verbatim}
The error you get \verb!invalid use of incomplete type! means
\begin{verbatim}
a forward declaration is not enough ... you need the class definition"
\end{verbatim}
The reason is that the time you define a friend method from class A, the
compiler needs to know the full definition of A, yet forward declaration
(Sect.\ref{sec:forward_declaration}) doesn't give that information.
\textcolor{red}{The only time that forward declaration is acceptable is when you
make pointers or a reference to that type, and you NEVER access to any of its
member}. Here, we use forward declaration and we access to the member function
\verb!A::method()!. So, the solution is to define the whole class A as
\verb!friend! of class B. 

\subsection{* friend method}

A similar to friend method
\begin{verbatim}
SYNTAX: 
friend return_type class_name::function(args); 
\end{verbatim}


\begin{verbatim}
class B{

    friend int A::methodA();
}
\end{verbatim}

Example:
\begin{verbatim}
class Node 
{
    private: 
    int data;
    int key;
    // ...

    friend int BinaryTree::find(); // Only BinaryTree's find function has access
};
\end{verbatim}

\begin{framed}
\textcolor{red}{Can a PRIVATE member function of class A a friend of class B?}
The answer is YES.
\end{framed}


\section{Uninheritable class or method (C++11 : 'final' specifier)}
\label{sec:OO_uninheritable-class}

We have learnt how we can extend a class using inheritance
(Sect.\ref{sec:OO_inheritance}). There are situations that we don't want a class
or a method to be inheritable. This class is called {\it final} (or frozen)
class. This can be done in C++11 by adding \verb!final! keyword to the class
definition or the method. [In Java, it uses 'final'; while C\# or C++ extension
in Windows it uses 'sealed']. 


Before C++11, there is some options: 
\begin{enumerate}
  \item  defining the class as \verb!public virtual!.  Here, we need a
  ClassSealer that help creating an uninheratable class.
\end{enumerate}

Example 1: 
\begin{Verbatim}
class ClassSealer {
private:
   friend class Sealed;
   ClassSealer() {}
};

class Sealed : private virtual ClassSealer
{ 
   // ...
};


class FailsToDerive : public Sealed
{
   // Cannot be instantiated
};
\end{Verbatim}



WHEN TO USE ? when we want an {\it immutable } class stays immutable
(Sect.\ref{sec:OO_immutable-mutable}).

RECOMMENDATION ? Generally, it's not recommended to use \verb!final! unless we
need to. If we use, we need to document the code for others to know why it's
uninheratable.

Example: uninheratable method
\begin{Verbatim}
class Base
{
    virtual void A() final; // final identifier marks this function as non-overrideable
};
 
class Derived: public Base
{
    virtual void A(); // trying to override final function Base::A() will cause a compiler error
};
\end{Verbatim}


Example: uninheratable class (put the \verb!final! specifier before the opening
\verb!{! if the class inherit from nothing, or before \verb!: base1, base2!)
\begin{Verbatim}
class Base final // final identifier marks this class as non-inheritable
{
};
 
class Derived: public Base // trying to override final class Base will cause a compiler error
{
};
\end{Verbatim}
or
\begin{Verbatim}
class Base1 
{
};
class Base2
{
};
 
class Derived final: public Base1, Base2  
{   // the final class
};


\end{Verbatim}
\section{Function object (functors)}
\label{sec:functor}
\label{sec:function-object}

In old time, a C function when being passed as an argument to other function in
the form of {\it function pointer} (Sect.\ref{sec:function_pointer}) and is
called a {\bf callback}.
It means that the callee is expected to {\it call back} (execute) the (function) argument at some
convenient time. However, there are some restriction using callback, e.g.
passing state into or returning out of the callback function sometime difficult.

\begin{verbatim}
/* Callback function */
int compareInts(void* a, void* b) {
  return *((int *)(a)) < *((int *)(b));
}
...
/* Declaration of C sorting function */
void sort(void* firstItem, size_t itemSize, void* lastItem, int (*cmpfunc)(void *, void *));
...
int main(void) {
    int items[] = {4, 3, 1, 2};
    sort((void *)(items), sizeof(int), (void *)(items + 3), compareInts);
    return 0;
}
\end{verbatim}

To resolve the issues, C++ offers a solution call a {\bf function object} (aka
functors, or functional) is used. A functor is an ordinary class/struct, with an
overload of \verb!()! operator. So, syntactically they behave like ordinary
function, under the hood they are  real objects, i.e. they are {\bf object +
()}; and as such it causes an overhead of object creation

Functors are any objects that can be used with {\bf ()} operator in the manner
of a function. In C++, we need to use the header file \verb!<functional>! to
write a functor. As a functor, you're allowed to
overload the ``function call'' operator (i.e. the operator ()). Here, the
function call operator can take any number of arguments of any types and return
anything it wishes to.


So, a normal function, a function pointer, or a class object can be a functor
for which the {\bf ()} operator is overloaded in the class implementation, i.e.
we need to define the function \verb!operator()()!.
\begin{verbatim}
struct IntComparator {
  bool operator()(int a, int b) const {
    return a < b;
  }
};
\end{verbatim}



C++11 provides a class template \verb!std::function!. Instances of this class,
or derivatives from this class can store, copy, and invoke any callable
target-functions, lambda expressions (expressions defining anonymous functions),
bind expressions (instances of function adapters that transform functions to
other functions of smaller arity by providing values for some of the arguments),
or other function objects. 
\footnote{\url{http://en.wikipedia.org/wiki/Functional_(C++)}}

\section{Mutable vs. Immutable objects}
\label{sec:OO_immutable-mutable}

In object-oriented (OO) and functional programming, an {\bf immutable object} is
an object from of an immutable class, i.e. the object's state CANNOT be modified
after it is created. This is in contrast to {\bf mutable objects}. To create an
immutable class, we need also to use \verb!final! keyword (from C++11 -
Sect.\ref{sec:OO_uninheritable-class}) to avoid the derived-class of it which
add a new function that could cause the class to be come mutable.

The reason of using {\bf immutable objects} is that they are thread-safe,
simpler to understand, higher security than mutable objects. In C++, we need to
provide 2 different ``get'' methods, e.g. getItems(). This concept in C/C++ is
called {\it const-correctness}.
\begin{verbatim}
template<typename T>
class Cart {
  private:
   std::vector<T> items;
 
  public:
   Cart(const std::vector<T>& v): items(v) { }
 
   std::vector<T>& getItems() { return items; }
   const std::vector<T>& getItems() const { return items; }
   int total() const { /* return sum of the prices */ }
 };
\end{verbatim}

\section{Template}
\label{sec:templat-C++}

A template enables the benefit of dynamic polymorphism, without the extra runtime costs of virtual functions.

\subsection{Non-type template}

\begin{verbatim}

template<auto n> struct B {  };


B<5>  b1;  // here we pass in the value '5', not a type name
\end{verbatim}

\subsection{Type template}

We can use \verb!class! of \verb!typename! interchangably. However, using \verb!typename! is the choice since 
C++17.

\begin{verbatim}
// NOTE: 'NAME' is optional
template <typename NAME, ...>

template <class NAME, ...>

// since C++11, we can mix using 'typename' and 'class'
template <typename|class NAME = default, ....>

\end{verbatim}

For type template, the argument at the time of use, must be a type-name (or
class name).

\subsection{Template template parametr}
 
\begin{verbatim}
template <typename T>
class my_array
{
}

// 2 type-template paramaters and 1 template-template parameter
template <typename K, typename V, template<typename> typename C  = my_array>
class Map
{
   C<K> key;
   C<V> value;
}
\end{verbatim}


\subsection{Where to put it}

We can put the template as part of the class (Sect.\ref{sec:class-template} -
which leads to creating multiple class), or inside the class for the function of
that class only (Sect.\ref{sec:member-template} - which leads to creating
multiple functions for one class).

\subsection{Member template}
\label{sec:member-template}
\label{sec:template-member}

\begin{verbatim}
template <class T>
class MyClass
{
  void MyFunc();
  
  // this is the member template
  template <class T2>
  static void MyFunc2(T2* data);
}


template <class T>
void MyClass<T>::MyFunc()
{
}

//ERROR: unable to match function definition to an 
// existing declaration
template <class T, class T2>
void MyClass<T>::MyFunc2(T2* data)
{
}
\end{verbatim}  

A template can be declared within a class, and is thus called a member template.
A member template can be defined within or outside its class definition or class
template definition.

\begin{enumerate}
  \item A {\bf member template of a class template} that is defined outside of its class template definition
  shall be specified with (1) the template-parameters of the class template, followed by (2) 
  the template-parameters of the member template,
  
\begin{verbatim}
// CORRECT code
template <class T>
template <class T2>
void MyClass<T>::MyFunc2(T2* data)
{
}
\end{verbatim}  

\end{enumerate}

\subsection{Class template}
\label{sec:template-class}
\label{sec:class-template}

Class template is the class that has \verb!template<class T1, class T2, ..>! 

\begin{verbatim}
template <class T1, class T2>
class ClassA{

  void method();
}


template <class T1, class T2>
void ClassA<T1, T2>::method()
{

}

\end{verbatim}


\textcolor{red}{template: typename vs. class}:
\label{sec:typename-vs.-class}
\label{sec:class-vs.-typename}


Why we support both \verb!class! and \verb!typename! within C++ to indicate a type parameter
since the keywords do not hold any platform significance? The reason for the two keywords is historical
\begin{enumerate}
  
  \item (In the original template specification): Stroustrup reused the existing
  class keyword to specify a type parameter rather than introduce a new keyword
  that might of course break existing programs
  
  Until the ISO-C++ standard, this was the only way to declare a type parameter.

  However, this causes problems
\begin{verbatim}
template <class T>

class Demonstration {

public:

void method() {

    T::A *aObj; // oops …

     // …

};
\end{verbatim}

aObj is intended by the programmer to be interpreted as the declaration of a
pointer to a nested type A within the type parameter T. 

BUT,  the language grammar interprets it as an arithmetic expression multiplying
the static member A of type T with aObj and throwing away the result


  \item To resolve the above issue, \verb!typename! specifier is added so that it knows that we're using nested type
  of a template parameter.

\begin{verbatim}
typename T::A* a6; // declare pointer to T’s A
\end{verbatim}
\end{enumerate}
\url{https://blogs.msdn.microsoft.com/slippman/2004/08/11/why-c-supports-both-class-and-typename-for-type-parameters/}


Based on the above explaination, \verb!typename! and \verb!class! are
interchangeable in the basic case of specifying a template: the two ways below
are equivalent.
\begin{lstlisting}
template < typename T >
class A {..}


template < class T >
class A {..}
\end{lstlisting}

However, there are scenarios that they are different:
\begin{enumerate}
  
 % \item if you use \verb!typename!, then \verb!T! can be anything, including built-ins (such as an array )
  
  
  
  \item  when you're refering to a nested type that depends on another template parameter, then you MUST use \verb!typename!
  
\begin{verbatim}
template<typename param_t>
class Foo
{
    typedef typename param_t::baz sub_t;
};
\end{verbatim}
  
  
  \item (until C++14) when you have a template is a class which itself is also a template
  class, then \verb!class! is used for a different purpose (referring to a
  template class)
  
\begin{lstlisting}

template < template < typename, typename > class Container, typename Type >
class Example
{
     Container< Type, std::allocator < Type > > baz;
};
\end{lstlisting}

(note: since C++17 both keywords are allowed in this case).

   \item use \verb!class! when instantiating a template
   
\begin{verbatim}
template class Foo<int>;
\end{verbatim}
\end{enumerate}


\section{Template vs. Abstract base class}
\label{sec:template-vs-abstract-base-class}

To make a class flexible (adaptable), we can use abstract base class
(Sect.\ref{sec:abstract_class_C++}). No object can be created from this class.
Instead, it defines common methods, that its derived class can choose a specific
implementation. Typically, with abstract class, we also have virtual member
function (Sect.\ref{sec:virtual_member-function}). The purpose of using abstract
class and virtual function is for late-binding (Sect.\ref{sec:late-binding}) and
is considered as dynamic polymorphism. For virtual function, a virtual table is
used to know what method to use for that object at run-time, which comes at
certain cost of vtable lookup.


\textcolor{red}{Example 01:}
{\small \begin{verbatim} 
class Brake {
public: virtual void stopCar() = 0;  
};

class BrakeWithABS : public Brake {
public: void stopCar() { ... }
};

class Car {
  Brake* _brake;
public:
  Car(Brake* brake) : _brake(brake) { brake->stopCar(); }
};
\end{verbatim}
}

\textcolor{red}{Example 02:}
\begin{verbatim}
class Logger
{
  void warning() { write()};
  virtual void write() = 0;
}
class File_Logger: public Logger
{
   virtual  /* optional to have this keyword - we use if there is a chance that File_Logger is also derived */
   void write() { /* write for File logger */};
}

class Network_Logger: public Logger
{
   virtual  /* optional to have this keyword - we use if there is a chance that File_Logger is also derived */
   void write() { /* write for Network logger */};
}
\end{verbatim}

% \textcolor{red}{The using of method 1 when we decide to select different
% 'brake' class at {\bf runtime}, i.e. runtime polymorphism or the choice is
% made at runtime}.

Another option is using Template (Sect.\ref{sec:template-C++98}) which is considered as 
static polymorphism.

\begin{verbatim}
template<class Brake>
class Car {
  Brake brake;
public:
  Car(){ brake.stopCar(); }
};
\end{verbatim}
\textcolor{red}{Using template is recommended. It's safer (compile-time error
rather than runtime error), and probably can give better optimization}. 

There is a strategy for converting from using virtual method to template class.
NOTICE: The Logger class is now a template class, and it accepts another class 
as template and this class is supposed to implement the \verb!write! function
\begin{lstlisting}
template <class T>
class Logger: public T
{
  void warning() { this->write()};
}

class File_Writer
{
   void write() { /* write for File logger */};
}

class Network_Writer
{
   void write() { /* write for Network logger */};
}

#typedef Logger<File_Writer>   File_Logger;
#typedef Logger<Network_Writer> Network_Logger;


\end{lstlisting}

The third option is using Template with private inheritance
(Sect.\ref{sec:inheritance_private})
\begin{verbatim}
template<class Brake>
class Car : private Brake {
  using Brake::stopCar;
public:
  Car(){ stopCar(); }
};
\end{verbatim}
\textcolor{red}{Private inheritance should be avoided in
general}\footnote{\url{http://www.gotw.ca/publications/mill06.htm}}.

\section{Method override}
\label{sec:method_overriding}

In OO programming, {\bf method overriding} is about overriding
(Sect.\ref{sec:override}) for a class's method.

This feature allow a subclass (child class) to provide a specific implementation
of a method that is already provided in one of its superclasses (parent
classes). One of the methods is in the parent class and the
other is in the child class.

When a method is overrided, by default, at runtime, the method is used by the
class pointer at declaration, not the exact object it points to.

To enable that which method to be invoked depends on the object, we declare it
using \verb!virtual! keyword.

\begin{lstlisting}
class animal
{
    void /*nonvirtual*/ move() { cout << "This animal moves in some way" << endl; }
    virtual void eat() {}
};

// The class "animal" may possess a deceleration for eat() if desired.
class fish : public animal
{
    // The non virtual function move() is inherited but cannot be overloaded
    void eat() { cout << "Fish ingest food through the mouth and break it down in the esophagus." << endl; }
    
    // we can use 'virtual' again
    //virtual void eat() { cout << "Fish ingest food through ..." << endl; } 
};
\end{lstlisting}

This is a more complicated case when a subclass has more than one parent.
\begin{lstlisting}
class marine {
   virtual void eat() { //eat something different };
}

// The class "animal" may possess a deceleration for eat() if desired.
class fish : public animal,
             public marine
{
    virtual void eat() { cout << "Fish ingest food through ..." << endl; } 
};
\end{lstlisting}


The parent class can be 


\section{Make the class printable with $cout \ll$}


Suppose that you have a class, and you want an object of that class to be used
in \verb!cout <<! statement
\begin{verbatim}
myClass inst;
cout << inst;
\end{verbatim}

How does the operator \verb!<<! know what to with \verb!inst! object?
In the class definition, we need to define a \verb!friend! method
\begin{verbatim}
namesapce SomeNamespace{

class myClass  //struct myClass
{
public:
  friend std::ostream &operator<<(std::ostream &, myClass const &)
  
private:
 char array[SOME_FIXED_SIZE];
}

std::ostream & operator<<(std::ostream & out, const MyClass &inst)
{ //we can decide what to print out here
   out << write(inst.array, SOME_FIXED_SIZE);
   return out;
}

} //end SomeNamespace
\end{verbatim} 
IMPORTANT: The operator \verb!<<! need to be defined in the same namespace as
the object as well.


\section{ref class/struct (C++/CLI)}
\label{sec:ref_class}

a \verb!ref! is required if we want to create a managed class, i.e. a class that
can be used with other .NET applications (e.g. C\#) and C++/CLI. A \verb!ref!
class/struct that should be allocated on the managed heap. In other words, it
creates a reference type managed by the C++/CLR. 

NOTE: In Managed C++, it uses \verb!__gc! keyword, and in C++/CLI, it uses
\verb!ref! keyword.
\begin{verbatim}
// Managed C++
__gc  class Block  {};

// C++/CLI
ref class Block {}; 
\end{verbatim}

\begin{framed}
CLR use a separate heap (managed heap) on which it implements a precise,
asynchronous, compacting garbage collection scheme. To work correctly, it must track all storage
locations that can point into this heap at runtime. \verb!^! provides a handle
(similar to the concept of a pointer) through which the garbage collector can
track a reference to an object on the managed heap, thereby being able to update
it whenever that object is moved. 
\end{framed}

\begin{verbatim}
ref class platformDependent
{
public:
	platformDependent(void);
	static void downloadLotsOfLocation ();
	static LocationList loc;
};
\end{verbatim}


\begin{framed}
\verb!out! is limited to C\# only. The out concepts is achieved via a mod opt I
believe and most languages ignore it. 
\end{framed}
\section{Run-time class-type comparison (std::is\_same)}
\label{sec:is_same}


\section{Assignment operator: operator=()}
\label{sec:operator=()}

Assignment, or operator=, is a function that requires careful consideration
before you make it virtual.

\begin{mdframed}

\textcolor{red}{Slicing issue}:
"Slicing" is where you assign an object of a derived class to an instance of a
base class, thereby losing part of the information - some of it is "sliced"
away. For example, class A { int foo; }; class B : public A { int bar; }; So an
object of type B has two data members, foo and bar .

\end{mdframed}

If you have a base TBase and its derived class TDerived, and TBase has virtual
operator=, then TDerived might need to override TBase::operator=, in addition to
defining its own operator=.
\begin{enumerate}

  \item If TDerived defines virtual operator=, its subclasses must override
  TBase::operator= and TDerived::operator=, in addition to defining their own
  operator=. And so on.
  
  Clearly this gets out of hand quickly. It only makes sense to make assignment
  virtual when there are additional restrictions that keep this geometric
  progression from occurring.
  
  \item If special case handling is not necessary, the derived version can just call the base class version:

\begin{verbatim}
TDerived& TDerived::operator=(const TDerived &d)
  {
      TBase::operator=(d);
  }
\end{verbatim}
\end{enumerate}
