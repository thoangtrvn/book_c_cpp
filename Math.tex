\chapter{Math operations}

NOTE: Using C++ standard template library, it provides many algorithms. 
\begin{Verbatim}
std::find
std::find_if
std::count
std::find
std::binary_search
std::equal_range
std::lower_bound
std::upper_bound
\end{Verbatim} 
\url{http://www.aristeia.com/Papers/CUJ_Dec_2001.pdf}

\section{Operator precedence}
\label{sec:precedence}


\begin{itemize}
   \item \verb!.! is higher than \verb!*!
\begin{verbatim}
*p.f
\end{verbatim}
\begin{minipage}[t]{0.5\textwidth}
EXPECT: (*p).f
\end{minipage} 
\begin{minipage}[t]{0.5\textwidth}
REALITY: *(p.f)

Take the $f$ offset from p, and use it as a pointer
\end{minipage} 
  
  \item \verb![]! is higher than \verb!*!
\begin{verbatim}
int *p[];
\end{verbatim}
\begin{minipage}[t]{0.5\textwidth}
EXPECT: int (*p)[]
\end{minipage} 
\begin{minipage}[t]{0.5\textwidth}
REALITY: int*(p[]);

An array whose elements are pointer to int.
\end{minipage} 


  \item function \verb!()! is higher than \verb!*!
\begin{verbatim}
int *f(int)
\end{verbatim}
\begin{minipage}[t]{0.5\textwidth}
EXPECT: f is a function pointer
\end{minipage} 
\begin{minipage}[t]{0.5\textwidth}
REALITY: f is a function, returning a pointer to int.
\end{minipage} 
   
   \item \verb!==! and \verb.!=. are higher than bitwise operations
\begin{verbatim}
(val&mask != 0)
\end{verbatim}

\begin{minipage}[t]{0.5\textwidth}
EXPECT:\verb!(val & mask) != 0!
\end{minipage} 
\begin{minipage}[t]{0.5\textwidth}
REALITY: \verb!val & (mask != 0)!
\end{minipage} 

   \item \verb!==! and \verb.!=. are higher than assignment operations
\begin{verbatim}
c=getchar() != EOF
\end{verbatim}
\begin{minipage}[t]{0.5\textwidth}
EXPECT: (c=getchar()) != EOF
\end{minipage} 
\begin{minipage}[t]{0.5\textwidth}
REALITY: c = (getchar() != EOF)
\end{minipage} 

   \item arithmetic is higher than shift
\begin{verbatim}
msb<<4 + lsb
\end{verbatim}
\begin{minipage}[t]{0.5\textwidth}
EXPECT: (msb<<4) + lsb
\end{minipage} 
\begin{minipage}[t]{0.5\textwidth}
REALITY: msb << (4+lsb)
\end{minipage} 

  \item comma (,) has lowest precedence of all operators
\begin{verbatim}
i =1,2;
\end{verbatim}
\begin{minipage}[t]{0.5\textwidth}
EXPECT: i = (1,2);
\end{minipage} 
\begin{minipage}[t]{0.5\textwidth}
REALITY: (i=1), 2;
\end{minipage} 

\end{itemize}

% \begin{minipage}[t]{0.5\textwidth}
% EXPECT: (*p).f
% \end{minipage} 
% \begin{minipage}[t]{0.5\textwidth}
% REALITY: *(p.f)
% \end{minipage} 

\section{Comparison}
\label{sec:math_comparison}

\verb!==! is used for comparison, but can be easily mistaken with \verb!=! which
is used for asignment.

POTENTIAL BUG: \verb!sizeof()! returns value of type \verb!size_t! which is an
unsigned int. During a comparision, the compiler does type cast promotion from
int to unsigned int (Sect.\ref{sec:type-cast-C}). So a value \verb!-1! in int
become a very large positive value in \verb!unsigned int!.

In a comparison between a type \verb!int! vs. \verb!unsigned! (e.g.
\verb!size_t!), this is the bug
\begin{verbatim}
int array[] = {2, 3, 4, , 5, 7};
#define TOTAL_ELEMENTS  (sizeof(array) / sizeof(array[0]))

int d = -1, x;

if (d <= TOTAL_ELEMENTS - 2) 
   x = array[d+1];
\end{verbatim}
To resolve this problem, put \verb!int! cast to keep everything in \verb!int!.
\begin{verbatim}
if (d <= (int) TOTAL_ELEMENTS - 2) 
   x = array[d+1];
\end{verbatim}

\section{Absolute value |x|}

NOTE: The function \verb!fabs()! return the absolute value |x| of x.
\begin{verbatim}
//C90
#include <math.h>
double fabs(double x)

//C99 - we have to call different names depending the type of argument
#include <tgmath>      // tg = type-generic
double       fabs
float        fabsf
long double  fabsl

//C++98  : use 'fabs' for all types in C99; but output is of the same
//         type as input
//  support double, float, and 'long double'
e.g. double fabs(double x);

//C++11 - cover C++98
//        also add an overload for integer, of template type 'T'
#include <cmath>
doube fabs(T x); //with all integer types of 'T' return result in double


\end{verbatim}


To use the generic function \verb!std::abs()! which accepts any other type of
input, we include
\begin{verbatim}
//C++98
#include <tgmath.h>
#include <math.h>

//C++11
#include <cmath>
\end{verbatim}

The function \verb!abs(n)! return the absolute value for integral type, i.e.
/n/ with $n$ is an integer type.
\begin{verbatim}
//C
int

//C++98
int
long int  abs()  //equivalent to 'labs(n)'

//C++11
int
long int  abs()  //equivalent to 'labs(n)'
long long int  abs()  //equivalent to 'llabs(n)'
\end{verbatim}


\section{Check divisibility}

Given little-endian architecture, to check if $a$ is divisible by $N$ =
2,3,4,5,\ldots?
\begin{enumerate}
  \item Using modulo (\%) operator \verb!a % N == 0!, if $N$ is a variable, then
  a division is generated, i.e. a calll to GCC EABI function which is about
  20-70 times slower than any logic operator  (ARM9 processor or X86/X64)
  \citep{granlund2012}. If $N$ is a constant, then  the compiler may choose a
  better algorithm itself. 
  
  \item If N is 4,8,16 we can use \verb!(v & N) == 0! test.
  
\end{enumerate}

\section{Minimum/Max/MinMax value}

\subsection{Min}

There is no min function is C. We can write the loop
\begin{verbatim}
small = element[0]
for each element in array, starting from 1 (not 0):
    if (element < small)
        small = element
\end{verbatim}

In C++, the STD has the header \verb!<algorithm>! that can work on containers,
e.g. std::vector, can use 
\begin{itemize}
  \item \verb!std::min_element()!: for more than 2 values
  \item \verb!std::min(val1, val2)!: for 2 values only
\end{itemize}
\url{http://en.cppreference.com/w/cpp/algorithm}

\begin{Verbatim}
#include <algorithm>
#include <iostream>
#include <vector>
 
int main()
{
    std::vector<int> v{3, 1, 4, 1, 5, 9};
 
    std::vector<int>::iterator result = std::min_element(std::begin(v), std::end(v));
    std::cout << "min element at: " << std::distance(std::begin(v), result);
}
\end{Verbatim}

\subsection{Max}

Similarly, there is no max function in C. We need to use the loop, similar to
that above.


In C++, if we use containers, e.g. std::vector, we can use
\verb!std::max_element()! function from \verb!<algorithm>! header file.
\begin{Verbatim}

\end{Verbatim}


\subsection{MinMax}

From C++11, we can use \verb!std::minmax_element()! from \verb!<algorithm>!
header file. The result value is of type \verb!std::pair!, or simply use
\verb!auto! (compiler automatically detect the type for declaration).

\begin{Verbatim}
#include <algorithm>
#include <iostream>
#include <vector>
 
int main()
{
    std::vector<int> v = { 3, 9, 1, 4, 2, 5, 9 };
 
    auto result = std::minmax_element(v.begin(), v.end());
    std::cout << "min element at: " << (result.first - v.begin()) << '\n';
    std::cout << "max element at: " << (result.second - v.begin()) << '\n';
}
\end{Verbatim}


\section{Power}

The simplest thing for $a^b$ with $b$ is a small integer, we can do
\begin{verbatim}
x = a * a * a; //b=3
\end{verbatim}
There  is no such function for \verb!pow(int, int)! before C++11. The reason is
that \verb!int! type can be promoted to any of float, double and can cause ambiguity.
\begin{verbatim}
call of overloaded `pow(int, int)' is ambiguous
\end{verbatim}
Since C++11, we have \verb!double pow(int base, int exponent)!.


\begin{framed}
NOTE: 2.0 is double, 2.0f is float.
\end{framed}

For floating-point power, we use \verb!pow()! in <cmath> library.
\begin{verbatim}
//pow(float, int);
pow(double, int); //C++98
pow(long double, int); //C++98

pow(float, float);  //C++98, C99, C++11
pow(double, double);  //C90, C++98, C99, C++11
pow(long double, long double); //C++98, C99, C++11
double pow(Type1 x, Typ2 y) //C++11 
\end{verbatim}
The additional overload provided by C++11 (Type1,Type2) which effectively cast
its argument to \verb!double! before doing calculation, except at least one of
the argument is \verb!long double!, the both are casted to \verb!long double!.
Other library
\begin{verbatim}
#include <cmath>  //requirement

#include <complex> --> complex pow      
#include <valarray> --> valarray pow   [power on all elements]
\end{verbatim}

We typically combine with \verb!static_cast! to choose the proper function to
use (NOTE: we should avoid C-style cast which is weak, un-checked that can lead
to many bugs)
\begin{lstlisting}
int side1Sqr = pow( static_cast<double>(i) ,i );
int side2Sqr = pow( static_cast<double>(a) ,a );
\end{lstlisting}
Error code in \verb!errno! in C90/C++98
\begin{itemize}
  \item EDOM: if domain error (e.g. $a$ is finite negative and $b$ is finite
  but not an integer value, or both $a,b$ are zero, or $a$ is zero and $b$ is
  negative)
  \item ERANGE: if pole or range error (e.g. $a$ is zero and $b$ is
  negative (depending on library implementation), or the result is too big or
  too small to be represented by a value in the returned type)
\end{itemize}
Since C99/C++11, \verb!math_errhandling! is considered which tell the error
handling mechanism employed by the functions in <cmath> (see references).


For big-precision calculation, we can use GMP library (\url{http://gmplib.org/})


\section{Times}


\subsection{System time}

Return the number of milliseconds has elapsed since the system was started.
\begin{enumerate}
  \item Windows
\begin{verbatim}
#include <Windows.h>

DWORD WINAPI GetTickCount(void);
\end{verbatim}
The time wraps around to zero if the system run continuously for 49.7 days. To
avoid this, use 64-bit version
\verb!GetTickCount64()!\footnote{\url{http://msdn.microsoft.com/en-us/library/ms724408(v=vs.85).aspx}}


  \item Linux
\begin{verbatim}
#include <ctime>

const long double sysTime = time(0);
const long double sysTimeMS = sysTime*1000;
\end{verbatim}

\end{enumerate}

\subsection{Elapsed processor time by the program}

To measure the period of time, in the number of clock-ticks, that processor
has served your program (excluding the time it uses for other processes), then
you can use
\begin{verbatim}
#include <time.h>       /* clock_t, clock, CLOCKS_PER_SEC */


clock_t clock (void);


clock_t t;
  int f;
  t = clock();
  printf ("Calculating...\n");

   //do something
   
  t = clock() - t;
  printf ("It took me %d clicks (%f seconds).\n",t,((float)t)/CLOCKS_PER_SEC);
\end{verbatim}


\subsection{Current date time}

\begin{verbatim}
time();
\end{verbatim}