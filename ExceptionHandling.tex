\chapter{Exception Handling}
\label{chap:Exception_Handling_C-C++}

\section{LBYL style: look before you leap}
\label{sec:LBYL}

Unlike the coding style EAFP in Python, the style LBYL in C requires the
programmers to check/test for pre-conditions before making calls or lookups.

In a multi-threaded environment, the LBYL approach can risk introducing a race
condition between "the looking" and "the leaping".
Example:
\begin{verbatim}
if key in mapping: return mapping[key] 
\end{verbatim}
can fail if another thread removes key
from mapping after the test, but before the lookup.

\section{Exceptions handling (try/catch)}
\label{sec:exception-handling}

An {\bf exception} is an error that is detectable by the code, and need
programmers to provide the adequate code to handle it.
 The code to handle this error is called {\bf exception handler}.

{\bf exception handling} = a mechanism that allows the program to
  save the current state of execution (in a predefined place), switching the
  execution to a given function (known as {\it exception handler}), with the
  exception as the input.

  The aim of an exception handler is to give the program to get a chance to
  fix the error, e.g. hoping to revert the error caused by the error.

\subsection{LISP}

  During the execution of the program, when a function is called, the parent
  function is put into {\it call stack} (execution stack, or run-time stack).
  The call  stack is used to keep track of the point to which each active 
  subroutine should return control when it finishes executing. It may  needs
  several information (e.g. return address, current state of the data..). To
  look for an exception handler, the runtime engine search  back through the
  call stack until the appropriate exception handler  is found. 
  
  For example:
  $f\cee{->[\text{call}]} g \rightarrow h$, $f$ contains a handler $H$ for
  exception E. If during the execution, the exception E occurs in $h$, as the
  handler for E, which is $H$, is in  $f$; then $g$ and $h$ are terminated; and
  $H$ in $f$ will handler  E. Here, the current state of the program is lost,
  and there is no  way to recover the execution by resuming current state of the
   program. 
   
 However, one programming language have this capability,  that is
 {\it Common Lisp} (CLisp) with Condition System. In CLisp,  it doesn't unwind
  the stack, so the current state is kept. So,  depending on the situation, the
  exception handler can resume the execution to the saved state.

% Exception are for exceptional conditions, i.e. the cases that cannot be
% easily handled during normal program execution. This allows the
% program not to get crashed or terminated. 
Exceptions were caught by ERRSET keyword in LISP, which return NIL in the case
of an error in 60s and 70s.
MacLisp introduced 2 keywords \verb!CATCH! and \verb!THROW! in 1972, and
reserving ERRSET and ERR for error handling. The first papers on structured
exception handling were written by Goodenough in 1975.


\subsection{C language}

C does not provide direct support for error handling (exception handling).
Typically, the programmers use the returned value as the indicator of error
code. The programmers need to check the return values from the function (use -1
or NULL values).

In the worst case, i.e. cannot recover from the error, then a C programmer
usually tries to write the error information out (log-file), and then terminate
the program. Some operating systems (e.g. Linux) provide \verb!<errno.h>! which
has the global variable \verb!errno! that we can print out the error description
using \verb!strerror(errno)!.

\begin{verbatim}
#include <include/asm-generic/errno.h>
\end{verbatim}

If an exception occur (divide by zero, interrupt, etc.), the host O/S (Unix,
Linux) of the *NIX processes can raise signal (Sect.\ref{sec:sig_atomic_t}). To
handle this, we use \verb!signal.h! (See example below).

\verb!longjmp! was in C, paired with \verb!setjmp! in 70s. THROW was in LISP in
the 70s (See example below).

Example: \url{http://en.wikibooks.org/wiki/C_Programming/Error_handling}
\begin{Verbatim}
#include <stdio.h>        /* fprintf */
#include <errno.h>        /* errno */
#include <stdlib.h>       /* malloc, free, exit */
#include <string.h>       /* strerror */
 
extern int errno;
 
int main(void)
{
 
    /* Pointer to char, requesting dynamic allocation of 2,000,000,000
     * storage elements (declared as an integer constant of type
     * unsigned long int). (If your system has less than 2 GB of memory
     * available, then this call to malloc will fail.)
     */
    char *ptr = malloc(2000000000UL);
 
    if (ptr == NULL) {
        int err = errno;        /* Preserve the errno from the failed malloc(). */
        puts("malloc failed");  /* This puts() could change the global errno. */
        puts(strerror(err));
    } else {
        /* The rest of the code hereafter can assume that 2,000,000,000
         * chars were successfully allocated... 
         */
        free(ptr);
    }
 
    exit(EXIT_SUCCESS); /* exiting program */
} 
\end{Verbatim}

Example: divide by zero error
\begin{Verbatim}
#include <stdio.h> /* for fprintf and stderr */
#include <stdlib.h> /* for exit */
int main( void )
{
    int dividend = 50;
    int divisor = 0;
    int quotient;
 
    if (divisor == 0) {
        /* Example handling of this error. Writing a message to stderr, and
         * exiting with failure.
         */
        fprintf(stderr, "Division by zero! Aborting...\n");
        exit(EXIT_FAILURE); /* indicate failure.*/
    }
 
    quotient = dividend / divisor;
    exit(EXIT_SUCCESS); /* indicate success.*/
}
\end{Verbatim}

Example: In *NIX process (e.g. Unix, Linux), we can stop the O/S from
terminating your ``divided by zero'' error by blocking SIGFPE signal. However,
you are recommendded not to interfer with it.
\begin{Verbatim}
#include <signal.h>
#include <stdio.h>
#include <stdlib.h>
 
static void catch_function(int signal) {
    puts("Interactive attention signal caught.");
}
 
int main(void) {
    if (signal(SIGINT, catch_function) == SIG_ERR) {
        fputs("An error occurred while setting a signal handler.\n", stderr);
        return EXIT_FAILURE;
    }
    puts("Raising the interactive attention signal.");
    if (raise(SIGINT) != 0) {
        fputs("Error raising the signal.\n", stderr);
        return EXIT_FAILURE;
    }
    puts("Exiting.");
    return 0;
}
\end{Verbatim}


Example: using setjmp
\url{http://en.wikibooks.org/wiki/C_Programming/Error_handling}

\url{http://www.on-time.com/ddj0011.htm}
\begin{Verbatim}
#include <stdio.h>
#include <setjmp.h>
 
jmp_buf test1;
 
void tryjump()
{
    longjmp(test1, 3);
}
 
int main (void)
{
    if (setjmp(test1)==0) {
        printf ("setjmp() returned 0.");
        tryjump();
    } else {
        printf ("setjmp returned from a longjmp function call.");
    }
}
\end{Verbatim}


\subsection{C++ language}

C++ added exception handling with \verb!try/catch! block.

We first put the block of code under exception inspection, i.e. put in the
\verb!try! block. Then, depending on what kind of exceptions occur (via the
\verb!throw! keyword), the control is transferred to the appropriate exception
handler, which is defined in each \verb!catch! block. For any unspecified
exceptions, the default handler is given inside the \verb!cath(...)! statement.

\begin{Verbatim}
try {
  // code here
}
catch (int param) { cout << "int exception"; }
catch (char param) { cout << "char exception"; }
catch (...) { cout << "default exception"; }
\end{Verbatim}


What kind of errors can be throwed?
\begin{enumerate}
  \item \verb!std::logic_error! : 
\end{enumerate}

