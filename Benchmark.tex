\chapter{Benchmark}


\section{Linux shell ``time": whole program}


\begin{lstlisting}
$ time ./a.out
\end{lstlisting}


\section{clock(): <time.h>}
\label{sec:clock()_C}


\verb!clock()! is standard in C language, whichmeasures the CPU time, and the
return a floating number of type \verb!clock_t!, as the number of clocks
ellapsed.
The number of clocks per second is architecture-specific, and can be checked via
\verb!CLOCKS_PER_SEC! constant.

\begin{lstlisting}
#include <time.h>

int main()
{
    clock_t tic = clock();

    my_expensive_function_which_can_spawn_threads();

    clock_t toc = clock();

    printf("Elapsed: %f seconds\n", (double)(toc - tic) / CLOCKS_PER_SEC);

    return 0;
}
\end{lstlisting}


\section{gettimeofday(): <sys/time.h>}	


\section{C++ Boost library: Chrono}

Sect.\ref{sec:Boost.Chrono}


\section{C++11: }