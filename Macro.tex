\chapter{Macro}
\label{chap:macro}


We use \verb!#define! (Sect.\ref{sec:define}),
as discussed in Sect.\ref{sec:macro_C}.

\section{Function-like macro}
\label{sec:function-like-macro}

We put a pair of parentheses immediately after the macro name.

\begin{verbatim}
#define lang_init()  c_init()

then
     lang_init()
          ==> c_init()

but  
     lang_init
         ==> lang_init
\end{verbatim}
So,   function-like macro is only expanded if its name appears with a pair of
parentheses after it. If you write just the name, it is left alone.

\url{https://gcc.gnu.org/onlinedocs/cpp/Function-like-Macros.html}


\section{Macro that can accept any number of arguments}


Sect.\ref{sec:macro_variadic}

\section{Using \# (stringize operator, number-sign) and \#\#}


C89 standard -- that was the standard the codified the \verb!#! and \verb!##!
operators in macros and mandates recursively expanding macros in \verb!args!
before substituting them into the body if and only if the body does not apply a
\verb!#! or \verb!##! to the argument. C99 is unchanged from C89 in this
respect.

\url{http://c-faq.com/ansi/stringize.html}

\subsection{\#}

The stringize operator or number-sign operator \verb!#! is used with a
function-like macro (Sect.\ref{sec:function-like-macro}) and converts a macro
parameters into string literal immediately, without further expanding it.

\begin{verbatim}
#define make_string(x) #x
\end{verbatim}
then the parameter token immediately to the right of the \verb!#! (ignoreing
any white-spaces) is transformed by the preprocessor into a string.
Leading and trailing white space are deleted.

Application:
\begin{verbatim}
#define COMMA ,

make_string(twas brillig);
make_string( twas brillig );
make_string("twas" "brillig");
make_string(twas COMMA brillig);
\end{verbatim}
print out
\begin{verbatim}
"twas brillig"
"twas brillig"  (leading and trailing spaces from the original token are
                 removed)

"twas COMMA brillig"   (if the tokens contain another macro name, it is not
                       expaned)
\end{verbatim}

\url{http://www.complete-concrete-concise.com/programming/c/preprocessor---understanding-the-stringizing-operator}

\url{http://stackoverflow.com/questions/2751870/how-exactly-does-the-double-stringize-trick-work}

 Example:
\begin{verbatim}
#define Str(x) x
#define Xstr(x) Str(x)
#define OP plus
char *opname = Xstr(OP);
\end{verbatim}
becomes
\begin{verbatim}
char *opname = Xstr(plus) = Str(plus) = plus;
\end{verbatim}

But:
\begin{verbatim}
#define Str(x) #x
#define Xstr(x) Str(x)
#define OP plus
char *opname = Xstr(OP);
\end{verbatim}
becomes
\begin{verbatim}
char *opname = Xstr(plus) = Str(plus) = "plus";
\end{verbatim}


\subsection{\#\# (token-pasting operator, token concatenation)}

We use \verb!##! to concatenate two tokens into one token.
Here, the values (rather than the names) of two macros are to be concatenated.

This is a quite common programming trick to add a prefix or suffix to
the all the giving functions. Example
\begin{verbatim}
// deps/libncurses/include/curses.h

/*
 * Extra extension-functions, which pass a SCREEN pointer rather than using
 * a global variable SP.
 */
#if 0
#undef  NCURSES_SP_FUNCS
#define NCURSES_SP_FUNCS 20120922
#define NCURSES_SP_NAME(name) name##_sp


/* Define the sp-funcs helper function */
#define NCURSES_SP_OUTC NCURSES_SP_NAME(NCURSES_OUTC)
typedef int (*NCURSES_SP_OUTC)(SCREEN*, int);

extern NCURSES_EXPORT(SCREEN *) new_prescr (void); /* implemented:SP_FUNC */

extern NCURSES_EXPORT(int) NCURSES_SP_NAME(baudrate) (SCREEN*); /* implemented:SP_FUNC */
extern NCURSES_EXPORT(int) NCURSES_SP_NAME(beep) (SCREEN*); /* implemented:SP_FUNC */
extern NCURSES_EXPORT(bool) NCURSES_SP_NAME(can_change_color) (SCREEN*); /* implemented:SP_FUNC */
extern NCURSES_EXPORT(int) NCURSES_SP_NAME(cbreak) (SCREEN*); /* implemented:SP_FUNC */
extern NCURSES_EXPORT(int) NCURSES_SP_NAME(vidputs) (SCREEN*, chtype, NCURSES_SP_OUTC); /* implemented:SP_FUNC */

#if 1
extern NCURSES_EXPORT(char *) NCURSES_SP_NAME(keybound) (SCREEN*, int, int);	/* implemented:EXT_SP_FUNC */
extern NCURSES_EXPORT(int) NCURSES_SP_NAME(assume_default_colors) (SCREEN*, int, int);	/* implemented:EXT_SP_FUNC */
extern NCURSES_EXPORT(int) NCURSES_SP_NAME(use_legacy_coding) (SCREEN*, int);	/* implemented:EXT_SP_FUNC */
#endif

#else

#undef  NCURSES_SP_FUNCS
#define NCURSES_SP_FUNCS 0
#define NCURSES_SP_NAME(name) name
#define NCURSES_SP_OUTC NCURSES_OUTC

#endif

\end{verbatim}




\begin{verbatim}
#define COMMAND(NAME)  { #NAME, NAME ## _command }

struct command commands[] =
     {
       COMMAND (quit),
       COMMAND (help),
       ...
     };
\end{verbatim}

When a macro is expanded, the two tokens on either side of each \verb!##! operator
are combined into a single token, which then replaces the \verb!##! and the two
original tokens in the macro expansion.

Application: the first \verb!#NAME! is replaced by a string, the
second \verb!NAME ## _command! is replaced by a concatenated value (not a
string).

\begin{verbatim}
struct command commands[] = 
   }
      {"quit", quit_command},
      {"shelp", help_command}
   }
\end{verbatim}
\url{https://gcc.gnu.org/onlinedocs/cpp/Concatenation.html}
 
Example:
\begin{verbatim}
#define IIF(cond) IIF_ ## cond
#define IIF_0(t, f) f
#define IIF_1(t, f) t
\end{verbatim}

\url{https://github.com/pfultz2/Cloak/wiki/C-Preprocessor-tricks,-tips,-and-idioms}
then
\begin{verbatim}
#define A() 1
IIF(A())(true, false) 
\end{verbatim}
becomes
\begin{verbatim}
IFF_A()(true,false)
\end{verbatim}
as \verb!A()! is not expanded to 1 due to the inhibition caused by the \verb!##!
operator.

% \begin{verbatim}
% IFF_1(true,false)  
%   ==> true
% \end{verbatim}
Two-stage replacment is used
\begin{verbatim}
#define CAT(a, ...) PRIMITIVE_CAT(a, __VA_ARGS__)
#define PRIMITIVE_CAT(a, ...) a ## __VA_ARGS__
\end{verbatim}

\section{Define an a lias name}

SYNTAX: \verb!#define <new-name> <existing-name>!

\begin{itemize}
  \item for a value

\begin{verbatim}
#define true 1
#define TRUE 1
\end{verbatim}
then we can use \verb!true! or \verb!TRUE! in the code.

  \item for a type
\begin{verbatim}
#define mybyte byte;
\end{verbatim}

  \item for a function
  
\begin{verbatim}
#define sin(x) __builtin_sin(x)
\end{verbatim}

\end{itemize}

\subsection{Detect O/S}


Sect.\ref{sec:macro-detect-OS}


\section{Detect CPU}

Sect.\ref{sec:macro-detect-CPU} and Sect.\ref{sec:macro-GNU-to-detect-CPU}

\section{Detect compiler}

Sect.\ref{sec:macro-detect-compiler} and Sect.\ref{sec:macro-detect-compiler-version}

