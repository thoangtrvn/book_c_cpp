\chapter{Project structure}

\section{Folder structure}

We should use one version manager, e.g. SVN. In one project, we should have the
following folders
\begin{verbatim}
branches
tools
trunk
vendor 
\end{verbatim}
and one file \verb!00README.TXT! to contains disclaiming information.
\begin{enumerate} 
  \item \verb!branches! should contain side-line development, experimental,
  development lines of multiple version of the same products (before putting in
  the stable release in \verb!trunk!.
  \item \verb!trunk! should contains only the main line of development
  \item \verb!vendor! to contain third-party libraries. When installing we
  should use the PREFIX=../../trunk/base.
\end{enumerate}


\subsection{trunk}

We should have the following folders
\begin{verbatim}
base
bin
doc
src
test
timing
verification
\end{verbatim}
where
\begin{itemize}
  \item \verb!base!: contains the folders of third-party libraries
  \item \verb!bin!: binary files
  \item \verb!doc!: documents (example or code manuals)
  \item \verb!test!: some test cases
  \item \verb!timing!:
  \item \verb!verification!: 
\end{itemize}

\subsection{src}

In the \verb!src! folder, we should have the following folders
\begin{verbatim}
arch
	- linux_laptop.mk
	- osx.mk
	- ...	
\end{verbatim}
and any other codes subfolders (to be discribed later). Important files
\begin{verbatim}
Makefile
Makefile.arch
\end{verbatim}
and other source codes files.

\subsection{Makefile}
\label{sec:Makefile_EP_EM}


We realize that using system name would help better select the proper setting,
rather using information about system architecture of 'uname' or any other
utilities. That's why we define the \verb!arc! folder, with files describing the
compiling configuration using the system name. The \verb!arch! should contains
different configuration setting for different system, e.g.
\verb!./arch/linux_laptop.mk! for the system \verb!LINUX_LAPTOP!. To know which
file is selected, \verb!Makefile! should include the file \verb!Makefile.arch!
first

Example: This is for EM
\begin{verbatim}
CXX=mpic++
CC=mpicc
LD=$(CXX)

#flags for defining macros
DFLAGS += -DWITH_MPI -DADD_ -D$(PLAT) -DUSE_CSTDIO_LFS

#base directory where to look for other third-party packages
BASE_DIR = $(PWD)/../base
# third-party packages
OOMPH_DIR = $(BASE_DIR)/oomph
SCOTCH_DIR = $(BASE_DIR)/scotch

##################
# Installation specific information
##################
# Flags for C pre-processor
EM_CPPFLAG = -DCARDIAC_BOOST -DCARDIAC_OPENMP -DWITH_MPI
AM_CPPFLAG = -DHAVE_CONFIG_H -I. -I$(OOMPH_DIR) -DOOMPH_HAS_MPI \
         -DOOMPH_HAS_TRIANGLE_LIB -DOOMPH_HAS_TETGEN_LIB \
         -DUSING_OOMPH_SUPERLU -DUSING_OOMPH_SUPERLU_DIST \
         -I$(OOMPH_DIR)/build/include -I$(SCOTCH_DIR)/build/include \
         -DLINUX_LAPTOP -I../base $(EM_CPPFLAGS)
         
# The solver is coded based on omph-lib         
# Include files and libraries for third_party
OOMPH-LIB_INCLUDE_DIR=$(OOMPH_DIR)/build/include
OOMPH-LIB_LIB_DIR=$(OOMPH_DIR)/build/lib
OOMPH-LIB_EXTERNAL_LIBS= -lHYPRE -lptscotch -lptscotcherr -loomph_hsl \
	-loomph_arpack -loomph_triangle -loomph_tetgen -loomph_superlu_3.0
	-loomph_parmetis_3.1.1 -loomph_superlu_dist_2.3 \
	-loomph_metis_from_parmetis_3.1.1 -lblas -llapack

EXTERNAL_DIST_LIBRARIES=-L$(BASE_DIR)/hypre/build/lib \
 	-L$(BASE_DIR)/scotch/build/lib
EXTERNAL_DIST_INCLUDE_DIR=$(BASE_DIR)/hypre/build/include

#Additional machine specific linking information that
# allows mixed-language compilation/linking
# e.g. Here is for RedHat Linux
FLIBS=-L/usr/lib/gcc/x86_64-redhat-linux/4.4.6 \
	-L/usr/lib64 -L/usr/lib/ -L/lib/ -L/lib64 \
	-lm -lgcc_s -L/reader_files
	
# Flags required for the shared libraries
SHARED_LIBRARY_PATH=-Wl, --rpath -Wl,$(OOMPH_DIR)/build/lib

CXXFLAGS_OPT = -g2 -O2 $(AM_CPPFLAGS) -I$(OOMPH-LIB_INCLUDE_DIR) \
	-I(EXTERNAL_DIST_INCLUDE_DIR)
CXXFLAGS_DEBUG = -g -O0 $(AM_CPPFLAGS) -I$(OOMPH-LIB_INCLUDE_DIR) \
	-I(EXTERNAL_DIST_INCLUDE_DIR)
\end{verbatim}
Depending on which system, we may need different APIs; what's why we need to
define a macro using the name of the system, e.g. \verb!LINUX_LAPTOP!. So, in
the code, if the API has different version, and to choose which one depending
on the system, we use 
\begin{verbatim}
#ifdef LINUX_LAPTOP
    call this_API
#else 
    call that_API
#endif
\end{verbatim}

The \verb!Makefile.arch! tells us the information about the \verb!ARCHGUESS!
macro, which is the name of the \verb!mk! file in the \verb!./arch! folder.
\begin{verbatim}
## Makefile.arch

#guess hostname with hostname command, stripping off all numbers
HOSTNAME := $(shell hostname -s | sed 's/[0-9]*//g')

#depending on the hostname, we define ARCHGUESS
ifeq ($(HOSTNAME), thunder)
  ARCHGUESS = thunder
endif
ifeq ($HOSTNAME), titan)
  ARCHGUESS = titan
 endif
 
# or we just simply override it
ARCHGUESS = linux_laptop

#Finally, if no result
ifndef ARCHGUESS
	$(error Unrecognized build host. Edit Makefile.arch)
endif
\end{verbatim}
the value of \verb!ARCHGUESS! will be used in \verb!Makefile! file as
\verb!ARCH! macro.

So \verb!Makefile! file
\begin{verbatim}
# Get name of architecture-specific incldue file 
SHELL = /bin/bash

# Define what are target-command
.PHONY: DEFAULT COMMON_TARGETS makedirs clean distclean depend

include Makefile.arch

ARCH ?= $(ARCHGUESS)

include arch/$(ARCH).mk

#keep version number of SVN
SVNVERSION := $(shell svnversion ./ )

#where to save binary (compiled file)
BINDIR = ../bin

HEARTSRC = \
	Cardioid.cc \
	... \  [here we list all source files]
	... \  [.cc = C++ source]

SPISRC= spi_impl.c
SIMDSRC= clooper.c TT06GatesSimd.c TT06Gates.c TT06NonGates.c \
	TT06NonGatesSimd.c workBound.c svd.c

DDCMD_FILES = \
	codata.h \
	ddcMalloc.c \
	ddcMalloc.h \
	... \
	...
	
DDCMDSRC = $(filter %.c, $(DDCMD_FILES))

HEARTOBJS = $(HEARTSRC:%.cc=$(OBJDIR)
\end{verbatim}


\verb!objs-$ARCH_FILENAME!

Example: This is for EP
{\small \begin{verbatim}
# Get name of architecture-specific include file from hostname
SHELL = /bin/bash

.PHONY: DEFAULT COMMON_TARGETS makedirs clean distclean depend

include Makefile.arch
# If defined, use architecture file set with ARCH variable
ARCH ?= $(ARCHGUESS)
include arch/$(ARCH).mk
\end{verbatim}}
which needs to read the compiling options from the architecture file, e.g. in
the case of system called \verb!peloton! with Intel microprocesors, we use
\verb!arch/peloton.mk!
\begin{verbatim}

CXX=mpic++ -fopenmp
CC=mpicc --std=gnu99 
LD=$(CXX)

DFLAGS = -DPELOTON -DWITH_PIO -DWITH_MPI \
    -DADD_ -DUSE_CSTDIO_LFS -D_LARGEFILE_SOURCE -D_FILE_OFFSET_BITS=64

INCLUDE =

CFLAGS_BASE = $(INCLUDE) $(DFLAGS)
CXXFLAGS_BASE = $(INCLUDE) $(DFLAGS)


CFLAGS_OPT =   $(CFLAGS_BASE) -g -O3
CFLAGS_DEBUG = $(CFLAGS_BASE) -g -ggdb -O0 -fno-inline
CFLAGS_PROF =  $(CFLAGS_BASE) -g -pg -O3 -DPROFILE

CXXFLAGS_OPT =   $(CXXFLAGS_BASE) -g -O3
CXXFLAGS_DEBUG = $(CXXFLAGS_BASE) -g -ggdb -O0 -fno-inline
CXXFLAGS_PROF =  $(CXXFLAGS_BASE) -g -pg -O3 -DPROFILE

LDFLAGS_OPT   = $(LDFLAGS_BASE) $(CFLAGS_OPT) $(CXXFLAGS_OPT)
LDFLAGS_DEBUG = $(LDFLAGS_BASE) $(CFLAGS_DEBUG) $(CXXFLAGS_DEBUG)
LDFLAGS_PROF  = $(LDFLAGS_BASE) $(CFLAGS_PROF) $(CXXFLAGS_PROF)
\end{verbatim} 

Now, the next content in the Makefile
 {\small \begin{verbatim}
SVNVERSION := $(shell svnversion ./ )

#------------------------------------------------------------------------------
#

BINDIR = ../bin

#NO main file is put here
HEARTSRC= \
    ThreadServer.cc \
    ThreadUtils.cc \
    stringUtils.cc \
    readCellList.cc \
    PioHeaderData.cc \
    ...

SPISRC=  spi_impl.c
SIMDSRC=  clooper.c TT06GatesSimd.c TT06Gates.c TT06NonGates.c TT06NonGatesSimd.c workBound.c svd.c

DDCMD_FILES = \
    codata.h \
    ddcMalloc.c \
    ddcMalloc.h \
    ddcMath.h \
    error.c \
    error.h \
    external.h \
    ...
DDCMDSRC = $(filter %.c, $(DDCMD_FILES)

\end{verbatim} }
All the source files of EP is \verb!HEARTSRC! macro, except the main file
(cardioid.cc). Now we add the main file
\begin{verbatim}
HEARTOBJS = $(HEARTSRC:%.cc=$(OBJDIR)/%.o)
HEARTOBJS += $(DDCMDSRC:%.c=$(OBJDIR)/%.o)
HEARTOBJS += $(SPISRC:%.c=$(OBJDIR)/%.o)
HEARTOBJS += $(SIMDSRC:%.c=$(OBJDIR)/%.o)

CARDIOIDSRC = $(HEARTSRC) cardioid.cc
CARDIOIDOBJS = $(CARDIOIDSRC:%.cc=$(OBJDIR)/%.o)
CARDIOIDOBJS += $(DDCMDSRC:%.c=$(OBJDIR)/%.o)
CARDIOIDOBJS += $(SPISRC:%.c=$(OBJDIR)/%.o)
CARDIOIDOBJS += $(SIMDSRC:%.c=$(OBJDIR)/%.o)
\end{verbatim}
As we can build for different architectures, we need to put the object files in
different folder, then we add the architecture as the prefix to the folder for
object files
\begin{verbatim}
OBJDIR := objs-$(ARCH)
OBJDIR_PRECOMPILED := objs_precompiled-$(ARCH)
\end{verbatim}
The executable file
\begin{verbatim}
EXENAME = cardioid
\end{verbatim}
Compiling options 
\begin{verbatim}
DFLAGS += -DDiff_Weight_Type_Single

DEFAULT: opt

COMMON_TARGETS: $(DDCMD_FILES) makedirs
\end{verbatim}
There are different build targets (default: opt = optimized binary). Other
options: \verb!debug!, \verb!profile! (generate profile to use with gprof),
\verb!clean! or \verb!distclean!. For ddcMD developers, i.e. those who want to
chage the I/O files, use \verb!ddcMD_dist! which compress the files into tar.gz
file for distribution. NOTE: \textcolor{red}{For each build target, a different
suffix is added to make a different binary files}
\begin{verbatim}
opt:
    @$(MAKE) --no-print-directory COMMON_TARGETS \
    $(BINDIR)/cardioid-$(ARCH) \
    BUILD_SUFFIX=$(ARCH) OBJDIR=objs-$(ARCH) \
    CFLAGS="$(CFLAGS_OPT)" CXXFLAGS="$(CXXFLAGS_OPT)" \
    LDFLAGS="$(LDFLAGS_OPT)"  EXENAME=cardioid-$(ARCH)
\end{verbatim}
Then the directives to compile the source codes (C and C++)
\begin{verbatim}
$(OBJDIR)/%.o: %.cc
    $(CXX) $(CXXFLAGS) -o $@ -c $<

$(OBJDIR)/%.o: %.c
    $(CC) $(CFLAGS) -o $@ -c $<
\end{verbatim}
There are other targets we may want to know for testing
\begin{verbatim}
testLoadBalancer:
    @$(MAKE) --no-print-directory makedirs testLoadBalancer-target \
    BUILD_SUFFIX=$(ARCH) OBJDIR=objs-$(ARCH) \
    CFLAGS="$(CFLAGS_OPT)" CXXFLAGS="$(CXXFLAGS_OPT)" \
    LDFLAGS="$(LDFLAGS_OPT)"  EXENAME=testLoadBalancer-$(ARCH)

testLoadBalancer-target: $(CARDIOIDOBJS) $(OBJDIR)/testLoadBalancer.o
    $(LD) $(DFLAGS) -o $(BINDIR)/$(EXENAME) $(OBJDIR)/testLoadBalancer.o $(CARDIOIDOBJS) $(LDFLAGS)

testGridRouter: makedirs $(CARDIOIDOBJS) $(OBJDIR)/testGridRouter.o
    $(LD) $(DFLAGS) -o testGridRouter $(OBJDIR)/testGridRouter.o $(CARDIOIDOBJS) $(LDFLAGS)

compareSnapshots: 
    @$(MAKE) --no-print-directory makedirs compareSnapshots-target \
    BUILD_SUFFIX=$(ARCH) OBJDIR=objs-$(ARCH) \
    CFLAGS="$(CFLAGS_OPT)" CXXFLAGS="$(CXXFLAGS_OPT)" \
    LDFLAGS="$(LDFLAGS_OPT)"  EXENAME=compareSnapshots-$(ARCH)

compareSnapshots-target: $(HEARTOBJS) $(OBJDIR)/compareSnapshots.o Version.cc
    $(LD) Version.cc $(DFLAGS) -o $(BINDIR)/$(EXENAME) $(HEARTOBJS) $(OBJDIR)/compareSnapshots.o $(LDFLAGS)
 
\end{verbatim}
and for profiling as well as performance analysis
\begin{verbatim}
profile:
    @$(MAKE) --no-print-directory COMMON_TARGETS \
    $(BINDIR)/cardioid-$(ARCH)-prof \
    BUILD_SUFFIX=$(ARCH)-prof OBJDIR=objs-$(ARCH)-prof \
    CFLAGS="$(CFLAGS_PROF)" CXXFLAGS="$(CXXFLAGS_PROF)" \
    LDFLAGS="$(LDFLAGS_PROF)"  EXENAME=cardioid-$(ARCH)-prof
mpip:
    @$(MAKE) --no-print-directory COMMON_TARGETS \
    $(BINDIR)/cardioid-$(ARCH)-mpip \
    BUILD_SUFFIX=$(ARCH)-mpip OBJDIR=objs-$(ARCH)-mpip \
    CFLAGS="$(CFLAGS_MPIP)" CXXFLAGS="$(CXXFLAGS_MPIP)" \
    LDFLAGS="$(LDFLAGS_MPIP)"  EXENAME=cardioid-$(ARCH)-mpip
flops:
    @$(MAKE) --no-print-directory COMMON_TARGETS \
    $(BINDIR)/cardioid-$(ARCH)-hpm \
    BUILD_SUFFIX=$(ARCH) OBJDIR=objs-$(ARCH)-hpm \
    CFLAGS="$(CFLAGS_OPT) -DNDEBUG -DNTIMING -DHPM" CXXFLAGS="$(CXXFLAGS_OPT) -DNDEBUG -DNTIMING -DHPM" \
    LDFLAGS="-L/usr/local/tools/mpitrace/lib -lmpihpm_smp $(LDFLAGS_OPT)"  EXENAME=cardioid-$(ARCH)-hpm 
oss:
    @$(MAKE) --no-print-directory COMMON_TARGETS \
    $(BINDIR)/cardioid-$(ARCH)-oss \
    BUILD_SUFFIX=$(ARCH) OBJDIR=objs-$(ARCH) \
    CFLAGS="$(CFLAGS_OPT)" CXXFLAGS="$(CXXFLAGS_OPT)" \
    LDFLAGS="$(LDFLAGS_OPT)"  EXENAME=cardioid-$(ARCH)-oss \
   LD="/usr/global/tools/openspeedshop/oss-dev/bgq/oss202/bgq/bin/osslink -c pcsamp $(CXX)"
\end{verbatim}


\section{EP}
\label{sec:EP}

EP is the project for doing electrophysiology, the main branch of Cardioid
project.  IMPORTANT: EP doesn't use oomph-lib nor hypre packages. The solver may
be self-contained in the code. It's finite difference method with explicit
operator-splitting method for numerical solving. The mesh is organized into
a grid of cubes.

% {\small \begin{verbatim}
% int main(int argc, char** argv)
% {
%    int npes, mype;
%    MPI_Init(&argc,&argv);
%    MPI_Comm_size(MPI_COMM_WORLD, &npes);
%    MPI_Comm_rank(MPI_COMM_WORLD, &mype);  
% 
%    // See units above.
%    units_internal(1e-3, 1e-9, 1e-3, 1e-3, 1, 1e-9, 1); 
%    units_external(1e-3, 1e-9, 1e-3, 1e-3, 1, 1e-9, 1); 
%    
%    profileInit();
%    profileStart("Total");
%    // heap_start moved to initializeSimulate so that the size can be set
%    // in the input deck.
% //   heap_start(500);
% 
%    if (mype == 0)
%      printBanner();
%    parseCommandLineAndReadInputFile(argc, argv, MPI_COMM_WORLD);
%    
%    timestampBarrier("Starting initializeSimulate", MPI_COMM_WORLD);
%    Simulate sim;
%    initializeSimulate("simulate", sim);
%    timestampBarrier("Finished initializeSimulate", MPI_COMM_WORLD);
% 
% #ifdef HPM
%   HPM_Start("Loop");
% #endif
%    timestampBarrier("Starting Simulation Loop", MPI_COMM_WORLD);
%    profileStart_HW("Loop");
%    switch (sim.loopType_)
%    {
%      case Simulate::omp:
%       simulationLoop(sim);
%       break;
%      case Simulate::pdr:
%      // printf("Cardioid pdr ptr=%p %p %p\n",sim.diffusion_->blockIndex(),sim.diffusion_->dVmBlock(),sim.diffusion_->VmBlock());  fflush(stdout); 
%       simulationLoopParallelDiffusionReaction(sim);
%       break;
%      default:
%       assert(false);
%    }
%    profileStop_HW("Loop");
%    timestampBarrier("Finished Simulation Loop", MPI_COMM_WORLD);
% #ifdef HPM
%   HPM_Stop("Loop");
% #endif
% 
%    profileStop("Total");
%    profileSetRefTimer("00:Loop");
% 
%    profileSetPrintOrder("Total");
%    profileSetPrintOrder("Assignment");
%    profileSetPrintOrder("Loop");
%    profileSetPrintOrder("parallelDiffReac");
%    profileSetPrintOrder("DiffusionLoop");
%    profileSetPrintOrder("ReactionLoop");
%    profileSetPrintOrder("Dummy");
%    profileSetPrintOrder("HaloExchMove2Buf");
%    profileSetPrintOrder("Integrator");
%    profileSetPrintOrder("ReactionWait");
%    profileSetPrintOrder("reactionL2Arrive");
%    profileSetPrintOrder("reactionL2Rest");
%    profileSetPrintOrder("Reaction");
%    profileSetPrintOrder("Reaction_nonGate");
%    profileSetPrintOrder("GateNonGateBarrier");
%    profileSetPrintOrder("Reaction_Gate");
% 
%    //profileSetPrintOrder("");
%    if (mype == 0)
%    {
%       profileDumpTimes(cout);
%       cout << "\n" << endl;
%    }
%    profileDumpStats(cout);
%    stringstream dirname;
%    dirname << "snapshot."<<setfill('0')<<setw(12)<<sim.loop_;
%    profileDumpAll(dirname.str());
%    heap_deallocate();
%    MPI_Finalize();
% 
%    return 0;
% }
% \end{verbatim}}

NOTE: The main() file in \verb!cardioid.cc!, after removing not important parts,
e.g. \verb!profile...! (to do performance test
Sect.\ref{sec:EP_performanceTest}).
\begin{enumerate}
  \item The program accepts maximum one argument, which is the name of the input
  file (by default: \verb!object.data!), and read in, if presence,
  \verb!restart! file as well using \verb!parseCommandLineAndReadInputFile()!
  function. 
  \item Read-in input files (Sect.\ref{sec:EP_input})
  \item Initialize the simulation (Sect.\ref{sec:EP_initialize})
  \item Run the loop (Sect.\ref{sec:EP_loop})
\end{enumerate}

{\small \begin{verbatim}
int main(int argc, char** argv)
{
   int npes, mype;
   MPI_Init(&argc,&argv);
   MPI_Comm_size(MPI_COMM_WORLD, &npes);
   MPI_Comm_rank(MPI_COMM_WORLD, &mype);  

   // See units above.
   units_internal(1e-3, 1e-9, 1e-3, 1e-3, 1, 1e-9, 1); 
   units_external(1e-3, 1e-9, 1e-3, 1e-3, 1, 1e-9, 1); 
   
//read-in input (object.data, restart)
   parseCommandLineAndReadInputFile(argc, argv, MPI_COMM_WORLD);
   
//initialize simulation   
   Simulate sim;
   initializeSimulate("simulate", sim);

//the loop of the simulation
   switch (sim.loopType_)
   {
     case Simulate::omp:
      simulationLoop(sim);
      break;
     case Simulate::pdr:
      simulationLoopParallelDiffusionReaction(sim);
      break;
     default:
      assert(false);
   }

//free memory
   heap_deallocate(); //ddcMD_files/src/heap.c
   MPI_Finalize();

   return 0;
}
\end{verbatim}}


\subsection{Read-in Input}
\label{sec:EP_input}

Depending on the settings, it can read-in different types of files. However, the
main one that it must read is the \verb!object.data! and, if presence,
\verb!restart! file.

Input file (\verb!object.data!): The input file \verb!object.data! has
  the format as described in Sect.\ref{sec:EM}. Here, the primary or root object
  is the SIMULATE object. There can be a \verb!restart! file that, if present,
  is read after the \verb!object.data! file. The \verb!restart! file is not a
  regular file, but a symbolic link (created by the command \verb!ln -s!) that
  point to the actual files somewhere else, e.g. the location of the checkpoint
  file, which is automatically updated when a new snapshort directory is
  created. So, we can easily choose which checkpoint file to restart the
  simulation by changing this symbolic link. This to guarantee that no actual
  data will be deleted, only pointers to the data.
\begin{verbatim}
object.data 
  // the main file tells the program what to do in 
  // the SIMULATE object 
  
\end{verbatim}

  \verb!ddcMD_files!: these are the files that read object.data file and
  parse it, as well as do parallel io (PIO). In EM code, the files are moved
  into \verb!./io! folder. The list of all I/O files are put in the macro of the
  Makefile \verb!DDCMD_FILES! (both header and source codes). Thus, to extract
  source code we use
  \begin{verbatim}
  DDCMDSRC = $(filter %.c, $(DDCMD_FILES)
  \end{verbatim}
  
  The files were extracted from ddcMD (a domain decomposition molecular-dynamics
  software package). The package is the first to run atomic-scale model of metal
  solidfication from the liquid phase with results that were {\it independent of
  system size}. The code runs on BlugeGene/L, BlueGene/P supercomputers. To
  evaluate the performance of nuclear weapon systems, scientists must understand
  how materials behave under extreme conditions (high pressures, temperatures),
  at which experiments are often difficult or impossible to conduct.
  Computational models to be used are those validated with experimental data
  \citep{streitz2006}.
  
In order to run the simulation, it may need to read-in the anatomy structure,
which is specified by the ANATOMY struture and \verb!anatomy! keywrod in the
SIMULATE structure. By default, the filenames are 

\begin{verbatim}
snapshot.initial/anatomy#....  
  //each file contain the anatomy information
  // each line has
  //
  // global_cellID cellType  [x1 .. x8] 0
  //
  // with global_cellID is the grid-point 1D-index in the
  // finite-difference mesh telling which grid-point contains the tissue, 
  // and also the cell-type for that tissue (100 (Epi),101 (M-cell),102 (Endo))
   
\end{verbatim}



{\small 
\begin{verbatim}
void usage_string_output()
{
  std::cout << "Usage: exec -t N [-b N2]" << std::endl;
  std::cout << "N: 1 - fiber problem, 2 - ecg wedge problem, 3 - ecg torso problem" << std::endl;
  std::cout << "4 - mechanics with pressure constraint, 5 - mechanics with volume constraint" << std::endl << "6 - mechanics test with bar" << std::endl;
  std::cout << "7 - infarct" << std::endl;
  std::cout << "8 - ecg wedge problem(parallel version), 9 - ecg torso problem(parallel version) " << std::endl;
  std::cout << "10 - phase_singularities, 11 - mixed bar" << std::endl;
  std::cout << "Default N = 5" << std::endl;
  std::cout << "-b is the option for the bar test, N2: 0 - pressure, 1 - rotation, 2 - contraction" << std::endl << "3 - body force"
            << std::endl;
}
\end{verbatim}
}

\begin{verbatim}
parseCommandLineAndReadInput(argc, argv, MPI_COMM_WORLD);

\end{verbatim}

\subsection{Initialize}
\label{sec:EP_initialize}

Everything is saved in the object of  \verb!Simulate! class.
\begin{verbatim}
//initialize simulation   
   Simulate sim;
   initializeSimulate("simulate", sim);
\end{verbatim}

NOTE: Currently, the information is read-in into a global static variable
\begin{verbatim}
static int nobject=0, mobject=0;
static niobject=0, miobject=0;
static OBJECT* object=NULL;
static OBJECT **object_list=NULL:
\end{verbatim}
GOAL: To remove this and put into regular class.


TRICK: The program use a static heap data
\begin{verbatim}
static void* heapBuffer=NULL;
\end{verbatim}
We specify the amount of heap memory (in bytes) to use (default 500 bytes).

The heap memory is supposed to be quad-aligned, i.e. the heap size should be a
multiple of 16.
\begin{verbatim}
static const int alignOffset = 16;

#define QUAD_WORD_ALIGN_HEAP
  _heapCapcity = n;
#if QUAD_WORD_ALIGN_HEAP
  _heapCapacity = (alignOffset - (_heapCapacity % alignOffset)) % alignOffset;
  ddcMallocAligned((void*) &heapBuffer, alignOffset, _heapCapacity);
#else
  _heapCapcity = ddcMalloc(_heapCapcity);
#endif
\end{verbatim}

Here, we also interpret data read-in from object.data file, the
\verb!object_cc.cc! file wraps the handling of read-in data written in
\verb!object.cc! to free user from data allocation/free management.
\begin{enumerate}
  \item At first, we need to detect the object structure, using the given
  classname, e.g. SIMULATE
\begin{verbatim}
OBJECT * obj = objectFind(name, "SIMULATE");
\end{verbatim}
  \item then we can check for the keyword=value[s]
  
The function
\begin{verbatim}
objectGet(obj, 'keyword-name', output-value, default-value, 
\end{verbatim}
can accepts [output-value, default-value] as
\begin{itemize}
  \item [string, string]
  \item [double, string]
  \item [bool, string]
  \item [int, string]
  \item [unsigned, string]
  \item [uint64\_t, string]
  \item [int64\_t, string]
  \item [string, vector<string>]
  \item [string, vector<int>]
  \item [string, vector<unsigned>]
  \item [string, vector<uint64\_t>]
  \item [string, vector<double>]
\end{itemize}

For the case [double, string], it accepts one more parameter unitConvertTo 
\begin{verbatim}
   doble& value,
   const string &defVal,
   const string & unitConvertTo
\end{verbatim}

\end{enumerate}

where the original function use
\begin{verbatim}
object(obj, 'keyword-name', output-value, type, 
\end{verbatim}
with \verb!type! can be \verb!WITH_UNITS!

\subsection{Output}

Output: The data are captured using PIO system in the form of  snapshots,
written in different directories, named \verb!snapshot.xxxxxxxx! (8-digits  the
simulation loop counter). Each folder contains the different checkpoint  files,
corresponding to different kinds of data you want to output.   
  


\subsection{Loop}
\label{sec:EP_loop}

\subsection{Performance Test}
\label{sec:EP_performanceTest}

It uses 2 pairs of files: PerformanceTimers.cc/hh and
\verb!PerformanceTimersBGQ.cc/hh!
\begin{verbatim}
PerformanceTimers.cc:void  profileInit()
PerformanceTimers.hh:void profileInit();
PerformanceTimersBGQ.hh:void profileInitBGQ(void)
PerformanceTimersBGQ.hh:void (*machineSpecficInit)() = profileInitBGQ; 
\end{verbatim}


\section{EM}	
\label{sec:EM}

EM is the project for solving a number of problems. The type of problems are
defined in the \verb!Params.cc/h! files.
 
IMPORTANT MACROS: used in compliling the code
\begin{verbatim}
-DWITH_MPI  : to tell if MPI is enabled
\end{verbatim}

\begin{enumerate}
  \item \verb!io!: 
  \begin{enumerate}
    \item  code for handling I/O in parallel (PIO), see
  \verb!ddcMD_files! in EP code; and 
    \item code for parsing input data in \verb!object.data!
  file, or \verb!restart! file (Sect.\ref{sec:object.data}). 
  \end{enumerate}
PLAN: switch to using Boost.Spirit lexer.  
  
  \item \verb!mixed!: contain code using mixed solver
  
\end{enumerate}

\subsection{Read-in Input}

The expected command-line for running the program.
{\small \begin{verbatim} 
Usage: your_excutable [input file]
\end{verbatim}}

The main object
\begin{verbatim}
problem PROBLEM {
	fiberProblem = fiber;
	//ecgProblem = ecg;
	//mechanicsProblem = mechanics;
	//infarctProblem = infarct;
	//psProblem = ps; //ps = phase singularities
\end{verbatim}
With
\begin{enumerate}
  \item fiberProblem (Sect.\ref{sec:fiber_generation})
\end{enumerate}

The subroutine is \verb!Params::read_input_file()! from \verb!Params.cc! file.
The main.cc file
\begin{verbatim}
int main (int argc, char **argv) {
   //oomph::CommandLineArgs is the namespace
   // setup(int argc, char **argv)
   oomph::CommandLineArgs::setup(argc, argv);
	
   MPI_Init (&argc, &argv);

   Params myParam;
   //each MPI task need to read in the same data
   myParam.read_input_file(argc, argv, MPI_COMM_WORLD);
   ....
}
\end{verbatim}


At first, it check how many
files passed to the program and read all of them.

By default, if there is no argument, at least one file must be used as the input
which is \verb!object.data! by default. We need to define 
\begin{verbatim}
Params.h / Params.cc
ProblemParams.h / ProblemParams.cc
\end{verbatim}
they are put in the namespace \verb!cardiac_mechanics!.
 {\small
\begin{verbatim}
#ifndef _PARAMS_H
#define _PARAMS_H

#include <string>

#ifdef WITH_MPI
#include <mpi.h>
#endif

namespace cardiac_mechanics
{
  class Params {
  public:
    enum ProblemType {FIBER = 0, ECG=1, MECHANICS=2, 
           INFARCT=3, PHASE_SINGULARITIES=4};
    Params(void);
    void read_input_file(int argc, char** argv, MPI_Comm comm);
    int get_problem_type(std::string &keyword, std::string &value);
  private:
    //a list of valid keywords in the PROBLEM class
      static const char* const problemClass[];
      ProblemType ptype_;
  }; 
}
   
#endif
\end{verbatim}
}

Here, we define different problem we need to solve: FIBER, ECG, MECHANICS,
INFARCT, etc. We can add more simulation type, for each type, we need to
specify a proper configuration in the object.data file
{\small
\begin{verbatim}
#include ``Params.h''
namespace cardiac_mechanics
{
  const char* const Params::problemClass[] = {``fiberProblem'',
  ``ecgProblem'', ``mechanicsProblem'', ``infarctProblem'', ``psProblem''} }
\end{verbatim} 
}
NOTE: The order of what we define here must match order of \verb!ProblemType!
enum. \textcolor{red}{What each problem does?}
  

The input to the simulation is always the \verb!object.data! file. To parse the
data we use codes in \verb!io! folder: \verb!objects.c/h! which define
functions like \verb!object_set()!, \verb!object_compile()!.

{\small
\begin{verbatim}
int Params::read_input_file(argc, argv, MPI_Comm comm)
{
  int myRank;
  std::string input_file(``object.data'');
  
  MPI_Comm_rank(comm, &myrank);
  if (argc !=2 && argc != 1){
    //to make sure only printing out on the root process
    //we check with myRank == 0
    if (myRank == 0){
      std::cout << ``Usage: your_executable [input-file] << std::endl;
    }
  }
  if (argc == 2) then {
     std::string argFile(argv[2]);
     input_file = argFile;
  }
  
  //parse input file 
  object_set(``files'', input_file.c_str());
  object_compile();
}
\end{verbatim} } 
  

 Then, depending on the information we parse from the \verb!object.data! files,
 i.e. the problem type, we choose a proper system setting through the
 \verb!ProblemParam! class.
\begin{verbatim}
 ### main()
 
 std::string obj_name;
 std::string obj_class;
 int ptype = myParams.get_problem_type(obj_class, obj_name);
 
  ProblemParams pinput = ProblemParams(obj_class, obj_name);
\end{verbatim}
  

The problem-type is read in
  {\small \begin{verbatim}
  ##main.cc
  std::string obj_name;
  std::string obj_class;
  int ptype = myParams.get_problem_type(obj_class, obj_name)
  
  int Params::
  get_problem_type(std::string &keyword, std::string &value)
  {
    if (!object_exists(``problem'', ``PROBLEM'')){
       std::cout << ``Cannot find PROBLEM block \n'';
       exit(-1);
    }
    OBJECT*obj = object_find(``problem'', ``PROBLEM'')
    
  }
  \end{verbatim} }
  
\section{Parsing object.data file}
\label{sec:object.data}
  
\subsection{Data format}

An example of the \verb!object.data! file
  {\small
  \begin{verbatim}
// An example of object.data
  
problem PROBLEM{
  mechanicsProblem = mechanics;
}
  
mechanics MECHANICSPROBLEM{
  type = barTest; //volume, barTest 
  mixed = 1; // choose solver:  0=SolidLinearSystem; 1=MixedLinearSystem 
  barTestType = 1; 
     //for barTest &  barMixedTest: choose either
     // pressure(0), rotation(1), contraction(2) or body force(3)
}
\end{verbatim} 
} 
This file follow C-style and C++-style comments. All identifiers are 
case-sensitive. 
  
The file is a number of 'object' blocks which is in the form
{\small \begin{verbatim}
ObjectName CLASSNAME{
  keyword1 = value1;
  keyword2 = value2;
  keyword3 = value3.1 value3.2 value3.3;
  ...
}
\end{verbatim}}
By convention, the object name are mixed-case, while the class-name are ALL
CAPS. 

Information contained in a block is a number of lines. Each line completes
with a semi-colon (;) and is in the form
\begin{verbatim}
keyword=value1 [ value2 [ value3]]; 
\end{verbatim}
So, a value here can be a string, float, integer, or a vector of scalar values.
The name set of keywords are open and arbitrary, i.e. programmers need to take
care how to interpret them. Unit can also be following the value
\begin{verbatim}
diffusionScale = 714 mm^3/mF;
\end{verbatim}

The order of these 'object' blocks are not important. If the two 'object' blocks
with the same name and Classname, then the keyword value lists are concatenated. If a
single keyword \verb!keywordX! appears multiple times in one block, the last
\verb!keywordX=valueY! is accepted.

In C code, to keep these information, we define OBJECT struct to keep one
'object' block
\begin{verbatim}
typedef struct object_st {
  char *name;
  char *objclass;
  char *value;
  char *valueptr;
  struct object_st *parent;
} OBJECT;
  \end{verbatim}
  
Eventually, all the read-in data are stored in a global static variable
\begin{verbatim}
static OBJECT *object = NULL;
static OBJECT **object_list = NULL;

static int nobject = 0, mobject = 0;
static int niobject= 0, miobject = 0;
\end{verbatim}

\subsection{Utilities}

Here is the code \verb!object.c/h!
  
{\small \begin{verbatim} 
#ifndef _OBJECT_H
#define _OBJECT_H

#include <stdio.h>

#ifdef __cplusplus
extern ``C''
{
#endif

void object_set(const char *get, ...)


#ifdef __cplusplus
}
#endif
\end{verbatim} }

There can be one or more files containing the structure like \verb!object.data!,
e.g. \verb!restart! file. The list of these files is read in by
\verb!object_set()!. \verb!object_set()! reads in ether (1) a single string that
contains one or more file names (separated by one or more white-spaces), or (2)
a number of arguments (each argument is a string of file name). The array of
filenames are saved into
\begin{verbatim}
static char *filenames[] = { "object.data", NULL, NULL, NULL, NULL, 
        NULL, NULL, NULL, NULL };
\end{verbatim}

Example 1: approach 1st (use in EP) 
\begin{verbatim}
// parse input file
string objectFile("object.data");
string restartFile("restart");
string inputFiles = objectFile;

if (filetest(restartFile.c_str(), S_IFREG) == 0)
   inputFiles += " " + restartFile;

object_set("files", inputFiles.c_str());
\end{verbatim}
 
\verb!object_compile()! (to parse the contents in all these files, and save into
appropriate data structure).
\begin{verbatim}
static OBJECT *object = NULL;
static OBJECT **object_list = NULL;
\end{verbatim}
What \verb!object_compile()! does is to loop through the files, call
\verb!object_compilefile(filename)! to read in the data from that file.
\begin{verbatim}
void object_compile(void){
  int k;
  ...
  for (k=0; k<nfiles; k++){
     object_compilefile(filename[k]);
  }
}
\end{verbatim}

\verb!object_compilefile()! then decide to read from which block to which block
in the file, i.e. \verb!object_compilefilesubset(filename, first, last)!. By
default, it reads from first block to the last block (by default). This is
actually the {\bf parser}.
\begin{verbatim}
void object_compilefile(const char* filename)
{
  // 0x7fffffff = the maximum number in INT32
  // NOTE: The min value (-,147,483,648) is 0x80000000
  object_compilefilesubset(filename, 0, 0x7fffffff);
}
\end{verbatim}
which save the opening file into the structure OBJECTFILE (filename and the
pointer to file handle (FILE*).
\begin{lstlisting}
typedef struct {
   FILE* file;
   char* name;
} OBJECTFILE

void object_comilefilesubset(const char* filename, int first, int last)
{ //read from line 'first' to line 'last' in the file
  char*line; 
  OBJECTFILE file;
  
  file = object_fopen(filename, "r");
  n = 0;
  
  if (file.file != NULL) {
  //here it doesn't read the single line
  //but considered the whole object as a single line
  //(as newline characters in between are removed)
     while ((line = object_read(file)) != NULL) {
   	   if (n >= first) { //read this line
   	     object_compilestring(line);
   	     if (++n > last) break; //exit the loop        
   	   }  
     }
  }
  object_fclose(file);
}
\end{lstlisting}

NOTE: It may be confused that $n$ represent the line-number. Indeed, it
represents the block-index. Here, each 'object' block is read-in using the
function \verb!object_read()!, reads in from the current cursor position to
the next '\}' character, and returns as a single string (i.e.
\textcolor{red}{all new-line characters are converted to white-space}). When we
have a presumably syntax-correct object block (NOTE: anything after // is
considered comment and are ignored), we parse the information inside using
\verb!object_compilestring(line)! which is actually the {\bf lexer}.
{\small \begin{verbatim}
char *object_read(OBJECTFILE ofile)
{
  FILE *file;
  static char *line = NULL;
  static int nline = 0; 
  int n, c, len, nitems, offset;
  file = ofile.file;
  n = 0; 
  c = getc(file);
  while (!feof(file))
  {  
    while (nline < n + 5) 
    {
      nline += 256; 
      line = (char*) Realloc(line, nline);
    }
    switch (c)
    {
    case '"': 
      line[n++] = (char)c;
      c = getc(file);
      int quoteStart = n; 
      while (c != '"') 
      {
        if (nline < n + 5 )
        {
          nline += 256; 
          line = (char*) Realloc(line, nline);
        }
        line[n++] = (char)c;
        c = getc(file);
        if (c==EOF) 
        {
          int lEnd = MIN(quoteStart+1,n);
          line[lEnd]='\0';
          error_action("String termination  for string in line starting with:\n\n****" , line  ,"****\n\n not found in object file : ", ofile.name, ERROR_IN("object_read", ABORT));
        }
      }
      break;
    case '/':
      c = getc(file);
      int commentStart = n;
      int m=n;
      switch (c)
      {
        case '*':
        c = getc(file);
        char clast = c;
        while (c != EOF)
        {
          if (c == '/'  && clast == '*') {c = ' '; break;}
          clast = c;
          c = getc(file);
          m++;
        }
        if (c==EOF)
        {
          int lEnd = MIN(commentStart+2,m);
          line[lEnd]='\0';
          error_action("Comment termination of (/*) in line starting with:\n\n****",line,"****\n\n not found in object file : ", ofile.name, ERROR_IN("object_read", ABORT));
        }
        break;
        case '/':
        clast = c = getc(file);
        while (c != EOF)
        {
          if (c == '\n'  && clast != '\\') {c = ' '; break;}
          clast = c;
          c = getc(file);
          m++;
        }
        if (c==EOF)
        {
          int lEnd = MIN(commentStart+2,m);
          line[lEnd]='\0';
          error_action("Comment termination of (//) in line starting with:\n\n****",line,"****\n\n not found in object file : ", ofile.name, ERROR_IN("object_read", ABORT));
        }
        break;
        default:
        ungetc(c, file);
        c = '/';
        break;
      }
      break;
    case '\n':
      c = ' ';
      break;
    case '\t':
      c = ' ';
    case ' ':
      break;
    case '{':
      break;
    case '}':
      break;
    case '\\':
      c = (char)getc(file);
      switch (c)
      {
      case 'n':
        c = '\n';
        break;
      case 't':
        c = '\t';
      }
      break;
    case '$':
      c = (char)getc(file);
      switch (c)
      {
      case 'S':
        nitems = fscanf(file, "%d", &offset);
        c = (char)getc(file);
        while (!isgraph(c))
        {
           if (feof(file)) return NULL;
          c = (char)getc(file);
        }
        long first = ftell(file) - 1;
        if (nitems > 0) fseek(file, offset, SEEK_CUR);
        while (c != ';' && c != '}')
        {
          if (feof(file)) return NULL;
          c = (char)getc(file);
        }
        long last = ftell(file) - 1;
        len = strlen(ofile.name);
        while (nline < n + len + 1 + 2 + 64)
        {
          nline += 256;
          line = (char*) Realloc(line, nline);
        }
        len = sprintf(line + n, "%s@%ld-%ld", ofile.name, first, last);
        n += len;
        break;
      case 'B':
        line[n++] = '$';
        break;
      }
      break;
    case ';':
      break;
    default:
      break;
    }
    line[n++] = (char)c;
    if (c == '}')
    {
      line[n++] = 0x0;
      trim(line);
      return line;
    }
    c = (char)getc(file);
  }
  return NULL;
}
\end{verbatim}}


To interpret the block
\begin{lstlisting}
void object_compilestring(char *string)
{
  OBJECT obj;
  char *line;
  int rc, i, l;
  
  line = strdup(string);
  rc = object_lineparse(line, &obj);
  if (rc < 2) {
    //do something
  }
}
\end{lstlisting}

Here, the work is done by \verb!object_lineparse()! function to
extract the object value+classname, as well as keywords and the associated
values.
\begin{verbatim}
typedef struct object_st {
  char *name;
  char *objclass;
  char *value;
  char *valueptr;
  struct object_st *parent;
} OBJECT;

int object_lineparse(char *line, OBJECT *object)
{
  char *tok, *last, *end;
  last = NULL;
  int rc, l;
  
  rc = 0;
  trim(line);
  object_check(line);
  end = line+strlen(line);
  *tok = '\0';
  tok++; 
  l = strlen(tok);
    if (tok[l-1] != '}' )  rc += 8;
    tok[l-1] = '\0'; 
    object->value = strdup(tok); 
    tok = strtok_r(line, " ",&last);
    object->name = strdup(tok);
    tok = strtok_r(NULL, " ",&last);
    object->objclass = strdup(tok);

    trim(object->name);
    trim(object->objclass);
    trim(object->value);
    if (object->value != NULL)
    {
        l = strlen(object->value);
        if (l>0 && object->value[l - 1] != ';')
        {         
            object->value = (char*) Realloc(object->value, l + 2);
            strcat(object->value, ";");
        }
    }
    if (object->name == NULL) rc += 4; 
    if (object->objclass == NULL) rc += 2;
   if (object->value == NULL) rc += 1;
    return rc;
  
}
\end{verbatim}

Eventually, all the read-in data are stored in a global static variable
\begin{verbatim}
static OBJECT *object = NULL;
static OBJECT **object_list = NULL;

static int nobject = 0, mobject = 0;
static int niobject= 0, miobject = 0;
\end{verbatim}


then we compile it 
\begin{verbatim}
object_compile();
\end{verbatim}

  
There are utilities function to deal with this data
\begin{enumerate}
  \item To check for the existence of an object, we look for ObjectName and
  CLASSNAME. The list of objects are saved in the global array
  \verb!object! with each element of type OBJECT
  
\begin{verbatim}
int object_exist(const char* name, const char *objclass)
{
   int i;
   for (i = 0; i < nobject; i++){
   if (!strcmp(object[i].name, name) &&
       !strcmp(object[i].objclass,objclass)) return 1; //found
   }
   return 0; //not found
}
\end{verbatim}
    
    \item 
  \end{enumerate}
  
The content of all these files are parsed using
\begin{verbatim}
object_compile()
\end{verbatim}
Then, in an MPI application, we only want the root process (rank=0) to parse,
and it will broadcast to all other process using \verb!object_Bcast()!
\begin{verbatim}
if (myRank == 0)
{
   object_set();
   object_compile();
}
#ifdef WITH_MPI
  object_Bcast();
#endif
\end{verbatim}
To send a package to other process, we need to pack in into a structure called
\verb!PACKBUF!, using \verb!object_pack()! where the child processes can read-in
using \verb!object_unpack()!
\begin{verbatim}
typedef struct pack_buf_st
{
    int n, nobject,mobject;
    char *buffer;
} PACKBUF;

void object_Bcast(int root, MPI_Comm comm)
{
    PACKBUF buf;
    buf.buffer=NULL;
    int myRank;
    MPI_Comm_rank(comm, &myRank);
    if (myRank==root) object_pack(&buf);
    MPI_Bcast(&buf,3,MPI_INT,root,comm);
    buf.buffer=Realloc(buf.buffer,buf.n);
    MPI_Bcast(buf.buffer,buf.n,MPI_CHAR,root,comm);
    if (myRank!=root) object_unpack(&buf);
    Free(buf.buffer);
}
\end{verbatim}


  
\section{How to code a problem EM}
  
  
There are two types of solvers that we can choose to solve: Solid-Linear and 
Mixed-Linear. Let's notice the common theme for codes using the same solver to
solve different problems.

To know the which problem to solve
\begin{verbatim}
std::string obj_name;
std::string obj_class;
int ptype = myParams.get_problem_type(obj_class, obj_name);
\end{verbatim} 

The steps to define in the constructor of a problem of type oomph::Problem
\begin{enumerate}
  \item timestep (if required)
\begin{verbatim}
Problem::add_time_stepper_pt(new SomeTimeStepper());
\end{verbatim}  

  \item the Mesh (which may requires passing an element type)
\begin{verbatim}
Problem::mesh_pt() = new SomeMesh<SomeElement>(...); 
\end{verbatim}  

  \item The boundary condition (by default: all nodal values are free and
  initial values are initialized to zeros, so we only need to 'pin' the values
  at boundary)
\begin{verbatim}
// at a node, for a value of zero
 Problem::mesh_pt()->node_pt(0)->pin(0); 
 
// at a ode, for a particular
val = 1.0
//using zero-based index
Problem::mesh_pt()->node_pt(0)->set_value(0,val); 
\end{verbatim}
  
  \item Add any further information to mesh element. The reason is that the
  constructor of mesh class accept no argument (to allow generic implementation
  of the mesh creation process). Thus, to add like global variable to each mesh
  element, we need to define a suitable access function to the mesh element, and
  pass the global variable to this function call.
  
  \item Assign global and local equation
\begin{verbatim}
// a generic functioin
Problem::assign_eqn_numbers(); 
\end{verbatim}  

  \item By default, the solver is assumed to solve a non-linear problem. With
  linear problem, there are some steps that can be avoided. To save time, we
  just set
\begin{verbatim}
bool oomph::Problem::Problem_is_nonlinear
\end{verbatim}  
to \verb!false! for solving a linear problem.

  \item Call the proper solver
\begin{verbatim}
// Solve the problem, using oomph-lib's default Newton solver
problem.newton_solve();
\end{verbatim}  
This is a black-box
\footnote{\url{http://oomph-lib.maths.man.ac.uk/doc/order_of_action_functions/html/}}.
During Newton iteration, sometimes we want to perform certain additional tasks,
e.g. to analyze the convergence (or divergence) of the iteration. At key stages
of this solving, \verb!oomph::Problem! already provides some \verb!virtual!
empty member functions so that users can override them.

The two pure virtual function need to be implemented, though they can be defined
as empty
\begin{verbatim}
// often used to update boundary condition
// when performing parameter studies
Problem::actions_before_newton_solve() 


// to perform post processing
// when a solution has been obtained
Problem::actions_after_newton_solve()
\end{verbatim}

  \item Synchronize
\begin{verbatim}
//synchronise the solution on different processors (on each submesh)
problem.synchronise_all_dofs();
\end{verbatim}


\end{enumerate}



\subsection{Volume Constraint}

\begin{verbatim}
void
calculate_mechanics_volume_constraint(ProblemParams& pinput, CardiacActivationType read_calcium_from_file = CardiacActivationType_Regular, int readerArg = 0)
{

  // Set up the problem, we are using active anisotropic elelements
  const bool read_tensors = false; 
  VentricularProblem<MechanicsElementFactoryBasicModel<ArtsFilamentModel> > problem(read_tensors, read_calcium_from_file);
  std::string sensor_file;
  pinput.get_value("sensorFileName", sensor_file, "sensor.txt");
  problem.volume_constraint_simulation(LumpedCirculatoryModel,
                                       read_calcium_from_file, readerArg, sensor_file);

}
\end{verbatim}
  
\subsection{BarTest- ! Mixed}
  
\subsection{BarTest - Mixed}
  
Consider simulate \verb!barTest! with mixed-linear solver (mixed=1).
  \begin{verbatim}
void
calculate_mechanics_test_with_bar_mixed(ProblemParams& pinput, BarTestType test)
{
  HBarProblem<HElementFactory<BasicActiveModel> > problem;
  problem.set_simulation_type(test);
  problem.initialize();
  problem.simulation();
}
  \end{verbatim}
with 'H' = hybrid.

We need to define the HBarProblem class
\begin{verbatim}
template <class FACTORY>
class HBarProblem :  public BlockMixedLinearProblem
//public BlockSolidLinearSystemProblem
{

protected:
  void rotation(double angle);
  void shift(double shift);
  virtual void initialize_const_law();
  CardiacMechanicsMesh<FACTORY>* get_mesh();

  double& get_pressure();
  GravityFct get_gravity();

public:

  HBarProblem():
    rotation_(false), smoothed_vectors_(false)
  {};


  void set_simulation_type(BarTestType test);

  virtual void initialize();

  void doc_solution(CardiacDocInfo& doc_info,
                    bool need_only_node_value = false, unsigned n_variables = 4);

  virtual void simulation();

  void set_rotation_test(bool rotation);

  void use_smooth_vectors() {
    smoothed_vectors_ = true;
  }
\end{verbatim}
with 
\begin{verbatim}
// mixed/BlockMixedLinearProblem.cc
class BlockMixedLinearProblem : public BlockSolidLinearSystemProblem
{

}
\end{verbatim}
witth 
\begin{verbatim}
// BlockSolidLinearSystemProblem
class BlockSolidLinearSystemProblem : public BlockLinearSystemProblem
{
public:
}
\end{verbatim}
with
\begin{verbatim}
class BlockLinearSystemProblem: public DistributedProblem
{
public
}
\end{verbatim}
with
\begin{verbatim}
class DistributedProblem: public oomph::Problem
{
public:
  virtual unsigned long assign_eqn_numbers(const bool &assign_local_eqn_numbers = true);
  
  void add_mesh(CardiacDistributedMesh *mesh) {
    submeshes_.push_back(mesh);
    Problem::add_sub_mesh(mesh);
  }

  unsigned get_n_submeshes() {
    return submeshes_.size();
  }
   
  CardiacDistributedMesh* get_submesh(unsigned index) {
    return submeshes_[index];
  }
    
private:
  std::vector<CardiacDistributedMesh*> submeshes_;
};
\end{verbatim}

\subsection{Heart Infarct}
  
\begin{verbatim}
void
calculate_infarct(ProblemParams& pinput)
{
  const bool read_tensors = true;
  ActiveModel::get_transverse_tension() = true;
  VentricularProblem<MechanicsElementFactoryBasicModelOrt<ArtsFilamentModel> > problem(read_tensors, CardiacActivationType_ActivationTime);
  std::string sensor_file;
  pinput.get_value("sensorFileName", sensor_file, "sensor.txt");
  problem.volume_constraint_simulation(LumpedCirculatoryModel,
                                       CardiacActivationType_ActivationTime, 0, sensor_file);
}
\end{verbatim}   

\subsection{Heart Infarct - Mixed}

\begin{verbatim}
void
calculate_mixed_infarct(ProblemParams& pinput)
{
  const bool read_tensors = true;
  ActiveModel::get_transverse_tension() = true;
  HVentricularProblem<HElementFactoryOrt<ArtsFilamentModel> > problem(read_tensors, CardiacActivationType_ActivationTime);
  std::string sensor_file;
  pinput.get_value("sensorFileName", sensor_file, "sensor.txt");
  problem.volume_constraint_simulation(LumpedCirculatoryModel,
                                       CardiacActivationType_ActivationTime, 0, sensor_file);
}
\end{verbatim}
  
\subsection{VentricularProblem.cc}
  
This file contains the code to solve 
\begin{verbatim}
void 
\end{verbatim}
  
\subsection{Generate Fibers (meshes + gradients)}
\label{sec:fiber_generation}

The class to deal with the problem is \verb!FiberGenerationProblem.cc/.h!. 
This class uses information from a mesh and a Poission function
\begin{verbatim}
FiberGenerationMesh* Ventricular_mesh_pt;
oomph::PoissonEquations<dimension>::PoissonSourceFctPt Source_fct_pt;
\end{verbatim}

Here is the main() code
\begin{lstlisting}
  Params myParams;
  //Parse the input to firgure out the problem type
  // Should call read_input_file exactly once for each MPI task
  myParams.read_input_file(argc, argv, MPI_COMM_WORLD);
  std::string obj_name;
  std::string obj_class;
  int ptype = myParams.get_problem_type(obj_class, obj_name);
// pinput will contain all the related parameters for the ptype problem
ProblemParams pinput = ProblemParams(obj_class, obj_name);

  switch(ptype) {
  
  case Params::FIBER :
    oomph::oomph_info << "Generating fibers ..."  << std::endl;
    generate_fibers(pinput);
    break;
  }
  
//FiberGenerationProblem.cc/h
void generate_fibers(ProblemParams& pinput );
\end{lstlisting}

The function uses \verb!FiberGenerationProblem! which is a child of
\verb!DistributedProblem! which is a child of \verb!oomph::Problem!.
\begin{verbatim}
void generate_fibers(ProblemParams& pinput)
{
  FiberGenerationProblem fg(pinput); 
  fg.fiber_generation();
}
\end{verbatim}
IMPORTANT: Every problem is a derivative of \verb!oomph::Problem!.

\begin{lstlisting}
FiberGenerationProblem:: public DistributedProblem
{
private:

  void setup_solver();

  void clear_boundary();
  void pin_boundary(unsigned boundary);
  void set_boundary_value(unsigned boundary, double value);
  void set_boundary(bool is_pin);

  static const unsigned dimension = 3;

  /// \short Tetgen ventricular mesh.
  FiberGenerationMesh* Ventricular_mesh_pt;
  oomph::PoissonEquations<dimension>::PoissonSourceFctPt Source_fct_pt;

  unsigned generation_phase_;
};

class DistributedProblem: public oomph::Problem
{ 
  void add_mesh(CardiacDistributedMesh *mesh) {
    submeshes_.push_back(mesh);
    Problem::add_sub_mesh(mesh);
  }
  
  unsigned get_n_submeshes() {
    return submeshes_.size();
  }
  
  CardiacDistributedMesh* get_submesh(unsigned index) {
    return submeshes_[index];
  }

private:
  std::vector<CardiacDistributedMesh*> submeshes_;
};
\end{lstlisting}
Here, we use distributed approach so that each node can process a region of the
mesh.

At first, we need to modify the constructor to read-in the input information
from the object file, and make necessary setup for the problem
\begin{verbatim}
//constructor
void FiberGenerationProblem::
FiberGenerationProblem(...) 
{
//Setup the solver
  this->setup_solver() ; //check below
  Newton_solver_tolerance = 1.0e-6;
  //this is required for Poisson problem
  Source_fct_pt = fiber_problem::source_function;

{
....//read in information from object.data
}


//Create the mesh from 3 files
// these three files are generated using TetGen
  this->Ventricular_mesh_pt = new FiberGenerationMesh('fiber.node', 
            'fiber.ele', 'fiber.face');
// Solid bulk mesh
  this->add_mesh(Ventricular_mesh_pt);
// oomph::Problem::build_global_mesh()
// to build  global mesh by combining all the submeshes
// which are passed as a vector of pointers to the submeshes
  this->build_global_mesh();
//NOTE: if one of the submesh has changed, we need to rebuild
//using oomph::Problem::rebuild_global_mesh()
}

class FiberGenerationProblem : public DistributedProblem
{
private:
  FiberGenerationMesh * Ventricular_mesh_pt;
  oomph::PoissonEquation<dimension>::PoissonSourceFctPt Source_fct_pt;
}
\end{verbatim}

The mesh to be generated is Tetgen which is defined in
\verb!FiberGenerationMesh!
\begin{verbatim}
class FiberGenerationMesh: public virtual CardiacDistributedMesh,
  public virtual AnisotropicMesh
{
 ...
private:
  unsigned phase_;
}

class CardiacDistributedMesh:
  public virtual MeshWithEnumeratedNodes
{ 

protected:
  oomph::TimeStepper *Time_stepper_pt;
  static const unsigned number_of_tet_vertices_ = 4;
  static const unsigned number_of_tet_edges_ = 6;
  static const unsigned number_of_edge_vertices_ = 2;

private:
  bool quadratic_;
}

class MeshWithEnumeratedNodes: public virtual oomph::Mesh
{  
protected:
  std::vector<std::pair<unsigned, oomph::Node*> > sorted_nodes;

private:
  unsigned n_vertex_nodes_;
};

class AnisotropicMesh : public virtual MeshWithDataAtIntegrationPoints
{
protected:
  /// \short Action after data initialization in integration points for element.
  ///
  /// We precalculate transformation tensors in integration points of anisotropic elemenent
  void after_initialization_data_at_integration_points(oomph::GeneralisedElement* const finite_element);

private:
  bool read_vectors_;
};

class MeshWithDataAtIntegrationPoints: public virtual MeshWithEnumeratedNodes
{
protected:
  /// \short Action after data initialization in integration points for element.
  virtual void after_initialization_data_at_integration_points(oomph::GeneralisedElement* const) = 0;
};
\end{verbatim}




\begin{lstlisting}
void FiberGenerationProblem::
setup_solver() 
{
  oomph::HypreSolver* hypre_linear_solver_pt = new oomph::HypreSolver;

  ...
  this->linear_solver_pt() = hypre_linear_solver_pt;
  this->Max_residuals = 100000; //the maximum desired residual (program exit if 
  //the maximum residual exceeds this value.
}
\end{lstlisting}


\textcolor{red}{\bf MAIN STAGE}: Here, a very important function
\verb!FiberGenerationProblem::fiber_generation()! which goes through a number of
steps
\begin{enumerate}
  \item Generate gradient from LV endo(0) to epi(1)

  \item Generte gradient from RV endo(0) to epi(1)
  
  \item Generate gradient from LV endo(0), RV endo(1), to epi(2)
  
  \item Map ventricles
  
  \item Smoothing between LV and RV 
  
  \item Smoothing trabecular vs. the bulk
  
  \item Trabecular fiber vectors
  
  \item Final fiber generations
  
  \item M-cellgeneration
  
  \item Smoothing M-cells
  
  \item Generating mesh (the most important part)
\end{enumerate}

Let's take one phase as example
\begin{verbatim}
const bool is_pin = true;
oomph::oomph_info << "################Gradient from endo (0) to epi (1) ##########" << std::endl;
  {
    // saving transmural gradeint magnitude
    generation_phase_ = 0;
    this->set_boundary(is_pin);
    unsigned long eqn_number = this->assign_eqn_numbers();
    oomph::oomph_info << "Phase 0, Number of equations: " << eqn_number << std::endl;

    this->newton_solve();
    this->doc_solution(doc_info, true);
    doc_info.number()++;
    this->synchronise_all_dofs();

    Ventricular_mesh_pt->calculate_gradient();
    const unsigned output_scalar_index = 3;
    Ventricular_mesh_pt->save_gradient_magnitude_to_scalar(output_scalar_index, this);
    Ventricular_mesh_pt->zero_node_values();
  } 
\end{verbatim}


The object structure
\begin{verbatim}
fiber FIBERPROBLEM {

}
\end{verbatim}
