\chapter{C/C++ APIs in Windows}
\label{chap:CCpp_Windows}

Here, we do not discuss the API for graphics, audio, video.
This is covered in another book - {\bf GUI development}.

To provide interface working with the Windows O/S, Windows API was developed
(Sect.\ref{sec:WinAPI}) which is in C language.
To provide object-oriented APIs via C++ language, MFC that wraps WinAPI was
developed (Sect.\ref{sec:MFC}).

To provide interface with .NET languages via C++ language, different
technologies have been developed: Managed C++ (Sect.\ref{sec:managed_C++}),
C++/CLI (Sect.\ref{sec:C++/CLI}).

Windows RT is the version of Windows running on ARM processors.
To support ARM processors in Windows 8.x, only a subset of current Windows APIs
is available (Sect.\ref{sec:WindowsAPI_Windows-RT})

To support the new WinRT APIs, C++/CX (Sect.\ref{sec:C++/CX}) was developed.

\section{C Runtime (CRT) Library}
\label{sec:CRT}
\label{sec:Standard-C-library-Windows}

Microsoft implemented the C standard library (Sect.\ref{sec:C-standard-library})
in the set of library files and call it {\bf C Runtime (CRT) library}. The
library files are built without \verb!iostream! or standard C++ library.

The library files are provided in 
\begin{itemize}
  
  \item single-threaded: compiled with /ML or /MLd option (\textcolor{red}{it is
  no longer available since VS 2005})

\begin{verbatim}
libc.lib
libcd.lib           (debug version)
\end{verbatim}  
  
  \item multi-threaded: compiled with /MD or \verb!/MDd! option.
  
First column: library you can use. Second column: the compiling option used to
compile that library, and it is the compiling option you need to use for able to
use the library in the first column. Third column: the macros that are defined
once the compiling option in the second column is used
  
\begin{verbatim}
            compile your code with      preprocessor directives
libcmt.lib       /MT                    _MT
        //This one is statistically linked
        
msvcrt.lib       /MD               _MT, _DLL
       /*This one serves as import library (dynamically linked)
       for msvcr71.dll or msvcr80.dll or msvcr90.dll
       depending on Visual C++ version
       */ 

libcmtd.lib      /MTd
msvcrtd.lib      /MDd
       /*This one serves as import library (dynamically linked)
       for msvcr71d.dll or msvcr80.dll or msvcr90.dll
       depending on Visual C++ version
       */
\end{verbatim}
The concept of {\it import library} is given in Sect.\ref{sec:import_library}

Since VS 2005, library files for managed code (C++/CLI) is added
\begin{verbatim}
msvcmrt.lib            /clr
             (import library for
             msvcm80.dll)
             
msvcurt.lib           /clr:pure
             (import library for
             msvcm80.dll)             
\end{verbatim}
\end{itemize}
\url{https://msdn.microsoft.com/en-us/library/vstudio/abx4dbyh(v=vs.71).aspx}

When you build a release version of your project, one of the basic C run-time
libraries (LIBC.LIB, LIBCMT.LIB, and MSVCRT.LIB) is linked by default, depending
on the compiler option you choose (single-threaded, multithreaded, or DLL).

\subsection{msvcrt.dll vs. msvcr**.dll}

\textcolor{red}{msvcrt.dll vs. msvcr**.dll}: Applications built with older
versions of Visual C++ (4.2 to 6.0) are linked to mscvrt.dll. This is not
recommended in newer versions of Visual C++ (8.0, 9.0, \ldots) It should use
static version of CRT library or DLL version of CRT library with a version
number as part of dll names(msvcr80.dll,msvcr90.dll,etc.).

\begin{mdframed}

\verb!msvcrt.dll! is a C Runtime Library but is supposed to be a system
component owned and built by Windows, and should not be used directly in an
application. Windows can continuously updates this library via service packs or
hot fixes.


Once you develop a C/C++ application in Windows, you should use
\verb!msvcr71.dll! (or another \verb!msvcr**.dll! file) and keep a copy of this
DLL in the application's folder. In Visual C++ 7.0, any application built with
Visual C++ .NET using the /MD switch will necessarily use msvcr71.dll.
\end{mdframed}

%If you have more than one DLL or EXE in the project, there is a chance that

NOTE: you link to the import library \verb!msvcrt.lib! and it will dynamically
link to the right DLL library depending on which Visual Studio version you use

\begin{enumerate}
  \item VS 2002: \verb!msvcr70.dll!
  \item VS 2003: \verb!msvcr71.dll!
  \item VS 2005: \verb!msvcr80.dll!
  \item VS 2008:  \verb!msvcr90.dll!
\end{enumerate} 
\url{http://support.microsoft.com/kb/KbView/154753}



\section{C++ Runtime Library}
\label{sec:C++_RT}

Since VS2003,  Visual C++ will no longer ship the old iostream libraries
(Sect.\ref{sec:iostream_library}).
\url{https://msdn.microsoft.com/en-us/library/vstudio/ct1as7hw(v=vs.71).aspx}

The library files are 
\begin{itemize}
  
  \item single-threaded: compiled with /ML or /MLd option (\textcolor{red}{it is
  no longer available since VS 2005})

\begin{verbatim}
libcp.lib
libcpd.lib           (debug version)

\end{verbatim}  
  
  
  \item multi-threaded: compiled with /MD or \verb!/MDd! option.
\begin{verbatim}
                compiled with      preprocessor directives
libcpmt.lib       /MT               _MT
        //This one is statistically linked
        
msvcprt.lib       /MD               _MT, _DLL
       /*This one serves as import library (dynamically linked)
       for msvcpr71.dll or mscvpr80.dll or msvcpr90.dll
       */ 

libcpmtd.lib      /MTd
msvcprtd.lib      /MDd
       /*This one serves as import library (dynamically linked)
       for msvcpr71d.dll or mscvpr80d.dll or msvcpr90d.dll
       */
\end{verbatim}
The concept of {\it import library} is given in Sect.\ref{sec:import_library}

\end{itemize}

\subsection{msvcpp.dll vs. msvcpr**.dl}
\label{sec:msvcpp.dll}

\textcolor{red}{msvcpp.dll vs. msvcpr**.dll}: Applications built with older
versions of Visual C++ (4.2 to 6.0) are linked to mscvpp.dll. This is not
recommended in newer versions of Visual C++ (8.0, 9.0, \ldots) It should use
static version of CRT library or DLL version of CRT library with a version
number as part of dll names(msvcpr80.dll, msvcpr90.dll,etc.).

%\section{Linking with static or dyna}
\subsection{iostream library (new vs. old)}
\label{sec:iostream_library}

Old iostream headers
\begin{verbatim}
fstream.h
iomanip.h
ios.h
iostream.h
istream.h
ostream.h
streamb.h
strstrea.h
\end{verbatim}

Since Visual Studio 2003, the new implementation of the iostream library do not have \verb!.h! suffix.
\begin{verbatim}
<fstream>
<iomanip>
<ios>
<iosfwd>
<iostream>
<istream>
<ostream>
<sstream>
<streambuf>
<strstream>
\end{verbatim}
\url{https://msdn.microsoft.com/en-us/library/aa984818(v=vs.71).aspx}


\url{http://support.microsoft.com/kb/KbView/154753}


NOTE: Replace (Sect.\ref{sec:iostream})
\begin{verbatim}
#include <iostream.h>
\end{verbatim}

with
\begin{verbatim}
using namespace std;

#include <iostream>
\end{verbatim}
\url{http://stackoverflow.com/questions/8506857/how-can-i-use-the-old-iostream-h-in-c-visual-studio-2010}


\section{Extension to C in Windows}
\label{sec:C_Windows}

\subsection{Windows API (WinAPI, Win32)}
\label{sec:WinAPI}

To provide interface working with the Windows O/S, Windows API is the
Microsoft's core set of API (WinAPI, Win32) written in C programming language.
Even though C has no object-oriented support, the WinAPI has sometimes been
described as object-oriented. Wrapper classes and extensions have been
developed that use WinAPI, e.g. MSFC (Sect.\ref{sec:MFC}), VCL, GDI+
(Sect.\ref{sec:GDI+}).

WinAPI is Unicode-friendly, i.e. it provides a macro for switching back and
forth between ANSI and Unicode (Sect.\ref{sec:string_Windows}). The
narrow-character version end with 'A', and the wide-character version end with
'W', and the generic version which is just a macro that call to either version
of the function by accepting \verb!TCHAR*! type.

\begin{verbatim}
DeleteFile   <-  Takes a TCHAR string (LPCTSTR)
DeleteFileA  <-  Takes a char string (LPCSTR)
DeleteFileW  <-  Takes a wchar_t string (LPCWSTR)
\end{verbatim}



\subsection{MSFC (MFC)}
\label{sec:MFC}

Microsoft Foundation Class Library (MSFC, MFC) wraps the WinAPI
(Sect.\ref{sec:WinAPI}) in C++ classes.
Thus, it's important to know WinAPI first.
MFC was first introduced in 1992 (Microsoft C/C++ 7.0 compiler for 16-bit
applications). Direct call to WinAPI is rarely used. Instead, programmers
create an object from one of MFC classes. To use MFC, all codes need to include
the pre-compiled \verb!stdafx.h! header file, and the functions/macros have
prefix \verb!Afx!.

\url{http://en.wikipedia.org/wiki/Microsoft_Foundation_Class_Library}

MFC is not included with the free version Visual C++ Express, but in the
commercial version Visual C++ 2010.
\begin{itemize}
  \item MFC 8.0 was released with Visual Studio 2005.

  \item MFC 9.0  was released with Visual Studio 2008. 
\end{itemize}
A light-weight alternative to MFC is Windows Template Library (WTL). Visual C++
Express use WTL, if Active Template Library is installed.

MFC is no longer widely used, as people changed to .NET framework.
 
\section{Extension to C++ in Windows}
\label{sec:VC++_Windows}

Microsoft Visual C++ IDE allows you to program code in different
languages: C, C++ and C++/CLI. Managed C++ is a deprecated feature
(Sect.\ref{sec:managed_C++}), and the current Windows version is C++/CLI
(Sect.\ref{sec:C++/CLI})

\subsection{Managed C++	 (deprecated)}
\label{sec:managed_C++}

Managed C++ (Managed Extension for C++) is a deprecated Microsoft set of
extension (grammar, keyword, attributes) that allows you to write both native
code and managed code in the same program.  In other words, you can take
advantage of the powerful C++ code, and can write managed code to run on .NET
CLR to take advantage of the garbage collection which relieves the programmer
from memory management.

Managed C++ is used in Visual Studio 2002 and 2003. Managed C++ is replaced by
C++/CLI since VS 2005 (Sect.\ref{sec:C++/CLI}). However, Managed C++ code can
also be compiled in the newer version of Visual Studio using
\verb!/clr:oldSyntax! option (Sect.\ref{sec:VC++_2005}).



\textcolor{red}{New using directive and new namespace so that .NET data type can
be used}: A Managed C++ code file needs to tell which .NET library to use and
which namespace of the file to use, i.e. having a new using directive and a new
using namespace directive
\begin{verbatim}
//hello.cpp
// compile with: /clr /LD

// this is required
#using <mscorlib.dll>

using namespace System;


// We can use .NET classes in Managed C++ code 
int main()  {
  Console::WriteLine("Hello, world!");
  return 0;
}
\end{verbatim}

The intrinsic managed type from .NET can be used in Managed C++. 
Example: we can use either \verb!UInt32! or \verb!unsigned int! in Managed C++
\begin{verbatim}
   C#         .NET type        Managed C++
   uint       System.UInt32     unsigned int
\end{verbatim}
\url{http://msdn.microsoft.com/en-us/library/0wf2yk2k.aspx}

NOTE: We can also import more Base Class Library (check C\# book)
\begin{verbatim}
#using <System.Windows.Forms.dll>

using namespace System::Windows::Forms;
\end{verbatim} 

\textcolor{red}{New compiling option}: \verb!/clr! enables any code referencing the .NET Framework to be compiled as CIL. {\bf To compile into DLL, we need to add /LD option. This is the known limitation.}

\begin{Verbatim}
cl.exe hello.cpp /clr

cl.exe hello.cpp /clr /LD
\end{Verbatim}



\textcolor{red}{New keywords}:  To further extend data types, i.e. to define
class, enums, pointers and references, under the control of CLR garbage
collector, Managed C++ defines some non-standard keywords.
% The managed code is under the garbage collection of the CLR (Common Language
% Runtime).
% Managed C++ uses non-standard keywords like \verb!__gc! or \verb!__value!.
\begin{itemize}
  \item \verb!__gc class! : to indicate a class is reference managed class
  
\begin{verbatim}
__gc class Block {};
\end{verbatim}
An object of this class is allocated on the CLR heap (managed heap) and is garbage collected.
  
  \item \verb!__value class! : to indicate a class is value managed class
  
An object of this class is considered as small, short-lived data items so there is no need for full garbage collection.
  
  \item \verb!__gc __interface! : to indicate a managed interface class
  
Any \verb!__gc! interface can be implemented by a \verb!__gc class!
\begin{verbatim}
// __gc_interfaces.cpp
// compile with: /clr /LD
#using <mscorlib.dll>
__gc __interface Ibase { 
   void f(); 
};
\end{verbatim}
\url{https://msdn.microsoft.com/en-us/library/aa712846(v=vs.71).aspx}  

NOTE: An interface is a class with no body code of its declared function prototypes.

  \item \verb!__value enum! : 
  
\begin{verbatim}
__value enum Color {red, green, blue}; 
\end{verbatim}

  \item \verb!__gc * ! : a pointer pointing to a managed object
  
\begin{verbatim}
__value struct V { int i;};
__gc struct G { V v; };

G __gc* pG = new G;  // pG point into CLR heap
V __gc* pV = &pG->v; 

\end{verbatim}  


  \item \verb!__nogc*! : a pointer cannot point to CLR heap (managed heap). By
  using this in a pointer definition, it helps the compiler to know explicitly
  whether the pointer is C++ pointer or CLR pointer.
  
\begin{verbatim}
V __nogc* pV2 = & v;   // pV2 cannot point into runtime heap 
\end{verbatim}

There are two types of pointers: \verb!__gc! pointers work on
.NET framework, \verb!__nogc! are normal C++ pointers. This helps programmers
to know which objects are under automatic garbage-collection by .NET and which
one need to be manually managed by the programmer.
  

  \item \verb!__gc __abstract! : to indicate a class is an abstract managed class
  
  \item \verb!__gc __sealed! : to indicate a derived class is sealed managed class
  
  \item \verb!__identifier! : to treat a C++ keyword as an identifier. This is
   required when working with class written in another language, and
  the name of this class is the C++ keyword.
  
Example: a class named \verb!operator!.
\begin{verbatim}
#using <mscorlib.dll>
#using "operator.dll"   // contains "operator" class

void () {
   __identifier(operator) *pO = new __identifier(operator);
}
\end{verbatim}
\url{https://msdn.microsoft.com/en-us/library/aa712812(v=vs.71).aspx}

  \item \verb!__delegate! : a delegate is a method pointer (the usage is similar to that in C++/CLI - Sect.\ref{sec:delegates})
  
First define a delegate type
\begin{verbatim}
__delegate int MyDelagate(String * str);
\end{verbatim}

then we define (1) the explicit method of the same interface of the delegate type, (2) and an object as the delegate pointing to the defined method
\begin{verbatim}
__gc class MyClass 
{
public:
	int MethodA(String *str) 
	{
		Console::WriteLine(S"MyClass::MethodA - The value of str is: {0}", str);
		return str->Length;
	}
}
MyDelegate *pDelegate = new MyDelegate(pMC, &MyClass::MethodA);


// to invoke the object's method via delegate
pDelegate->Invoke("Invoking MethodA");


\end{verbatim}
\url{http://www.codeproject.com/Articles/1077/Delegates-in-managed-C}


  \item \verb!__property! : to combine the get and set method into a single property block
  
  \item \verb!__event! : can be applied to method declaration, interface
  declaration, or data member declaration. The usage is similar to that in
  C++/CLI - Sect.\ref{sec:event_handler_C++/CLI}
  \url{https://msdn.microsoft.com/en-us/library/cb1dzt8t.aspx}
  
  \item \verb!__pin! : when a managed object is {\it pinned}, the CLR will not
  move the object location during garbage collection. This allows passing a
  pointer to a managed object as an actual parameter to an unmanaged function
  call.
\begin{verbatim}
G __pin * pG = new G;   // pG is a pinning pointer
   H * h = new H;
   h->incr(& pG -> i);   // pointer to managed data passed as actual
                         // parameter of unmanaged function call
\end{verbatim}  

Example:
\begin{verbatim}
Type __pin* name = dynamic_cast<Type *>(difName);
\end{verbatim}
NOTE: A pinning pointer can be implicitly converted to a \verb!__nogc! pointer.

\end{itemize}

Example:
\begin{verbatim}
__gc class Block {};                           // reference class
__value class Vector {};                       // value class
__gc __interface I {};                        // interface class
__gc __abstract class Shape {};                // abstract class
__gc __sealed class Shape2D : public Shape {}; // derived class


char WinName __gc[];  // declare an array
int object_location __gc[,];
\end{verbatim}

\begin{mdframed}

a \verb!__gc class! can 
\begin{itemize}
  \item have a constructor
  \item have a destructor
  \item implement any number of \verb!__gc! interface
\end{itemize}


Limitations: a \verb!__gc class! cannot
\begin{enumerate}
  \item  have multiple inheritance in managed code: due to the limitation of .NET CLR, i.e. a class managed under CLR's garbage collector cannot inherit more than one class.
  \item inherit a class that is not \verb!__gc! declared
  \item implement an interface that is not \verb!__gc! declared  
  \item no support ASP.NET web application in managed code
  \item no support for generic programming (aka templates)
  
  \item by default, be visible outside of its own assembly    
  
NOTE: To make a \verb!__gc! class visible outside, use \verb!public! keyword as well
\begin{verbatim}
public __gc class hey  { };
\end{verbatim}
%   \item    
% However, it had race
% conditions and other serious bugs. 
\end{enumerate}
\end{mdframed}
% Both have the same goal, i.e. creating .NET assemblies using C++ language. Typically if a
% company is developing .NET apps they don't lean toward C++/CLI, unless critical
% performance is necessary. 
\url{http://en.wikipedia.org/wiki/Managed_Extensions_for_C++\#Additional_or_amended_functionality_provided_in_Managed_C.2B.2B}

\subsection{C++/CLI}
\label{sec:C++/CLI}

Similar to Managed C++, C++/CLI allows you to write .NET code (managed code) and
C++ code (native code) in the same program. C++/CLI (Common Language
Infrastructure) is the replacement for Managed C++ since 2004
(Sect.\ref{sec:managed_C++}). C++/CLI is better in many ways
\begin{itemize}
  \item separate set keywords to use for managed codes, i.e. without sharing the same keyword with C++ code, e.g. to create a managed object we use \verb!gcnew! keyword.  
  
NOTE: C++/CLI was standardized as an extended version of C++ and ECMA defined
355 standards called CLR programs or C++/CLI programs.

  \item introduced the concept of generic (conceptually similar to C++ templates, but different in the implementation)
  
  \item finalizer syntax is introduced \verb.!ClassName(). - a special type of
  non-deterministic destructor that is only called as part of the garbage
  collection.
  
NOTE: the normal destructor is implemented through the IDisposable() interface;
while the finalizer is implemented to override the \verb!Finalize()! method of
the root \verb!Object! class. (Read C\# book to understand finalizable pattern
vs. disposable pattern).
  
  \item  \verb!typeid()! is added to help checking the class type name.
  
NOTE: typeid() does not support reference types. 

\end{itemize}


To use intrinsic .NET data type, we can use .NET types in C++/CLI code. If we
want to know the equivalent C\# data type, a good idea is to map the data type
between C\# and C++, we match with .NET data type from the two following links
\begin{itemize}
  \item C++/CLI and .NET:
  \url{http://msdn.microsoft.com/en-us/library/0wf2yk2k.aspx}
  
  \item C\# and .NET:
  \url{http://msdn.microsoft.com/en-us/library/ya5y69ds.aspx}
\end{itemize}

Example:
\begin{verbatim}
   C#         .NET type        C++/CLI
   uint       System.UInt32     unsigned int
\end{verbatim}


Support for C++/CLI is included in Microsoft Visual Studio 2005, 2008, 2010,
2012 (including Express edition).
A more detailed discusion is given in Sect.\ref{sec:VC++_C++/CLI}.
    
\url{https://msdn.microsoft.com/en-us/library/b23b94s7.aspx}


\subsection{stdafx.h}
\label{sec:stdafx.h}

Pre-compiled header file can greatly speed up the code compilation.
\textcolor{red}{This is only useful for large projects}.
The pre-compiled header file \verb!stdafx.h!  does \verb!#includes! all of the
.h files that you want to be precompiled. The file does not necessary name
\verb!stdafx.h! but the VS project template only picks that name. 

IMPORTANT:
\begin{itemize}
  \item use \verb!#include ``stdafx.h''!, instead of \verb!#include <stdafx.h>!

EXPLAIN: The file is supposed to locate in the current folder of the project, not in the include path.

  \item  using \verb!#include ``stdafx.h''! must be the first line in the .cpp files
  
  \item a file \verb!stdafx.cpp! must also be created, and the content is
\begin{verbatim}
#include "stdafx.h"
\end{verbatim}
The compilation of this file generate a \verb!*.pch! file. 
The name of the \verb!.cpp! file is \verb!stdafx.cpp!. The setting for this file is
different from other files, as it is built with \verb!/Yc! (i.e. create
pre-compiled header file option)
\footnote{\url{http://msdn.microsoft.com/en-us/library/7zc28563.aspx}}
\begin{verbatim}
/Yc[filename]
\end{verbatim}

Once that done, the compiler just use the \verb!.pch! file that is the result of
the first compilation.

% But before that can work, there must be one \verb!.cpp! file that
% \verb!#include! stdafx.h and get compiled first. 

  \item your project's setting must enable 
\begin{verbatim}
Project's properties
   Configuration: All Configurations
   
Configuration Properties
   C/C++
     Precompiled headers

  - Precompiled headers: choose /Yc or /Yu
  - Precompiled header file: "stdafx.h"
  - Precompiled header output file: "$(IntDir)\$(TargetName).pch" 
  (resulting in the compiler switch /Yu:stdafx.h).
 
\end{verbatim}

The corresponding settings for stdafx.cpp should be differentiated 
by the setting "Create precompiled header" (resulting in the compiler switch /Yc:stdafx.h).

\end{itemize}


\subsection{Windows Runtime Library files}


\begin{enumerate}
  \item C Runtime Library: Sect.\ref{sec:CRT}
  \item C++ Runtime Library: : Sect.\ref{sec:C++_RT}
  \item ATL: atl*.dll
  \item MFC: mfc*.dll
  \item VB 6.0 Run Time: MSVBVM60.DLL
  \item .NET: many libraries (mscorlib.dll, Systems.Windows.Forms.dll, \ldots)
\end{enumerate}

\url{http://en.wikipedia.org/wiki/Microsoft_Windows_library_files}


\section{APIs on Windows RT}
\label{sec:WindowsAPI_Windows-RT}
\subsection{WinRT (Windows 8)}
\label{sec:WinRT}

Windows 8, while also provide WinAPI (Sect.\ref{sec:WinAPI}), introduced the
new set of APIs called Windows Runtime (WinRT). WinRT is developed based on
C++, with object-oriented.
WinRT natively support ARM and x86 CPUs.

WinRT is based on Component Object Model (COM), and thus can interface with
different programming languages. WinRT supports development on C++/CX
(Sect.\ref{sec:C++/CX}), managed language C\#, VB.NET, JavaScript and
TypeScript. So, applications written in these language can run on Windows Phone
8 or Windows 8 easily.

WinRT applications run within a sandbox, and require user's approval to access
critical O/S resources and underlying hardware. File access is restricted to
some pre-determined places, e.g. Documents or Pictures.
\subsection{C++/CX}
\label{sec:C++/CX}

C++/CX (Component Extension) language is different from C++, also borrows some
syntaxs from C++/CLI  (Sect.\ref{sec:C++/CLI}), such as the hat character
\verb!^!. While C++/CX is syntactically similar to C++/CLI and thus looks almost
the same in many ways, it is semantically quite different, i.e. C++/CX is native code, no
CLR required. Other differences:
\begin{itemize}
  \item   For example, in C++/CX, the default interface on a class is specified
  through an attribute in the interface list while in C++/CLI it is an attribute
  on the class itself. 
  
  \item In C++/CLI, managed objects can move around in memory as the result of
  GC runs. There are several restrictions with managed class, e.g. we cannot
  put anything except primitive types (e.g. int) in the class, we cannot get
  the real address of a member (without pinning). In C++/CX, objects do not move
  around in memory and thus all of these restrictions are gone.
\end{itemize}

C++/CX has been designed to support the new API model in Windows 8 (WinRT -
Sect.\ref{sec:WinRT}). Initially, a new C++ library for Windows 8 called WRL
(Windows Runtime Library) was designed to support Windows 8. However, there were
so many difficulties to resolve.

\url{http://blogs.msdn.com/b/vcblog/archive/2011/10/20/10228473.aspx}

It enables the creation of Windows Store apps
(Sect.\ref{sec:Windows-store-apps}) and Windows Runtime component
(Sect.\ref{sec:Windows-Runtime-ABI}) that can easily interact with Visual C\#, Visual Basic and Javascript, and other languages that support
Windows Runtime. 

Programming in C++/CLI can be very challenging, as one must deftly juggle two
very different object models at the same time: the C++ object model with its
deterministic object lifetimes, and the garbage-collected CLI object model.


