\chapter{Executing string as code in other languages from C/C++}


\section{string as Python code}

compile and link your C program with \verb!-lpython2.7! (or a
proper version tag like \verb!-lpython3.3!)

inside your code, include \verb!#include <Python.h>!

write 



\section{strings as Scheme code (using libguile)}
\label{sec:libguile}

Libguile is the library which allows C programs to start a Scheme interpreter
and execute Scheme code. 
\begin{itemize} 
  \item All low-level public APIs have \verb!scm_! prefix.
  
  \item All high-level public APIs have \verb!gh_! prefix. This is more stable,
  as it is simpler.
\end{itemize} 

compile and linke your C program with \verb!-lguile!.


inside your code, include \verb!guile/gh.h>! 

initialize the code with 
\begin{verbatim}
gh_enter()  //to start Scheme interpreter
gh_enter(argc, argv, main_prog); //or pass to it the entry function
 
void main_prog(int argc, char *argv[])
{
gh_eval_str("(display \"hello Guile\")");

gh_eval_str("(define (square x) (* x x))");
gh_eval_str("(define (fact n) (if (= n 1) 1 (* n (fact (- n 1)))))");
gh_eval_str("(square 9)");
}      
\end{verbatim}




\url{https://www.gnu.org/software/guile/docs/master/guile-tut.html/What-is-libguile.html}