\chapter{Containers}
\label{chap:C++_containers}
\label{chap:containers}

\section{Introduction}
\label{sec:containers}

To avoid the tedious job of writing different classes, belonging to the same
type of data structure, e.g. vector or queue or stack, each can hold a different
type of data elements, C++ provides the so-called {\bf containers}.


Containers is an abstract way that allows programmers to organize data into some
common data structure, e.g. vector or queue or stack that can accept data
elements of any type, i.e. the data type of the data element is not defined for
the container until programmers uses the container.

This is the backbone for {\bf Generic Programming} paradigm which was proposed
and implemented by Alexander Stepanov in 1979. Although David Musser had
developed and advocated some aspects of generic programming already by 1971, it
was limited to a rather specialized area of software development.
At the time there was no real support in any programming language for generic
programming.

\begin{mdframed}

The first major language to provide such support was Ada (ANSI standard 1983),
with its generic units feature.

In 1985, the Eiffel programming language became the first object-oriented
language to include intrinsic support for generic classes, combined with the
object-oriented notion of inheritance.
 
In 1987 Stepanov and Musser had developed and published an Ada library for list
processing that embodied the results of much of their research on generic
programming.

Ada had not achieved much acceptance outside the defense industry and C++ seemed
more likely to become widely used and provide good support for generic
programming even though the language was relatively immature.

More importantly, C/C++ model of computation that allows very flexible access to
storage via pointers, which is crucial to achieving generality without losing
efficiency.

In 1993, Stepanov was asked, by Andrew Koenig of Bell Labs, to present that idea
in front of ANSI/ISO committee for C/C++ standardization.
Alex and Meng submitted for first draft of the proposal for the meeting in 1994.
After some modifications, it becomes part of the C++ standard, and the
implementation of such features is known as C++ standard template library (STL).

\end{mdframed}

A container can belong to one of the two categories
\begin{enumerate}
  \item \textcolor{red}{associative container}, i.e. there is no order between elements
  
  \item \textcolor{red}{seqiuence container}, i.e. there is an index associated with each element that enables to access
  the elements using an integer index.
\end{enumerate}

Any container needs an {\bf allocator} - default is Allocator object
(Sect.\ref{sec:Allocator}) - to handle memory management for elements in that
container.

There are two common different implementation of containers
\begin{enumerate}
  \item STL (part of C++ language) - Sect.\ref{sec:STL_containers}


The C++ Standard Template Library in C++ 98 implemented a number of containers
(Sect.\ref{sec:C++98_containers}), each provides homogeneous methods
(implemented using different algorithmic time) to insert, delete, and retreive
data element, regardless of the data type of the elements in that container.
  
  \item QTL (part of QT library): Sect.\ref{sec:QTL_containers}
  
  \item Boost library - Sect.\ref{sec:Boost-containers}: 
  
Boost added many more containers that has not been available in C++ 98 STL. Some of them are now part of C++11 STL.

\end{enumerate}

\subsection{STL Containers}
\label{sec:STL_containers}

The containers implemented in the C++ Standard Template Library (STL -
Sect.\ref{sec:C++98_STL}) are called STL containers.

So far, C++ STL lacks the feature to have heterogeneous containers that can hold
arbitrary element types like those supported in Python, Lisp/Scheme, ML and
Haskell.

C++03 added \verb!map! container - Sect.\ref{sec:C++03_containers}.

C++11 improved and added new containers - Sect.\ref{sec:C++11_containers}


\subsection{QTL Containers}
\label{sec:QTL_containers}

QTL containers are containers implemented in QT Library.
The QTL has a different philosophy from the STL, which is well summarized by J.
Blanchette: 

{\it "Whereas STL's containers are optimized for raw speed, Qt's
container classes have been carefully designed to provide convenience, minimal
memory usage, and minimal code expansion."}
 
\url{http://blog.qt.io/blog/2006/10/16/atomic-reference-counting-is-it-worth-it-2/}

\url{http://stackoverflow.com/questions/1668259/stl-or-qt-containers}

\subsection{Boost's containers}
\label{sec:Boost-containers}

Boost added many more containers that has not been available in C++ 98 STL.

Boost.Any - Sect.\ref{sec:Boost.Any}, Boost.Tuple




\section{Allocator (std::allocator)}
\label{sec:Allocator}
\label{sec:std::allocator}

When you allocate a very large chunk of memory region, it is typically better
just reserve the raw memory, but do not call the constructor. This is quicker,
as evoking also the constructor takes additional time, and there is a chance
that we don't use all of them.

Later on, e.g. when evoking an \verb!insert()! method, then it evokes a proper
constructor, one the raw memory region.

STL C++ provides and uses \verb!std::allocator! by default for all STL classes.
\verb!std::allocator! is used when you want to separate allocation,  and do
construction in two steps.

\begin{verbatim}
//template <class T> class allocator;
// class LifeNode { ... }

#include <memory>

// declare an allocator 
//   ...for integer values 
std::allocator<int> myAllocator; 
//   ...for LifeNode values
std::allocator<LifeNode> another_allocator; 

// allocate space for 
//  ... five ints 
int* arr = myAllocator.allocate(5); 
    
\end{verbatim}

\subsection{-- std::allocator C++11}

Now, we want to construct the object, by calling the object's constructor, e.g.
when \verb!push_back! or \verb!increase()! function is called.

\begin{lstlisting}


\end{lstlisting}

The job of the allocator is it to manage the lifetime of its elements. That
means to allocate and deallocate memory for its elements and to initialize and
destruct them.


\textcolor{red}{An allocator only does the job of allocating memory, it does not construct the objects.}
Constructing the object is the job of the constructor of a class, which needs to
use this allocator for memory allocation first, and the construct the object on
that memory second (Sect.\ref{sec:container-user-defined}).

\begin{mdframed}

Allocation and deallocation describe the acquiring and releasing of raw memory, aka storage. 

Construction and destruction, on the other hand, describe the beginning and
ending of an object's lifetime.

\begin{verbatim}
auto p = malloc(sizeof(T)); // allocation.
new (p) T();                // construction.
static_cast<T*>(p)->~T();   // destruction.
free(p);                    // deallocation.
\end{verbatim}

The correct answer is that an object should be destroyed prior to deallocating the memory it occupies.


\url{https://www.reddit.com/r/cpp_questions/comments/6avaey/difference_between_destruction_and_deallocation/}
\end{mdframed}

\begin{mdframed}

\url{http://www.angelikalanger.com/Articles/C++Report/Allocators/Allocators.html}

The STL allocator is one of the most overlooked topic in most C++ teachings.
They are rarely used explicitly, either via direct client code, or direct
construction of an allocator to be used in a container. It is important to
understand it, especially when you want to write a custom Allocator
(Sect.\ref{sec:allocator-user-defined})

Every standard library component, i.e. the container like \verb!std::vector!,
that may need to allocate or release storage does so through an {\bf Allocator}
(Sect.\ref{sec:container-user-defined}).
\url{http://en.cppreference.com/w/cpp/concept/Allocator}

STL allocators must satisfy a number of other requirements (Section 20.1.5 in
International Standard for C++, ISO/IEC 14882:2003), and the code to do so takes
roughly 80 editor lines.

\end{mdframed}

Allocators originally were introduced as part of the STL to handle the nasty
problem of different pointer types on PCs (such as near, far, and huge
pointers). 
\begin{itemize}

  \item  Alexander Stepanov and Meng Lee presented the Standard Template Library
  to the C++ standards committee in March 1994.

 Stepanov was requested to make the library containers independent of the
 underlying memory model (Sect.\ref{sec:memory_model}), which led to the
 creation of allocators.
 

  \item all of the STL container interfaces had to be rewritten to accept allocators - Sect.\ref{sec:container-user-defined}.
  
  Where the library had previously used pointer and reference types directly, it
  would now only refer to the types defined by the allocator.
  
  
  \item custom allocators could potentially be used to implement persistent storage STL containers,
  
  As the allocator can help to carry the state of the storage element, in theory, we can save them to disk and restore later on.
  This is not available, practically and efficiency, until C++11 standard.
  
  \item since C++11: 
  
The 2011 revision of the C++ Standard removed the weasel words requiring that
allocators of a given type always compare equal and use normal pointers. These
changes make stateful allocators much more useful and allow allocators to manage
out-of-process shared memory.

This give the programmer control over memory allocation within containers (e.g.
allocating on GPU, rather than CPU), rather than to adapt the address model of
the underlying hardware. (Sect.\ref{sec:allocator-GPU}).


  
  \item UNACCEPTED IDEA:
  the ability to use template arguments that are themselves templates
  
  Since these features could not be compiled by any existing compiler, there
  was, according to Stepanov, "an enormous demand on Bjarne [Stroustrup]'s and
  Andy [Koenig]'s time trying to verify that we were using these non-implemented
  features correctly."
  
  
\end{itemize}


Allocators now serve as a base for technical solutions that use certain memory
models, such as shared memory, garbage collection, and object-oriented
databases, without changing the interfaces. Allocators represent a special
memory model and are an abstraction used to translate the need to use memory
into a raw call for memory. They provide an interface to allocate, create,
destroy, and deallocate objects.

With allocators, containers and algorithms can be parameterized by the way the
elements are stored.
\begin{itemize}
  \item 20.1.5 section on ISO C++  [lib.allocator.requirements]
  \item 20.4.1 section [lib.default.allocator]

The provided \verb!std::allocator! in 20.4.1 is the only predefined and required
allocator imposed by [C++] on all C++ compiler implementations.

\end{itemize}
In fact, the most important part one should take note is in fact, section 20.1.5.1
\begin{verbatim}
The library describes a standard set of requirements for allocators, which are
objects that encapsulate the information about an allocation model. This
information includes the knowledge of pointer types, the type of their
difference, the type of the size of objects in this allocation model, as well as
the memory allocation and deallocation primitives for it. All of the containers
(_lib.containers_) are parameterized in terms of allocators.

\end{verbatim}

\url{https://www.codeproject.com/Articles/4795/C-Standard-Allocator-An-Introduction-and-Implement}

\subsection{std::allocator C++17}
\label{sec:std::allocator-C++17}


With C++17 the interface of std::allocator becomes a lot easier to handle. 
\begin{verbatim}
// Attributes
value_type                               T
propagate_on_container_move_assignment   std::true_ty
is_always_equal                          std::true_type

// Methods
constructor
destructor
allocate
deallocate

// Functions
operator==
operator!=
\end{verbatim}
\url{http://www.modernescpp.com/index.php/memory-management-with-std-allocator}


\section{Containers of homogeneous types}

Homogeneous containers is a template data structure that accepts a collection of
data elements of the same type. 
\begin{itemize}
  \item sequence container - Sect.\ref{sec:container_sequence}
  \item container adapter - Sect.\ref{sec:container-adaptor}
\end{itemize}

Depending on your needs, you should choose a proper container. Experts like
Scott Meyer, Bjarne Stroustrup suggest the use of \verb!queue! by default, while
others (Herb Sutter) suggests \verb!dequeue! (Sect.\ref{sec:stl:dequeue}).

\subsection{Iterate through data elements in a homogeneous container}
\label{sec:container-homogeneous-loop-through-data}

\subsection{** C++98}

In C++98, there are 3 ways to iterate through elements in a containers: 
% (1)
% using indices, (2) using an iterator, (3) using \verb!.at()! member function. 

\begin{enumerate}
  \item index-based iteration:

To make the code portable when using indexing, we should use \verb!size_type!
typedef of the corresponding container 
\begin{verbatim}
for(std::vector<int>::size_type i = 0; i != v.size(); i++) {
    /* std::cout << someVector[i]; ... */
}
\end{verbatim}
The reason is that a call to \verb!vector<T>::size()! returns a value of type
\verb!std::vector<T>::size_type!, not \verb!int!, \verb!unsigned int! or
otherwise. 
NOTE: Using \verb!size_t! as the index variable type is not portable (it can be
different from \verb!size_type! in some cases)


PROS: familiar to anyone familiar with C-style code, can loop using
  different strides (e.g. i += 2).
  
CONS: only for sequential random access containers (\verb!vector, array!,
\verb!deque!), doesn't work for \verb!list!, \verb!forward_list! or the
associative containers. Also the loop control is a little verbose (init, check,
increment). People need to be aware of the 0-based indexing in C++.

  
  \item {\it iterator} (Sect.\ref{sec:C++98_iterators})).

Exampe: replace T with the type of the data you store in the vector
\begin{verbatim}
std::vector<T>::iterator i = polygon.begin();
std::vector<T>::iterator end = polygon.end();

for(; i != end; i++){
    sum += *i;
}
\end{verbatim}

PROS: more generic, works for all containers (even the new unordered associative
containers, can also use different strides (e.g. \verb!std::advance(it, 2)!);
  
CONS: if index is needed, it needs extra work to get the index of the current
element (could be O(N) for \verb!list! or \verb!forward_list!). Again, the loop
control is a little verbose (init, check, increment).
  
  \item STL \verb!for_each! algorithm - Sect.\ref{sec:loop_for_each}
  
\begin{verbatim}
// NOTE: 'v' is the container containing data of type 'T'

std::for_each(v.begin(), v.end(), [](T const& elem) {
     // if the current index is needed:
     auto i = &elem - &v[0];

     // cannot continue, break or return out of the loop
});
\end{verbatim}

PROS: same as 2) plus small reduction in loop control (no check and increment),
this can greatly reduce your bug rate (wrong init, check or increment,
off-by-one errors).

CONS: same as explicit iterator-loop plus restricted possibilities for flow
control in the loop (cannot use \verb!continue!, \verb!break! or \verb!return!)
and no option for different strides (unless you use an iterator adapter that
overloads operator++).


  \item range-for loop
  
\begin{verbatim}
for (auto& elem: v) {
     // if the current index is needed:
     auto i = &elem - &v[0];

    // any code including continue, break, return
}
\end{verbatim}  
 
PROS: very compact loop control, direct access to the current element.

CONS: extra statement to get the index. Cannot use different strides.
 
  \item for-loop using \verb!.at()! member function.

\begin{lstlisting}
std::vector<int> items;

for(vector<int>::size_type i = 0; i < vs.size(); ++i)
{
   //Option 1
   vs[i] = "Am I faster?";
   //Option 2
   vs.at(i) = "Am I faster?";
}

\end{lstlisting}
NOTE: Using \verb!.at()! method is slowest; then come using indices, and using
iterator is the fastest. The reason is that \verb!.at()! also check for
out-of-bound indexing, so it's the slowest.

\end{enumerate}
\url{http://stackoverflow.com/questions/14373934/iterator-loop-vs-index-loop}
% 
% \begin{lstlisting}
% 
% //Option 3
% for(vector<string>::iterator it = vs.begin(); it != vs.end(); ++it)
% {
%    *it = "Am I faster?";
% }
% \end{lstlisting}
% 

% \begin{enumerate}
%   \item Use iterators
%   
% NOTE: We use \verb!const_iterator! if we don't want the data to be modified.
% 
% To loop through data elements in an STL container we use an Iterator
% (Sect.\ref{sec:C++98_iterators}).
%   
%   \item Use array index
%   
% \begin{verbatim}
% for(size_t i = 0; i < v.size(); ++i) {
%         n += v[i];
%     }
% \end{verbatim}
% \end{enumerate}
% 


\subsection{** C++11}

A simple form of the previous method is using \verb!auto! keyword
\begin{verbatim}
for ( auto i = v.begin(); i != v.end(); i++ ) {
    std::cout << *i << std::endl;
}
\end{verbatim}
or even simpler formula
\begin{verbatim}
for ( auto &i : v ) {
    std::cout << i << std::endl;
}
\end{verbatim}

We can use \verb!std::transform, std::copy, std::fill, std::for_each!
\begin{verbatim}
for(std::vector<int>::size_type i = 0; i != v.size(); i++) {
    /* std::cout << someVector[i]; ... */
}
\end{verbatim}

\subsection{** Boost}
\label{sec:foreach_boost}

\verb!BOOST_FOREACH! works for STL containers (Sect.\ref{sec:STL_containers}),
arrays, C-style strings.

\begin{verbatim}
#include <boost/foreach.hpp>

BOOST_FOREACH( vector_type::value_type& value, v ) {
    // do something with 'value'
}
\end{verbatim}

%\section{Operators}

\subsection{std::find()}
\label{sec:std::find}


 
There are really three overloaded versions of find().
\begin{enumerate}
 \item   
\end{enumerate}

\begin{lstlisting}

//use std::find to get iterator
auto itr=std::find(vct.begin(), vct.end(), 6);
if(itr==vct.end())
    return;
auto index=std::distance(vct.begin(), itr);
\end{lstlisting}

\subsection{std::upper\_bound() - binary search}
\label{sec:std::upper_bound}

Find the element \verb!val! in an ordered range \verb![first, last)!,
and returns the last position in that range where \verb!val! can be inserted
without violating the order.

PRECONDITION: \verb![first,last)! is ordered in ascending order (i.e.
increasing) according to the given operator
\begin{itemize}
  \item by default: \verb!operator<! is used for comparison

  \item we can use a different operator
  
\end{itemize}

\subsection{std::lower\_bound() - binary search}
\label{sec:std::lower_bound}

Find the element \verb!val! in an ordered range \verb![first, last)!,
and returns the first position in that range where \verb!val! can be inserted
without violating the order.

PRECONDITION: \verb![first,last)! is ordered in ascending order (i.e.
increasing) according to the given operator
\begin{itemize}
  \item by default: \verb!operator<! is used for comparison

  \item we can use a different operator
  
\end{itemize}

RETURN: The furthermost iterator 
\section{1. Sequence container}
\label{sec:container_sequence}


In sequence containers (vector, dequeue, and list): order of elements is
fixed, i.e. we can access a particular data element using the index.

\begin{itemize}
    \item vector: dynamic-size array, also accept insertion at any position
    \item dequeue (double-ended queue): provide fast insertions/deletions at
    beginning or end; and random access is quick - Sect.\ref{sec:std::dequeue}).

    \item list: provide fast insertions/deletions anywhere, but DONOT provide
    random access
  \end{itemize}

Examples of sequence containers: \verb!vector!, \verb!dequeue!, \verb!list!.

To access elements for a sequence containers, we can use
\begin{enumerate}
  \item \verb![]! operator; but there is no boundary checking for this choice
  (which can result in garbage data being returned) for std::dequeue

  \item \verb!at()!

NOTE: \verb!at()! returns a reference to the element, so we can modify the
element easily. 
\begin{verbatim}
dequeue<double> dq(5, 8.1);

int i = 3;
dq.at(i);  // which can throw 'out_of_range' exception 


using namespace std;
try {
   dq.at(i);
} catch (out_of_range&) {
  cout << "** out of range" << endl;
}

dq.at(2) = 5.2;
\end{verbatim}  


  \item \verb!front(), back()! operators: access the first and last element

  \item \verb!size()!, \verb!max_size()!, and \verb!empty()!: get the state of
  the containers
  
  \verb!max_size()! is a system and architecture-dependent value. 
  \begin{itemize}
    \item on 32-bit architecture: the max size for std::dequeue is $2^{32}-1$ 
    (which is theoretical limit of the addressable memory space on a 32-bit
    machine)
  \end{itemize}
  
  \item \verb!reserve()! : not available for std::dequeue; as the underlying
  structure for a dequeue is like a linked list rather than a raw array.
  
  
  \item \verb!push_front()!, \verb!push_back()!, \verb!pop_front()!,
  \verb!pop_back()!: in std::dequeue
  
  
  Push methods insert an element at either the back or the front of the deque,
  while pop methods remove an element:
  
  \item \verb!insert()!, \verb!erase()!, \verb!assign()!, \verb!resize()!,
  \verb!swap()!, and \verb!clear()!.
  
  
  The two methods most frequently called for performing multiple-element
  insertions and deletions are, not surprisingly, named insert() and erase().
  Both of these methods require the use of iterators, which behave much like
  pointers do for raw arrays.   

\begin{verbatim}
// delete one or (many continuous elements)
iterator erase( iterator location );
iterator erase( iterator start, iterator end );
\end{verbatim}  
  
  \item \verb!begin()!:  returns a random access interator to the first element
  of the deque.
\end{enumerate}
\url{http://www.dreamincode.net/forums/topic/35344-c-deque-tutorial/}

  
\subsection{Vector: std::vector}
\label{sec:std::vector}

Vectors are not arrays, yet they can be used to emulate like an array
(Sect.\ref{sec:stl_vector}). 
\begin{verbatim}
#include <vector>
using namespace std;
\end{verbatim}
Elements in a std::vector belong to a contiguous block of memory.


NOTE:
\begin{enumerate}
  \item Using vector to represent high dimensional array is not recommended as the
syntax is very clumsy
\begin{Verbatim}
void foo(int x, int y, int z) {
    vector< vector< vector<int> > > values(/*really painful expression here*/);
}
\end{Verbatim}

  \item its implementation uses template (Sect.\ref{sec:template-C++98}), i.e.
  they allocate their objects on the free-store exclusively (i.e.
  arrays's storage is not restricted to a specific storage class). 
  
  \item a vector object is defined with a given size. Then, a member function
  \verb!resize()! can be used to change the vector's size. This is different
  from VLA (Sect.\ref{sec:VLA}).
\end{enumerate}


\label{sec:stl_vector}

\subsection{-- insert}

To insert a new element at after a given position, as pointed by the iterator,
we use \verb!insert()!.
\begin{verbatim}
iterator insert(iterator where, const T& elem);
\end{verbatim}

To insert a new element at the end of the vector, use
\verb!push_back()! method
\begin{enumerate}
  
  \item if argument is a direct object (as opposed to a pointer), the vector
  makes a deep, independent copy of that object.
  
  \item if argument is a pointer: expect the pointer and nothing more to be
  replicated behind the scenes; any memory referenced by that pointer will be
  referenced from within the vector as well.

\end{enumerate}

\subsection{-- remove}

To remove an element at a given position, as pointed by the iterator, we use
\verb!erase()!.
\begin{verbatim}
iterator erase(iterator where);
\end{verbatim}

To remove an element at the end of the vector, use \verb!pop_back()! method.

\subsection{-- initialize/copy}

\textcolor{red}{\bf To initialize a vector}:
\begin{enumerate}
 \item C++11 standard: need to enable special flags \verb!-std=c++11!
\begin{verbatim}
std::vector<int> v {45, 63};

std::vector<int> v = {45, 63};
\end{verbatim}
or
\begin{verbatim}
std::vector<int> v(2);
v = {45, 63};
\end{verbatim}

  \item COPY (C++11):  \verb!begin()! and \verb!end()! are standard
  
\begin{verbatim}
std::vector<int> v(begin(arr), end(arr));
\end{verbatim}

  \item (traditional method) by converting from array (of contiguous elements)

CODE 01: must explicitly write the size  (take the first and last element in a
range and create a vector with everything in between)
\begin{verbatim}
int vv[2] = { 12,43 };
std::vector<int> v(&vv[0], &vv[0]+2);

int myints[] = {10,20,30,30,20,10,10,20};
std::vector<int> v(myints,myints+8);    
\end{verbatim}

CODE 02: no need to write the size
\begin{verbatim}
template <typename T, size_t N>
T* begin(T(&arr)[N]) { return &arr[0]; }
template <typename T, size_t N>
T* end(T(&arr)[N]) { return &arr[0]+N; }

int vv[] = { 12,43 };
std::vector<int> v(begin(vv), end(vv));
\end{verbatim}

  \item COPY
\begin{verbatim}
int pv[4] = { 4, 4, 4, 4};
vector<int> v (pv);
\end{verbatim}

If the size is unknown
\begin{verbatim}
int tmp[] = { 10, 20, 30 };
std::vector<int> v( tmp, tmp+3 ); // use some utility to avoid hardcoding the size here
std::vector<int> v( &tmp[0], &tmp[0]+3 ); // equivalent above

std::vector<int> v;  //assign to an existing vector
v.assign(&tmp[0], &tmp[0]+3);

vector<int> vec (tmp, tmp + sizeof(tmp) / sizeof(tmp[0]) );
\end{verbatim}

 \item Create your own function
\begin{verbatim}
template <typename T>
class make_vector
{
public:
   typedef make_vector<T> my_type;
   my_type& operator << (const T& val) {
   	data_.push_back(val);
    return *this;
  }
  operator std::vector<T>() const {
    return data_;
  }
  
private:
  std::vector<T> data_;
}



std::vector<int> v = make_vector<int>() << 1 << 2 << 3;
\end{verbatim}
\end{enumerate}

\subsection{-- convert to array}

\textcolor{red}{\bf Mapping a vector to an array}: 



\begin{enumerate}
  \item NO COPY: As vector guarantees to store their elements contiguously, i.e.
  \verb!&v[n] == &v[0] + n! for all $0\le n\le$ \verb!v.size()!. 
  
NOTE: use \verb!&v[0]! with \verb!v! is vector name.
\begin{verbatim}
std::vector<double> v;
double* a = &v[0];
\end{verbatim}
The pointer pointing to the first element in the vector.

IMPORTANT: This pointer is only valid as long as the vector is not re-allocated
or destroyed. Re-allocation happens automatically when you insert more elements
than it will fit into the vector remaining capacity, i.e. v.size() +
NumberOfNewElements >  v.capacity(). A solution with this is to reserve, before
getting the pointer, we can use \verb!v.reserve(NewCapacity)!.
\begin{verbatim}
std::vector<double> v;
v.reserve(40);

double* a =&v[0];
\end{verbatim}
  
Example: using a vector and pass as an array pointer
\begin{verbatim}
void f( double a[]);

//can be called with
vector <double> v;
v.push_back( 1.23 )
f( &v[0] );
\end{verbatim}
 

 \item COPY (more robust)
 
\begin{verbatim}
double arr[100];
std::vector<double> v;
std::copy(v.begin(), v.end(), arr);
\end{verbatim}
Make sure \verb!arr! is big enough

\end{enumerate}

\subsection{-- operation}

\textcolor{red}{\bf Check the existence of an element in the vector}:
\begin{enumerate}
  \item Unsorted vector: v=vector, value=we want to check
\begin{verbatim}
#include <algorithm>

if (std::find(v.begin(), v.end(), value) != v.end)
   ...//found
\end{verbatim}

  \item Sorted vector:
\begin{verbatim}
#include <algorithm>

if (std::binary_search(v.begin(), v.end(), value))
   ...//found
\end{verbatim}
\end{enumerate}

\subsection{-- multi-dimension}

\textcolor{red}{\bf Multi-dimensional vector}: 
\begin{verbatim}
std::vector< std::vector<data> > your_2D_vector;
\end{verbatim}
NOTE: We need to have a space between the two \verb!>! due to the limitation of
syntax detection in C++99 and earlier. Since C++11 standard, this limitation has
been resolved.

We can create a 2D vector and access its data like this
\begin{verbatim}
// Create
vector< vector<int> > vec(4, vector<int>(4));
    
    // Write
vec[2][3] = 10;
    // Read
int a = vec[2][3];
\end{verbatim}

To create a 2D vector
\begin{verbatim}
for (int i = 0; i < 10; i++) {
    std::vector< data> your_1D_vector;
    //add data
    your_1D_vector.push_back(10);
    ...
    
    your_2D_vector.push_back(your_1D_vector); // Add an empty row
}

\end{verbatim}

Or we can add a number of empty rows first
\begin{verbatim}
for (int i = 0; i < 10; i++) {
    vec.push_back(vector<int>()); // Add an empty row
}
\end{verbatim}

The loop through and add real data to each row
\begin{verbatim}

for (int j = 0; j < 20; j++) {
    for (int i = 0; i < vec.size(); i++) {
        vec[i].push_back(i * j); // Add column to all rows
    }

    // You could also use iterators:
    for (vector< vector<int> >::iterator it = vec.begin(); it != vec.end(); ++it) {
        it->push_back(j); // Add column to all rows
    }
}
\end{verbatim}

If all columns are the same length, then we can use 1D vector
\begin{verbatim}
const int rows = 10;
const int columns = 20;

vector<int> vec;
vec.resize(rows * columns);

for (int row = 0; row < rows; row++) {
    for (int col = 0; col < columns; col++) {
        vec[row * columns + col] = row * col;
    }
}
\end{verbatim}




\subsection{std::list (double linked list)}
\label{sec:std::list}

Lists are sequence containers that allow {\it constant time} insert and erase
operations anywhere within the sequence, and iteration in {\it both directions}.

List containers are implemented as doubly-linked lists.
If you really need single linked list, use \verb!forward_list!
(Sect.\ref{sec:std::forward_list}).

IMPORTANT:  In std::list, each of the elements they contain in different and
unrelated storage locations.

\subsection{std::forward\_list}
\label{sec:std::forward_list}


Forward lists \verb!std::forward_list! are implemented as singly-linked list.
It allow constant time insert and erase operations anywhere within the sequence.
For double-linked list, use \verb!std::list! (Sect.\ref{sec:std::list}).

You can write your own class for single-linked list as given in example:
\url{http://www.sanfoundry.com/cpp-program-implement-single-linked-list/}

\url{http://www.cplusplus.com/reference/forward_list/forward_list/}

\subsection{unordered\_set}
\label{sec:std::unordered_set}



\section{2. Container adaptor}
\label{sec:container-adaptor}

{\bf A container adaptor} are behavior added over sequence containers. The added
behavior can be a stricter behavior (i.e. stack that allows pop/push item on
the top, no random insert). A stack could be built over a \verb!std::vector!
(Sect.\ref{sec:std::vector}), a \verb!list! (Sect.\ref{sec:std::list}) or
\verb!dequeue!; GCC use \verb!dequeue!.

Examples of container adaptors: \verb!stack!, \verb!queue!,
\verb!priority_queue!.


\subsection{Stacks}

\subsection{Queues}

A queue is a container adaptor using FIFO approach (First-In-First-Out). 
\begin{lstlisting}
#include <queue>
\end{lstlisting}

To add new elements into the queue, we use \verb!push()! method
\begin{verbatim}
std::queue<data_type> myqueue;

myqueue.push(data_element);
myqueue.push(data_element);
\end{verbatim}
The first element to be added to the queue is called the \verb!front!, and the
last one is called the \verb!back!. To remove the first element from the queue,
we use \verb!pop()! method.

The reference to the first element can be done using \verb!fron()! method, and
to the last element can be done using \verb!back()! method.
\begin{lstlisting}
queue<int> myqueue;

myqueue.push(74);
myqueue.push(5);

std::cout << myqueue.front() << std::endl; // 74
myqueue.front() -= myqueue.back();

std::cout << myqueue.front() << std::endl; // 69
\end{lstlisting}
Modying this reference effectively modify the value contained in the queue;

\subsection{Dequeue}
\label{sec:stl:dequeue}
\label{sec:std::dequeue}

\verb!std::dequeue! is a sequence container (Sect.\ref{sec:container_sequence}),
with fast insertions/deletions at beginning or end; and random access is quick.
Accessing elements in a dequeue follows the APIs for sequence container
(Sect.\ref{sec:container_sequence})

This is different way to create/initialize a 'dequeue' object
\begin{verbatim}
//dequeue() ;  //default constructor
dequeue<int> dq1;   //create an empty dequeue of ints

//dequeue(const dequeue & d);  // copy constructor
dequeue<int> dq2(dq1) ; //create a copy 

dequeue<double> dq3(10); // 10 elements, all set to default (0.0)
dequeue<double> dq3(10, 5.1) ; // 10 elements; all set to 5.1

//dequeue(iterator start, iterator end);
double array1[] = {0.1, 4.0, 2.5, 2.5, 2.9};
dequeue<double> dq5(array1, array1+4) ; // create a dequeue using an array
\end{verbatim}

Elements in a std::dequeue are stored in small chunks of memory, rather than a
continuous blocks as in std::vector (Sect.\ref{sec:std::vector}). So, additional
memory can be allocated for the deque by allocating additional chunks, rather
than having to reallocate space for the entire container all at once.

\url{http://www.dreamincode.net/forums/topic/35344-c-deque-tutorial/}

\subsection{Priority queue}
\label{sec:priority-queue}

\cprotect\subsection{Heaps (\verb!priority_queue!)}

\subsection{trees (set)}
  
\section{3. Associative container}
\label{sec:container_associative}

Associative containers (set, multiset, map, and multimap)
it means there is an order between elements.
% is the one that you can look up a
% value, by writing
% \begin{verbatim}
% i = S.find("key"); //
% \end{verbatim}
 
Also, they support bidirective iterators
  \begin{itemize}

    \item map (dictionary): consist of a key/value pairs where the \verb!key! is
    used to order the sequence and unique. The associated data for each value
    has some meaning, e.e. it represents the number of times the key appears in
    the dataset.

    \item set: an ascending container of unique elements, i.e. like \verb!map!
    but there is no associated \verb!value! to each \verb!key!.
  \end{itemize}

  
\subsection{std::set}
\label{sec:std::set}


Sets are containers that store unique elements following a specific order.
The value is also a key, and uniquely identifies the  element.
Its siblings (those using keys to order data) are \verb!std::map!
(sect.\ref{sec:map}), \verb!std::multiset! (Sect.\ref{sec:multiset}), and 
multimap (Sect.\ref{sec:multimap}).

\begin{enumerate}
  \item Order of elements
  
  
  \item Access: 
set containers are generally slower than \verb!unordered_set!
(Sect.\ref{sec:std::unordered_set}) containers to access individual elements by
their key, but they allow the direct iteration on subsets based on their order.

\end{enumerate}

\verb!std::set! internal structure calls for a \verb!less! functor that given
two  objects to determine which should be placed in front of the other in an ordered list.
The default \verb!less! operator employ the '<' operator to compare two objects.
However, for a complicated data structure in that you haven't defined the '<' operator), then it will fails, and you need to extend the 
\begin{lstlisting}
template <class T> struct less {
  bool operator() (const T& x, const T& y) const {return x<y;}
  };
\end{lstlisting}

Example to implement that '<' operator for \verb!less! (Sect.\ref{sec:extend-data-support-for-container})
\url{http://cppisland.com/?p=275}


\url{http://lafstern.org/matt/col1.pdf}

\subsection{multi\_set}
\label{sec:multiset}


\subsection{std::map (associative arrays)}
\label{sec:map}
\label{sec:std::map}

A \verb!std::vector<X>! stores a whole bunch of X objects
(Sect.\ref{sec:std::vector}).
But if you have a \verb!std::map<X, Y>!, what it actually stores is a whole
bunch of \verb!std::pair<const X, Y>!s (Sect.\ref{sec:std::pair}). That's
exactly what a map is - it pairs together the keys and the associated values.
\begin{enumerate}
  \item an ordered structure w.r.t keys (i.e. keys always be increasing)
  
  \item unique keys only
  
  \item fast \verb!find()! method (with O(log n))
  
  \item unique indexing operator \verb!map_name[key]! which is also fast
\end{enumerate}


\begin{lstlisting}
map<string, int> portfolio;

portfolio.insert(make_pair(string("LU"), 400));
portfolio.insert(make_pair(string("AAPL"), 80));
portfolio.insert(make_pair(string("GOOG"), 6500));
\end{lstlisting}

When you iterate over a \verb!std::map!, you're iterating over all of these
\verb!std::pair!s. When you dereference one of these iterators, you get a
\verb!std::pair! containing the key and its associated value.

Now the type \verb!std::pair! gives you access to its elements through two
members:
\verb!first! and \verb!second!. So if you have a \verb!std::pair<X, Y>! called
p, \verb!p.first! is an X object and \verb!p.second! is a Y object.

\subsection{-- insert/add}

\textcolor{red}{There are different ways to insert/add a new map element}: 
\begin{enumerate}
  \item assignment: operator[]
  
  \item \verb!.insert()! method: not recommended
  
  The difference with operator[] is that the \verb!.insert()! can tell whether
  the element is inserted into the map. Also, if your class has no default
  constructor, you are forced to use insert.
  
  So insert will not change the value if the key already exists, the [] operator
  will.
\end{enumerate}

Example: Using \verb!int! as map index
\begin{verbatim}
#include <map>
#include <utility>

int main() {
  map<int, string> Employees;
// 1) Assignment using array index notation
   Employees[5234] = "Mike C.";
   Employees[3374] = "Charlie M.";
   Employees[1923] = "David D.";
   Employees[7582] = "John A.";
   Employees[5328] = "Peter Q.";
   
   for( map<int,string>::iterator ii=Employees.begin(); ii!=Employees.end(); ++ii)
   {
       cout << (*ii).first << ": " << (*ii).second << endl;
   }
}
\end{verbatim}
  

Example: Using \verb!std::string! as map index
\begin{verbatim}
 map<string, int> Employees;
\end{verbatim}

Example: We can also define our own comparison 
\begin{verbatim}
struct cmp_str 
{
   bool operator()(char const *a, char const *b) 
   {
      return std::strcmp(a, b) < 0;
   }
};

int main()
{
   map<char *, int, cmp_str> Employees;

   // 1) Assignment using array index notation
   Employees["Mike C."] = 5234;
   Employees["Charlie M."] = 3374;

   // 2) Assignment using member function insert() and STL pair
   Employees.insert(std::pair<char *,int>("David D.",1923));
 
   // 3) Assignment using member function insert() and "value_type()"
   Employees.insert(map<char *,int>::value_type("John A.",7582));

   // 4) Assignment using member function insert() and "make_pair()"
   Employees.insert(std::make_pair((char *)"Peter Q.",5328));

   cout << "Map size: " << Employees.size() << endl;

   for( map<char *, int, cmp_str>::iterator ii=Employees.begin(); ii!=Employees.end(); ++ii)
   {
       cout << (*ii).first << ": " << (*ii).second << endl;
   }
\end{verbatim}

\subsection{-- loop through key:value}

In C++17, we use structured binding (Sect.\ref{sec:structured-binding}).

\begin{lstlisting}

std::map< foo, bar > testing = { /*...blah...*/ };

for ( const auto& [ k, v ] : testing )
{
  std::cout << k << "=" << v << "\n";
}
\end{lstlisting}

In C++11, we use \ref{sec:loop_range-based-for-loop}

\begin{lstlisting}
map<K, V>::value_type

for (auto& kv : myMap) {
    std::cout << kv.first << " has value " << kv.second << std::endl;
}
\end{lstlisting}


\subsection{-- search}

std::find() if it applies to \verb!std::map! object, it searches the container
for an element with a key equivalent to k and returns an iterator to it if
found, otherwise it returns an iterator to map::end (Sect.\ref{sec:std::find}).

\begin{lstlisting}
std::map<char,int> mymap;
std::map<char,int>::iterator it;
  
it = mymap.find('b');

if (it != mymap.end())
    mymap.erase (it);
\end{lstlisting}  


NOTE: map::count, can be used to just check whether a particular key exists.
As a key is unique, map::count(key) can only returns either 0 or 1.
\begin{verbatim}
if (mymap.count('b') == 0)
  // not found
  mymap.
else
  // found
  
\end{verbatim}

\subsection{-- return index of a key}

There is no so-called index in a std::map, however, we can find the offset from
the current key to the beginning of the map. 

{\bf OPTION 2}: 
\begin{verbatim}
std::map<string, int>::iterator it = myMap.find("myKey");
it - vec.begin()
\end{verbatim}


{\bf OPTION 1}: We can use the \verb!std::distance! 
function to find the index

\begin{verbatim}
std::map<string, int> myMap;
// Populate myMap with a bunch of items...

myElement = myMap["myKey"];
// Now I need to get the index of myElement in myMap
std::distance(std::begin(myMap), myMap.find("myKey"))
\end{verbatim}


\cprotect\subsection{\verb!hash_map! and \verb!hash_set!}
\label{sec:hash_map}
\label{sec:hash_set}

While not part of the C++ Standard Template Library, 
\verb!hash_map! and
\verb!hash_set! are widely used to improve searching time.


\subsection{multimap}
\label{sec:multimap}



\subsection{std::list< std::pair<X, Y> >}
\label{sec:list_pairs}

This is different from std::map (Sect.\ref{sec:std::map})
\begin{enumerate}
  \item a simple sequence of std::pair; and data remains in the order we put it
  
  \item allow duplicate of the same (X,Y)
  
  \item finding a particular key is slower (i.e. O(N)) 
  
  \item offer no \verb!splice()! method
\end{enumerate}



\section{Containers of heterogeneous types}	


\subsection{Boost.Any}
\label{sec:Boost.Any}

\verb!Boost.Any! can do the job of holding arbitrary types, using inheritance,
but it cost performance drop (runtime costs), and as you don't have type informations,
you need to do dangerous type cast to get back the original type.

\subsection{Boost.Tuple}
\label{sec:Boost.Tuple}

\verb!Boost.Tuple! is another solution. A {\it tuple} is a basic data structure
that can hold an element of any data type. However, it lacks an algorithm to
store/rerieve data from tuples, and to pass them around. This can be achieved
with {\bf Boost.Fusion}.

\subsection{std::tuple (C++11)}
\label{sec:std::tuple}

A new data structure to hold a group of heterogeneous data structure is added to
C++11 called \verb!std::tuple!. Before this, you either have to use pairs of
pairs data structure (e.g. \verb!pair<T1, pair<T2,T3>>!); or write your own
class; or using a component in the Boost library called Boost.Tuple
(Sect.\ref{sec:Boost.Tuple}).

\begin{verbatim}
#include <tuple>
\end{verbatim}

Tuples are objects that hold a collection of elements, i.e. pack elements of
different types together in the same object.
\url{http://www.cplusplus.com/reference/tuple/}

\textcolor{red}{It's similar to plain-old type} \verb!struct! of C. However,
instead of accessing by name of data element; we access using the order in the
{\it tuple}.

Example:
\begin{lstlisting}
#include <tuple> //std::tuple, std::get, std:tie, std::ignore
                 //std::move

//create new variable either directly 
//      or using std::make_tuple()
std::tuple<int,char> foo (10,'x');
auto bar = std::make_tuple ("test", 3.1, 14, 'y');

// access element by index using std::get()
// which returns a reference to the data
// so we can read or write
// NOTE: To write, the tuple must be non-const
std::get<2>(bar) = 100;      // access element
std::tuple<int, int, int> tpl;
std::get<0>(tpl) = 1;
std::get<1>(tpl) = 2;
std::get<2>(tpl) = 3;

// unpack elements using std::tie
int myint; char mychar;
std::tie (myint, mychar) = foo;                            // unpack elements
std::tie (std::ignore, std::ignore, myint, mychar) = bar;  // unpack (with ignore)

// move items out of a tuple using std::move
auto movedTo = std::get<0>(std::move(myTuple));

\end{lstlisting}

\subsection{std::set< std::pair< X, Y> > }
\label{sec:set_pairs}

Set (sect.\ref{sec:std::set})

\subsection{pair}
\label{sec:std::pair}

\begin{verbatim}
#include <utility>
\end{verbatim}

\verb!std::pair! is a two-element tuple (Sect.\ref{sec:std::tuple}).

It is nothing more than a template
\verb!struct! with two fields
\begin{verbatim}
template<class U, class V>
struct pair
{
 U first;
 V second;
 pair(const U& first = U(), const V& second = V()) : first(first),
   second(second) { }

template <class U, class Y>
pair<U, V> make_pair(const U& first, const V& second
\end{verbatim}

So, \verb!std::pair<T1, T2>! can hold two objects of different types.

\begin{verbatim}
std::pair<int, int> mypair;

mypair = std::pair(4, 6);

mypair = std::make_pair(5,6);
\end{verbatim}

As a struct, when handling a pair , you're free to directly access the first and
second fields, since there are no private access modifiers in place to prevent
you.
All of the associative containers ( map , hash\_map , and multimap ) require a
pair be used to insert new data.


% \subsection{vector}

%Sect.\ref{sec}

%\subsection{queue}

\section{Define a new type using a specific value}


Template \verb!integral_constant! defines a type, keyword \verb!constexpr! defines a constant.

We can define a type that has only 1 value using \verb!std::integral_constant!.
Usecases:
\begin{enumerate}
  
  \item One of them is tag dispatch.

\begin{verbatim}
template <typename T>
auto foo(T value) {
    // Calls the correct implementation function, which return different types.
    // foo's return type is `int` if it calls the `std::true_type` overload
    // and `double` if it calls the `std::false_type` overload
    return foo_impl(value, std::is_arithmetic<T>{});
}

template <typename T>
int foo_impl(T value, std::true_type) {
    // Implementation for arithmetic values
}

template <typename T>
double foo_impl(T value, std::false_type) {
    // Implementation for non-arithmetic values
}
\end{verbatim}
 
  \item 
\end{enumerate}

For example \verb!std::true_type! is 
\begin{verbatim}
#include <type_traits>


std::integral_constant<bool, true>
\end{verbatim}

\begin{verbatim}
template<typename T>
void use_impl(const T&, std::false_type)
{
}

template<typename T>
void use_impl(const T&, std::true_type)
{
}

template<typename T>
void use(const T& v)
{
   use_impl(v, typename std::is_integral<T>::type());
}
\end{verbatim}

\url{http://ideone.com/469YTq}

Example:
\begin{lstlisting}

#include <type_traits>

typedef std::integral_constant<int, 2> two_t;

typedef std::integral_constant<int, 4> four_t;
        
\end{lstlisting}

\section{TIPS}

\subsection{Using array as an element in the STL}

These are wrong
\begin{verbatim}
std::pair <int, int[4]> mypair;

std::vector< int[3]> myvect;
\end{verbatim}

You can't use C-style array in STL container. The solution is
\begin{enumerate}
  \item use \verb!std::vector! instead
  \item use a map of pointers to array of 3 elements. NOTE: Don't map to an
  \verb!int[]!, but map to an \verb!int*!
\begin{verbatim}
int red[3] = {1,0,0};

std::map<int, int(*)[3]> colours;

colours.insert<std::pair<int, int(*)[3]>((GLUT_LEFT_BUTTON, &red));
\end{verbatim}

  \item use \verb!boost.tuples! instead of array of 3 elements 
  \item use a struct that wrap the array
\begin{verbatim}
struct Triple
{
  int color[3];
}

std::map<int, Triple> colours;

std::pair<int, Triple> colour;
\end{verbatim}
  
     \item use C++11 TR1 \verb!std::tr1::array!
\begin{verbatim}
typedef std::tr1::array<int, 3> Triple;
Triple red = { 1, 0, 0};

\end{verbatim}
\end{enumerate}


\subsection{Max/Min}

Use C++11/C++0X compiler flag
\begin{verbatim}
#include <algorithm>

auto it = max_element(std::begin(data), std::end(data);
\end{verbatim}

\subsection{Count number of elements, maching condition: std::count\_if}



\begin{verbatim}
bool is_greater_than_5(int value)
{
    return (value > 5);
}
 
 
vector<int> numbers { 1, 2, 3, 4, 5, 10, 15, 20, 25, 35, 45, 50 };


auto greater_than_5_count = std::count_if(numbers.begin(), numbers.end(), is_greater_than_5);
\end{verbatim}

It is possible to simplify the code by using a lambda expression to specify the user-defined predicate function object.
\begin{verbatim}

auto great_than_5_count = count_if(numbers.begin(), numbers.end(), [](int x) { return (x > 5); });
    
    
\end{verbatim} 
 
\subsection{Fast lookup}

Data structures with O(1) lookup (ignoring the size of the key)
\begin{enumerate}
  \item  Arrays have O(1) lookup. 
  
  \item Hashtable (\verb!std::unordered_map!) for c++11 has O(1) lookup. (Amortized, but more or less constant.)
\end{enumerate}

For complex types, balanced trees will be fine at O(log n), or sometimes you can
get away with a patricia trie at O(k).



\section{Extending the supporting data type for a container}
\label{sec:extend-data-support-for-container}

Suppose you have the data type
\begin{verbatim}
struct ProductKey
{
	std::string brand;
	std::string model;
	int type;
	int version;
};
\end{verbatim}

and you want to store many instances of such type int \verb!std::set! (Sect.\ref{sec:std::set})
\url{http://cppisland.com/?p=275}



\url{http://cppisland.com/?p=275}

\section{User-defined containers}
\label{sec:container-user-defined}


Every STL container is capable to take an allocator as argument. 
\begin{lstlisting}
template<
    class T,
    class Allocator = std::allocator<T>
> class vector;
\end{lstlisting}

Then allocations will be performed using this allocator, and not the standard
one (Sect.\ref{sec:std::allocator}).
\url{https://en.cppreference.com/w/cpp/container/vector}

This is useful e.g. for allocating objects of the same size in a pool, to
improve performance, or might be necessary if there's a special area of memory
where your objects need to live.

\url{https://stackoverflow.com/questions/31358804/whats-the-advantage-of-using-stdallocator-instead-of-new-in-c}


\subsection{new() operator vs. operator new()}
\label{sec:new()-operator}
\label{sec:operator-new}

The method for dynamically allocating memory in C/C++ is to use the \verb!new!
operator. This operator can be overloaded. To distinguish between the default
operator and an overloaded version, the default is called "new operator" and the
overloaded version is called "operator new".

The new operator starts by using operator new to allocate memory, but then it
invokes the constructor for the right type of object, so the result is a real
live object created in that memory.
 If that object contains any other objects (either embedded or as base classes)
 those constructors as invoked as well.
 
\begin{verbatim}
// "operator new"
class Foo
{
public:
        void* operator new( size_t );
}

// when you use it
// then  it is called "new operator"
// HERE
//    new Foo() calls Foo::operator new()
Foo* foo = new Foo();
\end{verbatim}


\verb!malloc! allocates raw memory (which works for raw pointer, and should not be used
for C++), while \verb!new()! allocates memory AND calls the constructor of the
object you're allocating memory for (which works for C++ class/objects).

\begin{verbatim}
void * ptr;

// void *operator new(size_t);
ptr = new T[N]; 
\end{verbatim}
doesn't allow you control over what constructors of \verb!T! class are called.

\verb!operator new()! is a function that allocates raw memory -- at least
conceptually, it's not much different from \verb!malloc()!. However, it's better
as we can cast to the type
 
\begin{verbatim}
  // create a pointer, pointing to a memory allocated for 100 elements of 'char' type
char *x = static_cast<char *>(operator new(100));
\end{verbatim}

There's also a separate operator \verb!new[]! that's used to allocate memory for arrays
-- but you're almost certainly better off ignoring that whole mess completely.




C++ offers the global operator \verb!new! in four variations.
\begin{enumerate}
  \item 

\begin{verbatim}
// the first two 
//    throw std::bad_alloc
//  if it cannot allocate memory
\end{verbatim}
\begin{lstlisting}
void* operator new  (std::size_t count );
void* operator new[](std::size_t count );

\end{lstlisting}

   \item 

The last two variations return a null pointer, it cannot allocate the memory.
   
\begin{lstlisting}
void* operator new  (std::size_t count, const std::nothrow_t& tag);
void* operator new[](std::size_t count, const std::nothrow_t& tag);

\end{lstlisting}

\end{enumerate}
The variants 2 and 4, which are designed for C arrays.

If we want to overrite the \verb!new! operator, it's quite convenient that is
sufficient to overload only the version 1 because the versions 2 - 4 use the
version 1 internally (Sect.\ref{sec:new()-overwrite-it}) and Sect.\ref{sec:new[]-operator}.


\subsection{new[]  vs. new() operator}
\label{sec:new[]-operator}

If you use new[], you must use delete[], but how can delete[] know how much
destructors to call? There might be an array of 2 instances, or one of 2000
instances? What some (possibly most or all) compilers do, is to store the number
of instances right before the memory it returns to you.
\begin{verbatim}
So if you call new[5], then new will allocate memory like this:

+---+-----------+-----------+-----------+-----------+-----------+
| 5 | instance1 | instance2 | instance3 | instance4 | instance5 |
+---+-----------+-----------+-----------+-----------+-----------+
\end{verbatim}
If you use new, then delete, the delete call will execute one destructor.

So \verb!delete[]! just call \verb!delete! 5-times; and \verb!new[]! calls
\verb!new! 5-times in this case.
If you overload \verb!new! then it is enough to effect \verb!new[]!.

\url{https://stackoverflow.com/questions/7510603/what-is-difference-between-new-and-new1}


Example:
\begin{lstlisting}
class MyClass{
  float* p= new float[100];
};

class MyClass2{
  int five= 5;
  std::string s= "hello";
};


int main(){
    
    int* myInt= new int(1998);  // one instance, get value 1998
    
    double* myDouble= new double(3.14);
    
    double* myDoubleArray= new double[2]{1.1,1.2}; // pointer pointing to the first instance of the
                               // two instances (adjacent to each other),
                               //and get values 1.1, 1.2 respectively
    
    MyClass* myClass= new MyClass;  
    MyClass2* myClass2= new MyClass2;
    
    delete myDouble;
    delete [] myDoubleArray;  // this is how we delete a pointer pointing to an array
    delete myClass;
    delete myClass2;
    
//  getInfo();
    
}
\end{lstlisting}
Sect.\ref{sec:delete-operator} discusses \verb!delete! operator.

\verb!new()! does allocation (raw memory) [using std::allocator], and construct
the object in one step.

Sometimes you want to separate allocation and construction into two steps (and
similarly to separate destruction and deallocation into two steps). NOTICE the
proper order
\begin{verbatim}

1. reserve the memory

2. create the object on that memory

create more if needed on that memory [if the memory is full, go to step 1. first]

3. destroy the object(s)

4. deallocate the raw memory [then other threads/processes can use the memory]

\end{verbatim}

Sect.\ref{sec:Allocator} talks about allocation and deallocation.


\subsection{placement operator new}

Technically, a placement \verb!operator new! is any operator new that takes additional
arguments besides the size of the memory needed.

For example, you want the memory to be allocated on a previously allocated memory region
\begin{lstlisting}
char* ptr = new char[sizeof(T)]; // allocate memory
T* tptr = new(ptr) T;            // construct in allocated storage ("place")
tptr->~T();                      // destruct
delete[] ptr;                    // deallocate memory
\end{lstlisting}
\url{https://en.cppreference.com/w/cpp/language/new}

In Visual C++:  the default implementation just returns the address passed into the call
\begin{lstlisting}
//from new.h
inline void* operator new( size_t, void* where )
{
   return where;
}
\end{lstlisting}

The C++ standard says
\begin{verbatim}
§18.4.​1.3 Placement forms

These functions are reserved, a C++ program may not define functions that
displace the versions in the Standard C++ library.

\end{verbatim}

The only purpose of the allocation and deallocation operators is to allocate and
deallocate memory, so when given memory nothing more should be done.
(The standard specifically notes that these functions "Intentionally perform no
other action.")



\subsection{overload new() operator}
\label{sec:new()-overload-it}



\verb!new! and \verb!delete! operators can be overloaded
(Sect.\ref{sec:overload}) globally or they can be overloaded for specific
classes.
\begin{itemize}

  \item   If these operators are overloaded using member function for a class,
  it means that these operators are overloaded only for that specific class.


  \item   If overloading is done outside a class (i.e. it is not a member
  function of a class), the overloaded ‘new’ and ‘delete’ will be called anytime
  you make use of these operators (within classes or outside classes). This is
  global overloading.

\end{itemize}

\url{https://www.geeksforgeeks.org/overloading-new-delete-operator-c/}


\subsection{overload new() operator per class}
\label{sec:new()-overload-it-per-class}

\begin{lstlisting}
// OVERLOAD

class MyClass 
{
   void * operator new(size_t size) 
    { 
    /* this overloaded 'new'
      do some extra work
      before calling the standard new:   ::new
    */
        cout<< "Overloading new operator with size: " << size << endl; 
        void * p = ::new student();  
        //void * p = malloc(size); will also work fine 
      
        return p; 
    } 
  
    void operator delete(void * p) 
    { 
        cout<< "Overloading delete operator " << endl; 
        free(p); 
    } 
}
\end{lstlisting}

Once we define such \verb!new! and \verb!delete! inside the class, then if we call 'new' for a pointer of that class
\begin{lstlisting}
int main() 
{ 
    MyClass * p = new MyClass("Yash", 24);  // it will call MyClass->new
  
    p->display(); 
    delete p; 
} 
\end{lstlisting}

\subsection{overload new() operator globally}
\label{sec:new()-overload-it-globally}

\begin{lstlisting}
#include<iostream> 
#include<stdlib.h> 
  
using namespace std; 
void * operator new(size_t size) 
{ 
    cout << "New operator overloading " << endl; 
    void * p = malloc(size); 
    return p; 
} 
  
void operator delete(void * p) 
{ 
    cout << "Delete operator overloading " << endl; 
    free(p); 
} 
  
int main() 
{ 
    int n = 5, i; 
    int * p = new int[3]; 
  
    for (i = 0; i<n; i++) 
    p[i]= i; 
  
    cout << "Array: "; 
    for(i = 0; i<n; i++) 
        cout << p[i] << " "; 
          
    cout << endl; 
  
    delete p; 
} 
\end{lstlisting}


\subsection{overrite new() operator}
\label{sec:new()-overwrite-it}


The only reason for overriding the operator \verb!new(size_t, void*)! could be to add
tracing information, but I think the need for that will be pretty low.


BEST PRACTICE: Never ever try to overload new/delete globally.
\begin{lstlisting}
// This is overload [NOT override]
void* operator new (unsigned int size, const char* filename, int line)
{
    void* ptr = new void[size];
    memleakfinder.AddTrack(ptr,size,filename,line);
    return ptr;
}

// "new x" calls the default new 
//     while "new (filename, line) x" calls your overloaded version
\end{lstlisting}
Either have them in a base class and derive all your objects from this class or
use a namespace or a template allocator parameter.
Because in case your program is more than a single file and using STL or other
libraries you are going to screw up.

BEST PRACTICE: It is also common to override

\begin{mdframed}
To write an overloaded, it's better to know how it is implemented originally.

In VS2005
\begin{lstlisting}
void * operator new(size_t size) _THROW1(_STD bad_alloc)
{       // try to allocate size bytes
   void *p;
   while ((p = malloc(size)) == 0)
    if (_callnewh(size) == 0)
     {       // report no memory
        static const std::bad_alloc nomem;
        _RAISE(nomem);
     }

     return (p);
}
\end{lstlisting}
\end{mdframed}

\textcolor{red}{Overload globally}: If we want to overrite the \verb!new!
operator, it's quite convenient that is sufficient to overload only the version
1 because the versions 2 - 4 use the version 1 internally.
\begin{verbatim}
void* operator new  (std::size_t count );
\end{verbatim}

If you overload \verb!new!, you'll also want/need to overload the matching
operator delete as well.

Example:
\begin{verbatim}
void * operator new(size_t size) 
{ 
    cout << "New operator overloading " << endl; 
    void * p = malloc(size); 
    return p; 
} 

void operator delete(void * p) 
{ 
    cout << "Delete operator overloading " << endl; 
    free(p); 
} 
  
int main() 
{ 
    int n = 5, i; 
    int * p = new int[3]; 
  
    for (i = 0; i<n; i++) 
    p[i]= i; 
  
    cout << "Array: "; 
    for(i = 0; i<n; i++) 
        cout << p[i] << " "; 
          
    cout << endl; 
  
    delete p; 
} 
\end{verbatim}


\textcolor{red}{Overload per-class (C++)}: 
\begin{lstlisting}
class student 
{ 
    string name; 
    int age; 
    
public: 
    student() 
    { 
        cout<< "Constructor is called\n" ;  
    } 
    
    void * operator new(size_t size) 
    { 
        cout<< "Overloading new operator with size: " << size << endl; 
        void * p = ::new student();  
        //void * p = malloc(size); will also work fine 
      
        return p; 
    } 
    void operator delete(void * p) 
    { 
        cout<< "Overloading delete operator " << endl; 
        free(p); 
    }     
};


int main() 
{ 
    student * p = new student("Yash", 24); 
  
    p->display(); 
    delete p; 
} 
\end{lstlisting}
\url{https://www.geeksforgeeks.org/overloading-new-delete-operator-c/}
\url{https://stackoverflow.com/questions/3675059/how-could-i-sensibly-overload-placement-operator-new}


\subsection{overwrite delete operator}
\label{sec:delete()-overwrite-it}

If we want to overrite the \verb!delete! operator, it's quite convenient that is
sufficient to overload only the version 1 because the versions 2 - 6 use the
version 1 internally (Sect.\ref{sec:delete-operator}).

\begin{verbatim}
void operator delete  (void* ptr); 
\end{verbatim}



\subsection{default allocator}
\label{sec:allocator-default}

With a raw memory allocator, you can allocate a certain amount of memory, which
determines your capacity. Then, as the user adds items to the vector (using the
constructor of their choice), you can construct objects in place in this memory.


%The default allocator uses operator new
%to allocate memory (Sect.\ref{sec:new()-operator}).
\begin{lstlisting}
std::vector<X> v;
v.reserve(4);        // (1)
v.push_back( X{} );  // (2)
v.push_back( X{} );  // (3)
v.clear();           // (4)
\end{lstlisting}

Here line (1) must allocate enough memory for four objects, but not construct
them yet. Then lines (2) and (3) must construct objects into the allocated
memory. Then line (4) must destroy those objects, but not deallocate the memory.
Finally, in the vector's destructor, all the memory can be deallocated.

This is often implemented as a thin layer around the C heap allocation
functions,[17] which are usually optimized for infrequent allocation of large
memory blocks. This approach may work well with containers that mostly allocate
large chunks of memory, like vector and deque.[15] However, for containers that
require frequent allocations of small objects, such as map and list, using the
default allocator is generally slow.[4][17] Other common problems with a
malloc-based allocator include poor locality of reference,[4] and excessive
memory fragmentation.

\ref{sec:malloc()}

\subsection{user-defined allocator}
\label{sec:allocator-user-defined}

A common question from programmers who have an intermediate amount of experience
with using the STL is, 'How do I write an STL allocator?' (Sect.\ref{sec:std::allocator}).

One of the main reasons for writing a custom allocator is performance. Utilizing
a specialized custom allocator may substantially improve the performance or
memory usage, or both, of the program.  

Any class that fulfills the allocator requirements can be used as an allocator,
i.e. a class, say name \verb!A!, capable of allocating memory for an object of
type \verb!T!, if

\textcolor{red}{(1) if this calss has the following types}
\begin{enumerate}
  \item for generically declaring objects and references (or pointers) to objects of type \verb!T!.

\begin{verbatim}
// types
A::pointer

A::const_pointer

A::reference

A::const_reference

A::value_type

\end{verbatim}

In STL, we have
\begin{verbatim}

#typedef A::pointer T*

#typedef A::const_pointer (T const*)
\end{verbatim}

In a user-defined allocator, the library implementors are encouraged to support more general allocators.

  
  \item an unsigned type which can represent the largest size for an object in the allocation model defined by \verb!A! class (as allocator).
  
\begin{verbatim}
A::size_type,
\end{verbatim}

   \item a signed integral type that can represent the difference between any two pointers in the allocation model.[11]
   
   
\begin{verbatim}
A::difference_type 
\end{verbatim}
\end{enumerate}


\textcolor{red}{(1) if this calss has the following member functions}: 

\begin{enumerate}
  
  \item  \verb!allocate()! and
  
The signature of the funtions
\begin{lstlisting}

// returns a pointer to the first element of a newly allocated array large
// enough to contain n objects of type T

A::pointer A::allocate(size_type n, A<void>::const_pointer hint = 0)


\end{lstlisting}

An optional pointer argument (that points to an object already allocated by A)
can be used as a hint to the implementation about where the new memory should be
allocated in order to improve locality. However, the implementation is free to
ignore the argument.
 
  
  \item   \verb!deallocate()!
  
The signature of the funtions
\begin{lstlisting}

//  accepts any pointer that was returned from a previous invocation of the
//       A::allocate member function and 
//       the number of elements to deallocate (but not destruct).

void A::deallocate(A::pointer p, A::size_type n)
\end{lstlisting}


  \item two member functions, A::construct and A::destroy
  
  
\begin{lstlisting}
template <typename T>
void A::construct(A::pointer p, A::const_reference t) { new ((void*) p) T(t); }

template <typename T>
void A::destroy(A::pointer p){ ((T*)p)->~T(); }
\end{lstlisting}


  \item \verb!A::max_size()!
  
  
member function returns the largest number of objects of type T that could be expected to be successfully allocated by an invocation of A::allocate;
\begin{verbatim}

template <typename T>
A::max_size()
{
  return A::size_type(-1) / sizeof(T);
}
\end{verbatim}

   \item \verb!A::address(A::reference)! member function returns an A::pointer denoting the address of an object, given an A::reference.
  
\begin{verbatim}

A::pointer  A::address(A::reference x)
{
  return ??? ;
}

\end{verbatim}
\end{enumerate}



A user-defined container need to have an allocator, e.g. we can use STL
allocator (Sect.\ref{sec:Allocator}). However,  as suggested by a member from
Microsoft, you should avoid from deriving a class from \verb!std::allocator!
(especially when it comes to rebinding and it's hard to debug since you don't
have the code to STL)
\footnote{\url{https://blogs.msdn.microsoft.com/vcblog/2008/08/28/the-mallocator/}}.

  
Example: write your ow allocator called {\bf Mallocator}
\begin{verbatim}
list<int, Mallocator<int> > l;
\end{verbatim}

{\tiny 
\begin{lstlisting}
template <typename T> class Mallocator {

public:

 

    // The following will be the same for virtually all allocators.

    typedef T * pointer;

    typedef const T * const_pointer;

    typedef T& reference;

    typedef const T& const_reference;

    typedef T value_type;

    typedef size_t size_type;

    typedef ptrdiff_t difference_type;

 

    T * address(T& r) const {

        return &r;

    }

 

    const T * address(const T& s) const {

        return &s;

    }

 

    size_t max_size() const {

        // The following has been carefully written to be independent of

        // the definition of size_t and to avoid signed/unsigned warnings.

        return (static_cast<size_t>(0)  -  static_cast<size_t>(1)) / sizeof(T);

    }

 

 

    // The following must be the same for all allocators.

    template <typename U> struct rebind {

        typedef Mallocator<U> other;

    };

 

    bool operator!=(const Mallocator& other) const {

        return !(*this == other);

    }

 

    void construct(T * const p, const T& t) const {

        void * const pv = static_cast<void *>(p);

 

        new (pv) T(t);

    }

 

    void destroy(T * const p) const; // Defined below.

 

 

    // Returns true if and only if storage allocated from *this

    // can be deallocated from other, and vice versa.

    // Always returns true for stateless allocators.

    bool operator==(const Mallocator& other) const {

        return true;

    }

 

 

    // Default constructor, copy constructor, rebinding constructor, and destructor.

    // Empty for stateless allocators.

    Mallocator() { }

 

    Mallocator(const Mallocator&) { }

 

    template <typename U> Mallocator(const Mallocator<U>&) { }

 

    ~Mallocator() { }

 

 

    // The following will be different for each allocator.

    T * allocate(const size_t n) const {

        // Mallocator prints a diagnostic message to demonstrate

        // what it's doing. Real allocators won't do this.

        std::cout << "Allocating " << n << (n == 1 ? " object" : "objects")

            << " of size " << sizeof(T) << "." << std::endl;

 

        // The return value of allocate(0) is unspecified.

        // Mallocator returns NULL in order to avoid depending

        // on malloc(0)'s implementation-defined behavior

        // (the implementation can define malloc(0) to return NULL,

        // in which case the bad_alloc check below would fire).

        // All allocators can return NULL in this case.

        if (n == 0) {

            return NULL;

        }

 

        // All allocators should contain an integer overflow check.

        // The Standardization Committee recommends that std::length_error

        // be thrown in the case of integer overflow.

        if (n > max_size()) {

            throw std::length_error("Mallocator<T>::allocate()  -  Integer overflow.");

        }

 

        // Mallocator wraps malloc().

        void * const pv = malloc(n * sizeof(T));

 

        // Allocators should throw std::bad_alloc in the case of memory allocation failure.

        if (pv == NULL) {

            throw std::bad_alloc();

        }

 

        return static_cast<T *>(pv);

    }

 

    void deallocate(T * const p, const size_t n) const {

        // Mallocator prints a diagnostic message to demonstrate

        // what it's doing. Real allocators won't do this.

        std::cout << "Deallocating " << n << (n == 1 ? " object" : "objects")

            << " of size " << sizeof(T) << "." << std::endl;

 

        // Mallocator wraps free().

        free(p);

    }

 

 

    // The following will be the same for all allocators that ignore hints.

    template <typename U> T * allocate(const size_t n, const U * /* const hint */) const {

        return allocate(n);

    }

 

 

    // Allocators are not required to be assignable, so

    // all allocators should have a private unimplemented

    // assignment operator. Note that this will trigger the

    // off-by-default (enabled under /Wall) warning C4626

    // "assignment operator could not be generated because a

    // base class assignment operator is inaccessible" within

    // the STL headers, but that warning is useless.

private:

    Mallocator& operator=(const Mallocator&);

};

 

// A compiler bug causes it to believe that p->~T() doesn't reference p.

 

#ifdef _MSC_VER

    #pragma warning(push)

    #pragma warning(disable: 4100) // unreferenced formal parameter

#endif

 

// The definition of destroy() must be the same for all allocators.

template <typename T> void Mallocator<T>::destroy(T * const p) const {

    p->~T();

}
\end{lstlisting}
}

\subsection{allocator for GPU}
\label{sec:allocator-GPU}


Suppose that you want to use \verb!std::vector!, and every element of that
vector must be allocate using \verb!cudaMallocManaged()!.

This is we can do, first write that allocator, name \verb!managed_allocator!,
and then pass it as the second argument to \verb!std::vector!, which allows us
to define a new type called \verb!managed_vector!.

\begin{lstlisting}
// create a nickname for vectors which use a managed_allocator
template<class T>
using managed_vector = std::vector<T, managed_allocator<T>>;
\end{lstlisting}
\url{https://github.com/jaredhoberock/managed_allocator}


Example: we can also use this vector using Thrust's APIs
\begin{lstlisting}
__global__ void increment_kernel(int *data, size_t n)
{
  size_t i = blockDim.x * blockIdx.x + threadIdx.x;

  if(i < n)
  {
    data[i] += 1;
  }
}

int main()
{
  size_t n = 1 << 20;

  managed_vector<int> vec(n);

  // we can use the vector from the host
  std::iota(vec.begin(), vec.end(), 0);

  std::vector<int> ref(n);
  std::iota(ref.begin(), ref.end(), 0);
  assert(std::equal(ref.begin(), ref.end(), vec.begin()));

  // we can also use it in a CUDA kernel
  size_t block_size = 256;
  size_t num_blocks = (n + (block_size - 1)) / block_size;

  increment_kernel<<<num_blocks, block_size>>>(vec.data(), vec.size());

  cudaDeviceSynchronize();
  
  
  // by default, the Thrust algorithm will execute on the host with the managed_vector
  thrust::fill(vec.begin(), vec.end(), 7);
  assert(std::all_of(vec.begin(), vec.end(), [](int x)
  {
    return x == 7;
  }));

  // to execute on the device, use the thrust::device execution policy
  thrust::fill(thrust::device, vec.begin(), vec.end(), 13);
}
\end{lstlisting}

\subsection{stateful allocator}

If you want to write a stateful allocator, you would have had to reimplement not
only allocate() and deallocate(), but also the ctor, copy ctor (unless the implicitly
defined copy ctor did the right thing), dtor (unless the implicitly defined dtor
did the right thing), rebinding ctor, rebind struct (see above), and equality
test.    


\section{User-defined iterator}
\label{sec:iterator-user-defined}


First, we should be aware that there are different types of iterators (Sect.\ref{sec:iterator}).

If we define a container (Sect.\ref{sec:container-user-defined}, then we can
also define an iterator for that container.
\begin{enumerate}
  
  \item it must be a template class (to avoid code duplication iterator) with
  one or two types passing to it:

\begin{lstlisting}
template <typename PointerType> class MyIterator 
{


}
\end{lstlisting}
  
  \item suppose the container's name is \verb!Array! class, then the iterator we
  can name as \verb!ArrayIterator!.
  
  \item define typedef in your custom container class
  
  be parameterized by 'value type', 'pointer type', 'reference type' or
  all of them (depends on implementation) - Sect.\ref{sec:iterator-user-defined}.

Example: we use the two names \verb!iterator_type! and
\verb!const_iterator_type! for the names of the non-const and const iterators.

\begin{lstlisting}
template <class T>
class MyClass
{
  typedef MyIterator<T> iterator_type;
  typedef MyIterator<const T> const_iterator_type;
  friend class MyIterator<T>;
  friend class MyIterator<const T>;

}
\end{lstlisting}
   
The more widely-adopted names for iterators are \verb!iterator! and \verb!const_iterator!
\begin{lstlisting}

template <class T>
class MyClass
{
  typedef MyIterator<T> iterator;
  typedef MyIterator<const T> const_iterator;
  friend class MyIterator<T>;
  friend class MyIterator<const T>;

}
\end{lstlisting}

   \item define at least the two methods \verb!begin()! and \verb!end()! in your custome container class
   
\begin{lstlisting}
template <class T>
class MyClass
{
  typedef MyIterator<T> iterator;
  typedef MyIterator<const T> const_iterator;
  friend class MyIterator<T>;
  friend class MyIterator<const T>;
  
  T* m_data;
  int m_size;
  
  /* NOTE: 'iterator' is a type */
  iterator begin()
  {
     return iterator(&m_data[0]);
  }
  
  const_iterator cbegin()
  { 
     return const_iterator(&m_data[0]);
  }
  iterator end()
  {
     return iterator(&m_data[m_size-1]);
  }
  
  const_iterator cend()
  { 
     return const_iterator(&m_data[m_size-1]);
  }
  
  // if we want to support reverse iterator, i.e. start from the end, we can do
  typedef MyReverseIterator<T> reverse_iterator;
  typedef MyReverseIterator<const T> const_reverse_iterator;
  iterator rbegin()
  {
     return reverse_iterator(&m_data[m_size-1]);
  }
  
  const_iterator crbegin()
  { 
     return const_reverse_iterator(&m_data[m_size-1]);
  }  
  iterator rend()
  {
     return reverse_iterator(&m_data[0]);
  }
  
  const_iterator crend()
  { 
     return const_reverse_iterator(&m_data[0]);
  }
}
\end{lstlisting}
     
     \item Typically, it is a good idea to make the custom iterator derived from
     STL standard iterator categories to let STL algorithms know the type of
     iterator we've made (Sect.\ref{sec:iterator})
     
     
\end{enumerate}

\subsection{Example: iterator is a class inside the container}

\url{https://gist.github.com/jeetsukumaran/307264}


\subsection{Example: iterator in a class outside the Container}
Example: typically, an iterator starts from the beginning (i.e. the containers
support 2 methods begin() and end()), as below, but we can also define a
ReverseIterator (with 2 new methods for the containers rbegin() and rend() )
\begin{lstlisting}
template <typename T>
class MyIterator : public std::iterator<std::random_access_iterator_tag, T, ptrdiff_t,
                                        T*,
                                        T&>
{
protected:
   T*   m_ptr;   // the pointer pointing to the current data element of the container 
                 // that this iterator is associated with
public:
    MyIterator(T& ptr = nullptr)
    {
       m_ptr = ptr;
    }
    
    MyIterator(const MyIterator<T>& rawItetaror) = default;
    ~MyIterator(){};
    
    //now many operators need to be implemented
    // as inherited from std::iterator, we are freed from reimplemented some of them
    // by defining them as '= default'
    
    //1. assignment operators
    MyIterator<T>&   operator=(const MyIterator<T>& rawIterator) = default;
    
    MyIterator<T>&   operator=(T* ptr) { m_ptr = ptr; return (*this)};
    
    //2. logical operator
    operator      bool() const
    {
       if (m_ptr)
          return true;
       else
          return false;
    }
    
    //3. comparison operators    
    bool    operator== (const MyIterator<T>& rawIterator) const { return (m_ptr == rawIterator.getConstPtr());};
    bool    operator!= (const MyIterator<T>& rawIterator) const { return (m_ptr != rawIterator.getConstPtr());};
    
    //4. incremental operators, e.g. 
    //   others: 			it += 4; it -= 3;
    //   pre-increment (return reference):  ++it;  --it;
    //   post-increment (return value): it++; it--; 
       // 4.1. others
    MyIterator<T>&       operator+=(const ptrdiff_t& movement) { m_ptr += movement; return (*this);};
    MyIterator<T>&       operator-=(const ptrdiff_t& movement) { m_ptr -= movement; return (*this);};
       // 4.2. pre-increment (increase the iterator, then return reference to the changed iterator)
    MyIterator<T>&       operator++(						) { ++m_ptr; return (*this);};
    MyIterator<T>&       operator--(						) { --m_ptr; return (*this);};
       // 4.3. post-increment (return an iterator that points to the current data, then increment the given iterator)
    MyIterator<T>       operator++(ptrdiff_t /*dummy*/) { auto temp(*this); ++m_ptr; return temp;};
    MyIterator<T>       operator--(ptrdiff_t /*dummy*/) { auto temp(*this); --m_ptr; return temp;};
       //   a = it+5;  // does not change 'it', but return the iterator pointing to the value at 5 increment from the value pointed by 'it'
    MyIterator<T>       operator+(const ptrdiff_t& movement) { auto oldPtr = m_ptr; m_ptr += movement; auto temp(*this); m_ptr = oldPtr; return temp;};
    MyIterator<T>       operator-(const ptrdiff_t& movement) { auto oldPtr = m_ptr; m_ptr -= movement; auto temp(*this); m_ptr = oldPtr; return temp;};
    
    //5. support finding difference between 2 iterators
    ptrdiff_t           operator-(const MyIterator<T>& rawIterator){return std::distance(rawIterator.getPtr(), this->getPtr();}
    
    T&                  operator*(){ return *m_ptr;}
    const T&            operator*() const { return *m_ptr;}
    T*                  operator->(){ return m_ptr;}
    
    // supporting APIs
    T*                  getPtr() { return m_ptr;}
    const T*            getConstPtr() const { return m_ptr;}

}
\end{lstlisting}

Example: 
\begin{lstlisting}
template <typename T, typename NonConstT>

class ArrayIterator 
{


}
\end{lstlisting}


Example:  we can also define a ReverseIterator (so that the container can
support 2 new methods for the containers rbegin() and rend() )
\begin{lstlisting}
template <typename T>
class MyReverseIterator : public MyIterator
{
}
\end{lstlisting}
\url{https://stackoverflow.com/questions/3582608/how-to-correctly-implement-custom-iterators-and-const-iterators}