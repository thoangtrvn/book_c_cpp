\chapter{GObject system}
\label{chap:GObject}

\section{GObject}
\label{sec:GObject}

To help bridging C language and an object-oriented language like C++, the
concept of an "object" needs to be introduced. 

After GIM 0.6 (Sect.\ref{sec:GTK+}), the designers realized that for future
development and progress, the entire toolkit needed to be rewritten to be
object-oriented. However, they don't want to use C++, so a new system is
developed to add an object oriented system to C.

There are several reasons why GObject is used
\begin{enumerate}
  \item to avoid the controversial idea of multiple inheritance in C++
  
 GObject system is built to only support single inheritance, which can
 drastically simplify the inheritance hierarchies.
  
  \item to make language interoperability easier
  
C++ just isn't that great at interop. However, interoperability with C is almost
taken for granted.

  \item to avoid many overkill feature of C++; while GTK+ only needs
  some object oriented features in C.
  
  \item to be language independent
  
GObjects is intended to be language independent. It has dynamic typing, and you
should it compare with a run-time system like COM, .NET or CORBA, rather than
with specific languages. If you go into languages, then the features are more at
the Objective-C than the C++ side.

The GObject type system does things that you can't do easily in C++ (at that
day). For one thing, it allows creation of new classes at runtime, and does this
in a way that is language-independent - you can define a new class in Python at
runtime, then manipulate instances of that class within a function written in C,
which need not even be aware that Python was ever involved.
\end{enumerate}
\url{http://stackoverflow.com/questions/9747468/why-was-the-gobject-system-created}


% For example: java.lang.Object,
% GObject, QObject, python's base Object, or whatever.


