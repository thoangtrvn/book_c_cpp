\chapter[Intel Cilk]{Intel Cilk (parallel programming language)}
\label{chap:Intel_Cilk}

Like Intel TBB, Intel Cilk (Cilk++, CilkPlus) also target parallelism on
multicore processors. Intel Cilk Plus offers a competitive alternative for
parallel programming that is far easier to use than MPI.  
It is intended to complement TBB and OpenMP, enabling programmers to parallelize
applications that may be too complex or cumbersome to fit into one of these frameworks.
The tradeoff is that Cilk Plus may not be as finely tuned for specific
programming patterns or environments as TBB or OpenMP. 

\begin{mdframed}

In general, one should not expect to see performance improvements by convert an
existing parallel code in TBB or OpenMP to Cilk Plus.   TBB, OpenMP and MPI
continue to be good choices for writing High Performance Computing applications.

\end{mdframed}

Cilk is a pure programming language, started from MIT since 1990s, originally
based on ANSI C, with additional keywords target to parallelism.
Cilk became Cilk++ as the product of a new company, Cilk Arts, in 2006, and then
acquired by Intel in 2009. The final name is Intel Cilk Plus.

Intel Cilk Plus 
\begin{itemize}
  \item  eliminate the need for several of the original Cilk keywords while adding the
ability to spawn functions and to deal with variables involved in reduction operations.

  \item add array extensions, being incorporated in a commercial compiler (from
Intel), and compatibility with existing debuggers
  
\end{itemize}

The idea of Cilk: {\it the programmers expose and tell the compiler which parts
(elements) that can be safely parallelized; and the run-time environment (i.e.
the scheduler) will decide during execution how to actually divide the work
between processors}, i.e. the program runs without rewriting on different number
of co-processors (including 1).

Keywords: we need to include the header file, and use
\begin{verbatim}
_Cilk_spawn 
_Cilk_sync
cilk_spawn 
cilk_sync
\end{verbatim}
The functions that uses this must be defined with \verb!cilk! keyword.

The two main purposes of the keywords are (1) spawn a new thread, (2) sync
between the threads.

\section{\_Cilk\_spawn}

\verb!_Cilk_spawn!

\begin{verbatim}
cilk int fibonacy(int n)
{
  if (n<2) return n;
  else
  {
    int x, y;
    x = spawn fibonacy(n-1)
    y = spawn fibonacy(n-2)
    
    sync;
    return x+y;
  
  }

}
\end{verbatim}

% 
% \section{Inlet (advanced)} (no longer available in Cilk Plus)
% 
% By default, the spawn process put the returned value into the memory variable to
% the parent's frame. The alternate way is to use {\bf inlet}.
% 
% 
% An inlet is a function internal to a Cilk procedure which handles the results
% of a spawned procedure call as they return.
% A function is an inlet if it has the \verb!inlet! keyword.
% If an \verb!abort! keyword is used inside the inlet, it tells the scheduler that
% any other procedures spawned off by the parent procedure can safely be aborted.
% 
% IMPORTANT: All the inlets of a procedure are guaranteed to operate atomically
% with regards to each other and to the parent procedure, thus avoiding the bugs
% that could occur if the multiple returning procedures tried to update the same
% variables in the parent frame at the same time.
% 

\section{Hyperobjects}
\label{sec:CilkPlus_hyperobjects}

