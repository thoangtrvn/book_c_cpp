% \chapter{Data type: Date time}
% \label{chap:date_time}
% 


% \section{Date-time}
% \label{sec:date_time}

\section{Year 2038 problem}

Due to the complication of using date time (local vs. GMT time, Year 2038 bug
problems), you are suggested to use 64-bit data type (i.e. \verb!long long! in
GNU C or POSIX/SUS-systems), and use GMT-time all the times. Suggestions
\begin{enumerate}
  \item  Using local time,
even though easier for a human to know the time of the event in referenced to
the time of current geographical area of the user, it's hard to compare with
other time due to the complexity of converting them to the same
references due to daylight saving available in some countries and the saving
varies from countries to countries, as well as the GMT offset complexities.

  \item When you use more than 2 libraries, make sure the data types are of the
same size. This will allows the code to be recompiled easily
  
\end{enumerate}

Example: after it passes the largest number, the 32-bit data
is interepted as year 1901 \url{http://en.wikipedia.org/wiki/Year_2038_problem}
\begin{verbatim}
Hexa (32-bit): 0x7FFFFFFF
Binary :  01111111  11111111  11111111  11111111  
Decimal: 2147483647
Date (interpreted)   : 2038-01-19 03:14:07 (UTC)
Date (true)  : 2038-01-19 03:14:07 (UTC)

//after it pass the maximum value --> it becomes negative
Binary :  10000000  00000000  00000000  00000000
Decimal: -2147483648
Date (interpreted)   : 1901-12-13 20:45:52 (UTC)
Date (true)  : 2038-01-19 03:14:08 (UTC)
\end{verbatim}

This is known as {\bf Unix Millenium bug} which is a serious problems in
embedded systems which still use 8-bit, 16-bit or 32-bit microprocessors.
Embedded systems are being used every where: mirowave ovens, gas stations, car
fuel injection computers, radios, \ldots


There is no universal solution to the Year 2038 problem. A solution is to use
struct \verb!tm! (Sect.\ref{sec:struct_tm}) or using the new 64-bit definition
for \verb!time_t!.
\begin{verbatim}
  // in Cygwin
#typedef unsigned long long time_t


// /usr/include/bits/types.h
typedef long int __time_t;
typedef __time_t time_t;
\end{verbatim}


Another options is to write your own function:
\url{http://www.2038bug.com/developers.html}


\section{struct tm}
\label{sec:struct_tm}

The header file is \verb!<time.h>! (in C) or \verb!<ctime>! (in C++).

In C90 (C++98), the struct \verb!tm! has 9 members, in any order
\begin{verbatim}
int tm_sec; //second (0..61*) [FIXED 0..60 in C99/C++11]
int tm_min; // minutes (0..59)
int tm_hour; //hours since midnight (0..23)
int tm_mday; //day of the month (1..31)
int tm_mon;  //month since Jan (0..11)
int tm_year; // year since 1900
int tm_wday; // day since Sunday (0..6)
int tm_yday; // days since Jan-1 (0..365)
int tm_isdst; //daylight saving flag (0 or 1)
\end{verbatim}
NOTE: \verb!tm_sec! is generally 0..59. However, the extra value is to
accommodate for leap seconds in certain years. [NOTE: {\bf leap seconds} is
one-second adjustment applied to UTC in order to keep the ]

\verb!tm! struct is being used in some functions
\begin{verbatim}
time_t mktime(struct tm*	)		// convert from tm to time_t (local time)
struct tm localtime( )  // return current local time (left or right of GMT)
                        // and inclusive any daylight-saving offsets.
struct tm gmtime( )    // return current GMT-time
char* asctime(const struct tm*	)  //convert from struct tm to string 
\end{verbatim}
The string returned by \verb!asctime()! is in the same format as \verb!ctime()!
function, which uses the information from these fields: 
\begin{verbatim}
tm_wday
tm_mon
tm_mday
tm_hour
tm_min
tm_sec
1900+tm_year
\end{verbatim}

Example:
\begin{verbatim}
  static const char wday_name[][4] = {
    "Sun", "Mon", "Tue", "Wed", "Thu", "Fri", "Sat"
  };
  static const char mon_name[][4] = {
    "Jan", "Feb", "Mar", "Apr", "May", "Jun",
    "Jul", "Aug", "Sep", "Oct", "Nov", "Dec"
  };
  static char result[26];
  sprintf(result, "%.3s %.3s%3d %.2d:%.2d:%.2d %d\n",
    wday_name[timeptr->tm_wday],
    mon_name[timeptr->tm_mon],
    timeptr->tm_mday, timeptr->tm_hour,
    timeptr->tm_min, timeptr->tm_sec,
    1900 + timeptr->tm_year);
  return result;
\end{verbatim} 


\section{time\_t}
\label{sec:time_t}

ISO C define \verb!time_t! as an arithmetic type, but doens't specify any
particular type, or the range, resolution or encoding of it. Also, it doesn't
specify the meaning of arithmetic operations on the data of this type. So, it's
pretty much implementation dependent. The header file is \verb!<time.h>! (in C)
or \verb!<ctime>! (in C++). To check how \verb!time_t! is defined, we write a
simple code and compile like this
\begin{verbatim}
#include <time.h>

int main(int argc, char** argv)
{
        time_t test;
        return 0;
}
\end{verbatim}
and compile
\begin{verbatim}
gcc -E test.c | grep 'time_t'
\end{verbatim}

The function that deals with \verb!time_t! data treat the value as \verb!local!
time. The value of using \verb!local! time, which can changes depending where
the machine is running on the globe (i.e. GMT time or Greenwich Mean Time or UTC
- Coordinated Universal Time). So, if two machines talk to each other, the
current time between them, using this function, can be a few hours difference.
To make sure it's the same, they should be referenced to the same place, i.e.
UTC time. This can be done using \verb!gmtime()! function.

\begin{enumerate}
  \item \verb!time()!: return the current calendar time (in seconds) compared to
  the intial time (as described below). 
  
\begin{verbatim}
time_t time(time_t *);
\end{verbatim}  
If the argument is not a NULL value, then the current calendar time is returned
to this value as well.

   \item \verb!ctime()!: convert the time (in local reference time) to string.

\begin{verbatim}
char* ctime(const time_t * timer); 
\end{verbatim}
This is equivalent to \verb!asctime(localtime(timer)!
   
The C-string is in the format \verb!Www Mmm dd hh:mm:ss yyyy!, followed by a
character \verb!'\n'!, and terminated by a NULL-character.
\begin{verbatim}
Wwww = weekday
Mmm  = month (in letters)
dd   = day of the month
hh:mm:ss = time 
yyyy = year
\end{verbatim}

   \item \verb!gmtime():!
\end{enumerate}

{\bf UNIX and POSIX-compliant system}: 
\begin{verbatim}
   // 32-bit or 64-bit wide depending which CPU architecture
   //                       and operating system
#typedef signed integer time_t
\end{verbatim}
The value of this type tells the number of seconds since the start of the Unix
epoch, i.e. {\it midnight UTC of January 1, 1970} (not counting leap seconds). 

If it's a 32-bit value, then consider the largest value and it causes Year 2038
problem
\begin{verbatim}
#include <stdio.h> /* printf */
#include <time.h>  /* time_t */

time_t biggest_in_seconds = 0x7FFFFFFF;

printf("biggest time in the future = %s ", ctime(&biggest));
\end{verbatim}

\section{Convert to string}
\label{sec:datetime_to_string}

We can use \verb!ctime(), ctime_r()! 

We can use \verb!asctime()!

We can use \verb!strftime()! function, which interprets the data in
\verb!timptr! to the string pointed by \verb!s! using the format given by
\verb!format!. The maximum characters (including the NULL-character) of the
string pointed by \verb!s! is given in \verb!maxsizes!.
\begin{verbatim}
#include <time.h>

size_t strftime(char *s, size_t maxsize, const char *format,
    const struct tm *timptr);
\end{verbatim}
The function returns the total number of characters (not including the
NULL-character) copied to \verb!s! if the total length copied is less than
\verb!maxsizes! (i.e. successful). Otherwise, it returns 0 (which can be used
to detect error) and the contents of the string pointed by \verb!s! is
indeterminate.

\url{http://pubs.opengroup.org/onlinepubs/007908799/xsh/strftime.html}


Example:
\begin{Verbatim}
#include <stdio.h>      /* puts */
#include <time.h>       /* time_t, struct tm, time, localtime, strftime */

int main ()
{
  time_t rawtime;
  struct tm * timeinfo;
  char buffer [80];

  time (&rawtime);
  timeinfo = localtime (&rawtime);

  strftime (buffer,80,"Now it's %I:%M%p.",timeinfo);
  puts (buffer);

\end{Verbatim}


\section{Thread-safe functions}

MinGW provides both \verb!*_r()! and \verb!*_s()! functions. The \verb!*_r()!
are the re-entrant (thread-safe) versions

\subsection{POSIX}

Added to SUS (Single Unix Specification) version 2: \verb!gmtime_r()!,
\verb!asctime_r()!, \verb!ctime_r()!, and \verb!localtime_r()!.

Example: The function \verb!gmtime()! is not thread-safe as it
use a pointer to an internal object of type \verb!struct tm!. A thread-safe
version is \verb!gmtime_r()! (on standard POSIX-compliant system) or
\verb!gmtime_s()! (on Windows system). Using these thread-safe are non-portable
\begin{verbatim}
  // the order of the arguments are different
  
  // return a pointer pointing to the value the same as the second argument
  // or return a NULL pointer if failed
struct tm* gmtime_r(const time_t*, struct tm*)

  // return 0 (successful) 
  // or errno_t value if failed
int gmtime_s(struct tm*, const time_t*)
\end{verbatim}

\subsection{Windows}

These functions are added to Windows' Secure CRT (Secure C run-time):
\verb!gmtime_s()!, \verb!asctime_s()!, \verb!ctime_s()!, and
\verb!localtime_s()!. 

NOTE: Thread-safety is not an issue with \verb!gmtime()! and \verb!localtime()!
in Windows, and the output is allocated per thread. Here, the two new functions
added to do Secure CRT's parameter validation (e.g. checking if the parameter
are NULL to evoke error handling or not).


\section{<chrono> (C++11)}
\label{sec:chrono-header-file}

Since C+=11, the header file \verb!<chrono>! has been added and contains
elements to deal with time.
\begin{enumerate}
  \item measure duration, i.e. time spans
  
  \item a specific time point
  
  \item relate time point to real physical time
\end{enumerate}

This is adopted from Boost::Chrono library (Sect.\ref{sec:Boost.Chrono})

\url{http://www.cplusplus.com/reference/chrono/}

To convert from
\begin{verbatim}
std::chrono::system_clock::time_point
\end{verbatim}
to \verb!time_t!, we use

\begin{lstlisting}
time_t  data = std::chrono::system_clock::to_time_t(...)
\end{lstlisting}

Finally, use the normal C function \verb!ctime()! or strftime to format it.

\begin{verbatim}
std::chrono::system_clock::time_point tp = std::chrono::system_clock::now();

std::time_t time = std::chrono::system_clock::to_time_t(tp);
std::tm timetm = *std::localtime(&time);
std::cout << "output : " << std::put_time(&timetm, "%c %Z") << "+"
          << std::chrono::duration_cast<std::chrono::milliseconds>(tp.time_since_epoch()).count() % 1000 << std::endl;
          
\end{verbatim}

\url{http://www.cplusplus.com/reference/ctime/strftime/}