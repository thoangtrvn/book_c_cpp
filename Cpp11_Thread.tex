\chapter{C++11 multithreading (std::thread)}
\label{chap:multithreading_C++11}

\url{https://solarianprogrammer.com/2011/12/16/cpp-11-thread-tutorial/}

Before C++11 and C11, there was no support for concurrency in the language; the
standard libraries and the compilers are free to give some guarantees, and the
programmers need to use explicit tools to handle this
(Chap.\ref{chap:PThreads}, Chap.\ref{chap:OpenMP}, Chap.\ref{chap:OOMPH-Lib}
and other libraries from the Parallel Programming book).

What news in C++11 and C11?
\begin{enumerate}  
  \item New keyword \verb!thread_local! for defining variables with
  thread-specific values.
  
  \item New memory models (Sect.\ref{sec:memory_model}): update on two different
  data objects by two different threads are indepenedent of each other, i.e. 2
  different variables are guarantees to reside at two different physical  memory
  addresses.
   
  \item Support start multiple threads, including passing arguments, return
  values, and exception across thread boundaries, synchronize multiple threads
  (synchronize both control flow and data access).
\end{enumerate}
that can provides concurrency at 2 levels: high-level programming, low-level
programming

\section{high-level programming}

C++11 STL has \verb!std::async()! and class \verb!std::future<ReturnType>:! to
create a future-callable object.

\begin{enumerate}
  \item \verb!std::async()!	accept a {\it callable object} (e.g. a function,
  member function, function object, or lambda (Sect.\ref{sec:C++11_lambda})) and
  create a new thread to run it in the background, if possible. 
  \begin{verbatim}
    std::future<long> result1 = std::async(func1)
  \end{verbatim}
  By default, the system itself define when to start the background task. 
  \begin{itemize}
    \item  To create a thread and force it run immediately (if not possible,
    throw \verb!std::system_error!), we pass \verb!std::launch::async! policy
  \begin{verbatim}
  std::future<long> result1 = std::async(std::launch::async, func1)
  \end{verbatim}
  The error code, if occurs, is \verb!resource_unavailable_try_again!
  (equivalent to POSIX errno EAGAIN).
  
%   There are different launch policy:
% \begin{itemize}
%   \item \verb!std::async(std::launch::async,...)!
  \item \verb!std::async(std::launch::deferred, ...)! (postpone the execution
  of the callable object until \verb!get()! method is called)
  \begin{lstlisting}
  auto f1 = std::async( std::launch::deferred, task1);
  auto f2 = std::async( std::launch::deferred, task2);
  ...
  auto val = thisOrThatIsTheCase() ? f1.get() : f2.get();
  \end{lstlisting}
% \end{itemize}
  
    \item Later on, if not start, trigger it, and wait and get the result
    back using \verb!get()! metod
    \item Later on, if not start, trigger it, and wait until it completes (no
    need to use the result) using \verb!wait()! method
    \item Later on, wait a certain limited time using \verb!wait_for()! method
    (don't force the thread to start)
    \item Later on, wait until a specific time point is reached using
    \verb!wait_until()! method (don't force the thread to start)
  \end{itemize}
  The two later methods return either (1) \verb!std::future_status::deferred!
  (the background task has not started yet), (2)
  \verb!std::future_status::timeout! (the task started but hasn't finished
  yet), (3) \verb!std::future_status::ready! (the task completed).
  
\begin{mdframed}
Using \verb!auto! keyword, we can reduce the amount of code to write
(Sect.\ref{sec:C++11_auto})
\begin{verbatim}
auto result1(std::async(func1))
\end{verbatim}
\textcolor{red}{A function can accept arguments either by copy or by reference}.
We also have example for this, but it's RECOMMENDED to pass by value; unless
copying it TOO expensive. If we have to pass by reference, it's RECOMMENDED to
pass by constant reference and \verb!mutable! is not used.

NOTE: If we expect the background task to return nothing, we use
\verb!std::future<void>!.
\end{mdframed}
  
  NOTE: EAGAIN is raised when performing non-blocking I/O but resources is
  temporarily unavailable, try again later. 
  
  \item \verb!std::future<>! class allows you to wait for the thread to be
  finished and provides access to its outcome (a returned value or an
  exception), if any. \verb!get()! method can be used ONLY once for a
  \verb!std::<future>! instance. After the call to get() method, the object is
  in an invalid state (which can be checked using \verb!valid()! method for the
  instance) [Compilers are encouraged to throw \verb!std::future_error!].
  \textcolor{red}{Result can be used only once}.
  
  \item \verb!std::shared_future<>! allows using \verb!get()! method multiple
  times. The results from all shared futures are put into a memory region called
  {\it share state} so that can be reused later.
  \begin{lstlisting}
  std::shared_future<int> f = async(func1);
  
  //or using 'auto'
  auto f = async(func1).share();
  \end{lstlisting}
  then we define a function that accept a pointer to \verb!shared_future!
  object, so that it can call \verb!get()! method as many time as we want (see
  Example below)
\end{enumerate}

The using of \verb!get()! method in \verb!shared_future<>! is different
from \verb!future<>! in the following situations

\begin{minipage}[t]{0.5\textwidth}
\begin{verbatim}
//return a copy of the result
T future<T>::get()    


//specialization for references
T& future<T&>::get() 
//specialization for void
void future<void>::get()
\end{verbatim}
\end{minipage}
\begin{minipage}[t]{0.5\textwidth}
\begin{verbatim}
//return a reference to the result
//stored in the 'share state' memory region
const T& shared_future<T>::get()   


T& shared_future<T&>::get()
    
void shared_future<void>::get()  
\end{verbatim}
\end{minipage}

\begin{mdframed}
Without using \verb!get()! method and no \verb!async! launch policy, the program
can exit without running the background thread.

Without using \verb!get()! method, but \verb!async! launch policy is used, at
the end of the code, the program don't exit but waits for the background
task to end. However, it's suggested to use \verb!get()! method before a program
ends. 
\end{mdframed}

Example: to find the sume of the returned value of two functions
\begin{verbatim}
a = func1() + func2()
\end{verbatim}
If func1() and func2() are data-independent; then they can be executed
concurrently. This can be done in C++11, in a multi-processor system

Example: func1() run in the background and func2() is the foreground task
\begin{lstlisting}
#include <future>
#include <thread>
#include <iostream>
#include <exception>
using namespace std; // so that we don't have to repeat std:: everytime

int func1() {
  ..
  return a;
}
int func2() {
  ...
  return b;
}

int main() {
  std:cout << "..." << std::endl;
  
  // start a function asynchronously (now or later or never)
  std::future<int> result1(std::async(func1));
  
  // call func2() synchronously (here and now)
  int result 2 = func2(); 

  // wait for func1() to finish and add it to the result of func2()
  int result = result1.get() + result2;
    
  std::cout << "Result is: " << result
            << std:endl;
\end{lstlisting}
EXPLANATION: using \verb!std::async()! to start func1() in a new thread
(without blocking the execution of the current process), and put the result to
an object of class \verb!std::future<>!. The using of \verb!std::future<>! is
IMPORTANT for 2 reasons: (1) when C++ code see object of this class, it
understand that it should wait until the result is available (which is a
'future' outcome of some function); (2) force the background task to start (if
it has not started yet, when we really need the result of it). The method
\verb!get()! does a few things: check if func1() has been started and completed
(easy, just get the result); if func1() started but not completed (it blocks
and wait for the result); if func1() not started (it forces to start, and
blocks the foreground, wait to get the result).

HACKER'S TIPS: The order is important to achieve a better performance. If we put
func2() to the same line (as given below), there's a chance that your code
runs as sequential again
\begin{verbatim}
  int result = func2() + result1.get() ;
\end{verbatim} 
\textcolor{red}{The evaluation on the right-hand side of the assignment
statement is unspecified; so when result1.get() is called before  func2(); the
code is sequential.} TIPS: Maximize the distance between calling async() and
calling get(), i.e. call early and return late.

Example:
\begin{lstlisting}
auto f(std::async(func1))

...
f.wait();    // trigger func1 if it has not been start; and 
             // wait until it completes
             
f.wait_for(std::chrono::seconds(10)); //wait at most 10 seconds

f.wait_until(std::system_clock::now()+std::chrono::minutes(1));
\end{lstlisting}

Example: We use \verb!wait_for()! or \verb!wait_until()! under the situation
that we have an certain amount of time, and we need some result with as much
accurate as possible. So, we have two threads, one do a task that
gives high-accuracy and the other one give lower accuracy. After the waiting
time, i.e. we can't wait longer, we check if thread A completed or not, if yes,
get the high-accuracy result; otherwise, it's okay, get the less accuracy result
\begin{lstlisting}
int quickComputation();
int accurateComputation();

std::future<int> f;  

int bestResultInTime() {
  auto tp = std::chrono::system_clock::now() + std::chrono::minutes(1);
  
  f = std::async (std::launch::async, accurateComputation);
  int guess = quickComputation();
  
  auto s = f.wait_until(tp);
  
  if (s = std::future_status::ready) {
     return f.get();
   }else{
     return guess;
   }
}
\end{lstlisting}
NOTE: If we declare 'f' inside the function, at the end of the function, the
destructor of 'f' will have to wait until \verb!accurateComputation()! to
complete. This is not what we expect.

Example: When we have a parallel code that we don't know when it completes; we
can design the program to to do something, while waiting for it to complete.
Bad code: the following code can be endless if the background task has not
started yet
\begin{lstlisting}
while (f.wait_for(chrono::seconds(0) != future_status::ready)) {
   //do something
   ...
}
\end{lstlisting}
Solution: using \verb!std::launch::async! policy or reschedule so that other
threads to run (call \verb!std::this_thread::yield()! inside the loop) or check
for \verb!future_status::deferred! (see below)
\begin{lstlisting}
std::future<int> f(async(func1)) ; 

//check if it has started
if (f.wait_for(std::chrono::seconds(0)) != future_status::deferred) {
  // do something else while waiting for it
  while (f.wait_for(std::chrono::seconds(0) != future_status::ready) {
    ...
  }
}

auto result = f.get();
\end{lstlisting}


Example: Using \verb!async()! with lambda, i.e. we have a block of code to run,
or the function need to receive arguments. 
\begin{enumerate}
  \item  The argument can be literal
\begin{lstlisting}
void doSomething(char c) {
   ...
}


auto f1 = std::async( [] { doSomething('.'); });
auto f2 = std::async( [] { doSomething('+'); });
\end{lstlisting}

   \item Pass by value 
   \begin{lstlisting}
   char c = '+';
   auto f = std::async( [=] { doSomething(c); });
   \end{lstlisting}
   The use of capture as [=] means a {\it copy} of c and all other visible
   objects to the lambda.
   
   \item Pass by reference, even a function pointer using \verb!std::future<>!
   \begin{lstlisting}
void doSomething(const char& c) {
    ...
}
   
char c = '.';
auto f = std::async(doSomething, c);
   //pass by reference
auto f = std::async([&] {doSomething(c); }); //risky
auto f = std::async(doSomething, std::ref(c); //risky

f.get() ; //variable 'c' must be existing after this point
\end{lstlisting}
   It's risky as we need to make sure the lifetime of the arguments passed
   longer the time the execution of the background task completed.
   
   \item Pass by reference an instance of \verb!std::shared_future<>! 
   \begin{lstlisting}
   using namespace std;
   void doSomething(char c, shared_future<int> f) {
     //wait for number
     int num = f.get(); 
     ...
   }
   
   
   int queryNumber();
   void doSomething(char c, const shared_future<int>& f);
   
   auto f = async(queryNumber).share();
   
   auto f1 = async(launch::async, doSomething, '.', std::ref(f))
   auto f2 = async(launch::async, doSomething, '+', std::ref(f))
   
   f1.get();
   f2.get();
   \end{lstlisting}
   The above using multiple shared-future objects (all sharing the same {\it
   share state}), we can use a single shared-future object, and perform
   \verb!get()! multiple times.
   \begin{lstlisting}
   f1.get();
   f1.get();
   \end{lstlisting}
   In any way, passing by reference is risky (as we need to make sure the
   lifetime of 'f' is longer than 'f1'); but copying a shared future is
   expensive. So be careful to use !
   
   \item A pointer to a class member function
   \begin{lstlisting}
class X {
   public: 
      void mem (int num);
      ...
};

X x;
auto a = std::async(&X::mem, x, 32);
\end{lstlisting}
First is the pointer to member function; Second is the instance object, and from
Third are arguments.
\end{enumerate}









\section{low-level programming (std::thread, mutex, atomic)}

We declare an object of class \verb!std::thread! and pass the desired task an
initial argument, which automatically start the background task at the time the
object is created (otherwise, it throws \verb!std::system_error! with error
code \verb!resource_unavailable_try_again!). To be able to pass data, we need to
use \verb!std::promise<>! 
\begin{verbatim}
#include <thread>

void doSomething();

std::thread t(doSomething); // start the background task

...

t.join();
\end{verbatim}
To synchronize the code, i.e. wait until the thread complete, we use
\verb!join()! method. Compare with \verb!async()!, we
\begin{enumerate}
  \item DON't have the control to data; but know the thread ID
  using \verb!get_id()! method
  \begin{lstlisting}
  std::thread::id id = t.get_id();
  \end{lstlisting}
  The class constructor \verb!std::thread::id()! yields a unique ID representing
  'no thread' [NOTE: No guarantee that the main thread has id 1, or 'no
  thread' has id 0].
  
  \item NEED to make sure before the end of the foreground thread, either (1)
  join() is called; or (2) \verb!detach()! is called on the background task.
  Otherwise, the program aborts (\verb!std::terminate()!). When  \verb!detach()!
  is called, the background task is free to run without any  control.
  \item When there are 'detached' background tasks, and the program run to the
  end of \verb!main()!, then these background tasks continue to run. If they
  access global or static objects that are already destroyed or under
  destruction (as it's the end of the program), the result is undefined
  behavior. If detached threads need to use global or static objects, a solution
  (to make sure these data are not destroyed before datached objects finish) is
  \begin{enumerate}
    \item Use a condition variable, that detached threads use to signal that
    they have finished; and before calling exit() or end of main() code, signal
    these condition variables
    \item Call \verb!quick_exit()! at the end of the program (end the program
    without calling the destruction for global or static objects)
    \item Force the main thread to wait until detached threads complete using
    \verb!..at_thread_exit()!.
  \end{enumerate}
\end{enumerate}

\begin{mdframed}
HACKER'S TIP: Detached threads can be come a problem if they use non-local
resources. Thus, passing data to a thread by reference is always a risk. It's
strongly recommended passing by value.
\end{mdframed}

Example: function with arguments
\begin{lstlisting}
void doSomething(int num, char c);

std::thread t(doSomething, 5, '.');
\end{lstlisting}



To protect concurrent access to a shared data, mutexes or atomic functions need
to be used.

\section{std::thread() vs. boost::thread()}


One of the biggest change to the C++ language is the addition of multithreading
support since C++11.

To compile code, make sure use 
\begin{verbatim}
# Linux
g++ -std=c++11 -pthread file_name.cpp

# Mac XCode
clang++ -std=c++11 -stdlib=libc++ file_name.cpp

# Windows, use commercial library just::thread
\end{verbatim}




