\chapter{Program Constructs}

\section{Structured binding (C++17)}
\label{sec:structured-binding}

Like a reference, a structured binding is an alias to an existing object. Unlike
a reference, the type of a structured binding does not have to be a reference
type.

Structured Bindings give us the ability to declare multiple variables
initialized from a \verb!tuple! (Sect.\ref{sec:tuple}) or \verb!struct!.

Example: a function returns multiple values, packed in a \verb!tuple!.
If a function returns a \verb!tuple!, then we want to be able to 
declare multiple variables receiving the result

\begin{verbatim}
tuple<T1,T2,T3> f(/*...*/) { /*...*/ return {a,b,c}; }

auto {x,y,z} = f(); // x has type T1, y has type T2, z has type T3


struct S {
    int x1 : 2;
    volatile double y1;
};
S f();
 
const auto [x, y] = f(); // x is a const int lvalue identifying the 2-bit bit field
                         // y is a const volatile double lvalue
\end{verbatim}

\textcolor{red}{We can use \verb!auto! with \verb!&! and \verb!&&!}
\begin{lstlisting}
int a[2] = {1,2};
 
auto [x,y] = a; // creates e[2], copies a into e, then x refers to e[0], y refers to e[1]
auto& [xr, yr] = a; // xr refers to a[0], yr refers to a[1]
\end{lstlisting}

\url{https://en.cppreference.com/w/cpp/language/structured_binding}

\url{http://www.open-std.org/jtc1/sc22/wg21/docs/papers/2015/p0144r0.pdf}

\section{Loops}

\subsection{finite loop}

{\small \begin{verbatim} 
for ( variable initialization; condition; variable update ) {
  Code to execute while the condition is true
}
\end{verbatim}}

\textcolor{red}{\bf IMPORTANT}: Though we often use \verb!init! to represent the
type of the index, when using template we should use \verb!<template>::size_type!, e.g.
\verb!std::vector<int>::size_type! (Sect.\ref{sec:size_type}).


Before C99, i.e. C89 and earlier, we cannot declare the variable inside the
\verb!for! loop
\begin{verbatim}
int ii; 
for (ii = 0; ii < 5; ii++)
{

}

//otherwise error:
// error: 'for' loop initial declarations are only allowed in C99 mode
\end{verbatim}

Since C99 extension, i.e. we need to compile the code with the flag
\verb!-std=c99!, we can do
\begin{verbatim}
for (int ii = 0; ii < 5; ii++)
{

}
\end{verbatim}

\begin{verbatim}
#include <iostream>

using namespace std; // So the program can see cout and endl

int main()
{
  // The loop goes while x < 10, and x increases by one every loop
  for ( int x = 0; x < 10; x++ ) {
    // Keep in mind that the loop condition checks 
    //  the conditional statement before it loops again.
    //  consequently, when x equals 10 the loop breaks.
    // x is updated before the condition is checked.    
    cout<< x <<endl;
  }
  cin.get();
}
\end{verbatim}





\subsection{++ii or ii++ ?}

\begin{itemize}
  \item pre-increment: increase the value of ii and evaluate to the new
  incremented value
  
  \item post-increment: increase the value of ii, yet evaluate to the
  original non-incremented value, i.e. a temporary variable need to be used
\end{itemize}
In C++, the pre-increment (++ii) is usually prefered where you can use either.

This is because if you use post-increment, it can require the compiler to have
to generate code that creates an extra temporary variable. This is because both
the previous and new values of the variable being incremented need to be held
somewhere because they may be needed elsewhere in the expression being
evaluated.

So, in C++ at least, there can be a performance difference which guides your
choice of which to use.
\url{http://stackoverflow.com/questions/484462/difference-between-i-and-i-in-a-loop}


NOTE: In C\# and Java, the bytecode shows no difference between pre-increment
and post-increment.

% 
%  using ++i instead of i++ to increase a counter (assuming you do not need the
% result of the increment operator itself). Your compiler might optimise, but you
% lose nothing using ++i, and might win something, so you are better off using
% ++i.


\subsection{infinite loop}

\subsection{** while loop}

If we want to check the condition first
{\small \begin{verbatim} 
while ( condition ) { Code to execute while the condition is true }
\end{verbatim}}

If we want the body to be executed at least once
{\small \begin{verbatim}
do {
} while ( condition );
\end{verbatim}}

Example:
\begin{verbatim}
  int x = 0;  // Don't forget to declare variables
  
  while ( x < 10 ) { // While x is less than 10 
    cout<< x <<endl;
    x++;             // Update x so the condition can be met eventually
  }
\end{verbatim}

Example:
\begin{verbatim}
  int x;

  x = 0;
  do {
    // "Hello, world!" is printed at least one time
    //  even though the condition is false
    cout<<"Hello, world!\n";
  } while ( x != 0 );
  cin.get();
\end{verbatim}

\subsection{** range-based for loop (C++11)}
\label{sec:loop_range-based-for-loop}
\label{sec:for-loop-ranged-based}



\subsection{** for\_each loop}
\label{sec:loop_for_each}

Check Sect.\ref{sec:foreach_boost} in BOOST library, or since C++98, we can use 
\verb!for_each! loop for a container (Chap.\ref{chap:containers})

NOTE: The third argument can be a function pointer to a regular unary function
or (since C++11) a {\bf move constructible} function object
(Sect.\ref{sec:move-constructor-C++11}). If the function returns value, the
value is ignored.

\begin{lstlisting}
#include <algorithm>    // std::for_each
#include <vector>       // std::vector

void myfunction (int i) {  // function:
  std::cout << ' ' << i;
}


struct myclass {           // function object type:
  void operator() (int i) {std::cout << ' ' << i;}
} myobject;

void main()
{
  std::vector<int> myvector;
  
  for_each (myvector.begin(), myvector.end(), myfunction);
}

\end{lstlisting}
A single argument function is used with the argument is of the same type as the
data element in the container.

\url{http://www.cplusplus.com/reference/algorithm/for_each/}



\section{Conditional statements}

\subsection{'if' statements}

NOTE: In C, there is no logical data type, \verb!bool! is an alias to integer.
In C, 0 means false, and any non-zero value means true.

\begin{lstlisting}
if ( statement is TRUE )
    Execute this line of code;

if ( statement is TRUE ) 
{
    Execute line 1;
    Execute line 2;
}
   
if ( TRUE ) {
}
    /* Execute these statements if TRUE */
else {
    /* Execute these statements if FALSE */
}

if ( age < 100 ) {      /* If the age is less than 100 */
    printf ("You are pretty young!\n" ); /* Just to show you it works... */
}
else if ( age == 100 ) {    /* I use else just to show an example */ 
    printf( "You are old\n" );   
}
else {
    printf( "You are really old\n" );     /* Executed if no other statement is */
}
\end{lstlisting}

COMMON MISTAKES:

Example: using assignment instead of comparison, the assignment return 
\begin{verbatim}
// What will happen?
if (a=b) c; /* a always equals b, but c will be executed if b!=0 */
\end{verbatim}


Example: using multiple comparision, here the first \verb!0<a! is tested first
which returns either 0 or 1; the value is then compared with 5 which is always
true regardless of the value is 0 or 1.
\begin{verbatim}
  if( 0 < a < 5) c;      /* this "boolean" is always true! */
\end{verbatim}

Example:
\begin{verbatim}
 if( a =! b) c;      /* this is compiled as (a = !b), an assignment, rather than
                     (a != b) or (a == !b) */
\end{verbatim}



\subsection{Short-hand conditions}

A traditional C 'if' statement
\begin{verbatim}
if (a > b) {
    result = x;
} else {
    result = y;
}
\end{verbatim}
can be written as
\begin{verbatim}
result = (a>b)? x : y;
\end{verbatim}

Examples:
\begin{lstlisting}
int a(5), b(10);
(a > b)? std::cout << a : std::cout << b; // operation
int c = (a > b)? a : b;                   // assignment
int d = (a > b)? ((a > 10)? 10 : a) : b;  // nested
\end{lstlisting}

\begin{mdframed}
GNU C-extension allows
\begin{verbatim}
a = x ? x : y;
\end{verbatim}
to be written as
\begin{verbatim}
a = x ?  : y;
\end{verbatim}
This is important if 'x' is an expression whose state can be changed (by some
other side-effects) that make the second evaluation give a different result.
\end{mdframed}

C++ allows using the short-hand in left-side of the assignment
\begin{verbatim}
int a = 0;
int b = 0;

(argc > 1 ? a : b) = 1;
\end{verbatim}

\subsection{switch\ldots case}

NOTE: \verb!switch! statement is just a fancy kind of \verb!goto! statement (Sect.\ref{sec:goto}).  

We should not use \verb!switch ... case! statement on string data
\begin{verbatim}
  switch(std::string("raj")) //Compilation error - switch expression of type illegal 
    {
    case"sda":
    }
\end{verbatim}
The reason is that in order for C/C++ compiler to generate code for a switch
statement, it must know how to generate the code for value comparison. 
C++ language does not have a string type, i.e. std::string is part of the
Standard Library, but not part of the language. As a result, for string, it's
harder as it can be case sensitive, insensitive, culture aware, etc. 
% Additionally, C/C++ switch statements are typically generated as branch
% tables. It's not nearly as easy to generate a branch table for a string style switch. 


A common compiling error is
\begin{verbatim}
error: jump to case label
\end{verbatim}

Example: 
\begin{verbatim}
switch(foo) {
  case 1:
    int i = 42; // i exists all the way to the end of the switch
    dostuff(i);
    break;
  case 2:
    dostuff(i*2); // i is *also* in scope here, but is not initialized!
}
\end{verbatim}
The reason is that variables declared in one \verb!case! block are still visible 
in the subsequent \verb!case! unless an explicit \verb!{  }! block is used.

The solution is to enclose each \verb!case! block in braces \verb!{}!.
\begin{verbatim}
switch(foo) {
  case 1:
    {
        int i = 42; // i only exists within the { }
        dostuff(i);
        break;
    }
  case 2:
    dostuff(i*2); // Now you cannot use i accidentally --> error detected by compiler
}
\end{verbatim}

\url{http://stackoverflow.com/questions/5685471/error-jump-to-case-label}

\section{Logical operators}

\begin{verbatim}
Relational and comparison operators ( ==, !=, >, <, >=, <= )

Logical operators ( !, &&, || )

Conditional ternary operator ( ? )

Comma operator ( , )
\end{verbatim}
\url{http://www.cplusplus.com/doc/tutorial/operators/}

Example:
\begin{verbatim}
7==5 ? 4 : 3     // evaluates to 3, since 7 is not equal to 5.
\end{verbatim}

Example: The comma operator (,) is used to separate two or more expressions that
are included where only one expression is expected. When the set of expressions
has to be evaluated for a value, only the right-most expression is considered.  
\begin{verbatim}
 a = (b=3, b+2);
   // b is assigned the value 3
   // and then a is assigned the value of (b+2)
\end{verbatim}

\subsection{Bitwise operator}

\begin{verbatim}
Bitwise operators ( &, |, ^, ~, <<, >> )
\end{verbatim}

\subsection{Condition for string}

We cannot use 'switch' to compare string (or \verb!char*!). Some people
suggest using hash table using your own hash function. 
\begin{verbatim}
h=_myhash (mystring);
switch (h)
{
case 66452:
   .......
case 1342537:
   ........
}
\end{verbatim}
If you want to avoid the hashed-value, you can use
\begin{verbatim}
if (hash(str) == HASH("some string") ..
\end{verbatim}
An example of hash table implementation:
\url{https://developer.gnome.org/glib/stable/glib-Hash-Tables.html}.

You can use 'if..elseif...' statement, instead.


\subsection{Condition for function call}

You can use a map, with key='string', and value=pointer to the function-call.

\begin{verbatim}
typedef void (*funcPointer)(int);
\end{verbatim}
and create multiple functions
\begin{verbatim}
void String1Action(int arg);
void String2Action(int arg);
\end{verbatim}
and to map string content to function choice, we use
\begin{verbatim}
std::map<std::string, funcPointer> stringFunctionMap;
\end{verbatim}
and
\begin{verbatim}
stringFunctionMap.add("string1", &String1Action);
\end{verbatim}

We can also use hash table and function object which give O(1) insertion and
lookup
\begin{verbatim}
void func1() {
}
std::unordered_map<std::string, std::function<void()>> hash_map;
hash_map["Value1"] = &func1;
// .... etc
hash_map[mystring]();
\end{verbatim}

