\chapter{C++/CLI}

\section{C++/CLI}
\label{sec:VC++_C++/CLI}

A brief discussion of C++/CLI is given in Sect.\ref{sec:C++/CLI}.
To create a C++/CLI project, since Visual C++ 2005: (1) File/New/Project.. (2)
select Visual C++, and template CLR Empty
project.\footnote{\url{http://www.functionx.com/cppcli/classes/Lesson15c.htm}} 

\textcolor{red}{Compiling options}: A C++/CLI project needs to be compiled with
\verb!/clr! option (Sect.\ref{sec:clr_option}) to control the compiling at
module-level, and optionally using \verb!#pragma managed! to control at the
function-level to tell which function should be compiled to run on CLR
(Sect.\ref{sec:pragma_managed}).
\begin{itemize}
  \item  \verb!stdafx.h! is the default precompiled header for the console project to use
the extended features. Some other files can also be automatically generated. You
can remove them, not manually, but by turning off in the project properties.
  
\end{itemize}


\textcolor{red}{Example: Managed C++ vs. C++/CLI}
\begin{verbatim}
// Managed C++
__value class Vector {};  // value class (C++ context, compared to reference
                         //  class which is .NET context)
__interface I {};  //interface class
__gc __abstract  class Shape {};   // abstract class
__gc __sealed class Shaped2D : public Shape {} ; //derived class
System::ComponentModel::Container __gc *components;
Button* button1 = new Button; // managed heap
int *pi1 = new int;           // native heap
Int32 *pi2 = new Int32;       // managed heap


// C++/CLI
value class Vector {};
interface class I {};
ref class Shape abstract {};
ref clas Shaped2D sealed :: Shape {};
System::ComponentModel::Container ^ components;
Button ^ button1 = gcnew Button;
int * pi1 = new int;   // (no changed)
Int32 ^ pi2 = gcnew Int32; 
\end{verbatim}
\url{http://msdn.microsoft.com/en-us/library/ms235209.aspx}

Compared to Managed C++ (Sect.\ref{sec:managed_C++}), there are major changes in
C++/CLI, especially the new keywords are standardized, and no longer use
preceding two underscores \verb!__!

\begin{itemize}
  \item  standardize the keywords: 
  
  \begin{enumerate}
    \item removing double underscore in front of all keywords, with one
  exception: \verb!__identifier! is retained. 

Example: \verb!delegates! keyword : Sect.\ref{sec:delegates}  
\url{http://msdn.microsoft.com/en-us/library/ms235298.aspx}

     \item \verb!__gc! is replaced by \verb!ref! keyword in class (Sect.\ref{sec:ref_class}).

NOTE: To define a managed class or managed struct, we use \verb!ref class! or
  \verb!ref struct!.

NOTE: \verb!delete! to delete a managed object (which also call Dispose()
  implicitly), and make sure assigning \verb!nullptr! to it after that
\begin{verbatim}
delete pictureBox1->Image;
pictureBox1->Image = nullptr;

//reload
pictureBox1->Image = pictureBox1->Image->FromFile(...);
\end{verbatim}

     \item \verb!__gc *! is replaced by hats \verb!^! keyword, to declare
  a handle, i.e. a reference to a managed object. 

{\bf NOTE}: There are two types of pointers in Managed C++: \verb!__nogc!
pointers (normal C++ pointers), and \verb!__gc! pointers (referencing to managed
object - .NET reference types). In C++/CLI, there is only one pointer, i.e. the
normal C++ pointers. To access managed objects in C++/CLI, we use a {\it handle}
with the new syntax \verb!ClassName ^!, rather than \verb!ClassName *!.
This C++/CLI handle has nothing to do with Win32 HANDLEs and this handle is a
reference to a managed object on the CLI heap.

% \verb!^! symbol: a type decorator, which roughly correlates to \verb!*!
%   in C++. However, \verb!^! means the object is an C++/CLR object, not a
%   traditional C++ object.
\begin{verbatim}
Dog ^ mydog = gcnew Dog();

 // pass by address
void MyFunction(Dog ^ dog)
\end{verbatim}

The object is allocated on the so-called {\bf managed heap}, and passing
argument using \verb!^! means passing by address, similar to by reference like
in pointers.
\begin{verbatim}
// pass by reference
void MyFunction(Dog &dog);
\end{verbatim}
  
Accordingly, to allocate the real memory for that handle, \verb!new! is replaced
by \verb!gcnew! (and in C++/CLI, \verb!new! has only one meaning - allocating a
regular C++ pointer).
  
{\bf IMPORTANT}: Structure are value types, so we don't use \verb!gcnew!. 
  
  \end{enumerate}

% 
%   
%   \item  \verb!ref! keyword : reference class, 
  

  \item \verb!%! symbol: tracking referenence to handle (roughly correlates to \verb!&! in C++ (Sect.\ref{sec:reference-to-pointer}))
  
Example:
\begin{verbatim}
void func(ClassA^% thObj)
{
  //Modify what obj2 is referencing, to g_obj
  thObj=g_obj; // g_obj is a global object.
  
  //You can instantiate a new class
  thObj=gcnew ClassA();
  
  //Modify the variable thObj is referencing, through a its member function, SetInt().
  thObj->SetInt(3);
}


int main()
{
  ClassA^ obj = gcnew ClassA;
  ClassA^% obj2=&obj;
  
  func(obj2);
  ....
  return 0;

    array<String^> ^arr = gcnew array<String^>(10);
    int i = 0;
 
   // String^ is a handle
   // String^% is a reference to handle
    for each(String^% s in arr)
        s = i++.ToString();
} 
\end{verbatim}
  
  

  \item Type cast: C++/CLI provides 3 different operators \verb!static_cast!,
  \verb!dynamic_cast!, \verb!safe_cast!.
  
  \begin{itemize}
    \item \verb!static_cast! : the fastest of the thre operators, but it is the
    most dangerous (as it assumes the programmer know what he/she is doing, and
    it does no validity)
 \begin{verbatim}
 static_cast<int> (var);
 
 static_cast<ClassA ^> (var);
 \end{verbatim}
 
    \item \verb!dynamic_cast!: it is slower than \verb!static_cast! as it
    verifies that the type casting is valid, i.e. if it is not a valid
    conversion, return \verb!nullptr!. 
    
    \item \verb!safe_cast!: the closest match the conversion behavior of all
    other .NET supported languages.
\begin{verbatim}
if (dynamic_cast<ClassA ^> (ClassB) != 0)
{
  //ClassB is of type ClassA
}
\end{verbatim}
  \end{itemize} 
\end{itemize}

Example:
\begin{verbatim}
ref class Block {};                // reference class
value class Vector {};             // value class
interface class I {};        // interface class
ref class Shape abstract {};       // abstract class
ref class Shape2D sealed: Shape{}; // derived class
\end{verbatim}

% The changes from Managed C++ to C++/CLI
% \begin{itemize}
%   \item the double score \verb!__! in front of all keywords have been removed
%   
%   \item 
% \end{itemize}


In C++/CLI, managed objects can move around in memory as the garbage collector
runs. This means you can't get the real address of a member (without pinning) or
even embed anything except primitive types (i.e. int) into your class.  


{\bf SUMMARY}: To declare and define a reference object of a data type to be
managed by CLR, we need to declare that reference object using (1) .NET type,
(2) using hat keyword (\verb!^!), and (3) allocate the memory for that object
using \verb!gcnew!. 


\subsection{Mapping C\# to C++/CLI}

Table.\ref{tab:compare_C++/CLI_Csharp} shows the comparison
between using C\# code and C++/CLI for managed code.


\begin{table}[hbt]
\begin{center}
\caption{Techniques and its purpose}
\begin{tabular}{p{4cm}cc} 
\hline
Operation & C++/CLI & C\# \\
\hline\hline
namespace separation & \verb!::! & \verb!.! \\
access method & \verb!::! & \verb!.! \\
access data member & \verb!->!  & \verb!.! \\
 & \verb!this->components! & \verb!this.components! \\
create managed object & \verb!gcnew! & \verb!new! \\
create unmanaged object & \verb!new! & (not availale) \\ 
pragma & \verb!#pragma region RegionName! & \verb!#region RegionName! \\
 & \verb!#pragma endregion!   & \verb!#endregion! \\
object of reference type & \verb!Type ^ obj = gcnew Type()! & \verb!Type obj = new Type()! \\
after scope modifier & \verb!private: Type ^obj;! & \verb!private Type obj;! \\
floating literal & use dot at least & suffix \verb!f! or \verb!F! \\ 
      & \verb!8.! or \verb!8.2! & \verb!8F! or \verb!8.2F! \\
      & \verb!8.e0! & \\
type of an object & \verb!ClassType::typeid! & \verb!typeof(ClassType)! \\    
null & \verb!0! (Managed C++ only) & null \\
  & \verb!#include <stddef.h>! (add NULL definition) & \\
  & \verb!nullptr! (C++11 - use in C++/CLI) & \\   
\end{tabular}
\end{center}
\label{tab:compare_C++/CLI_Csharp}
\end{table}

To indicate a null object, 
\begin{itemize}
  \item Managed C++ allows using either 0 or NULL
  \item C++/CLI requires using \verb!nullptr!. 
  In the new syntax, 0 no longer represents a null address but is treated as an integer.
\end{itemize}
\url{https://msdn.microsoft.com/en-us/library/ms235261.aspx}

Example
\begin{verbatim}
// C#
System.ComponentModel.Container components;
this.components = new System.ComponentModel.Container();

// C++/CLI
System::ComponentModel::Container ^components;
this->components = gcnew System::ComponentModel::Container();

\end{verbatim}
\verb!gcnew! is for .NET reference objects; objects created with \verb!gcnew!
are automatically garbage-collected; it is important to use \verb!gcnew! with
CLR types

Example: to implement a callback
\begin{verbatim}
// C#
private void loadFieldToolStripMenuItem_Click(object sender, EventArgs e)
{
}

// C++/CLI
private: System::Void loadFieldToolStripMenuItem_Click(System::Object ^sender,
\ System::EventArgs ^e) {
}

\end{verbatim}

Example: 
\begin{verbatim}
// C#
System::ComponentModel::ComponentResourceManager resources =
 new System::ComponentModel::ComponentResourceManager(typeof(Form1));
this.imFruitSmall.ImageStream = 
(<System::Windows::Forms::ImageListStreamer)(resources->GetObject(L"imageList.ImageStream")));



// C++/CLI (old)
System::ComponentModel::ComponentResourceManager^ resources =
 gcnew System::ComponentModel::ComponentResourceManager(typeid<Form1>);
this->imFruitSmall->ImageStream = 
(stdcli::language::safe_cast<System::Windows::Forms::ImageListStreamer^ >
 (resources->GetObject(L"imageList.ImageStream")));

// C++/CLI (new)
System::ComponentModel::ComponentResourceManager^ resources =
 gcnew System::ComponentModel::ComponentResourceManager(Form1::typeid);
this->imFruitSmall->ImageStream = 
(cli::safe_cast<System::Windows::Forms::ImageListStreamer^ >
 (resources->GetObject(L"imageList.ImageStream")));

\end{verbatim}

IMPORTANT: 
\begin{enumerate}
  \item \verb!stdcli::language! namespace is removed in C++/CLI 2012, and is
  replaced by \verb!cli!
  \item \verb!typeid<object>! is replaced with \verb!object::typeid!
  \item \verb!->! is replaced by \verb!.!

\begin{verbatim}
// C#
this->staticColorFV->DrawingArea =
((System::Drawing::RectangleF)(resources->GetObject("staticColorFV->DrawingArea")));

// C++/CLI
this->staticColorFV->DrawingArea =
(cli::safe_cast<System::Drawing::RectangleF>(resources->GetObject(L"staticColorFV.DrawingArea")));
\end{verbatim}  
\end{enumerate}
\url{https://social.msdn.microsoft.com/forums/vstudio/en-US/4e75804b-b803-463f-8334-9d3ce5afa331/unable-to-use-c-express-beta1-project-in-beta2}

When you compile, you may get the error
\begin{verbatim}
DataVisualization is not a member of 
\end{verbatim}
even though you already add
\begin{verbatim}
using
\end{verbatim}
in the code. The reason is that you need to add the reference to the assemblies
\begin{verbatim}
using namespace System::Windows::Forms::DataVisualization;
using namespace System::Windows::Forms::DataVisualization::Charting;
\end{verbatim}

When you compile, you may get the error
\begin{verbatim}
operator = 'function is unavailable in ....
\end{verbatim}
this is because you declare the variable of .NET type without using \verb!^!
prefix.

To check the type of an object, we use \verb!::typeid! in C++/CLI, while in C\#
we use \verb!typeof()! operator.
\begin{verbatim}
MyObject^ mo = gcnew MyObject();
Object^ o = mo;

if( o->GetType() == MyObject::typeid )
{
    // Do somethine with the object
}
else
{
    // Try something else
}
\end{verbatim}
\url{http://stackoverflow.com/questions/2410721/how-to-check-an-objects-type-in-c-cli}

You need to use dynamic cast
\begin{verbatim}
// C#
((System.ComponentModel.ISupportInitialize)(this.trackBar1)).BeginInit();

// C++/CLI
(dynamic_cast<System::ComponentModel::ISupportInitialize^>(this->trackBar1))->BeginInit();
\end{verbatim}
\url{http://msdn.microsoft.com/en-us/library/system.componentmodel.isupportinitialize.aspx}


Reference: 
\begin{enumerate}
  \item \url{http://blogs.msdn.com/b/hsutter/archive/2003/11/23/53519.aspx}

  \item \url{http://en.wikipedia.org/wiki/C++/CLI}
\end{enumerate}

\subsection{namespace}
\label{sec:C++/CLI_namespace}

Using .NET namespace in C++/CLI can be done using \verb!using namespace! keyword
and the namespace are separated using \verb!::!

\begin{verbatim}
// C#
using Tietronix.SolarDataTypes;

// C++/CLI
using namespace Tietronix::SolarDataTypes;
\end{verbatim}


\subsection{enum class vs. enum}
\label{sec:enum-class-C++/CLI}

\verb!enum class! is a new feature from C++11 (Sect.\ref{sec:enum_class-C++11}).
To create a .NET-compatible enumeration, we use \verb!enum class!, not the
old C++ \verb!enum! type (Sect.\ref{sec:enum-C}). Also, later Visual C++ requires
using
\begin{verbatim}
public enum class
\end{verbatim}
to disambiguate from the \verb!enum class! contextual keyword which is part of
C++11 (and therefore creates a native type).

Example:
\begin{verbatim}
// C++/CLI
public enum class X { A, B, C };

// to use it in C++/CLI
X::A
\end{verbatim}

\subsection{array from C\# to C++/CLI}

The way to define an array in C\# is using []
\begin{verbatim}
// C#
System::Windows::Forms::DataGridViewColumn 
  object = new System::Windows::Forms::DataGridViewColumn[] {...};

object abc = new object[] {"None", "By Pies"};   
\end{verbatim}

However, in C++/CLI, we need to use \verb!array<ValueDataType>! or
\verb!array<ReferenceDataType^>!

\begin{verbatim}
System::Windows::Forms::DataGridViewColumn 
 ^object = gcnew array<System::Windows::Forms::DataGridViewColumn^> {...};
 
Object ^abc = gcnew array<Object^> {"None", "By Pies"}; 
\end{verbatim}


Example:
\begin{verbatim}
using namespace System;

array<double> ^ SingleFamilyValues = { 550550, 1215080, 625550, 850450 };

array<double> ^ SingleFamilyValues =
	gcnew array<double> { 550550, 1215080, 625550, 850450 };
	
	
array<double, 2> ^ MarketValues = gcnew array<double, 2>(2, 4);
	
    MarketValues[0,0] = 550550;
    MarketValues[0,1] = 1215080;
    MarketValues[0,2] = 625550;
    MarketValues[0,3] = 850450;

    MarketValues[1,0] = 483045;
    MarketValues[1,1] = 912834;
    MarketValues[1,2] = 328555;
    MarketValues[1,3] = 600425;	
\end{verbatim}

To explicitly indicate an array is one-dimensional array, we can use either (\textcolor{red}{a single list of items})
\begin{verbatim}
array<DataType, 1> ^ VariableName = gcnew array<DataType>(Dimension);

array<DataType, 1> ^ VariableName = gcnew array<DataType, 1>(Dimension);
\end{verbatim}
Example:
\begin{verbatim}
 array<__wchar_t, 1> ^ HouseTypes = gcnew array<__wchar_t>(4)
	{ L'S', L'C', L'S', L'T' };
    array<int, 1> ^ Bedrooms = gcnew array<int>(4) { 5, 2, 3, 3 };
    array<Byte, 1> ^ Stories = gcnew array<Byte>(4) { 3, 1, 3, 2 };
    array<double, 1> ^ Bathrooms = gcnew array<double>(4)
	{ 3.5, 1, 2.5, 1.5 };
    array<bool, 1> ^ HasGarage = gcnew array<bool>(4)
	{ true, true, false, false };
    array<double, 1> ^ Values = gcnew array<double>(4)
	{ 550500, 115000, 425000, 350000 };
\end{verbatim}


To define two dimensional array, we use (\textcolor{red}{two lists of items})
\begin{verbatim}
array<DataType, 2> ^ VariableName = { { Members-List1 }, { Members-List2 } };

array<DataType, 2> ^ VariableName = gcnew array<DataType, 2>(2, Dimension);
\end{verbatim}
The value 2 is called the rank of the array. The rank of an array specifies the number of sub-lists that the array contains. 
The array elements are created inside of curly brackets.
The main array is in fact made of two sub-lists, each sub-list must be created inside its own curly brackets.  

Example:
\begin{verbatim}
 array<double, 2> ^ MarketValues = {
              // List 1 
             { 550550, 1215080, 625550, 850450 },
             // List 2
             { 483045,  912834, 328555, 600425 } 
         };

array<double, 2> ^ MarketValues = gcnew array<double, 2>
    {
        { 550550, 1215080, 625550, 850450 },
        { 483045,  912834, 328555, 600425 }
    };
    
    
\end{verbatim}

To define multi-dimensional array, we can define 
\begin{enumerate}
  \item using the technique of a two-dimensional array
  
\begin{verbatim}
array<DataType, 2> ^ VariableName = gcnew array<DataType, 2>(Rank, Dimension)
{
    { Members-1 },
    { Members-2 },
    { Members-n } };
\end{verbatim}
The Rank factor specifies the number of sub-lists that the array will contain. The Dimension factor represents the number of elements that each sub-list will contain
  
  \item using the rank in the \verb!<>! operator to specify the number of pages of an array
\begin{verbatim}
array<DataType, Lists> ^ Variable = 
	gcnew array<DataType, Lists>(Lists, Sub-Lists, Elements)
\end{verbatim}

Example: the array will have 3 lists. Each list will have 2 sub-lists and each
sub-list will have 4 elements.
\begin{verbatim}
array<double, 3> ^ MarketValues = gcnew array<double, 3>(3, 2, 4)
\end{verbatim}
\end{enumerate}
then assign: To initialize the array when creating it, create the square
brackets for each list. Inside the square brackets of a list, create the square
brackets for the sub-list(s). Inside the square brackets of the sub-list,
specify the value of each element, separating them with commas.

\begin{verbatim}
   array<double, 3> ^ MarketValues = gcnew array<double, 3>(3, 2, 4)
    {
        // First List
        {
            // First Sub List
            {
                2250550, // 1st List - 1st Sub List - 1st Element
                1215080, // 1st List - 1st Sub List - 2nd Element
                1625550, // 1st List - 1st Sub List - 3rd Element
                850450   // 1st List - 1st Sub List - 4th Element
            },
            // Second Sub-List
            {
                558750, // 1st List - 2nd Sub List - 1st Element
                503150, // 1st List - 2nd Sub List - 2nd Element
		362650, // 1st List - 2nd Sub List - 3rd Element
		435755  // 1st List - 2nd Sub List - 4th Element
            }
        },
        // Second List
        {
            // First Sub-List
            {
                483045, // 2nd List - 1st Sub List - 1st Element
	        622440, // 2nd List - 1st Sub List - 2nd Element
                808445, // 2nd List - 1st Sub List - 3rd Element
		600425  // 2nd List - 1st Sub List - 4th Element
	    },
	    // Second Sub-List
	    {
		283045, // 2nd List - 2nd Sub List - 1st Element
                412834, // 2nd List - 2nd Sub List - 2nd Element
		328555, // 2nd List - 2nd Sub List - 3rd Element
		300425  // 2nd List - 2nd Sub List - 4th Element
	    }
	},
	// Third List
	{
	    // First Sub-List
	    {
		 624550,  // 3rd List - 1st Sub List - 1st Element
		 85655,   // 3rd List - 1st Sub List - 2nd Element
		 1250755, // 3rd List - 1st Sub List - 3rd Element
		 904685   // 3rd List - 1st Sub List - 4th Element
	    },
	    // Second Sub-List
	    {
		 324550, // 3rd List - 2nd Sub List - 1st Element
		 15655,  // 3rd List - 2nd Sub List - 2nd Element
		 147576, // 3rd List - 2nd Sub List - 3rd Element
		 92885   // 3rd List - 2nd Sub List - 4th Element
	    }
	}
    };
\end{verbatim}
\url{http://www.functionx.com/cppcli/arrays/multidimension.htm}

\subsection{fixed (C\#) to pin\_ptr or interior\_ptr (C++/CLI) to \_\_pin (Managed C++)}

The location of a managed object can be changed at any time once the garbage
collection is carried out. Thus, it is impossible to use native pointer pointing
to managed object. To resolve this problem, there are some solutions, either in C\#, C++/CLI or Managed C++.

Managed C++ used \verb!__pin! to indicate the CLR not moving the location of the managed object during garbage collection. 

C\# allows pinning the location of a managed object during the execution of a block of statement (not the whole program)
\begin{verbatim}
Point pt = new Point();
// Using fixed allows the address of pt members to be
// taken, and "pins" pt so it isn't relocated.
fixed ( int* p = &pt.x )
{
    *p = 1; 
}
\end{verbatim}
\url{https://msdn.microsoft.com/en-us/library/f58wzh21(v=vs.80).aspx}

C++/CLI provides \verb!pin_ptr! keyword and \verb!interior_ptr! keyword
\begin{itemize}
  
  \item interior pointer: the value of the pointer is updated at runtime to
  reflect the new location of the object's that is pointed to every time the
  managed object is relocated.
\begin{verbatim}
interior_ptr< type > var = [address];
\end{verbatim}  
  The physical address pointed to by the interior pointer never remains the same, but it always points to the same object. 
  
  \item pinning pointer: the value of the pointer is not changed, and it also prevents the GC from relocating the object, i.e. it always points to the same object.
  
 NOTE: We should use pinning pointer if we want to pass the poitner to an unmanaged function. 
   
\end{itemize}

\textcolor{red}{interior\_ptr}
\begin{Verbatim}
ref struct CData
{
    int age;
};

  CData^ d = gcnew CData();
    d->age = 100;
    
interior_ptr<int> pint = &d->age; // ((2))
\end{Verbatim}

With System::String, we can use \verb!PtrToStringChars! helper function
(declared in header file \verb!<vcclr.h>!) to get an interior pointer to the
underlying string buffer of a System::String object.
\begin{verbatim}
#include <vcclr.h>
\end{verbatim}

As it returns a \verb!const! interior pointer to the first character of the
System::String object, we need to \verb!const_cast! to convert to a
non-\verb!const! pointer, which allows us to loop through the string using
\verb!while! loop

\begin{Verbatim}

String^ str = "Nish wrote this book for Manning Publishing";
interior_ptr<Char> ptxt = const_cast< interior_ptr<Char> >(
    PtrToStringChars(str)); // ((1))

interior_ptr<Char> ptxtorig = ptxt; // ((2))

while((*ptxt++)++); // ((3))
Console::WriteLine(str); // ((4))
while((*ptxtorig++)--); // ((5))
Console::WriteLine(str); // ((6))    
\end{Verbatim}

\begin{verbatim}
#using <mscorlib.dll>
#include <vcclr.h>  // PtrToStringChars()

__const_Char_ptr p = PtrToStringChars(str);
\end{verbatim}

\textcolor{red}{pin\_ptr}
\begin{verbatim}
// compile with: /clr 

array<int>^ arr;   // CLR integer array
arr = gcnew array<int>(SIZE);

pin_ptr<int> p = &arr[0];   // pin pointer to first element in arr

// using unmanaged pointer 
int* np = p;   // pointer to the first element in arr

// and pass to the unmanaged function
native_function(np);   // pass pointer to native function
\end{verbatim} 

The difference:
\begin{itemize}
  \item \verb!fixed! (C\#) creates its own scope of execution
  
  \item \verb!pint_ptr! (C++/CLI) scope is from its declaration until the end of the enclosing block
\end{itemize}

\subsection{<app.config>: persistable settings}

In .NET, <app.config> is the file that can be used to store configuration
settings for a given .NET assembly. You first need to add the file to the
project, add some data to it. After successfully compiling the project, the file
is automatically copied to the target folder of the project and renamed into the
name \verb!<appName>.exe.config!.

In C++/CLI, we can also create this file
\begin{verbatim}
Right-clock on Project
   -> add -> new item 
       -> configuration file (app.config)
\end{verbatim}
and add some settings in XML-based format. The major predefined
configuration section is \verb!<configuration></configuration>!

Example:
\begin{verbatim}
<configuration>
  <appSettings>
    <add key="name" value="pyright"/>
    <add key="firstname" value="nico"/>
  </appSettings>
</configuration>
\end{verbatim}
The information is stored in the form of key/value pair. For more information on
this file structure, please read the C\# Manual Book.

In the code, we can read the setting from the file, or save to it
\begin{verbatim}
// read the value using 'key' name as the index 
String ^name = Configuration::ConfigurationManager::AppSettings["name"];
String ^firstName = 
    Configuration::ConfigurationManager::AppSettings["firstname"];
Console::WriteLine("My name's {0} {1}", firstName, name);
\end{verbatim}
We can also use the numerical index, but it can be confused if the order change
\begin{verbatim}
String ^name = Configuration::ConfigurationManager::AppSettings[0];
\end{verbatim}
To iterate all values, we do
\begin{verbatim}
for each(String ^aValue in ConfigurationManager::AppSettings)
{
    Console::WriteLine("User: {0} - {1}", 
        aValue, ConfigurationManager::AppSettings[aValue]);
}
\end{verbatim}

Unfortunately, C++/CLI has no knowledge about configuration file management, and
thus the file, if created, is not automatically copied to the target
destination and renamed properly. To enable this in C++/CLI, we need to add
\verb!System.Configuration! reference to your code
\begin{verbatim}
right click on project -> references 
    -> add new references.
        at .NET tab, choose System.Configuration
\end{verbatim}
and then (3) add post-build events
that copy the file, rename it into the target folder. 
\begin{verbatim}
right-click project properties 
   -> configuration properties -> build events 
         -> post build event 
\end{verbatim}
at \verb!Command Line! add 

\begin{verbatim}
copy app.config "$(TargetPath).config"
\end{verbatim}


\subsection{byte in C\# to unsigned char in C++/CLI}

C\# has two types: byte or System.Byte. However, C++ does not have \verb!byte!
data type; instead it has \verb!unsigned char! which is equivalent to
\verb!System.Byte!. So we need to either (1) switch to using \verb!System.Byte!
or \verb!unsigned char!, or (2) define new type in C++/CLI code

\begin{verbatim}
typedef unsigned char byte;
\end{verbatim}

\subsection{string}

In .NET, \verb!string! and \verb!System.String! are the same, and both are
reference type. 

In C++, \verb!std::string! is ISO C++ data type. In C++/CLI, we use
\verb!System::String^! 

Suppose we have two functions:
\begin{verbatim}
void display( String^ );
void display( const string& );
\end{verbatim}
when we pass a string literal, which one is used?
\begin{verbatim}
display( "which one?" );
\end{verbatim}

Under C++/CLI, a string literal goes through a hierarchical sequence of standard
conversions, as follows (slightly simplified),
\begin{enumerate}
  \item as \verb!const char*! is used for a string literal, the compiler search
  for the version of the function that accept \verb!const char *! first. If not,
  then it discard \verb!const! and try to find the version of the function that
  accept \verb!char *!.
  
  In terms of string literal, C++/CLI by default uses the same as ISO-C++, i.e.
\begin{verbatim}
const char *
\end{verbatim}
   
  \item If there is no version accepting \verb!char *!, then it searchs for
  \verb!String^!.
  
  \item If there is no version accepting \verb!String^!, then it searchs for the
  one that accepts a \verb!String^! base classes, e.g. \verb!Object^! or an
  interface. 
  
  \item If there is none of the above, then it searchs for the best matches as
  a \verb!bool! parameter. 
\end{enumerate}
\url{http://blogs.msdn.com/b/slippman/archive/2004/05/20/135879.aspx}
% However, if we pass a string literal in a context in which a
% \verb!System::String^! Unicode string literal is required

{\bf Empty string}: An empty string is a zero length string, a string that is
equal to null (""), or not assigned. C++/CLI uses
\verb!String::IsNullOrEmpty(s)! to check if a string is empty. 
\url{http://msdn.microsoft.com/en-us/library/system.string.isnullorempty(v=vs.110).aspx}
To check for null string, we compare with \verb!nullptr!

\begin{verbatim}
String^ s;
//s = ""; //Uncomment to test 2nd case.
if (String::IsNullOrEmpty(s))
{
  MessageBox::Show("empty string");
}

static String^ GetValue(String^ section, String^ key, String^ defaultValue)
{
    if (defaultValue == nullptr)
        // special case handling
    ....
}
\end{verbatim}
\url{http://www.prestwood.com/ASPSuite/KB/CrossRef.asp?SyntaxID=56&LangID=14}
\subsection{Construct a decimal value}

\begin{verbatim}
// C#
int[ ] bits;
decimal decimalNum = new decimal( bits );
decimal decimalNum = new decimal( new int[ ] { 0, 0, 0, 0 } );


// C++/CLI
array<int> ^bits;
Decimal decimalNum = Decimal(bits);
Decimal decimalNum = Decimal(gcnew array<Int32> {0,0,0,0};);
 
\end{verbatim}
\url{http://msdn.microsoft.com/en-us/library/t1de0ya1(v=vs.80).aspx?cs-save-lang=1&cs-lang=cpp#code-snippet-4}

Example:
\begin{verbatim}
array<String^>^ gc1 = gcnew array<String^>{"one", "two", "three"};
array<String^>^ gc2 = {"one", "two", "three"};
\end{verbatim}
\url{http://msdn.microsoft.com/en-us/library/dtbydz1t.aspx}


\subsection{Default value: method argument}

If the method belong to a managed type, then it is not allowed, or you will get
C3222 error \url{http://msdn.microsoft.com/en-us/library/zk03y5fe.aspx}
\begin{verbatim}
// error C3222
public ref class ManagedType
{
	void SaveFieldConfigurationAs(bool saveFieldSubset = false)
	{
	}
}
\end{verbatim}
The solution is to create 2 overloadded functions: one without full parameters,
and one without optional arguments. Inside this second overloaded function,
call the first one and pass the default value.

\begin{verbatim}
public ref class ManagedType
{
  private:
	void SaveFieldConfigurationAs()
    {
		SaveFieldConfigurationAs(false);
	}
	void SaveFieldConfigurationAs(bool saveFieldSubset)
	{
	}
}
\end{verbatim}
% , and It is not allowed to pass
% default value to managed type
% \begin{verbatim}
% // COMPILER ERROR
% private: void SaveFieldConfigurationAs(Boolean saveFieldSubset = false)
% \end{verbatim}
% 
% We can use default value for unmanaged type only
% \begin{verbatim}
% private: void SaveFieldConfigurationAs(bool saveFieldSubset = 0)
% \end{verbatim}

We can use C\# 4.0 feature, i.e. the new attribute
\verb!Optional! and C\# 2.0 feature, i.e. \verb!DefaultParameterValue!
attribute. However, we cannot use that in C++/CLI, e.g. the following code
\begin{verbatim}
public ref class MyClass1
{
 public:
  MyClass1([System::Runtime::InteropServices::Optional]
           [System::Runtime::InteropServices::DefaultParameterValue(2)]
           int myParam1)                                            
  {
    System::Console::WriteLine(myParam1);
  }
};
\end{verbatim}

The reason is that the C\# compiler don't use the [DefaultParameterValue] for
setting the correct default value in the metadata; it uses \verb!.param!
directive to get the value embedded in the metadata (Check the specs
in Partition II, Chapter 15.4.1; but 15.4.1.4 is silent about it. The .para m
directive) that is visible in C\#.
So in C\# the deault parameter is visible but only with the system default value. 
In the case of boolean type, \textcolor{red}{a workaround solution is using
Nullable} as for nullable type, the default is always ``null'' (.HasValue ==
false)

\begin{verbatim}
// in C++/CLI header file
String^ test([Optional] Nullable<bool> boolTest);

// in C++/CLI .cpp file
String^ YourClass::test(Nullable<bool> boolTest) 
{ 
if (!boolTest.HasValue) { boolTest = true; } 
return (boolTest ? gcnew String("True") : gcnew String("False")); 
}


// when we use it in C#
MessageBox.Show(YourClass.test());
\end{verbatim}
\url{https://visualstudio.uservoice.com/forums/121579-visual-studio/suggestions/5639618-c-cli-support-for-optional-parameters-for-using}



\url{http://stackoverflow.com/questions/15454394/why-c-cli-has-no-default-argument-on-managed-types}

\subsection{Event handler}
\label{sec:event_handler_C++/CLI}

In C++/CLI, the EventHandler delegate type requires an object of type
\verb!EventArgs! as the second argument, not a string.

There are 4 options:
\begin{enumerate}
  \item create an instance of \verb!System::EventHandler!, if the underlying
  delegate is a global handler and has the standard signature for .NET events
  (object + event args)
  
\begin{verbatim}
delegate void MyOwnEventHandler(Object^ sender, EventArgs^ e) { }  
myEventSource->MyEvent += gcnew EventHandler(MyOwnEventHandler); 
\end{verbatim}
  
  \item create an instance of \verb!System::EventHandler! if the underlying
  delegate has the standard signature for .NET events and the event handler is a
  class method
\begin{verbatim}
ref class EventReceiver {
public:
   void Handler(Object^ sender, EventArgs^ e) {  }
   
   void something()
   {
      // (1)  use inside the class
      myEventSource->MyEvent += gcnew EventHandler(this, \
                               &EventReceiver::Handler);
   }
};

// (2) use outside the class
myEventSource->MyEvent += gcnew EventHandler(myEventReceiver, \
                               &EventReceiver::Handler);
\end{verbatim}
  
  \item define a delegate and create the instance of that delegate, if the event
  is a custom one (you define yourself as a class member)
  
\begin{verbatim}
delegate void Del(int, float);
ref class EventReceiver {
public:
    void Handler(int i , float f) {  }
};


myEventSource->MyEvent += gcnew Del(myEventReceiver, &EventReceiver::Handler);
\end{verbatim}

  \item create an instance of \verb!System::EventHandler! generic that takes
  a \verb!MyEventArgs! arg as parameter
\begin{verbatim}
ref class EventReceiver {
public:
   void Handler(Object^ sender, MyEventArgs^ e) {  }
};
myEventSource->MyEvent += gcnew EventHandler<MyEventArgs^>(this, &EventReceiver::DataReceived);
\end{verbatim}
\end{enumerate}
BEST:
\url{http://stackoverflow.com/questions/4942311/how-do-i-assign-an-event-handler-to-an-event-in-c-cli}


\begin{verbatim}
// C#
someObj.SomeEvent += new EventHandler(Blah_SomeEvent);

private void Blah_SomeEvent(object sender, EventArgs e)
{
}

// C++/CLI
someObj->SomeEvent+= gcnew EventHandler(this, &Blah_SomeEvent);
\end{verbatim}

\url{http://manski.net/2011/04/cpp-cli-cheat-sheet/}
\begin{verbatim}
// C#
class SomeClass 
{
  private void loadFieldToolStripMenuItem_Click(object sender, EventArgs e)
  {
  }
  public void SomeClass::RegisterHandler(){
    textBox.TextChanged += gcnew EventHandler(this->TextChanged_Handler); 
  }
}

// C++/CLI
ref class SomeClass 
{
  private: void loadFieldToolStripMenuItem_Click(Object ^sender, EventArgs ^e)
  {
  }
  
  public: void SomeClass::RegisterHandler(){
    textBox->TextChanged += gcnew EventHandler(this, &SomeClass::TextChanged_Handler);
  }
}  
\end{verbatim}

\url{http://msdn.microsoft.com/en-us/library/58cwt3zh.aspx}


\subsection{User interface}

It is possible to use Winforms in C++/CLI applications (Check the book on GUI
Development, and chapter on Winforms)

\begin{verbatim}
#include "MainForm.h"

using namespace System;
using namespace System::Windows::Forms;

[STAThread]
void main(array<String^>^ arg) 
{
   Application::EnableVisualStyles();
   Application::SetCompatibleTextRenderingDefault(false);
   
   //Create the form instance
   CUDAFluxGUI::MainForm form;
   
   Application::Run(%form);

}
\end{verbatim}

\subsection{Delete a UI control}

During design, we saw that, if you have added a control to your application but
you don't want that control anymore, you could delete it. In some rare cases,
you may also want to allow the user to delete a control (instead of hiding it).
When a control is deleted, the Control::ControlRemoved event is fired.   

\begin{verbatim}
ublic: event ControlEventHandler^ ControlRemoved;
\end{verbatim}
This event is carried by a ControlEventHandler delegate through the
ControlEventArgs class (Sect.\ref{sec:event_handler_C++/CLI}).

\subsection{Link .NET dll to C++/CLI project}

You have two options: (1) \verb!#using! statement; (2) add references.

The first option requires
\begin{itemize}
  \item Name the DLL, with or without path (recommend not to use path as the
  location of the DLL during compiling and running of the application can be
  different)
  
\begin{verbatim}
#using "CUDAFluxGUIComponent.DLL"
\end{verbatim}
You may need to add \verb!/AI! and path name to the DLL in your project's
property
\begin{verbatim}
C/C++
  CommandLine Arguments
\end{verbatim}
or 
\begin{verbatim}
C/C++
  General
    Using (#using) directive
\end{verbatim}
\end{itemize}

For the second option, you just need to add (1) compile dependencies, and (2)
add references to the DLL. Then the Visual Studio implicitly add \verb!/FU!
compiling option and the folder path to the DLL. 

In order to use the class from the DLL, make sure it is \verb!public!.
Otherwise, you may get the error
\begin{verbatim}

\end{verbatim}

\subsection{Link native LIB to C++/CLI}

Just add the reference to the .LIB file.

\url{http://stackoverflow.com/questions/4002269/linking-unmanaged-c-dll-with-managed-c-class-library-dll?lq=1}

\subsection{Link native DLL to C++/CLI  project}

Suppose you have a C++ code in another DLL project
\begin{verbatim}
// In the unmanged file (in the unmanged dll project)
using std::vector;
namespace populationwin32 
{
  class PopulationWin32
  {
  public:
    PopulationWin32();
  // ...
  };
}
\end{verbatim}


\textcolor{red}{IMPORTANT: }

In your current C++/CLI project, you want to call the class defined in that
project. You may get error 
\begin{verbatim}
Error   1   error C2653: 'populationwin32' : is not a class or namespace name
\end{verbatim}

The reason is that you need to tell the compiler that the code being used is
defined in another library, and the interface (i.e. header file) to that code
\begin{enumerate}
  \item Add the path to the header files
\url{http://msdn.microsoft.com/en-us/library/t9az1d21(v=vs.80).aspx}

  \item Add \verb!#include! to the header file in your C++/CLI code
\begin{verbatim}
#include "PopulationWin32.h"
\end{verbatim}
\end{enumerate}


The header file of the DLL
\begin{verbatim}
#ifdef  MYDLL_EXPORTS 
#define DLLCALL __declspec(dllexport)   /* Should be enabled before compiling 
                                           .dll project for creating .dll*/
#else
#define DLLCALL __declspec(dllimport)  /* Should be enabled in Application side
                                          for using already created .dll*/
#endif

// Interface Class
class ImyMath {
public:
    virtual ~ImyMath() {;}
    virtual int Add(int a, int b) = 0;
    virtual int Subtract(int a, int b) = 0;
};

// Concrete Class
class MyMath: public ImyMath {
public:
    MyMath() {}
    int Add(int a, int b);
    int Subtract(int a, int b);
    int a,b;
};

//  Factory function that will return the new object instance. (Only function
//  should be declared with DLLCALL)
extern "C" /*Important for avoiding Name decoration*/
{
    DLLCALL ImyMath* _cdecl CreateMathObject();
};

// Function Pointer Declaration of CreateMathObject() [Entry Point Function]
typedef ImyMath* (*CREATE_MATH) ();
\end{verbatim}

The DLL source file
\begin{verbatim}
#include "dllHeader.h"

// Create Object
DLLCALL ImyMath* _cdecl CreateMathObject() {
    return new MyMath();
}

int MyMath::Add(int a, int b) {
    return a+b;
}

int MyMath::Subtract(int a, int b) {
    return a-b;
}
\end{verbatim}

The code to use the DLL
\begin{verbatim}
 #include <iostream>
#include <windows.h>
#include "dllHeader.h"

int main()
{
    HINSTANCE hDLL = LoadLibrary(L"MyDLL.dll"); // L".\Debug\MyDLL.dll"

    if (hDLL == NULL) {
        std::cout << "Failed to load library.\n";
    }
    else {
        CREATE_MATH pEntryFunction = (CREATE_MATH)GetProcAddress(hDLL,"CreateMathObject");
        ImyMath* pMath = pEntryFunction();
        if (pMath) {
            std::cout << "10+10=" << pMath->Add(10, 10) << std::endl;
            std::cout << "50-10=" << pMath->Subtract(50, 10) << std::endl;
        }
        FreeLibrary(hDLL);
    }
    std::cin.get();
    return 0;
}
\end{verbatim}
\url{http://stackoverflow.com/questions/27998/exporting-a-c-class-from-a-dll/28022#28022}

\subsection{Link a DLL that uses CUDART with C\#/C++/CLI}

Suppose you have C++/CLI code or C\# code and want to use CUDA-capable Thrust
library. As soon as you include Thrust's header file (no need to call a
function), you may get linker errors.

In C\# or C++/CLR, we cannot link anything compiled with /MT. 

Using \verb!/MD cudart!, we can use \verb!cudart! in a DLL that we then link
with a C\#, C++/CLR application. 
An easier way is to use P/Invoke into any native DLL, regardless of linker
setting. However, if it really does get rid of the P/Invoke
overhead (especially when Marshalling is required to use), that could be a
pretty big deal for using CUDA through .NET.
\url{https://devtalk.nvidia.com/default/topic/408277/-mt-versus-md-dlls-for-cuda/}




