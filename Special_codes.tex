\chapter{Special Codes}
\label{chap:special_code}


\section{CRC (Cyclic Redundancy Check)}
\label{sec:CRC}

It's important to check the integrity of a raw-data after being transmitted via
a digital network. The reason is that data is transmitted in blocks via a noisy
(error-introducing) channel. To avoid error, the transmitter generate a value
(i.e. the {\bf checksum}) based on a given function with the input is the
transmitted datablock
\begin{verbatim}
CRC-value = f(datablock)
\end{verbatim}
and then append this to the message. 

When the message (datablock + the CRC-value) is received at the other end, it
uses the same function, generate a new value using the datablock, and compare
with the CRC-value. If the two values are the same, then the datablock is
correctly received.

The worst case is that the datablock and/or CRC-value is corrupted, but the
final calculation gives a match CRC-value. It's impossible to avoid this
completely, but we can minimize the chance to occur by increasing the amount of
information in the checksum function, e.g. using two bytes instead of one byte. 

The basic idea of an CRC algorithm is to treat the datablock as an enormous
binary number, divided it by another fixed binary number, and then get the
remainder from this division as the CRC-value.

 Examples of check-sum function
\begin{enumerate}
  \item the datablock is viewed as sequence of positive integers
  
Example: simply the sum of the bytes in the message mode 256
  
\begin{verbatim}
   Message                    :  6 23  4
   Message with checksum      :  6 23  4 33
   Message after transmission :  6 27  4 33
\end{verbatim}
Here, the chance for error to be undetected is 1/256.

NOTE: A better approach is to widen the bytes, e.g. using 16-bit register
instead of 8-bit, and then sum the bytes mod 65536 instead of 256. Here, the
chance for error to be undetected is 1/65536, which is much smaller.
 
 
  \item the datablock is viewed as a (message) polynomials with binary
  coefficients, e.g. each number is treated as a bit-string whose bits are the
  coefficient of a polynomial, i.e. the polynomial whose coefficients are either
  0 or 1. Then the CRC-value is the remainder of this operation: the polynomial
  representation of the datablock is multiplied by $x^n$, and then divided
  by a given CRC-polynomial. The coefficients of the remainder polynomial is the
  CRC-value.
\begin{equation}
M(x) \times x^n = Q(x) \times G(x) + R(x)
\end{equation}
with M(x) is the representation of the datablock, G(x) is the CRC-polynomial,
and R(x) is the remainder or the CRC-value.

Example: the number 23 (decimal) is 10111, so each bit is the coefficient
\begin{verbatim}
1*x^4 + 0*x^3 + 1*x^2 + 1*x^1 + 1*x^0
\end{verbatim}
or simply
\begin{verbatim}
x^4 + x^2 + x^1 + x^0
\end{verbatim}
  
  Here, \verb!x! is assumed 2. By changing the rule how the coefficients work,
  different CRC functions were developed. Selecting a good polys is very
  important. The goal is to avoid: SINGLE-BIT ERROR, TWO-BIT ERROR, ERROR WITH
  AN ODD NUMBER OF BITS, BURST ERRORS
 \end{enumerate}


Using poly is important in developping CRC algorithms. It turns out that the
reflection of a good poly (e.g. G=11101) is also a good poly, e.g. if
G=10111. The set of standard good polys are defined as CRC-CCITT

\begin{verbatim}
   X25   standard: 1-0001-0000-0010-0001
   X25   reversed: 1-0000-1000-0001-0001

   CRC16 standard: 1-1000-0000-0000-0101
   CRC16 reversed: 1-0100-0000-0000-0011
\end{verbatim}


References:
\begin{enumerate}
  \item \url{http://www.zlib.net/crc_v3.txt}
  \item
  \url{http://www.drdobbs.com/implementing-the-ccitt-cyclical-redundan/199904926}
\end{enumerate}
