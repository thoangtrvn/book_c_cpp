\chapter{GMP library}
\label{chap:GMP-lib}

There are two header files that you can choose
\begin{itemize}
  \item C lang: \verb!gmp.h!
  
  \item C++ lang: \verb!gmpxx.h!
\end{itemize}
To make sure GMP support C++, compile with \verb!--enable-cxx! 
\begin{verbatim}
./configure --prefix=~/gmp --enable-cxx
make
make install
\end{verbatim}

In Code
\begin{verbatim}
#include <stdio.h> // required for FILE* parameter

#include <stdarg.h> // required for va_list parameter
                    // e.g. gmp_vprintf()
#include <obstack.h>  // required for 'struct obstack' parameter
                   // e.g. gmp_obstack_printf()
#include <gmp.h> 
\end{verbatim}


Compile: 
\begin{verbatim}
 // 
g++ mycxxprog.cc -lgmp

// GMP C++ functions
g++ mycxxprog.cc -lgmpxx -lgmp
\end{verbatim}

NOTE: If the GMP library is installed to non-standard location, use '-I' and
'-L' compiler options to point to the right directories.

\section{Data type }
\label{sec:GMP-datatypes}

GMP uses a different name for intrinsic data type, e.g. integer usually means a
multiple precision integer.

GMP uses 
\begin{enumerate}
  \item \verb!mpz_t! type for integers with multiple precision
  
  \item \verb!mpq_t! type for rational numbers with multiple precision
  
  \item \verb!mpf_t! type for floating-point numbers with arbitrary precision
  mantissa with a limited precision exponent. 
  
  
  \item \verb!mp_limb_t! type:
  
  A limb means the part of a multi-precision number that fits in a single
  machine word. (We chose this word because a limb of the human body is
  analogous to a digit, only larger, and containing several digits.) Normally a
  limb is 32 or 64 bits. 

  \item \verb!mp_size_t! type: which counts of limbs of a multi-precision
  number. Currently this is normally a long, but on some systems it's an
  \verb!int! for efficiency, and on some systems it will be \verb!long long! in
  the future.
  
  \item \verb!mp_bitcnt_t! type: which counts of bits of a multi-precision
  number. Currently this is always an \verb!unsigned long!, but on some systems
  it will be an unsigned long long in the future.
  
  \item \verb!gmp_randstate_t! type: represents an algorithm selection and
  current state data.
  
  
\end{enumerate}
In general \verb!mp_bitcnt_t! is used for bit counts and ranges, and
\verb!size_t! is used for byte or character counts.

Example:
\begin{verbatim}
mpz_t sum;

struct foo { mpz_t x, y; };

mpz_t vec[20];
\end{verbatim}


\subsection{integer}

\url{https://gmplib.org/manual/Initializing-Integers.html}


\subsection{floating-point}



\section{IMPORTANT: before using data}

As data type in GMP is defined as a single data element, it is important to
initialize them before using it
\begin{verbatim}
mpz_t data;
mpz_init(data);

// use 'data'
\end{verbatim}


\section{Types of functions}

There are 6 classes of functions in GMP which can be distinguish based on the
prefix
\begin{enumerate}
  \item \verb!mpz_! functions (about 150 functions): those manipulate signed
  integer arithmetic:
  
  \item \verb!mpq_! functions (about 35 functions): 
  
  \item \verb!mpf_! functions (about 75 functions)
  
  \item \verb!mpn_! functions (about 60 functions):
  operate on natural numbers, mainly works with \verb!mp_limb_t! datas, and hard
  to use.
  
  
  \item miscellaneous functions: those for generating random numbers; and
  custome allocations.
\end{enumerate}

\section{Classes}

Instead of using intrinsic data types in Sect.\ref{sec:GMP-datatypes}, we can
use a class-form of these types by adding \verb!gmpxx.h! header file.

Three classes
\begin{verbatim}
Class: mpz_class
Class: mpq_class
Class: mpf_class
\end{verbatim}
The standard operators and various standard functions are overloaded to allow
arithmetic with these classes.

\begin{verbatim}
mpz_class a, b, c;
...
mpz_gcd (a.get_mpz_t(), b.get_mpz_t(), c.get_mpz_t());
\end{verbatim}

There are no namespace setups in gmpxx.h, all types and functions are simply put
into the global namespace. This is what gmp.h has done in the past, and
continues to do for compatibility. The extras provided by gmpxx.h follow GMP
naming conventions and are unlikely to clash with anything.


\section{Convert back to C data type}

IMPORTANT: All operations in GMP mainly utilized data of GMP data types. 
At the end, we need to convert back to C types for further processing using
other APIs.

\begin{itemize}
  \item for integer:
  \url{https://gmplib.org/manual/Converting-Integers.html}
\end{itemize}

\section{Passing data type as function parameter}

\url{https://gmplib.org/manual/Parameter-Conventions.html}

When a GMP variable is used as a function parameter, it's effectively a
call-by-reference, meaning if the function stores a value there it will change
the original in the caller. Parameters which are input-only can be designated
const to provoke a compiler error or warning on attempting to modify them.

How could this happens? For interest, the GMP types \verb!mpz_t! etc. are
implemented as one-element arrays of certain structures. This is why declaring a
variable creates an object with the fields GMP needs, but then using it as a
parameter passes a pointer to the object. Note that the actual fields in each
\verb!mpz_t! etc are for internal use only and should not be accessed directly
by code that expects to be compatible with future GMP releases.

