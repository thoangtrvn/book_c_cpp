\chapter{Pseudo-random number generators}


A true random number sequence is a sequence that has no repetition,
i.e. the cycle is infinity. Indeed, you can never have a true random
number generator. However, when the cycle (period length) is big
enough, it is statistically indistinguishable from a true random
sequence. So, such sequences is called a {\it pseudorandom} (or
quasi-random) number sequence. The generator is called pseudo-random
number
generator\footnote{\url{http://infohost.nmt.edu/tcc/help/lang/fortran/random.html}}.

Each generator need one initial value or a sequence of values called
the {\it seed} numbers. With a given seed number, the generator will
produce a unique sequence of pseudorandom values.
\textcolor{red}{Two sequences from the same seed number(s) are the
  same; thus is convenient for debugging purpose or performance
  comparisons}.

Labs doing research on RGN:
\begin{itemize}
\item \url{http://random.mat.sbg.ac.at/}: Univ. of Salzburg
\end{itemize}

While a long period is not a guarantee of quality in a random number generator,
short periods (such as the $2^{32}$ common in many older software packages) can be
problematic. 

At the lowest level, the solution of generating random numbers is the solution
that generate bits (0 and 1) at different positions of the $k$-bit words at the
same likelihood (for uniform distribution). This is the role of the {\bf engine} (term used in C++11).
\textcolor{red}{From the sequence of uniform distribution, we can generate the sequence of any other distribution.}
This is the role of the {\bf distribution} component.



\section{Mersene Twistter}
\label{sec:PRNG_Mersenne-Tiwstter}

{\bf Mersenne Twister} (developed 1997 by Matsumoto and Nishimura) is the most widely used PRNG methods.
It is the default PRNG for many programming languages (R, Python, IDL, MatLab, MS Visual C++)

It was the first to provide (1) fast generation of high-quality PRNs, i.e. overcome most of the flaws found in older PRNGs.

The sequence is based on the Mersenne prime, a prime number in the form $M_n =
2^n-1$, i.e. a prime number which is one less than a power of two.
\url{http://en.wikipedia.org/wiki/Mersenne_prime}

The commonly used version is 
\begin{itemize}
  \item MT19937: use 32-bit word length with very long sequence of $n=19937$
  
  \item MT19937-64: use 64-bit word length, and generate a different sequence.
\end{itemize}

For many applications, the state space $2^{252}-1$ is enough.
In 2011, Saito \& Matsumoto proposed a version of the Mersenne Twister to
address this issue. The tiny version, TinyMT, uses just 127 bits of state space.

The seed is a $k$-bit word length, which is then used to generate integers in the range $[0, 2^k-2]$.


\subsection{SFMT, dSFMT}
\label{sec:sfmt}
\label{sec:dSFMTs}

SFMT, the Single instruction, multiple data-oriented Fast Mersenne Twister, is a
variant of Mersenne Twister, introduced in 2006, designed to be fast when it
runs on 128-bit SIMD. Intel SSE2 and PowerPC AltiVec are supported by SFMT.

SFMT stands for SIMD-oriented Fast Mersenne Twister. SFMT generates
single-precision data. For double precision, it has dSFMT.
dSFMT only works on CPUs which use IEEE 754 format double precision
floating point numbers.

The current version of SFMT (1.3.3) supports the periods from
$2^{607}-1$ to $2^{216091}-1$.

The current version of dSFMT (2.1) supports the periods from
$2^{521}-1$ to $2^{216091}-1$.

References:
\begin{itemize}
\item
  \url{http://www.math.sci.hiroshima-u.ac.jp/~m-mat/MT/SFMT/index.html}
\item \url{http://www.math.sci.hiroshima-u.ac.jp/~m-mat/MT/emt.html}
\item \url{http://www.codeproject.com/KB/DLL/SFMT_dll.aspx}
\end{itemize}



\subsection{MTGP}
\label{sec:mtgp}

MTGP is a variant of Mersenne Twister optimised for graphics processing units (GPU)
published by Mutsuo Saito and Makoto Matsumoto.

MTGP stands for Mersenne Twister for GPU Processor that supports both
32-bit and 64-bit (1) unsigned integer and (2) floating point numbers
based on IEEE 754 formats. Thus, we have totally 2 set of files,
representing by their prefixes: \verb!mtgp32-! and \verb!mtgp64-!.
For each version, correspondingly, there are two different
implementations:
\begin{itemize}
\item simple version: representing by their suffixes, \verb!-ref.c!
\item fast version : representing by their suffixes, \verb!-fast.c!
\begin{verbatim}
mtgp32-fast.h               mtgp64-fast.h
mtgp32-fast.c               mtgp64-fast.c
mtgp32-param-fast.c         mtgp64-param-fast.c
\end{verbatim}
\end{itemize}

MTGP can generate uniformly distributed pseudo-random sequences of
some given periodic length \verb!mexp!, e.g. ${23209}, {11213}$...
\begin{itemize}
\item 64-bit RNG has periods of either $2^{23209}-1$, $2^{44497}-1$
  and $2^{110503}-1$.
\item 32-bit RNG has periods of either $2^{11213}-1$, $2^{23209}-1$
  and $2^{44497}-1$.
\end{itemize}
Each version support some predefined set of seed values for some given
periods \verb!mexp!. In version 1.0.2, each period has total 128
different sets of seed values, each set will help generate one random
sequence. All sets of seed data are stored in two files: one for the
simple version (\verb!mtgp64-param-ref.c!), one for the fast version
(\verb!mtgp64-param-fast.c!).

\begin{framed}
  If you want more parameter set, i.e. more random sequences, you need
  to use MTGP Dynamic Creator (MTGPDC) which can create $2^{32}$
  parameter sets for the following periods $2^{3217}, 2^{4423},
  2^{11213}, 2^{23209}, 2^{44497}$ (ver. 1.0.2). Download the source
  code from the same website.
\end{framed}

We will discuss how to work with the ``fast'' 64-bit version using one
of the 128 given parameter sets for seeding.  The file
\verb!mtgp64-param-fast.c! contains three arrays
\begin{verbatim}
mtgp64_params_fast_23209
mtgp64_params_fast_44497
mtgp64_params_fast_110503
\end{verbatim}
of the derived type \verb!mtgp64_params_fast_t!.

\begin{framed}
  The code only works on systems that support IEEE 754 floating-point
  format. Its, however, run very slow on CPU; and fast on GPU.
\end{framed}

What we need to do
\begin{enumerate}
\item \verb!params! point to the correct structure element in one of
  the three above arrays (given via \verb!mexp! and \verb!no! values)
\begin{lstlisting}
mtgp64_params_fast_t * params;

int mexp; // periods, either 44497, 23209 or 110503
          //+ for double precision
int no;   // should be within 0 to 127

read_in(&mexp, &no);

switch (mexp) {
case 44497:
  params = mtgp64_params_fast_44497;
  break;
case 23209:
  params = mtgp64_params_fast_23209;
  break;
case 110503:
  ...
default:
  ...
}

if (no >= 128 || no < 0) {
  printf("error");
}
\end{lstlisting}

\item initialize internal state
  \begin{itemize}
  \item (CPU version) given by the variable \verb!mtgp64!
  \begin{itemize}
  \item  \verb!mtgp64-init! allocates and init the
    internal state (first argument) given \verb!params! (2nd argument)
    and a single integer seed (third argument)
  \item  \verb!mtgp64_init_by_array! .... and an array of
    seed values (3rd argument, with length information is the 4th
    argument)
  \item  \verb!mtgp64_init_by_str! .... and the string is
    terminated by zero (3rd argument)
  \end{itemize}
\begin{lstlisting}
mtgp64_fast_t mtgp64;

/* Seed value */
/* --> scalar or an array or a string array */
uint64_t seed = 1;
uint64_t seed_ar[4] = {1, 2, 3, 4};
char seed_str[] = "\01\02\03\04";

params += no;
rc = mtgp64_init(&mtgp64, params, seed);

rc = mtgp64_init_by_array(&mtgp64, params, seed_ar, 4);
\end{lstlisting}

\item (GPU version)  (\verb!d_status)!
  \begin{itemize}
  \item  \verb!mtgp64_init_state! allocates and init the
    internal state (first argument of type \verb!uint_64!) given
    \verb!params! (2nd argument) and a single integer seed (third
    argument). This function is called internally by
    \verb!make_kernel_data!
  \item \verb!make_kernel_data (d_status, params)!
  \end{itemize}
\begin{lstlisting}
#define MEXP 23209
#define LARGE_SIZE (THREAD_NUM*3)
#define BLOCK_NUM_MAX 200
#define TBL_SIZE 16

mtgp64_kernel_status_t * d_status;

int block_num;
int block_num_max;

\end{lstlisting}
\end{itemize}


\item  generate random data
  \begin{itemize}
  \item (CPU version) (given the internal state \verb!mtgp64!)
  \begin{itemize}
  \item \verb!mtgp64_genrand_uint64(mtgp64)! : generate a 64-bit
    single 64-bit unsigned integer.
  \item \verb!mtgp64_genrand_close1_open2! : generate a
    double-precision pseudo-uniformly distributed in the range [1,2]
  \item \verb!mtgp64_genrand_close_open! : generate a double-precision
    pseudo-random uniformly distributed in the range $[0,1)$
  \item \verb!mtgp64_genrand_open_close! : generate a double-precision
    pseudo-random uniformly distributed in the range (0,1]
  \item \verb!mtgp64_genrand_open_open! : generate a double-precision
    pseudo-random uniformly distributed in the range (0,1)
  \end{itemize}

\begin{lstlisting}
/* Generate 100 random values */
for (i=0; i< 100; i++) 
   rnd_value[i] = mtgp64_genrand_uint64(mtgp64);
\end{lstlisting}

\item (GPU version) generate random data (given the internal state
  \verb!d_status!)
  \begin{itemize}
  \item \verb!mtgp64_uint64_kernel(mtgp64_kernel_status *d_status!,
    \verb!uint64_t* data, int size)!: generate an array of 64-bit
    unsigned integer.
  \item \verb!mtgp64_double_kernel(d_status, *d_data, int size)!:
    generate an array of double precision floating point numbers in
    \verb!d_data!. 
  \item \verb!make_constant(const params[], int block_num)!: set
    constants in device memory
  \item \verb!make_texture(const params[], *d_texture_tbl[3],!  int
    \verb!block_num)!: set constants in device memory

  \item CALL FROM HOST: \verb!make_uint64_random(*d_status!, int
    \verb!num_data, int block_num)! - generate 64-bit unsigned random
    numbers, \verb!make_double_random(*d_status, int num_data!,
    \verb!int block_num)!  - generate double precision random numbers
  \end{itemize}
\end{itemize}
\end{enumerate}

\subsubsection{Sample codes}
\label{sec:sample-codes}

In the folder \verb!/cuda_sample!, there are some files
\begin{itemize}
\item \verb!mtgp32-*!: for 32-bit
\item \verb!mtgp64-*!: for 64-bit
  \begin{itemize}
  \item \verb!mtgp64-cuda13-tex.cu!: CUDA 2.3, with texture and SM1.3
    (support double precision)
  \item \verb!mtgp64-cuda-tex.cu!: CUDA 2.3, with texture and
    single-precision only
  \item \verb!mtgp64-cuda.cu!: CUDA 2.2
  \item \verb!mtgp64-cuda-common.c!: to be included by the above files
  \end{itemize}
\end{itemize}


References:
\begin{itemize}
\item \url{http://www.math.sci.hiroshima-u.ac.jp/~m-mat/MT/MTGP/index.html}
\end{itemize}

\section{Libraries}

CUDA:
\begin{itemize}
  \item CURAND: pseudo-random number generators
  \item SARU: Sect.\ref{sec:saru}
\end{itemize}


\section{Generate data in parallel}
\label{sec:gener-data-parall}

\subsection{CUDA Sample MersenneTwister}
\label{sec:cuda-mersennetwister}

CUDA C has a sample implemented the MersenneTwister with cycle
$2^{607}-1$. Each twister is run for each thread

\subsection{SPRNG}
\label{sec:sprng}

SPRNG stands for Scalable Parallel Number Generator. It use MPI for
parallel data generator.

References:
\begin{itemize}
\item \url{http://sprng.cs.fsu.edu/}
\end{itemize}


\subsection{SARU}
\label{sec:saru}

SARU is a 32-bit PRNG that can provide a single-precision or
double-precision uniform distributed random number in the range
$[0,1)$. This base PRNG can be used to generate many other
distributions. 

SARU was developed by Steve Worley.