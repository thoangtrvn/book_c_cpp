\chapter{Compile}
\label{chap:compiler}


The C source code files can be compiled into different targets (static lib,
shared lib, executable file), for different platform (32-bit, 64-bit, ARM).
In general, compilation goes through 2 steps:
{\bf compiling} and {\bf linking}. 

First, the compiler convert the source code (the .cpp file) to object file
(*.o). At this time, the code may have reference to non-completed types, i.e.
those whose declaration are available via header files; but whose definition are
not available yet.
In the first step, the compiler takes a {\bf compilation unit}
(Sect.\ref{sec:compilation-unit}).

At linkage stage, then, the linker takes all the object files, together with
statistically-linked libraries, to create a binary program.

It's important to know how to best organize the code (Sect.\ref{sec:modules-in-C++}).

Different IDE tools allows us to do this easily. Chap.\ref{chap:Visual_C++}
describes how to compile C/C++ projects using Visual Studio.
Chap.\ref{chap:compile_Csharp} how to use Visual Studio for C\# language. To
compile your code into a library, you should also read Chap.\ref{chap:library}.

\section{Manage complexity of a C/C++ project}
\label{sec:modules-in-C++}

C and C++ programs normally take the form of a collection of separately compiled
modules.
As a big project is developed, an new executable can be built rapidly if only
the changed modules need to be recompiled.

\textcolor{red}{\bf What is a module in C/C++?}:
In C++, the contents of a module consist of structure type (struct)
declarations, class declarations, global variables, and functions.
The functions themselves are normally defined in a source file (a “cpp” file).

Each source (.cpp) file has a header file (a “.h” file) associated with it
that provides the declarations needed by other modules to make use of this
module.

The idea is that other modules can access the functionality in module X simply
by \verb!#include!'ing the “X.h” header file, and the linker will do the rest.

The code in X.cpp needs to be compiled only the first time or if it is changed;
the rest of the time, the linker will link X’s code into the final executable
without needing to recompile it, which enables the Unix make utility and IDEs to
work very efficiently.

Usually, the main module (i.e. the one with \verb!main()! function) does not
have a header file, since it normally uses functionality in the other modules,
rather than providing functionality to them.

IMPORTANT: compilers and linkers need help in enforcing the One Definition Rule
(Sect.\ref{sec:one-definition-rule}).

\textcolor{red}{Well-designed header file to reduce the coupling}:

Reduce the amount of “coupling” between components by minimizing the number of
header files that a module’s header file itself \verb!#include!s. On very large
projects (where C++ is often used), minimizing coupling can make a huge differ-
ence in “build time” as well as simplifying the code organization and debugging.

\subsection{Design a good module}
\label{sec:module-in-C++-good-design}
\label{sec:compilation-unit}

Each module with its .h and .cpp file should correspond to a clear piece of
functionality.
The Standard Library modules <cmath> and <string> are good examples of clearly
distinct modules.
{\it Don’t force together into a module things that will be used or maintained
separately, and don’t separate things that will always be used and maintained
together.}


Always use “include guards” in a header file. Choose a guard symbol based on the
header file name, since these symbols are easy to think up and the header file
names are almost always unique in a project.
For example “Geometry_base.h” would start with:
\begin{verbatim}
#ifndef GEOMETRY_BASE_H
#define GEOMETRY_BASE_H
\end{verbatim}
and end with:
\begin{verbatim}
#endif
\end{verbatim}
 
 
If module A needs to use module X’s functionality, it should always 
\begin{verbatim}
#include “X.h”
\end{verbatim}	
, and never contain hard-coded decla- rations for structs, classes,
globals, or functions that appear in module X.

The reason is that: If module X is changed, but you forget to change the
hard-coded declarations in module A, module A could easily fail with subtle
run-time errors that won’t be detected by either the com- piler or linker.
  

\textcolor{red}{RULE 6}:
Set up global variables for a module with an extern declaration in the header
file, and a defining declaration in the .cpp file.
\begin{verbatim}
// in header file
// place an extern declaration in the .h 
// it means that any files included this header file
//  can use 'g_number_of_entities' (without asking where it is)
//  it's location/definition will be provided at linkage time
//  by linking the object file (compiled form the .C file containing the unique definition
//  of 'g_number_of_entities' (below)

extern int g_number_of_entities;
\end{verbatim}
and in source file
(the .cpp file for the module) must include this same .h file, 
and near the beginning of the file, a defining declaration should appear.
This declaration both defines and initializes the global variables

\begin{verbatim}
// the source file
// where unique-definition occurs

// static/global variables are initialized to zero by default; but initializing
// explicitly to zero is customary because it marks this declaration as the
// defining declaration, meaning that this is the unique point of definition.

int g_number_of_entities = 0;
\end{verbatim}

NOTE: different C compilers and linkers will allow other ways of setting up
global variables, but this is the accepted C++ method for defining global
variables and it also works for C to ensure that the global variables obey the
One Definition Rule.
 
\url{http://umich.edu/~eecs381/handouts/CppHeaderFileGuidelines.pdf}



\section{Translation units}

\label{sec:translation-unit}


A {\bf translation unit} is not "a header and a source file". It could include a
thousand header files (and a thousand source files too).
A translation unit is typically a pre-processed source file (i.e a source file
with all the contents of all header files)

A translation unit is simply what is commonly known as "a source file" or a
``.cpp file'' after being preprocessed. If the source file \#includes other
files the text of those files gets included in the translation unit by the
preprocessor. There is no difference between C and C++ on this matter.

 

\section{Locale of the source files}
\label{sec:locale_source-files}

The source code can be stored in ASCII scheme or Unicode scheme or some other
encoding scheme (Sect.\ref{sec:locale}). Before you can compile your program,
the compile must be able to read the source code. So, it's important to tell the
compiler which encoding scheme of the source file
\begin{itemize}
  \item ILE (Integrated Language Environment) C/C++ compiler from IBM can
  compile code written in single--byte EBCDIC CCSID (Coded Character Set
  Identifier).
  
IBM i (previously named OS/400 then i5/OS) is EBCDIC-based O/S runs on IBM Power
Systems and IBM PureSystems. Extended Binary Coded Decimal Interchange Code
(EBCDIC)  is an 8-bit character encoding used mainly on IBM mainframe and IBM
midrange computer O/S. 
  
  There are two supported locales: *CLD and *LOCALE. Use either but not both
  \begin{itemize}
    \item *LOCALE: based on IEEE POSIX P1003.2, ISO/IEC 9899:1990/Amendment
    1:1994[E], and X/Open Portability Guide standards for global locales and
    coded character set conversion. 
    \item *CLD: for ILE C use only (not ILE C++). Programs compiled prior to
    V3R7 use the *CLD locale support
  \end{itemize}
  \url{http://publib.boulder.ibm.com/iseries/v5r2/ic2924/books/c092712331.htm}
  
  \item GNU C/C++ : 
\end{itemize}

\section{C Preprocessor cpp}
\label{sec:cpp-preprocessor}

\verb!cpp! is the GNU C macro processor, that is used automatically by the C
compiler (or C++ compiler, or Objective-C compiler) to transform your source
code, before compilation.

How it is used: it expects 2 file names as arguments (one for infile, and one
for outfile).
It reads the \verb!infile! file, along with ALL files specified via the
\verb!#include! statement in that file. 
\begin{itemize}
  \item we can substitue one or both with \verb!-! (a dash) which means standard input (i.e. from terminal) or standard output (i.e. the terminal)
\end{itemize}

\begin{verbatim}
cpp <options>
    infile  outfile
\end{verbatim}
Options to use:

\begin{enumerate}
  \item what ISO standard of the code:
  
\begin{verbatim}
-std=c90

-std=c99

-std=c11
\end{verbatim}
  
Also use \verb!-pedantic! to get all the mandatory diagnostics.  
  
  \item any macro to be defined (so that the check \verb!#ifdef MACRO_NAME! or 
  \verb!#if defined(MACRO_NAME)!) returns True
  
\begin{verbatim}
-D MACRO_NAME
\end{verbatim}
you define any name in place of \verb!MACRO_NAME!

  \item any macro to be defined with a given value
\begin{verbatim}
-D MACRO_NAME=value
\end{verbatim}  
  
  
\end{enumerate}


\subsection{Preprocessed files}
\label{sec:preprocessed-file}

To stop after the preprocessing stage, and do no compilation, we use \verb!-E!
option to \verb!g++! or \verb!gcc!. Those who don't need preprocessing are
ignored. The preprocessed output is written to the standard output. 


\subsection{Preprocessed + Compilation}

If we want to stop at compilation, and do no assemble, we use \verb!-S! option.
Those who don't need compilation are ignored. The output is an assembler code
file for each input file specified. By default, the assembler code file has the
same name with the input file, with extension is \verb!.s!


\subsection{Preprocessed + Compilation + Assemble}

If we want to stop at compilation, and do no linking, we use \verb!-c! option.
Unrecognized input files, not requiring compilation or assembly are ignored. The
output is in the form of an object file for each input file, with extension
\verb!.o!


\subsection{Preprocessed + Compilation + Assemble + Linking}
\label{sec:preprocessed_compile_assemble_linking}

When the symbol (function or global identifier) is {\it internal linkage}, it means that it's
only accessible in one translation unit. An {\it external linkage} means that
the symbol is accessible throughout the program.

A linkage to a symbol can be controled as internal or external using
\verb!static! and \verb!extern!, respectively. By default, \verb!const! symbols
are \verb!static!, and non-\verb!const! symbols are \verb!extern!.

\begin{verbatim}
// in namespace or global scope
int i; // extern by default

const int ci; // static by default
extern const int eci; // explicitly extern

static int si; // explicitly static

// the same goes for functions (but there are no const functions)
int foo(); // extern by default
static int bar(); // explicitly static 
\end{verbatim}

For internal linkage, instead of using \verb!static! keyword, it's suggested to
use {\bf anonymous namespaces}

\section{How a C-code runs: -nostartfiles}

When you run a binary C-program, unlike assembly language, some basic
pre-requisites need to be satisified before the control is transferred to this
C-code. These setups include
\begin{enumerate}
  \item stack
  \item global variables: initialized, uninitialized
  \item read-only data
\end{enumerate}

\begin{verbatim}
static int arr[] = { 1, 10, 4, 5, 6, 7 };
static int sum;
static const int n = sizeof(arr) / sizeof(arr[0]);

int main()
{
        int i;

        for (i = 0; i < n; i++)
                sum += arr[i];
}
\end{verbatim}

Stack is used for storing local (auto) variables (Sect.\ref{sec:heap_stack}).
The memory grows downward (i.e. toward lower address). The memory location of
the initial stack pointer is stored in the {\bf r13} register or the alias {\bf
sp} in assembly
\begin{verbatim}
ldr sp, =0xA4000000
\end{verbatim}
The \verb!.data! section contains the initialized global data by the compiler.
The C-language guarantees all uninitialized global variables will be initialized
to zero. The section \verb!.bss! is used for uninitialized variables. The
global \verb!const! data are saved in \verb!.rodata! section (e.g. string
constants); the content of this section cannot be modified.

Once the pre-requisites are satisfied.

\subsection{How a C-code initialize: \_start(), \_\_libc\_start\_main(), \_\_libc\_csu\_init(), main()}	
\label{sec:_start()}
\label{sec:__libc_start_main()}

In the program with blank \verb!main()!
\begin{verbatim}
int main() {}
\end{verbatim}

Once you compile, you can use \verb!objdump -d! on the binary a.out, you will see it evokes two subprograms
\begin{itemize}
  \item \verb!_start()!: setup aligment, push arguments on stack, 
  and then call \verb!__libc_start_main()!.
  
  \item \verb!__libc_start_main()!: the first line to execute is 
  
\begin{verbatim}
jmp *0x8049658
\end{verbatim}
which is the indirect branch instruction, which actually jumping back to \verb!__Libc_start_main()! at this time
is the location loaded in RAM at runtime. The real RAM address of \verb!__libc_start_main()! is found in 
DYNAMIC RELOCATION RECORDS table, which is created in RAM by the dynamic loader when the program is loaded. This is called {\bf lazy loading} mechanism.

This function, once loaded into RAM, evokes \verb!__libc_csu_init()!, main(), and \verb!exit()! in that order.

\verb!__libc_csu_init()! is the constructor of the program, it is called by
\verb!__libc_start_main()! before actually running the user-defined main() function.

\end{itemize}

\url{dbp-consulting.com/tutorials/debugging/linuxProgramStartup.html}


\subsection{main(), wmain()}
\label{sec:main_wmain}

Once the program has been loaded into the memory (i.e. after the execution of
\verb!__start()!, \verb!__libc_start_main()!), the program entry point 
\verb!main()! is executed (both Windows and Linux).
\begin{verbatim}
int main();
int main(int argc, char* argv[]);
\end{verbatim}

In Windows, to adhere to Unicode
programming model, we use \verb!wmain()! to allows the program to handle
wide-character command lines
\footnote{\url{http://msdn.microsoft.com/en-us/library/bky3b5dh.aspx}}
\begin{verbatim}
int wmain( int argc, wchar_t* argv[]);
int wmain( int argc, wchar_t *argv[ ], wchar_t *envp[ ] )
\end{verbatim}
If we use \verb!wmain()!, then all parameters and arguments need to follow
Unicode (Sect.\ref{sec:WindowsDataTypes}).

While Windows applications prefer UTF16, Unix applications still want UTF8 for
Unicode strings encoding. Even not defined in C++, Windows also provide an
extension to entry point, called \verb!_tmain()! to allows easy switching
between multi-byte or Unicode.
\footnote{\url{http://stackoverflow.com/questions/895827/what-is-the-difference-between-tmain-and-main-in-c}}

\begin{verbatim}
 int _tmain(int argc, _TCHAR *argv[]) 
\end{verbatim}
which can be mapped to
\begin{verbatim}
int wmain(int argc, wchar_t *argv[])
\end{verbatim}
or
\begin{verbatim}
int main(int argc, char *argv[])
\end{verbatim}

NOTE: All the identifiers in the code needs to have \verb!_t! prefix, e.g.
\verb!_TCHAR, _tcout!. 
\begin{verbatim}
#include <iostream>
#include <tchar.h>

#if defined(UNICODE)
    #define _tcout std::wcout
#else
    #define _tcout std::cout
#endif

int _tmain(int argc, _TCHAR *argv[]) 
{
   _tcout << _T("There are ") << argc << _T(" arguments:") << std::endl;

   // Loop through each argument and print its number and value
   for (int i=0; i<argc; i++)
      _tcout << i << _T(" ") << argv[i] << std::endl;

   return 0;
}
\end{verbatim}

\subsection{without main()}
\label{sec:without_main()}

The only places we see a C application without main() are either embedded
applications or device drivers. For embeded applications, we can define its own
entry point.
\begin{verbatim}
#include<stdio.h>
#include<stdlib.h>
_start()
{
   exit(my_main());
}
int my_main()
{
   printf("Hello");
   return 0;
}
\end{verbatim}
and then compile with
\begin{verbatim}
gcc  -nostartfiles  hello.c 
\end{verbatim}


The latter one tend to be loaded into a running program.

\subsection{-nostdlib}

Some additional work is required if you want to use standard C library. So
\verb!-nostdlib! means do NOT use it. Then, only the libraries you specify will
be passed to the linker, and any options specifying linkages to the system
libraries (e.g. -static-libgcc, -shared-libgcc) are ignored.

\subsection{How a C-code terminate: exit(), atexit(), \_Exit(), std::quick\_exit (since C++11)}	
\label{sec:exit()}
\label{sec:atexit()}
\label{sec:_Exit()}
\label{sec:quick_exit()}

In C, \verb!exit()! terminates the calling process, and returns the error code
as passed via the argument to the function.

\begin{verbatim}

exit(0); 

exit(1);
\end{verbatim}
\verb!exit()! does some cleaning before termination of the program, e.g.
\begin{itemize}
  \item connection termination
  
  \item flush the buffers
  
  \item \ldots
\end{itemize}
You can registered a list of functions to get called \verb!

In C11 and C++11, it introduces \verb!_Exit()! which terminates the program
immediately, i.e. without performing any cleanup tasks nor running function
registered with \verb!atexit()!.
Example: you won't see any output to the terminal 
\begin{lstlisting}

#include <bits/stdc++.h>  // needs for atexit()
#include <stdlib.h> //needs for _Exit()

int main(void)
{
  int exit_code = 0;
  print("Terminating using _Exit"); //this is buffered, and is not flushed to the terminal right away
  _Exit(exit_code);  //exit, and not flush the buffer
}
\end{lstlisting}


Example: we can register a function, which will be evoked at normal completion of main() or at \verb!exit()! calls
\begin{lstlisting}
void func(void)
{
   std::cout << "Exiting";
}

int main(void)
{
  atexit(func);
  
  exit(10);  //which will evoke func() before termination
}
\end{lstlisting}

\verb!std::atexit()! then calls \verb!__cxa_atexit()! to register that function.
This is not the source standard, and it is only binary standard.
On the other hand, \verb!atexit()! is not in the binary standard, but it is in
the source standard.


Since C++11, it supports \verb!std::quick_exit()! function which causes normal
program termination without complete.y cleaning the resources. What it actually
does is to evokes all the functions registered using
\verb!std::at_quick_exit()!, and in the reverse order of their registration.
If any exception occurs in such functions, \verb!std::terminate()! is called. If
all registered functions complete with no exception, then calls to
\verb!std::_Exit()! occurs, using the given exit code.
IMPORTANT: Functions registered with \verb!std::atexit()! are not called by
\verb!quick_exit()!.


\begin{lstlisting}
#include <cstdlib>

void quick_exit(int exit_code) noexcept;
\end{lstlisting}

\begin{mdframed}

WHY adding \verb!std::quick_exit()!?

The feature was added to specifically deal with the difficulty of ending a
program cleanly when you use threads.
As \verb!exit! is started by a highly asynchronous event, e.g. user closing the
GUI, or admin shutting down the machine, \ldots without regarding to the states
of the threads the program has started, such threads are almost always in a
highly unpredictable state.




\end{mdframed}

\subsection{Threads created with MPI\_Init}


An MPI program, using a minimal 1 process, needs at least 2 threads. It means
that after \verb!MPI_Init()! is evoked, two threads are created by the MPI
library, and after \verb!MPI_Finalize()!, the two threads are closed.

\subsection{When a global scope variable get destructed?}

What is the lifetime of these different file-scope, and probably namespace-scope, variables
\begin{lstlisting}


MyClass myobj;  //global-scope

static MyClass myobj; 

const MyClass myobj;

static const MyClass myobj;


static constexpr MyClass myobj; //since C++11
\end{lstlisting}


For static function-scope variable:
\begin{lstlisting}
void somefunc()
{
   static MyClass myobj; //when its initialization actually occured?
}


\end{lstlisting}


The destructor of file or namespace-scope objects get called when the control
flow leaves main(). So, if the destruction requires a third-party library, which
is already unloaded before the destructor get called. This is a problem.


\section{include (header) files}
\label{sec:include-files}

There are two different ways to include a header file
\begin{verbatim}
// Option 1
#include  "headerfile.h"

// Option 2 
#include <headerfile.h>
\end{verbatim}

\begin{enumerate}
  \item Option 1:
  the preprocessor searches in for the file in folders that are
  implementation-defined
  \begin{verbatim}
  //POSIX-based implementation
  search order
  1. current folder: ./headerfile.h
  2. the directories as specified in -I before -iquote option (or -I- for
  older gcc)  [default: -iquote come last, if not specified] 
  3. directories as used with <> in option2
  
NOTE: -I- (deprecated; use -iquote instead) can be put at any point in the list
of -I options 
  Effect 1:
    Directories appearing before the -I- in the list are searched
    only for headers requested with quote marks.
    Directories after -I- are searched for all headers
  Effect 2:
    the directory containing the current file is not searched for anything,
    unless it happens to be one of the directories named by an -I switch
    
    
NOTE: special folder name '-'
  write -I./-    
  \end{verbatim}
  and causes the replacement of that directive by the
  entire contents of the header. This method is normally used to include programmer-defined header files.
  
NOTE: 
\begin{verbatim}
"mypath/myfile" 
     is short for 
./mypath/myfile
\end{verbatim}  
  
  If this search is not supported, or if the search fails, the directive is
  reprocessed as if it read \verb!#include <headerfile.h>!
  
  \item Option 2:
   searches in an implementation dependent manner, normally in search
  directories pre-designated by the compiler/IDE, 

\begin{verbatim}
  //GCC
  search order
1.
(C & C++) directories specified via -I option
   (if multiple, then search folders specified from left-to-right order)

2.     
(C++ only) libdir/../include/c++/<version>

3.
(C & C++)
     /usr/local/include
     libdir/gcc/<target>/version/include
     /usr/target/include
     /usr/include
     
NOTE: 
  target is the canonical name of the system GCC was configured to compile
code for; often but not always the same as the canonical name of the system it runs on. 
  version is the version of GCC in use.
  
NOTE: You can prevent GCC from searching any of the default directories with the
-nostdinc option
\end{verbatim}
  and causes the replacement of
  that directive by the entire contents of the header.. This method is normally
  used to include standard library header files.
  
NOTE:
\begin{verbatim}
<mypath/myfile> 
     is short for 
<defaultincludepaths>/mypath/myfile
\end{verbatim}
  
\end{enumerate}
\url{https://gcc.gnu.org/onlinedocs/cpp/Search-Path.html}

Use of \verb!-I! option: To tell the directory where the header file (.h) reside, you
use the -I option or specify the full path to the header file
\begin{verbatim}
gcc .... -I /usr/local/pvm/include/
\end{verbatim}

In GNU GCC, an environment variable equivalent to \verb!-I! is 
\verb!CPATH! (which is used regardless of the language is being processed).
Search order: path in \verb!-I! first, and then path in \verb!CPATH!. For
language-specific, we can use the below environment variables:
\begin{itemize}
  \item \verb!C_INCLUDE_PATH!: C-lang (C header files) 
    
  \item \verb!CPLUS_INCLUDE_PATH!: C++-lang (C++ header files)
  
  \item \verb!OBJC_INCLUDE_PATH!: ObjectiveC-lang
  
  \item \verb!CPATH!: for all languages
\end{itemize}

% In the second case, the INCLUDE statement must be the first
% statement\footnote{\url{http://www.netlib.org/pvm3/book/node140.html}}.

A good habit is to define these macros in the beginning of the makefile.
\begin{verbatim}
MAKE =	make
CC   =	gcc
RM   =	rm -rf

OBJS =	main.o foo.o bar.o

prog:	$(OBJS)
	$(CC) -o prog $(OBJS)

clean:
	$(RM) $(OBJS) prog

.c.o:
	$(CC) -c $<
\end{verbatim}


\section{Source files}
\label{sec:source-files}

C/C++ sourfe files are in free-form, rather than column-based or text-line-based
restrictions like FORTRAN 77.

\textcolor{red}{IMPORTANT: The A.c file should first include ``A.h''
file, and then any other headers required for}.

Read Sect.\ref{sec:extern_''C''} to know how to call C++ code from C.

\subsection{C language} 

In C language, typically, you have \verb!main.c!
\begin{lstlisting}
#include <stdio.h>
#include <stdlib.h>
#include "source.h"
int main(void){
    printf("%d", returnSeven());
    return 0;
}
\end{lstlisting}

Source files are typically in pair of source/header, e.g.
\verb!source.c!/\verb!source.h!.
\begin{verbatim}
// header.h
#ifndef __HEADER__H
#define __HEADER__H

int returnSeven(void);
/// ...

#endif
\end{verbatim}
and
\begin{lstlisting}
#include "header.h"

int returnSeven(void){
    return 7;
}
\end{lstlisting}

In order to call a function, it has to be 'declared' before you use it. There
are two ways to 'declare' a function: to write a prototype or to write the
function itself (the definition). If you use a function, just by providing the
prototype, not the definition, it's called {\bf forward declaration}
(Sect.\ref{sec:forward_declaration}).
In C, to tell that a variable is declared somewhere else so that the compiler doesn't
allocate the memory for it, we use \verb!extern! keyword.
\begin{verbatim}
int x; //allocate the memory for 'x' as int type
extern int y;  // don't, as 'y' is assumed to be defined somewhere else
\end{verbatim}
A good programming style is to put all
function definition in the \verb!.c! file, and the function prototype in the
header \verb!.h! file. So, when one source file want to use the function defined
in another file, it just \verb!#include <header-file.h>! the proper header file
name.

In the old C (K\&R style), the function doesn't have prototype
\begin{lstlisting}
main (argc, argv)

  int argc;
  char **argv;

{
  ...
}
\end{lstlisting}
Using C89 and later, it has prototype
\begin{lstlisting}
main (int argc, char **argv) { ... }
\end{lstlisting}
In C89/C90, without the function prototype, it's implicitly assumed as function
taking undefined list of arguments and returning \verb!int!.


\subsection{C++ language}

There is no fixed rule of what extension your code must use. Example: MSVC uses
.cpp; while on Linux, it uses .cc as the extension. Other choices: .cxx. 

For the header files, it can be .hpp (MSVC), .hh (Linux), .hxx, .H, or even
using .h (Microsoft Visual C++).


Example: Boost libraries use .hpp/.cpp. 

References: \url{http://www.parashift.com/c++-faq-lite/hdr-file-ext.html}

\section{Linking}
\label{sec:linking}

Once you compile the code, you may need to link to external libraries (.so, .a). 
The location of these libraries are found (1) first in fodler specified in
\verb!-L! compiler option
\begin{verbatim}
gcc -L/home/mylib/ -lmylib
\end{verbatim}
then (2) from folders specified in the environment variable
 \verb!LIBRARY_PATH!

\section{Online compilers}

When testing your code (with different compiler's versions), you can paste your
code to an online editor, select the compiler to use, and then see the result
quickly.

\begin{enumerate}
  \item \url{http://www.comeaucomputing.com/tryitout/}
\end{enumerate}

\section{Compilers}

The most popular C/C++ compiler in UNIX-liked operating systems is GCC,
Sect.\ref{sec:GCC}.


Some compilers provide enhanced features. To write code that utilize features
from a certain compiler, we can add the macro to detect the present of the
compiler being used as given in Sect.\ref{sec:macro-detect-compiler}.

To write code that utilize the ehnahced features from a certain version of
language, we can use macros to detect the version of the compiler from that
it can be used to infer if a certain language feature is supported -
Sect.\ref{sec:macro-detect-compiler-version}). 
Example, for C++11, to check which features are supported by a given GCC
version: take a look at
\url{http://wiki.apache.org/stdcxx/C++0xCompilerSupport}




\section{Compile MPI code}

When you compile the code with MPI, you can either
\begin{enumerate}
  \item  using the compiler you needs
(e.g. gcc, clang, intel cc, ..) with options that link to libmpi library as well
as path to MPI header file

  \item using a wrapper, e.g. mpiCC (for C++ code), mpicc (for C code), 
  
 The wrapper would call the compiler that was selected when the MPI library,
 e.g. OpenMPI, was installed. These compilers are found during running \verb!configure! script of building say OpenMPI, or
 using those as indicated in these following environment variables, if defined by user
 
\begin{verbatim}
CC
CXX
F77
FC
\end{verbatim}
  
 \item using the wrapper, and also override the default setting of chosen compilers
 
 By using a text files containing this configuration information,
 \begin{verbatim}
 /path/to/openmpi_library/share/openmpi/mpicc-wrapper-data.txt
 \end{verbatim}
 file  [\textcolor{red}{we rarely modify this file, but it's good to know the default setting as stored in these files}],  
 or by setting selected environment variables of the form
 \verb!OMPI_value!, with valid values \begin{verbatim}
 CPPFLAGS
     flags to used when invoking the preprocessor
     
 LDFLAGS
     ... linker 
 
 LIBS
     libraries added when invoking the linker
 
 CC
     chose the C compiler
 CFLAGS
     choose the flag for that C compiler
     
 CXX
 
 CXXFLAGS
 
 FC
 
 FCFLAGS
 \end{verbatim}
 
pass the option to specify the
gcc/g++ version using
\begin{verbatim}
-cc : use with mpicc to change the compiler used
-CC : .....    mpicxx (or mpic++, mpiCC)
-fc            mpif77
-f90           mpif90
\end{verbatim}
\url{https://www.dartmouth.edu/~rc/classes/intro_mpi/compiling_mpi_ex.html}

\end{enumerate}

\section{GCC (gcc, g++)}
\label{sec:GCC}
\label{sec:g++}

GNU compiler system includes
\begin{itemize}
  \item  \verb!gcc! : compile C code
  \item  \verb!g++! : compile C++ code
\end{itemize}

Remember that there are different versions of C or C++ language specification. 
You can force the GNU compiler to use the non-default version by passing to the 
\verb!-std=! option (Sect.\ref{sec:-std-option-GNU}).
\begin{itemize}
  \item since GCC 4.7:  you can find the default version of C++ standard supported
  
\begin{verbatim}
g++ -dM -E -x c++  /dev/null | grep -F __cplusplus

// g++ 7.4
#define __cplusplus 201402L

// g++ 4.8
#define __cplusplus 199711L
\end{verbatim}

NOTE: \verb!__cplusplus! should be so defined
according to ISO C++ section 16.8 
\url{https://gcc.gnu.org/bugzilla/show_bug.cgi?id=1773}

   \item C++14 is the default for GCC 6.1 and later:
   
   \url{https://gcc.gnu.org/projects/cxx-status.html#cxx14}
   
   \item 
\end{itemize}

These compilers utilize a library file that implements the C standard library
(Sect.\ref{sec:C-standard-library}) or C++ standard library
(Sect.\ref{sec:C++-standard-library}).

GCC is very widely used in Linux/Unix world. It compiles the source-code into
binary files targetted to a particular hardware architecture, i.e. the binary
code is in assembly from a given instruction set (e.g. MMX, SSE, SSE2) of CPU.
You can select the target CPU using \verb!-march=cpu-type! option
(Sect.\ref{sec:GCC_CPU_target})

To install multiple versions of GCC in your system, we
can 
\begin{itemize}
  \item use Modules for version selection - Sect.\ref{sec:modules_gcc}
  \item use update-alternatives selection - Sect.\ref{sec:update-alternative}
\end{itemize}

\subsection{Versions}

GCC 1.0 released in 1987, and was extended to support C++ in the same year.
Later on, other languages were also supported (Objective-C, Objective-C++, Fortran, Ada,
Java, etc.).

By 1991, GCC 1.x reached stability but not extendable to incorporate some new
features. 

In 1997, a new project formed to incorporate g77 (FORTRAN), PGCC (P5
Pentium-optimized GCC), many new C++ improvements, and many new architectures
supports. The project is called EGCS (Experimental/Enhanced GNU Compiler
System). To work with it, programmers call \verb!gcc!, a driver to interpret
command arguments, and decide which programming languages to use for each input
files. The files are pre-processed, and then run by the assembler, and finally
the {\bf linker} to produce the complete binary file. There's a low-level
runtime library, \verb!libgcc!, which is required to handle arithmetic
operations that the target processors cannot handle directly.

\begin{mdframed}
Previously, GCC was compiled by C compiler. In May 2010, GCC was compiled by C++
compiler, as it now has a subset of C++ features (destructors and generics). In
Aug. 2012, the transition is completed and GCC is now being implemented by C++
totally (minimum C++ 2003 features).
\end{mdframed}

Prior to GCC 4.0, g77 which only support FORTRAN 77 was used. In newer versions,
g77 was dropped; and \verb!gfortran! was used as the front-end as it supports
Fortran 95, and part of Fortran 2003.

\subsection{-std= option}
\label{sec:-std-option-GNU}


To specify a particular version of language standard to use, we use \verb!std=!
\begin{verbatim}
gcc -std=c89 

  //C99 mode
gcc -std=c99 
gcc -std=gnu99 

\end{verbatim}
In C mode, \verb!-ansi! is equivalent to \verb!std=c89! (as C89 was referred to
as ANSI C; even though nowadays C11 is now the official ANSI C standad).

\begin{verbatim}
g++ -std=c++98

g++ -std=gnu++98
		which is c++98 PLUS GNU extensions

   // C++0x draft or C++11 standard
g++ -std=c++0x

g++ -std=c++11
		//disable GNU extension

g++ -std=gnu++11
		//also enable GNU extension

g++ -std=gnu++1y

g++ -std=gnu++14
\end{verbatim}

To know which features in C++11 supported by GCC, check
\url{http://gcc.gnu.org/projects/cxx0x.html}. 


Since GCC 4.7, C++11 standard is officially suported. 

\subsection{write code to support different versions of C/C++ languages}

In the code, to support code working with different versions of the GNU
compiler, we can  detect if C++11 standard or C++00x draft is used via 
the macro (Sect.\ref{sec:macro-detect-compiler})
\begin{verbatim}
#if defined(__GXX_EXPERIMENTAL_CXX0X__) || __cplusplus >= 201103L
   ...
#else
   ... //old code
#endif      

NOTE: __cplusplus = 199711L (C++98/03)
                  = 201103L (C++11)
\end{verbatim}

You can check the difference using 
\begin{verbatim}
g++ -E -dM -std=c++0x -x c++ /dev/null >b && \
g++ -E  -dM -std=c++98 -x c++ /dev/null >a && \
diff -u a b | grep '[+|-]^&#define' && rm a b
\end{verbatim}

Other macros can be used
\begin{verbatim}
#if __GNUUC__ < 4 || (__GNUC__ == 4 && __GNUC_MINOR__ < 4) // <= gcc 4.4
  #define BOOST_NO_CXX11_HDR_THREAD
#elif (__GNUC__ == 4 && __GNUC_MINOR__ < 8) // gcc 4.8
  #if !(defined(_GLIBCXX_HAS_GTHREAD) &&
  	   defined(_GLIBCXX_USE_C99_STDINT_TR1) &&
  	   defined(_GLIBCXX_USE_SCHED_YIELD) &&
  	   defined(_GLIBCXX_USE_NANOSLEEP) &&
  	   defined(_GLIBCXX_USE_EXPERIMENTAL_CXX0X__)
  	   )
  	 #define BOOST_NO_CXX_HDR_THREAD
#elif __cplusplus < 201103L
     #define BOOST_NO_CXX11_HDR_THREAD
#endif       
\end{verbatim}

\subsection{Install multiple versions of GCC}
\subsection{-- using Modules }
\label{sec:modules_gcc}

Use Modules for version management (Check Sys-admin book). Then we need to
install each gcc version into /modules/gcc/ folders. For example:
\begin{verbatim}
/modules/gcc/4.2
/modules/gcc/4.7
\end{verbatim} 

To be able to compile gcc from source, first we need essential packages 
\begin{verbatim}
 sudo apt-get install g++ gawk m4 gcc-multilib
\end{verbatim}

Next, we need to install packages that GCC depends on, i.e. three other
libraries gmp, mpc and mpfr.
\begin{itemize}
  \item GCC 5.4: needs libgmp 4.2+ (Sect.\ref{sec:GMP-datatypes}), MPFR 2.4+,
  MPC 0.8+.

SUGGEST: GMP 5.0, MPFR 4.3 and MPC 1.0

NOTE:
\begin{verbatim}
MPFR 4.0+ depends on and needs GMP 5.0+

\end{verbatim}
  
\end{itemize}

Then select the source code of the right version of GCC to download and compile:
\url{https://gcc.gnu.org/releases.html}
and \url{https://ftp.gnu.org/gnu/gcc/}
Make sure use \verb!--prefix=! option of \verb!./configure! command as
\begin{verbatim}
./configure --prefix=/packages/gcc/


../configure                           \
    --prefix=/packages/gcc/5.4                           \
    --disable-multilib                 \
    --with-system-zlib                 \
    --enable-languages=c,c++,fortran,go,objc,obj-c++ \
    --with-isl=/packages/isl/lib &&
\end{verbatim}
for each version.

\subsection{-- using update-alternative}
\label{sec:update-alternative}

Suppose the default gcc/g++ version is 4.8, and you want to use gcc/g++ 4.4,
you need to search for the older version and install it.
You can use \verb!aptitude! and search for 
\verb!gcc-4.4! and \verb!g++-4.4!.
NOTE \verb!+! is a special character in regular expression, so make sure to use
\verb!\+! to indicate the plus.

Once the older version is installed, you need to specify the version to use.

\begin{verbatim}
sudo add-apt-repository ppa:ubuntu-toolchain-r/test
sudo apt-get update
sudo apt-get install gcc-4.7 g++-4.7
\end{verbatim}
and then update-alternatives
\begin{verbatim}
sudo update-alternatives --install /usr/bin/gcc gcc /usr/bin/gcc-4.6 60 --slave /usr/bin/g++ g++ /usr/bin/g++-4.6 
sudo update-alternatives --install /usr/bin/gcc gcc /usr/bin/gcc-4.7 40 --slave /usr/bin/g++ g++ /usr/bin/g++-4.7 
sudo update-alternatives --config gcc
\end{verbatim}



\subsection{libgcc\_s.so}
\label{sec:libgcc_s.so}

\verb!libgcc_s.so! is GCC's runtime library. It contains some low-level
functions that GCC emits calls to (like long long division on 32-bit CPUs).
\url{https://gcc.gnu.org/onlinedocs/gccint/Libgcc.html}

\url{http://refspecs.linuxbase.org/LSB_4.1.0/LSB-Core-generic/LSB-Core-generic/libgcc-s.html}


\subsection{Target CPU}
\label{sec:GCC_CPU_target}

\verb!-march=cpu_type! option generates instructions (binary code, assembly
code) for a given CPU type. The generated code thus may not run on other
processors. However, \verb!-mtune=cpu_type! only tunes the generated code for
that CPU type, but the code still be able to run on other processors. NOTE:
\verb!-march=cpu_type! implies \verb!-mtune=cpu_type!.

Values for \verb!cpu_type!
\begin{itemize}
  \item 'native': generate code for the CPU in the local machine (by default)
  \item 'i386':
  \item 'i486':
  \item 'i586':
  \item 'pentium':
  \item 'pentium-mmx': based on Pentium core with MMX instruction set
  \item 'pentiumpro':
  \item 'i686':
  \item 'pentium2': based on Pentium Pro core with MMX instruction set.
  \item 'pentium3': 
  \item 'pentium3m': based on Pentium Pro core with MMX and SSE 
  \item 'pentium-m':  based on Pentium III CPU with MMX, SSE and SSE2
  \item 'pentium4':
  \item 'pentium4m': based on Pentium 4 with MMX, SSE and SSE2
  \item 'prescott':
  \item 'nocona':
  \item 'core2':
  \item 'corei7':
  \item 'corei7-avx':
  \item 'core-avx-i':
  \item 'atom':
  \item 'k6': AMD K6 CPU with MMX
  \item 'k6-2':
  \item 'k6-3': AMD K6 CPU with MMX and 3DNow!
  \item 'athlon':
\end{itemize}
\url{http://gcc.gnu.org/onlinedocs/gcc-4.8.3/gcc/i386-and-x86-64-Options.html}

\subsection{GCC extension: \_\_attribute\_\_}
\label{sec:__attribute__}

One of the best (but little used) feature of GCC extension (since GCC 2.95.3)
\footnote{\url{www.unixwiz.net/techtips/gnu-c-attributes.html}}) is
\verb!__attribute__! mechanism, which tell the compilers to do better
optimization. There is always two sets of parantheses surrounding the content.
\begin{verbatim}
__attribute__((myattribute))
\end{verbatim}

You also need to compile your code with \verb!-Wall! to enable this. This can
apply to either (1) function, (2) variable, and (3) data type.
\begin{enumerate}
  \item Function: 8 keywords ``noreturn'', ``const'', ``format'', ``section'',
  ``constructor'', ``destructor'', ``unused'' and ``weak''
  \item Variable: ``section''
  \item Type: \ldots
\end{enumerate}
NOTE: We may also use attribute keyword with \verb!__! preceding, and follwing
each keyword, e.g. \verb!__noreturn__! and \verb!noreturn! are the same.

{\bf noreturn}: During optimization, the compiler need to consider the case when
'fatal' can occur. However, there are functions we know that this never happens.
So, we can tell the compiler to ignore this, which can help producing better
code, as well as avoidig spurious warnings of uninitizlied variables. An example
is that 
\begin{verbatim}
extern void exit(int) __attribute__((noreturn))

extern void abort(void) __attribute__((noreturn))
\end{verbatim}
IMPORTANT: Functions that use ``noreturn'' must be 'void' function, i.e.
function to return 'void'
\footnote{\url{docs.freebsd.org/info/gcc/gcc.info.Function_Attributes/html}}.



Before C11, using \verb!__attribute__! causes the code to be non-portable. One
solution is to use
\begin{verbatim}
#ifdef __GNUC__
#define UNUSED __atribute__((unused))
#else
#define UNUSED
#endif

void function(void) UNUSED;
\end{verbatim}
However, since C11, the keyword \verb!_Noreturn! has been introduced in the
standard to indicate that a function never turns. This is being used in
functions like \verb!longjmp()!, \verb!exit()!, \verb!abort()!,
\verb!quick_exit()!, and \verb!thrd_exit()!.

To make the code portable, use ANSI C's \verb!#pragma!.  Compilers that are NOT
GCC compatibles use \verb!#pragma! to achieve the similar goals. However, it's
NOT recommended. Since C99, it has \verb!_Pragma! operator tha allows you to
spew pragmas in the middle of your code, and to compose the contents of a pragma
with macros.

\subsection{GCC extension: \_\_builtin\_ prefix}
\label{sec:__builtin_C99functions}

GCC provides built-in versions of the ISO C99/posix. These functions has a
prefix \verb!__builtin_!, e.g. The \verb!<string.h>! functions
like \verb!memcpy! becomes \verb!__builtin_memcpy!.

TIPS: The \verb!<string.h>! functions are only replaced when the size of the
source argument can be known at compile time or even at the first level of
optimization (-O1 or -O). In which case the call to libc is replaced directly by
unrolled code.
\url{http://stackoverflow.com/questions/11747891/when-builtin-memcpy-is-replaced-with-libcs-memcpy}

To avoid using \verb!__builtin_! extension, you compile the code with
\verb!-fno-builtins!, -ansi, -std=c89, or something similar.


\url{https://gcc.gnu.org/onlinedocs/gcc-3.4.6/gcc/Other-Builtins.html}


\subsection{GCC 3.4}

GCC 3.4 uses LSB 3.0 (Sect.\ref{sec:LSB_3.x}).

\subsection{GCC 4.5 (Ubuntu Natty)}

\url{https://wiki.ubuntu.com/NattyNarwhal/ToolchainTransition}

\subsection{GCC 4.6 (Ubuntu 12.04)}

It has 3 minor versions: 4.6.1, 4.6.2, and 4.6.3 (March-2012). 

\subsection{GCC 4.7}

NOTE: gcc-4.7 supports a number of C++11 features.
 \url{https://gcc.gnu.org/gcc-4.7/cxx0x_status.html}

To enable C++11 features, build with
\verb!-std=c++11! or \verb!-std=gnu++11! compiler
options; the former disables GNU extensions.

\subsection{GCC 4.8.1}

This is a major release with all major features C++11 are supported
(Sect.\ref{sec:C++11}). Even though GCC 4.8.1 is C++ complete, libstdc++ (the
standard library that goes with GCC - Sec.\ref{sec:libstdc++}) is not; so one way to
utilize full C++11 is by compiling and linking with libc++ (Sect.\ref{sec:libc++})
\url{https://gcc.gnu.org/projects/cxx0x.html}

In many Linux distro, we need to download and compile it manually to have C++11
features (check sys-admin book). 
\ref{sec:C++11}

\subsection{GCC 5.0 (Ubuntu 14.10)}
\label{sec:GCC-5.0}

Since GCC 5.x, the default is C++11 (Sect.\ref{sec:C++11}), i.e.
\verb!-std=gnu11! rather than C++98 \verb!-std=gnu98!. Also, C++ runtime library
(libstdc++) provides a dual ABI (ABI C++98 and ABI C++11), and uses a new ABI by
default (Sect.\ref{sec:ABI-compatibility-C++98-C++11}).

New extensions:
\begin{enumerate}
  \item OpenACC 2.0a specification (Sect.\ref{sec:OpenACC})
  
  \item OpenMP 4.0 specification
  
  \url{http://openmp.org/mp-documents/OpenMP4.0.0.Examples.pdf}
  
  
  \item Cilk Plus
  
  \item Go language 1.4.2 
\end{enumerate}

New CPUS
\begin{enumerate}
  \item Atmel AVR: ATtiny4/5/9/10/20/40. 
  
  \item 
\end{enumerate}
\url{https://gcc.gnu.org/gcc-5/changes.html}


%Sect.\ref{sec:glibc}

\subsection{GCC 6.0 (Ubuntu 16.x)?}
\label{sec:GCC-6.0}

Since GCC 6.x, the default is C++14 (Sect.\ref{sec:C++14}), i.e.
\verb!-std=gnu14!.

New extensions:
\begin{enumerate}
  \item OpenACC 2.0a specification (Sect.\ref{sec:OpenACC}): improved
  implementation
  
  \item OpenMP 4.5 specification
  
  \url{http://openmp.org/mp-documents/OpenMP4.5.0.Examples.pdf}  
\end{enumerate}

New CPUS
\begin{enumerate}
%  \item Atmel AVR: ATtiny4/5/9/10/20/40. 
  
  \item 
\end{enumerate}

\url{https://gcc.gnu.org/gcc-6/changes.html}


How to install:
\url{https://gist.github.com/application2000/73fd6f4bf1be6600a2cf9f56315a2d91}


\subsection{GCC 7.x}

GCC 7.0 was release in May 2017

\begin{verbatim}
sudo add-apt-repository ppa:ubuntu-toolchain-r/test
sudo apt-get update
sudo apt-get install gcc-7 g++-7
gcc-7 --version
\end{verbatim}


\subsection{GCC 7.2 (Ubuntu 17.10)}

Ubuntu 17.10 has GCC 7.2 and clang 4.0 as default.

\subsection{GCC 7.3}

\url{https://gcc.gnu.org/ml/gcc/2018-01/msg00197.html}

GCC 7.3 has more than 99 bugs fixed since GCC 7.2; besides the adding of new
options to generate safe code - preventing Spectre vulnerabilities.

Retpoline is a spectre v2 mitigation technique, and was added to GCC 7.3 (after
GCC 8.1).

\verb!-mindirect-branch-loop=! option to control loop filler in call
and return thunks generated by \verb!-mindirect-branch=!

\begin{verbatim}
'lfence' uses "lfence" as loop filler.  

'pause' uses "pause" as loop filler.  
'nop' uses "nop"  as loop filler. 

The default is 'lfence'.
\end{verbatim}
As per AMD architects, using "lfence" in "retpoline" is better than "pause" for our targets.
So please allow filler to use "lfence". 

To install GCC 7.3 in Ubuntu 16.04
\begin{verbatim}
sudo apt-get install -y software-properties-common python-software-properties

sudo add-apt-repository ppa:ubuntu-toolchain-r/test
sudo apt update
sudo apt install g++-7 -y

sudo update-alternatives --install /usr/bin/gcc gcc /usr/bin/gcc-7 60 \
                         --slave /usr/bin/g++ g++ /usr/bin/g++-7 
sudo update-alternatives --config gcc
gcc --version
g++ --version
\end{verbatim}
\url{https://gist.github.com/jlblancoc/99521194aba975286c80f93e47966dc5}


\section{Clang (clang, clang++)}
\label{sec:clang}

Clang is an LLVM-based compiler that targets to C/C++/Objective-C/Objective-C++.
The compiler uses libc++ (Sect.\ref{sec:libc++}) which is a 100\% complete C++11
implementation on Apple's OS X.

It uses LLVM as the back-end, and Clang is designed to be able to replace the
full GNU Compiler Collection (GCC). It can  produce code that run faster than
GCC, low memory use and GCC-compatible. Clang is safe in Linux to use since is
configured as a GCC replacement. Its standard library is \verb!libc++!
(Sect.\ref{sec:libc}).


It has two parts
\begin{enumerate}
  \item the driver
  
  \item the front-end:  Sect.\ref{sec:clang-front-end}
  
  The -cc1 argument indicates that the compiler front-end is to be used, and not
  the driver.
\end{enumerate}


Clang is released as part of regular LLVM releases, and thus Clang is designed
to be built as part of an LLVM build. Assuming that the LLVM source code is
located at \verb!$LLVM_SRC_ROOT!, then the clang source code should be installed
as:   \verb!$LLVM_SRC_ROOT/tools/clang!.
\url{https://github.com/llvm-mirror}


Note that functionality provided by <atomic> is only functional with clang.

\subsection{clang front-end}
\label{sec:clang-front-end}

The -cc1 argument indicates that the compiler front-end is to be used, and not
the driver (Sect.\ref{sec:clang-driver}).  It uses LLVM as its back end and has
been part of the LLVM release cycle since LLVM 2.6.

The options for the front-end is meant to be used by developer.


\subsection{clang driver}
\label{sec:clang-driver}

By default, if you run \verb!clang!, the driver is used. If you add \verb!-cc1!,
the front-end is used. A clang driver is a production quality compiler driver
with a compilar chain and tools, using the compatible options with gcc driver.
(Sect.\ref{sec:GCC}). \url{https://clang.llvm.org/docs/DriverInternals.html}

ALL OPTIONS: \url{https://clang.llvm.org/docs/genindex.html}

STAGE 1: parsing source file (i.e. string of texts)
\begin{verbatim}
The clang driver can dump the results of this stage using the 
-### flag (which must precede any actual command line arguments). For example:

clang -### -Xarch_i386 -fomit-frame-pointer -Wa,-fast -Ifoo -I foo t.c
\end{verbatim}


FOR CROSS-COMPILATION:
\begin{verbatim}
  //  <arch><sub>-<vendor>-<sys>-<abi>
"-triple", "x86_64-apple-macosx10.11.0",
"-target", "x86_64-apple-macosx10.11.0",

NOTE:
arch = x86_64, i386, arm, thumb, mips, etc.
sub = for ex. on ARM: v5, v6m, v7a, v7m, etc.
vendor = pc, apple, nvidia, ibm, etc.
sys = none, linux, win32, darwin, cuda, etc.
abi = eabi, gnu, android, macho, elf, etc.
\end{verbatim}
\url{https://clang.llvm.org/docs/CrossCompilation.html}

CHANGE DEFAULT PREFIX to ROOT INCLUDE FOLDER (default: /):
\begin{verbatim}
-isysroot /path/to/build
      , which makes all includes for your library relative to the build
      directory.
\end{verbatim}

Example: if you don't want \verb!<mylib.h>! to be found in
\verb!/usr/include/mylib.h!, then you pass the custom folder to 
\verb!-isysroot! 
\begin{verbatim}
-isysroot /Developer/SDKs/MacOSX10.4u.sdk 
           will look for mylib.h in
/Developer/SDKs/MacOSX10.4u.sdk/usr/include/mylib.h.

SO: all default include paths should be prefixed with
         /Developer/SDKs/MacOSX10.4u.sdk
\end{verbatim}
\url{https://clang.llvm.org/docs/UsersManual.html}


\subsection{clang-tidy}
\label{sec:clang-tidy}

\url{http://clang.llvm.org/extra/clang-tidy/}


clang-tidy is a clang-based C++ “linter” tool. Its purpose is to provide an
extensible framework for diagnosing and fixing typical programming errors, like
style violations, interface misuse, or bugs that can be deduced via static
analysis
\begin{verbatim}

sudo apt-get install clang-tidy-6.0
\end{verbatim}


\subsection{compilation database}
\label{sec:clang-compilation-database}

\url{https://clang.llvm.org/docs/JSONCompilationDatabase.html}

\subsection{.clang file, .clang.ow file}
\label{sec:clang-dot-clang}

Each project can have a dot file at his root, containing
the compiler options. This is useful if you're using some
non-standard include paths.
        
        
Vim-clang uses a file named ".clang" (same as \verb!.clang_complete! for
\verb!clang_complete!) in the project root to save Clang options.

The format of the .clang file is either
\begin{verbatim}
  Each flags option can have one or more compiler arguments. A .clang file can
  have multiple flags options. They will be concatenated in the order of their
  appearance.  
flags=<flags>

flags=-I/home/arakshic/.local/include -DNDEBUG
flags=-I/../src
\end{verbatim}
or
\begin{verbatim}
  // requires the path (relative to the .clang file) to
  // a clang JSON compilation database format specification
compilation_database = "<path to compilation_database>"
\end{verbatim}

The file ".clang" only contains: "-I." to include files in directory "./".
So that the source file should use \verb!#include "yacl/xxx.h"! to include
files.

A new similar file named ".clang.ow" is added to deal with the special
case that one want to overwrite all clang options, which means the one don't
need automatically generated options for clang by the plugin.

NOTE: all options in ".clang" and ".clang.ow" must be safe to be used by the
'shell', that means special chars should always be escaped correctly.

\subsection{.clang-format file}
\label{sec:clang-dot-clang-format}

.clang-format file contains format style

\begin{verbatim}
-style=file to load style configuration from
                .clang-format file located in one of the parent
                directories of the source file (or current
                directory for stdin).
                Use -style="{key: value, ...}" to set specific
                parameters, e.g.:
                  -style="{BasedOnStyle: llvm, IndentWidth: 8}"
\end{verbatim}
\url{https://clang.llvm.org/docs/ClangFormat.html}

An easy way to create this file
\begin{verbatim}
clang-format -style=llvm -dump-config > .clang-format
\end{verbatim}


\subsection{Clang versions}

\subsection{-- Clang 7.0}
\label{sec:clang-7.0}

\begin{verbatim}
sudo apt-get install libedit-dev

// if you get error on ppc64le architecture of ldap-common not found
sudo aptitude safe-upgrade '-o APT::Get::Fix-Missing=true'
\end{verbatim}



\begin{verbatim}

\end{verbatim}

\subsection{-- Clang 6.0}
\label{sec:clang-6.0}

On Ubuntu 16.04, it only has clang 3.8 by default: 

To install Clang 6.0
\begin{verbatim}
wget -O - https://apt.llvm.org/llvm-snapshot.gpg.key | sudo apt-key add -
sudo apt-add-repository "deb http://apt.llvm.org/xenial/ llvm-toolchain-xenial-6.0 main"
sudo apt-get update
sudo apt-get install -y clang-6.0


update-alternatives --install /usr/bin/clang++ clang++ /usr/bin/clang++-3.8 100
update-alternatives --install /usr/bin/clang++ clang++ /usr/bin/clang++-6.0 1000
update-alternatives --install /usr/bin/clang++ clang /usr/bin/clang-3.8 100
update-alternatives --install /usr/bin/clang clang /usr/bin/clang-3.8 100
update-alternatives --install /usr/bin/clang clang /usr/bin/clang-6.0 1000
update-alternatives --config clang
update-alternatives --config clang++
\end{verbatim}
\url{https://blog.kowalczyk.info/article/k/how-to-install-latest-clang-6.0-on-ubuntu-16.04-xenial-wsl.html}

\subsection{-- Clang 4.0}
\label{sec:clang-4.0}

\subsection{-- Clang 3.9}
\label{sec:clang-3.9}

Added
\begin{verbatim}
CXTranslationUnit_KeepGoing
\end{verbatim}


\subsection{-- Clang 3.8}
\label{sec:clang-3.8}

Added
\begin{verbatim}
CXTranslationUnit_CreatePreambleOnFirstParse
\end{verbatim}

\subsection{-- Clang 3.6}
\label{sec:clang-3.6}

\subsection{Download and build }
\label{sec:clang-install}

Use the script
\begin{verbatim}
wget https://raw.githubusercontent.com/Microsoft/ChakraCore/master/tools/compile_clang.sh

// Update LLVM_VERSION="3.9.1" at line 7 to LLVM_VERSION="4.0.0"

sudo ./compile_clang.sh
\end{verbatim}

First download LLVM: LLVM in git: \url{https://github.com/llvm-mirror/llvm}

\begin{verbatim}
git clone ...
\end{verbatim}

Checkout Clang into a folder inside llvm
\begin{verbatim}
cd llvm/tools

// checkout to llvm/tools/clang
git clone ... 
\end{verbatim}
CLANG in git: \url{https://github.com/llvm-mirror/clang}
\footnote{\url{https://github.com/llvm-mirror/clang/blob/979aa4018b0fb1b7acee5c4cdb0b4354816b5a9a/INSTALL.txt}}

NOTE: If using SVN to download the source we can follow:
\url{https://solarianprogrammer.com/2013/01/17/building-clang-libcpp-ubuntu-linux/}

{\bf Folder structure:}
\begin{verbatim}
$LLVM
  llvm
    tools/clang
  build
\end{verbatim}

NOTE: Newer llvm use cmake
\begin{verbatim}
cd $LLVM
mkdir build && cd build
cmake -G <generator> [options] <path to llvm sources>

// 
\end{verbatim}



NOTE: Older llvm use configure and make. To build in location
\verb!/usr/clang_3_4!
\begin{verbatim}
cd $LLVM
mkdir build && cd build
../configure --prefix=/usr/clang_3_4 --enable-optimized --enable-targets=host --disable-compiler-version-checks
make -j 8
\end{verbatim}

References: CPU0
\url{http://jonathan2251.github.io/lbd/about.html}

\subsection{* Build LLVM/CLANG in RedHat}
\label{sec:LLVM-build}
\label{sec:CLANG-build}

As RedHat 6.x only has gcc 4.4.7 which is not supported to build LLVM/CLANG.

You need to download gcc 4.8.2 source and then compile it using gcc 4.4.7.

As we install gcc in a non-standard location, we need to tell the system to
looks for the libraries in non-standard locations as well.

Example: for 
\begin{verbatim}
> /build/share/gcc/4.8.2/bin/g++ -E -x c++ - -v < /dev/null
\end{verbatim}

\begin{enumerate}
  \item Install gcc 4.8.2
  \item Install glibc with right version 2.15+ 
  \item Download the \verb!build-gcc.sh! script

\url{http://btorpey.github.io/blog/2015/01/02/building-clang/}
  
\begin{verbatim}
// follow the steps in the link above first to download

// then
export
EXTRA_CMAKE_ARGS="-DCMAKE_C_COMPILER=/gsa/yktgsa/home/t/m/tmhoangt/bin/gcc/4.8.2/bin/gcc -DCMAKE_CXX_COMPILER=/gsa/yktgsa/home/t/m/tmhoangt/bin/gcc/4.8.2/bin/g++ -DGCC_INSTALL_PREFIX=/gsa/yktgsa/home/t/m/tmhoangt/bin/gcc/4.8.2 -DCMAKE_CXX_LINK_FLAGS='-L/gsa/yktgsa/home/t/m/tmhoangt/bin/gcc/4.8.2/lib64 -Wl,-rpath,/gsa/yktgsa/home/t/m/tmhoangt/bin/gcc/4.8.2/lib64' -DCMAKE_INSTALL_PREFIX= -DLLVM_ENABLE_ASSERTIONS=ON -DCMAKE_BUILD_TYPE=Release -DLLVM_TARGETS_TO_BUILD=X86 -DPYTHON_LIBRARY=/home/tmhoangt/bin/lib/libpython2.7.so -DPYTHON_EXECUTABLE=/home/tmhoangt/bin/bin/python"

mkdir build
cd build
../llvm/configure --prefix=/home/tmhoangt/bin/ --enable-optimized
--enable-targets=host --disable-compiler-version-checks

./build-gcc.sh_short
\end{verbatim}
\end{enumerate}



\section{Intel CC (icc)}
\label{sec:icc}
\label{sec:Intel-C-Compiler}

\section{XL CC (xlcc)}
\label{sec:xlcc}


\section{Visual C++}
\label{sec:VisualC++-Microsoft}
Visual C++ 2010 implemented many C++0x specification. Visual C++ 2012 expands
to include many C++11 features
\footnote{\url{msdn.microsoft.com/en-us/library/vstudio/hh567368.aspx}}.

Extensions:
\begin{enumerate}
  \item SCARY iterators
  \item \verb!<filesystem>! from TR2 proposal. If offers working with
  file/folder intuitively using \verb!path! object, a directory
  iterator \verb!recursive_directory_iterator!.
  This is based on Boost.Filesystem V2. (NOTE: Boost.Filesystem V3 has not been implemented yet
  in Visual C++ 2012)
\end{enumerate}


\section{Intermediate files}

During the compilation, there can be different files are generated, and are
automatically deleted at the end of the compilation.
You can tell the compiler to keep these files using a compiling option.

The content of these files can be useful for debugging purpose
(Sect.\ref{chap:debug}).

\url{https://msdn.microsoft.com/en-us/library/fwkeyyhe.aspx}

\subsection{Listing file (/FA)}
\label{sec:Listing_file_/FA}


A listing file is the file that contains source code, assembler code, or machine code.

\begin{verbatim}

          default file extension      content
/FAc      .cod                       machine code
 
/FAs      .asm                      assembly code

/FAcs     .cod                      source code, machine code, assembly code

/FAsc
\end{verbatim}

\subsection{Map file (/Fm, /MAP)}


A mapfile is the file containing segments of code in the order in which they appear in the final EXE or DLL file.
The content of a map file
\begin{itemize}
  \item module name
  \item timestamp from program header file
  \item list of segments: 
  \item list of public symbols
  \item entry point : \verb!section:offset! address
\end{itemize}

The default name is the base name of the program with the extension \verb!.map!
\begin{verbatim}
/Fm:path

/MAP:filename
\end{verbatim}
\url{https://msdn.microsoft.com/en-us/library/8byw5bh9.aspx}

\url{https://msdn.microsoft.com/en-us/library/8byw5bh9.aspx}

\subsection{Object file (/Fo)}
\label{sec:object_file_/Fo}


\subsection{Hot patchable file (/hotpatch)}
\url{https://msdn.microsoft.com/en-us/library/ms173507.aspx}

\section{Linker}
\label{sec:linker}

Sect.\ref{sec:how-linux-run-a-program} explains how Linux launch a program, and
load the code into memory.

Compiling your code goes through two stages:
\begin{itemize}
  \item compile the code, as organized into a single source file (.c, .cpp, .cxx) into a single object file 
  
  If the code utilize functions in another library, or object file, it only need header files 
  
  So, \verb!-I /path/to/include/! is a commonly used option.
  
   Other options to use at compiling stages: see 
  
  \item (optional) generate an archive file (*.a in Linux, *.lib in Windows)
  that contains many or all object files (*.o in Linux, *.obj in Windows)
  
  
  \item link the main source file, with either (1) all the object files together, (2) with the archive file
  
 So, we have the options: (1) compile everything into a .lib file and then link that .lib file to the executable, 
 (2) link the executable directly against all the .obj file
  
 At this time, it needs to also know the external library, i.e. \verb!-lcudart! (for libcudart.so library), and
 the path to the library, i.e. \verb!-L/path/to/lib! option.  

 Other options to use at linking stages: see below
\end{itemize}

Suppose your program 
\begin{itemize}
  
  \item  if the code call a function - the dynamic object, by loading an external library, using 
  \verb!dlopen! (Sect.\ref{sec:dlopen}), but such dynamic object refers back to the symbols, e.g. the function, 
  defined by the current program, then we need to link such library to the program using option \verb!-rdynamic!.  
  
  Another choice is using \verb!-Wl,--export-dynamic!, this will cause passing
  \verb!--export-dynamic! to the linker.
  
\begin{mdframed}

The \verb!-Wl,xxx! option for gcc passes a comma-separated list of tokens as a space-separated list of arguments to the linker

\begin{verbatim}
gcc -Wl,aaa,bbb,ccc

// turns into 
ld aaa bbb ccc
\end{verbatim}

So, to have 
\begin{verbatim}
ld -rpath

// if we use gcc, then we need
gcc -Wl,-rpath
\end{verbatim}

\end{mdframed}
  
  \item link a static library against your executable, unreferenced symbols
  
  
\verb!-Wl,--whole-archive! if passed to the linker, then for each archive (*.a
file) mentioned on the command line, after the \verb!--whole-archive! option, 
linker is forced to include every object file in the archive in the link, rather
than searching the archive for the required object files.

WHEN TO USE THIS: This is normally used in Makefile's target for buidling a
shared library, i.e. to turn an archive file (*.a) into a shared library, i.e.
forcing every objects to be included in the resulting shared library.

NOTE: In Visual Studio C++ 2015 (Update 2), it use \verb!/WHOLEARCHIVE! flag,
which has equivalent functionality to the \verb!--whole-archive! option to
\verb!ld!.  

NOTE: In Visual Sdtudio, use \verb!/INCLUDE! flag to force inclusion of unused symbols.


   \item if you get the error 

\begin{verbatim}
ld bad rpath option
\end{verbatim}
It means that the linker cannot find the path to the shared library being used, i.e. we need to add \verb!-L/path/to/lib! 
where it can find the library

\end{itemize}



NOTE: The usual gcc switch for passing options to linker is -Wl, and it is
needed because gcc itself may not understand the bare -rpath linker option. 
When using \verb!-rpath! the path is saved in \verb!DT_RPATH!.



\section{Compile your program using a non-standard glibc}

By default, the compiler links to the glibc version being used by the Linux
kernel. If, for some reason, you want to compile the code to use a different
version of glibc
\begin{enumerate}
  \item first, you need to install that version of glibc (Sect.\ref{sec:glibc-install})
  
  \item 
 Then, when you compile your program
\begin{verbatim}
g++ main.o -o myapp ... \
   -Wl,--rpath=/path/to/newglibc \
   -Wl,--dynamic-linker=/path/to/newglibc/ld-linux.so.2
\end{verbatim}

\end{enumerate}

The usual gcc switch for passing options to linker is -Wl,, and it is needed
because gcc itself may not understand the bare -rpath linker option.
The -rpath linker option will make the runtime loader search for libraries in
/path/to/newglibc (so you wouldn't have to set \verb!LD_LIBRARY_PATH! before
running it), and the -dynamic-linker option will "bake" path to correct
ld-linux.so.2 into the application.


	
Now you can use a convenient utility patchelf (nixos.org/patchelf.html), which
allows you to modify rpath and interpreter of already compiled ELF
(Sect.\ref{sec:ELF}).
