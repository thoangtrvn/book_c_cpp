\chapter{Files/Folders}

\url{http://www.gnu.org/software/libc/manual/html_node/File-System-Interface.html}

\section{Check file/folder existence and create/delete}

\subsection{struct stat (C) and mkdir() - Linux}

\verb!struct stat! is a system struct that is defined to store information about
files. The struct has different fields
\begin{verbatim}
#include <sys/stat.h>

struct stat buf;
it error_code = stat("filename.txt", &buf); //buf contains information

if (rc >= 0) 
{ //file exist
  
}
\end{verbatim}

NOTE:  The "sysstat.h" header deals with broken versions of <sys/stat.h> and can
be replaced by <sys/stat.h> on modern Unix systems (but there were many issues
back in 1990).

Example 2:
\begin{verbatim}
#include <sys/types.h>
#include <sys/stat.h>
#include <unistd.h>

struct stat st = {0};

if (stat("/some/directory", &st) == -1) {
    mkdir("/some/directory", 0700);
}
\end{verbatim}



\subsection{SHCreateDirectoryEX  (Win32/C++, Windows XP, 2003)}

In pure Win32/C++ and for Windows XP and 2003 only
\begin{verbatim}
inline void EnsureDirExists(const std::wstring& fullDirPath)
{
    HWND hwnd = NULL;
    const SECURITY_ATTRIBUTES *psa = NULL;
    int retval = SHCreateDirectoryEx(hwnd, fullDirPath.c_str(), psa);
    if (retval == ERROR_SUCCESS || retval == ERROR_FILE_EXISTS || retval == ERROR_ALREADY_EXISTS)
        return; //success

    throw boost::str(boost::wformat(L"Error accessing directory path: %1%; win32 error code: %2%") 
        % fullDirPath
        % boost::lexical_cast<std::wstring>(retval));

    //TODO *djg* must do error handling here, this can fail for permissions and that sort of thing
}
\end{verbatim}
\url{http://stackoverflow.com/questions/1680836/create-directory-sub-directories}

\subsection{<filesystem> header: VS C++ 2012}

The <filesystem> header is not part of C++11; it is a proposal for C++ TR2 based
on the Boost.Filesystem library. Visual C++ 2012 includes an implementation of
the proposed library.


\subsection{Boost.Filesystem V2 (cross-platform)}
\label{sec:boost_filesystem_V2}

This is outdated, use V3 instead (Sect.\ref{sec:boost_filesystem_V3}).
\verb!Boost.Filesystem! provides a convenient and cross-platform interface for
file and directory traversing and manipulation. The intent is not to compete
with Python/Perl/ or shell script, but to provide C++ the feature to do when
it's needed. We need to install this as a separately compiled library
\begin{verbatim}
#include "boost/filesystem.hpp"
using boost::filesystem; //or

namespace fs=boost::filesystem;
\end{verbatim}
and compile with
\begin{verbatim}
-I$BOOST_INCLUDE_DIR -L$BOOST_LIB_DIR -lboost_system lboost_filesystem
\end{verbatim}

A class \verb!path! object
\begin{verbatim}
// use either regular or wide-character
// Input : const char* 
//         const std::string &
path my_path("some_dir/file.txt");  
wpath my_path("some_dir/file.txt"); 
   //both returns a path object 
  //that works for API of any operating system that accepts either
  // some_dir::file.txt
  // [some_dir]file.txt
  // some_dir/file.txt
  // etc.
\end{verbatim}
We can also use \verb!operator/! to append paths
\begin{verbatim}
ifstream file1( arg_path / "foo" / "bar");
ifstream file1( arg_path / "foo/bar");
\end{verbatim}

\begin{enumerate}
  \item if file/folder exists
\begin{verbatim}
if (fs::exists(fs::status(data_dir)))       // true - directory exists
if (fs::exists( data_dir ))     
\end{verbatim}  
which nees an argument of type boost::filesystem::path or something that is
implicitly convertible to it, e.g. std::string

Check subfolder
\begin{verbatim}

boost::filesystem::path config_folder(Config::CONFIG_FOLDER_NAME);

//use / operator
if(!(boost::filesystem::exists(
    config_folder / Config::fmap[Config::current_hash_function]));

//or 
if( !(boost::filesystem::exists(
   std::string(boost::format("%s/%s") 
   % config_folder 
   % Config::fmap[Config::current_hash_function])));
{
\end{verbatim}

  \item create directory
\begin{verbatim}
create_directory("my_directory");
\end{verbatim}  
  \item remove
\begin{verbatim}
remove_all ("something");
\end{verbatim}

  \item copy file/folder
\begin{verbatim}
//NOTE: We should provide destination filename, 
//      not just the folder
path user_path( "C:\\My Folder" );
path dest_file( "C:\\My Folder\\file.txt" );

boost::filesystem::create_directory( user_path );
path file(  "C:\\Another\\file.txt" );

//WRONG
boost::filesystem::copy_file( file, user_path );
//CORRECT
boost::filesystem::copy_file( file, dest_file );
\end{verbatim}  

However, an exception is thrown if the file existed in the destination folder.
To copy with overwrite
\begin{verbatim}
fs::copy_file(sourcefile, destfile, fs::copy_option::overwrite_if_exists);
\end{verbatim}
NOTE: In older version of Boost 1.46 (before rev.67067), the problem is that 
if the dest file exists and is smaller than the source, then the function only
overwrite the first \verb!size(filesource)! bytes in the target file.

Another choice
\begin{verbatim}
if (exists (destfile))
	remove(destfile);
copy_file(sourcefile, destfile);	
\end{verbatim}
However, this raises the problem if the two files are the same??? If we delete
the target file then we also delete the source file.

Or to use try..catch
\begin{verbatim}
try
{
	fs::copy_file(sourcefile, destfile);
}catch (const fs::filesystem_error& e)
{
	std::cerr << "Error: " << e.what() << std::endl;
}
\end{verbatim}

   \item iterate through files
   
   \item find files
\begin{verbatim}
find_files
\end{verbatim}
\end{enumerate}


Even though it's not part of C++11; yet it's a proposal for C++11/TR2
\footnote{\url{www.open-std.org/jtc1/sc22/wg21/docs/papers/2012/n3335.html}}.
Visual C++2012 contains the implementation for this proposed library.

References:
\begin{enumerate}
  \item \url{www.boost.org/doc/libs/1_41_0/libs/filesystem/doc/index.htm}
\end{enumerate}

\subsection{Boost.Filesystem V3 (cross-platform, C++11)}
\label{sec:boost_filesystem_V3}

This is the major revision of Boost.Filesystem V2
(Sect.\ref{sec:boost_filesystem_V2}) to work better with C++11 new data type
\verb!char16_t! and \verb!char32_t!
\footnote{\url{www.boost.org/doc/libs/1_49_0/libs/filesystem/v3/doc/v3_design.html}}.
Boost.Filesystem V3 is the default in Boost 1.60.0+.

In this version, there's no need to use \verb!path! and \verb!wpath!, we only
need a single class \verb!path! that supports different character type
(\verb!char!, \verb!wchar_t! \verb!char16_t!, \verb!char32_t!).
The class \verb!path! also have new members
\begin{verbatim}
has_stem()

has_extension()

is_absolute()

is_relative()

make_preferred()
\end{verbatim}
New or improved functions
\begin{verbatim}
absolute()

create_symlink()

read_symlink()

resize_file()

unique_path()
\end{verbatim}

To create folder:
\begin{verbatim}
#include <boost/filesystem.hpp>
//...
boost::filesystem::create_directories("/tmp/a/b/c");
\end{verbatim}
which returns \verb!true! is creating folder successfully; and \verb!false!
otherwise.


\begin{enumerate}
  \item
  \url{http://www.boost.org/doc/libs/1_49_0/libs/filesystem/v3/doc/v3.html}
  
  \item
  \url{http://www.boost.org/doc/libs/1_46_1/libs/filesystem/v3/doc/reference.html}
\end{enumerate}

\section{Temporary filename: tmpnam(), mkstemp()}
\label{sec:tmpnam}
\label{sec:mkstemp}
\label{sec:tmpfile}

\verb!std::tmpnam()! is defined in \verb!<cstdio>!:
\begin{verbatim}
char* tmpnam(char* filename);
\end{verbatim}
create a unique filename that is different from any currently existing file, and
the name is stored in the argument. \textcolor{red}{CONS: not thread-safe; it is possible (though with low chance) that 
during the time it starts, to it returns, another process also uses that name}. 

\textcolor{red}{BETTER CHOICE (i.e. thread-safe)}: std::tmpfile or (POSIX) mkstemp. 


On POSIX systems, we can use \verb!mkstemp()!


For portable, we can use
\begin{verbatim}
namespace fs=boost::filesystem;


//return folder
fs::path tmppath = temp_directory_path();

//return file
filesystem::unique_path()? 

\end{verbatim}



\section{Count number of files}

There are different options, some are not platform portable.
Don't use \verb!system()! as it only returns the exit status of the UNIX shell
(Sect.\ref{sec:system()}).

\begin{enumerate}
  \item If it's all the files, then use \verb!opendir! and \verb!readdir!
  \item Try \verb!execv!
  \item Try \verb!glob!, if we use file pattern to search
  \verb!http://linux.die.net/man/3/glob!
  \item Try \verb!popen/pclose! (Linux) and
  \verb!_popen/_pclose! (Windows) \url{http://linux.die.net/man/3/popen} which
  gives us a FILE* of the \verb!stdout!. 
  \begin{verbatim}
#include "stdio.h"

int main(){
    FILE* fp;
    char result [1000];
    fp = popen("ls -al .","r");
    fread(result,1,sizeof(result),fp);
    fclose (fp);
    printf("%s",result);
    pclose(fp);
    return 0;
}    
  \end{verbatim}
  
  \begin{verbatim}
#include <string>
#include <iostream>
#include <stdio.h>

std::string exec(char* cmd) {
    FILE* pipe = popen(cmd, "r");
    if (!pipe) return "ERROR";
    char buffer[128];
    std::string result = "";
    while(!feof(pipe)) {
    	if(fgets(buffer, 128, pipe) != NULL)
    		result += buffer;
    }
    pclose(pipe);
    return result;
}
  \end{verbatim}
  
  \item Try \verb!dirent.h! (Linux and Windows)
\begin{verbatim}
     #include <stdio.h>
     #include <sys/types.h>
     #include <dirent.h>
     
     int
     main (void)
     {
       DIR *dp;
       struct dirent *ep;
     
       dp = opendir ("./");
       if (dp != NULL)
         {
           while (ep = readdir (dp))
             puts (ep->d_name);
           (void) closedir (dp);
         }
       else
         perror ("Couldn't open the directory");
     
       return 0;
     }
\end{verbatim}
with the order of files appear in the directory are fairly random.

or
  \begin{verbatim}
DIR *dir;
struct dirent *ent;
if ((dir = opendir ("c:\\src\\")) != NULL) {
  /* print all the files and directories within directory */
  while ((ent = readdir (dir)) != NULL) {
    printf ("%s\n", ent->d_name);
  }
  closedir (dir);
} else {
  /* could not open directory */
  perror ("");
  return EXIT_FAILURE;
}
\url{http://www.softagalleria.net/download/dirent/}  
  \end{verbatim}
  
  \item Try \verb!Boost::Filesystems! with \verb!directory_iterator! and
  \verb!std::count_if!)
\end{enumerate}

\url{http://stackoverflow.com/questions/6318077/read-perl-script-from-c}

\url{http://stackoverflow.com/questions/478898/how-to-execute-a-command-and-get-output-of-command-within-c}

\subsection{system()}
\label{sec:system()}

\verb!system()! enables execution of a Linux shell command. However, it only
returns the error status code.
Also, \verb!system()! is a non-reentrant function (i.e. not safe for
multi-threaded application).


\subsection{execl()}
\label{sec:execl()}

Using \verb!execl()! is not recommended in a multi-threaded application; as it
is non-reentrant.

The \verb!execl()! functions are variadic: they take a variable number of
parameters so that you can pass a variable number of arguments to the command.
The functions need to use a null pointer as a marker to mark the end of the
argument list. 
The safe thing to do especially if you have a 64-bit architecture is to
explicitly use (void *)0 which is defined to be compatible with any pointer and
not rely on any dodgy \verb!#defines! that might happen to be in the standard
library. NOTE: We also can use \verb!(char*)0!.

NULL isn't a null pointer, it's a null pointer constant. And a null pointer
constant is required by the C++ standard to have an integral type.
When NULL (or 0) is used in a context which requires a pointer type, it is
implicitly converted to a null pointer.

The first argument is the path of the executable. The following arguments appear
in argv of the executed program. The list of these arguments is terminated by a
(char*)0; that's how the called function knows that the last argument has been
reached. Posix standard says: the last argument must be (char*)0, or you have
undefined behavior.
More generally, the C standard says that it is undefined behavior if a varargs
function tries to extract a type other than the type which was passed. execl is
documented to require a char*.


\begin{verbatim}
execl( "/bin/ls", "ls", "-l", (char*)0 );

execl( "/bin/ls", "ls", "-l", (void*)0 );

 // not recommended
execl( "/bin/ls", "ls", "-l", NULL);

\end{verbatim}
In the above example, for example, the executable at "/bin/ls" will
replace your code; in its main, it will have argc equal 2, with argv[0] equal
"ls", and argv[1] equal "-l".



Immediately after this function, you should have the error handling code.
\verb!excl()! always returns -1; so no need to test it.  And it only returns if
there was some sort of error.
\url{http://stackoverflow.com/questions/12677120/execl-arguments-in-ubuntu}