\chapter{C++ Pointers}

\section{::operator new, ::operator new[]}

\subsection{C++: notice}

In C++, we can allocate raw memory using the C-based functions (malloc(),
calloc(), \ldots - Sect.\ref{sec:malloc()}) for backward compatibility.
In C++ compiler, the header file to those functions is \verb!<cstdlib>!.


However, these C-based functions return a generic pointer \verb!void*!. The lack
of a specific pointer type returned is considered as type-unsafe by many
programmers. Also, the allocation is based on number of bytes, not the number of
objects or data elements, which is not straight forward convention in C++.  

\begin{mdframed}

For backward compatibility with C code, to dynamically allocate memory, in C++,
we can use C functions,  \verb!malloc!, \verb!calloc!, \verb!realloc! and
\verb!free! by including the <cstdlib> header file (Sect.\ref{sec:malloc()}).

However, it is important to know that C++ does not allow implicit type-cast;
instead, programmers need to call dynamic-cast function to cast from
\verb!void*! to a specific type.

\end{mdframed}


As C++ is a strong-type language, so type-casting is always recommended.
In C++, it's more convenient to think the number of dynamically objects, rather
than the total number of bytes all these objects occupy.

C++ languages thus provides a few new functions to allocate/free memory, e.g.
new or new[], delete or delete[]. Sect.\ref{sec:operator-new-C++}).

Also, C++ DOES NOT allow implicit casts from void * to other pointer types.
This prevents most C code from compiling under a C++ compiler, particularly if
the C code contains calls to malloc(). To resolve that issue, C++ introduced some
new operators for dynamic type-casts (Sect.\ref{sec:type-cast-C++}) such as 
\verb!reinterpret_cast! (Sect.\ref{sec:reinterpret_cast}).

 
\subsection{C++11: new operator  vs. ::operator new(...) }
\label{sec:new-operator}

C++ supports dynamic allocation of objects using the \verb!new! operator. The \verb!new!
operator allocate memory for objects from a pool called the free store. Behind the scene, the new
operator calls the special function \verb!::operator new! (Sect.\ref{sec:operator-new-C++}).
 
Example: \verb!new! operator allocate one object, which may takes a few number
of bytes.
The 'new' expression which knows how many bytes it needs, then pass that value
to 'operator new' to allocate raw memory, then after that, it evokes the object's constructor next.

\begin{verbatim}
Apple * p = new Apple(); //new expression using the constructor Apple()
 
Apple * p = new Apple("gala"); //new expression using the constructor Apple(std::string)

\end{verbatim}

  
%\section{C++: 'new' operator}
\subsection{::operator new (C++)}
\label{sec:operator-new-C++}


SYNTAX: We optionally can use \verb!::! at the front which explicitly referring to the global-scope operator 
\verb!new!.
\begin{verbatim}
#include <new>

[::] "new" ["(" expression-list ")"] {new-type-id | "(" type-id ")"} ["(" expression-list ")"]
[::] "delete" ["[]"] cast-expression

void *operator new (size_t);
void operator delete (void *);
void *operator new[] (size_t);      // for arays
void operator delete[] (void *);
\end{verbatim}

\verb!::operator new! below just allocate raw memory, by telling it the number
of bytes; and no object's constructor is called [which is evoked by the new
operator] (Sect.\ref{sec:new-operator}).

\begin{verbatim}
void* mem = operator new(sizeof(apple)); //just like calling malloc()
// or
// void* mem = ::operator new(sizeof(apple)); //just like calling malloc()

//now the constructor is next

apple* p2 = new(mem) apple(1); //call construct here using placement new.

 \end{verbatim}

\url{https://stackoverflow.com/questions/1885849/difference-between-new-operator-and-operator-new} 


\url{https://en.cppreference.com/w/cpp/header/new}


\begin{verbatim}
https://en.cppreference.com/w/cpp/memory/new/operator_new

void* operator new  ( std::size_t count );
void* operator new[]( std::size_t count );


(since C++17)
void* operator new  ( std::size_t count, std::align_val_t al);
void* operator new[]( std::size_t count, std::align_val_t al);
\end{verbatim}

This is different from the \verb!new! operator.

So, it is recommended to use the \verb!new! operator in C++ -


\subsection{-- usecase 01}

C++ provides a \verb!new! operator to create a pointer to an
object, or an array of object. If the
allocation fails, it throws an exception of type \verb!std::bad_alloc!, i.e. we
don't need to check for the result.

To avoid throwing exception, i.e. if it fails, return a NULL pointer and
continue the execution, we use a special object called \verb!nothrow! declared
in header file \verb!<new>!. 

\begin{verbatim}
#include <new>

bobby = new (nothrow) int [5]; 
if (bobby == 0) {
  // error assigning memory. Take measures.
  }; 
\end{verbatim}

NOTICE: \verb!nothrow! method requires more work than using exception method,
since the value returned have to be checked by the programmer after each and
every memory allocation. So, we're not recommended to use too often.


\subsection{-- usecase 02: default constructor and default value}

Syntax:
\begin{verbatim}
type_name *p_var;
p_var = new typename;
\end{verbatim}
with \verb!typename! can be an intrinsic type, \verb!enum!, \verb!class! and
\verb!struct!. 

If \verb!typename! is a \verb!class! type, then the default constructor of the
class is called, i.e. the constructor with no argument.


We can also assign intial value to the data
\begin{verbatim}
p_var = new typename(initializer);
\end{verbatim}
with \verb!initializer! plays the role of input to the constructor in the case
of \verb!class! object.

Example: the integer to which intp points gets initialized to the value 7. 
\begin{verbatim}
int *intp = new int (7);

  foo *foop = new foo;

delete intp;
  
  
  delete foop;

\end{verbatim}

\subsection{-- usecase 02: array}

Arrays are allocated slightly differently, and must be deleted with delete[]


We can create an array
\begin{verbatim}
p_var = new typename [size];
\end{verbatim}

Example:
{\small 
\begin{verbatim}
int *p_scalar = new int(5); //allocates an integer, set to 5. (same syntax as constructors)
delete p_var;
p_var = 0;

int *p_array = new int[5];  //allocates an array of 5 adjacent integers. (undefined values)
delete[] p_var;
p_var = 0;


int *p_array = new int[5](5);  //allocates an array of 5 adjacent integers, and calling the constructor, accepting '5' as argument
                           // which in this case, initialize the elements to value '5'
delete[] p_var;
p_var = 0;

\end{verbatim}}


\textcolor{red}{It's impossible to reallocate the memory with 'new[]' operator.
Instead, we need to 'new[]' an array of bigger size, copy over the data, and
remove the old object.}


\begin{mdframed}
Simple rules: every time you type 'new', type 'delete', and then put the working
code in between them.

\begin{verbatim}
Foobar *foobar = new Foobar();
.... //code here
delete foobar; // TODO: Move this to the right place.
\end{verbatim}
\end{mdframed}

\subsection{-- overload global operator new}

The operator new function can be overloaded, provided that it always returns
void * and has a first argument of type \verb!size_t!.

\begin{verbatim}
[::] "new" ["(" expression-list ")"] {new-type-id | "(" type-id ")"} ["(" expression-list ")"]
[::] "delete" ["[]"] cast-expression
\end{verbatim}
\begin{itemize}
  \item  The first optional 'expression-list':
  
  passes additional arguments to overloaded operator new functions.
   
  For instance, the standard C++ library's default new throws an exception when it runs out memory.
   
  
  \item The second optional 'expression-list': 
  
   This is a list of arguments passed to the allocated object's constructor. 

\end{itemize}
Note, however, that operator delete cannot be overloaded. 


NOTE: the standard C++ library's default new throws an exception when it runs
out memory.


Example: An overloaded version of new returns a NULL pointer instead. This overloaded new is defined as:
\begin{verbatim}
class foo {... };

struct nothrow_t {};
extern const nothrow_t nothrow;
void *operator new throw() (size_t, const nothrow_t&);

// use it
foo *fp = new (nothrow) foo(5);  //it means '5' is passed to the object's constructor - class foo's constructor
               // 'nothrow' is passed to the operator new

  if (!fp) {
    /* Deal with being out of memory */
  }

\end{verbatim}



\subsection{-- operator new as part of a class}

In addition to overloaded global new functions, each class can have its own set
of operator new and delete functions.

There is at most one operator delete and one operator delete[] per class.
While the global ::operator delete and delete[] take one argument of type void
*, per-class operators can optionally take a second argument of type \verb!size_t!.
This second argument will contain the size of the deleted chunk of memory.



It can be a member function or a global function
\begin{verbatim}
// member
void* T::operator new(size_t x);
void* T::operator new[](size_t x);
void T::operator delete(void* x);
void T::operator delete[](void* x);

// global
void* operator new(size_t x);
void* operator new[](size_t x);
void operator delete(void* x);
void operator delete[](void* x);
\end{verbatim}

\subsection{-- placement new (using allocator)}

Suppose you already allocate memory before, now you can use that region of
memory for several 'placement-new' of smaller size.
This ensures the memory region returned by the placement-new is contiguous from
each other, for example.

Another use case is consider an arena allocator that allocates objects from a
large pool of memory by simply incrementing a pointer. All objects are freed at
once by deleting the entire pool.
 

The so-called "placement-new," doesn't allocate memory at all, but can be called
to invoke an object's constructor on an arbitrary piece of memory. It is
typically defined as:

SYNTAX:
\begin{lstlisting}
inline void *
operator new(size_t, void *p)
{
  return p;
}

\end{lstlisting}

Example: 
\begin{verbatim}
int * p = new [10];

int *x = new(p) int;
int *y = new(p+1) int;
\end{verbatim}

Allocator - Sect.\ref{sec:Allocator}

\begin{verbatim}
class arena {
  /* ... */
  arena (const arena &);            // No copying
  arena &operator= (const arena &); // No copying
 public:
  arena ();
  ~arena ();
  void *allocate (size_t bytes, size_t alignment) { /* ... */ }
};

inline void *
operator new (arena &a, size_t bytes)
{
  return a.allocate (bytes, sizeof (double));
}

inline void *
operator new[] (arena &a, size_t bytes) // This function is bad news
{
  return a.allocate (bytes, sizeof (double));
}

class foo { /* ... */ };

void
f ()
{
  arena a;

  char *string = new (a) char[80]; // opreator new[] (arena &, size_t)
  int *intp = new (a) int;         // opreator new (arena &, size_t)
  foo *fp = new (a) foo;           // better hope foo::~foo() isn't useful

  /* No need to free anything before returning */
}

\end{verbatim}

Example:
\begin{verbatim}
class arena{

}

arena a;

\end{verbatim}

Consider what happens during 
\begin{verbatim}
foo *x = new (a) foo;
\end{verbatim}

Two functions are invoked: the arena allocator
\begin{verbatim}
operator new (arena &, size_t)
\end{verbatim}
and foo's default constructor, 
\begin{verbatim}
foo::foo()
\end{verbatim}

Operator new does not know what type is being allocated, while foo::foo() doesn't know what memory allocator is being used.
So, it is important to ensure that  there is NO dynamically allocation inside foo::foo().

If there is dynamic memory allocation inside foo::foo() constructor; then we need to explicitly call the destructor first; before 
freeing the allocator 'arena'

\begin{verbatim}
  arena a;

  // ...
  foo *fp = new (a) foo;           // must be destroyed
  // ...
  fp->~foo ();
  
\end{verbatim}

A BETTER CHOICE: If \verb!foo:~foo()! only needs to free memory (as opposed to other
resources such as file descriptors or locks), foo could provide an overloaded
constructor that takes an arena reference and ensures all further memory is
allocated from that arena.

\begin{verbatim}
  arena a;

  // ...
  foo *fp = new (a) foo (a);       // Tell foo about our arena
  // foo no longer needs its destructor invoked
\end{verbatim}

PITFALL: A user need to know the details of the implementation of the allocator
arena and the implementation of foo.


ANOTHER PITFALL:
A potential problem may arise if the constructor for foo throws an exception.
Because constructors have no return values in C++, exceptions are the simplest
way for them to indicate failure. When constructors throw exceptions, the
compiler will try to free the memory allocated by new. While this may ordinarily
be a good thing, it would have disastrous consequences in the case of the arena
allocator, where calling ::operator delete on a pointer in the middle of the
arena would have unpredictable results.

One can avoid this problem through so-called "placement-delete" functions that,
after the first argument, match the argument types of particular placement new
operators. For the arena type, we should therefore declare an empty placement
delete:

\begin{verbatim}
inline void operator delete (void *, arena &) {}
\end{verbatim}



\subsection{delete vs. delete[]}
\label{sec:delete-operator}


To delete a pointer, allocated by \verb!new! for a scalar
(Sect.\ref{sec:new()-operator}), we use \verb!delete! operator.
\begin{verbatim}
int * abc = new int();

delete bc;
\end{verbatim}


To delete a pointer, allocated by \verb!new[]! for an array, we use
\verb!delete[]! operator
\begin{verbatim}
int * abc = new int[4]();

delete [] abc;
\end{verbatim}


C++ offers six variations for operator delete
\begin{lstlisting}
void operator delete  (void* ptr); 	
void operator delete[](void* ptr);
void operator delete  (void* ptr, const std::nothrow_t& tag);
void operator delete[](void* ptr, const std::nothrow_t& tag);
void operator delete  (void* ptr, std::size_t sz);
void operator delete[](void* ptr, std::size_t sz);
\end{lstlisting}
The remaining 5 uses the first one, so we just need to override the first one.


\section{Allocate large array}



When you want to allocate 1Gb as a single contiguous array, then that can only
happen if there is a contiguous block of memory available in the 2Gb of process
memory. Chances for that happening aren't high, so it'll fail.

\url{http://forums.codeguru.com/showthread.php?546231-Best-way-to-allocate-large-memory}

If you are using a 64-bit operating system, then malloc should be able to
allocate the large size with no problem.


\url{https://baptiste-wicht.com/posts/2012/12/cpp-benchmark-vector-list-deque.html}



\section{boost::shared\_ptr (Boost library)}

\url{http://www.boost.org/doc/libs/1_46_1/libs/smart_ptr/shared_ptr.htm}


\section{vtkSmartPointer}

vtkSmartPointer is similar to \verb!shared_ptr! as it use internal reference
counting of vtkObject. Check VTK chapter in ``Data Visualization'' book.


\section{Smart pointer (C++11)}
\label{sec:smart_pointer}

A brief introduction of smart pointers was given in
Sect.\ref{sec:c++11_smart_pointer}. 

There are other types of smart pointers, and we discuss here when to use each
one
\begin{enumerate}
  \item local variable: \verb!std::auto_ptr! - Sect.\ref{sec:auto_ptr}
  \item class members: use copied pointer
  \item STL containers: use garbage collected pointers 
  \item explicit ownership transfer: owned pointer
  \item big objects: copy on write
\end{enumerate}

\subsection{auto\_ptr (deprecated)}
\label{sec:auto_ptr}

NOTE: Using this is deprecated in C++11, see explanation below. It is completely
removed from C++17. Use \verb!unique_ptr! instead (Sect.\ref{sec:unique_ptr}).

\verb!std::auto_ptr<T>! is a class template that can be used to provide as a
wrapper to a C++ raw pointer, and ensure automatic memory cleanup, i.e. RAII
(Sect.\ref{sec:RAII}).


The header file \verb!<memory>!
\begin{verbatim}
#include <memory>
\end{verbatim}
with the implementation
\begin{lstlisting}
template <class T> 
class auto_ptr
{
    T* ptr;
public:
    explicit auto_ptr(T* p = 0) : ptr(p) {}
    ~auto_ptr()                 {delete ptr;}
    T& operator*()              {return *ptr;}
    T* operator->()             {return ptr;}
    // ...
}
\end{lstlisting}

What \verb!auto_ptr! does is (1) own a dynamically allocated pointer,
and (2) perform automatic cleanup when the object is no longer needed.
\begin{multicols}{2}
\begin{lstlisting}
void f()
{
  T* pt( new T );

  /*...more code...*/

  delete pt;
}
\end{lstlisting}
\columnbreak
\begin{lstlisting}
void f()
{
  std::auto_ptr<T> pt( new T );

  /*...more code...*/

} // no 'delete' is required
\end{lstlisting}
\end{multicols}

USAGE:
\begin{lstlisting}
  T* pt1 = new T;  // regular pointer  
  // pass ownership to an auto_ptr
  auto_ptr<T> pt2( pt1 );
  *pt2 = 12;       // same as "*pt1 = 12;"
  pt2->SomeFunc(); // same as "pt1->SomeFunc();"      

  // use get() to see the pointer value
  assert( pt1 == pt2.get() );

  // use release() to take back ownership
   T* pt3 = pt2.release();
                  
  // delete the object ourselves, since now
  // no auto_ptr owns it any more
  delete pt3;


    // by calling '.reset'
    //  an auto_ptr can discard the current pointer it own (including delete it)
    //  and own a new pointer
  auto_ptr<T> pt (new T());
  pt.reset( new T(2) );
  // deletes the first T that was
  // allocated with "new T(1)"
\end{lstlisting}

\textcolor{red}{\bf WEAKNESS}: The problem with \verb!auto_ptr! is that it can loose the
ownership of the raw pointer, via copy constructor or copy assignment
\begin{verbatim}
 std::auto_ptr<int> x (new int);
 *x = 12; 
 printf(*x); //  print '12'
 std::auto_ptr<int> y(x);   //copy constructor
 
 printf(*x);  // fail, as 'x' raw pointer is NULL now
 printf(*y);  // print '12'
 
 
 // or 
 std::auto_ptr<int> y(new int);
 y = x;   //copy assignment 
          // y's previous raw pointer is nolonger tracked
          // y now hold the raw pointer that x previously hold [and 'x' loss this raw pointer]
          // x's holding of a raw pointer is turned into NULL
  
\end{verbatim}

EXPLAIN: Only one \verb!auto_ptr! can hold the ownership of a regular (raw)
pointer at a time, so when we do a copy of an \verb!auto_ptr! pointer (both copy
construction and copy assignment), the ownership is transferred to the new one,
and the \verb!auto_ptr! on the right-hand-side no longer hold the pointer (i.e.
set back to a null state), and also, if the \verb!auto_ptr! on the
left-hand-side is holding another pointer; that pointer is discard so that the
left-hand-side \verb!auto_ptr! can hold the new pointer.


SO, due to this technical issue, \verb!auto_ptr! cannot be used to store objects
in most containers. So, since C++11, \verb!std::auto_ptr! will be deprecated in
favor of \verb!std::unique_ptr! (Sect.\ref{sec:unique_ptr}).

% 
% It's simply a wrapper around a regular pointer. Here, the destructor takes care
% of deleting the pointer. How to use?
% \begin{lstlisting}
% void foo()
% {
%     auto_ptr<MyClass> p(new MyClass);
%     p->DoSomething();
%     
%     /* we can avoid
%     MyClass* p(new MyClass);
%     p->DoSomething();
%     delete p;
%     */
% }
% \end{lstlisting}

{\bf APPLICATIONS}:
\begin{enumerate}
  \item wrap data member as pointer

Old way: the destructor needs to add code to destruct the data member as pointer
\begin{verbatim}
class C
    {
public:
  C();
  ~C();
  /*...*/
private:
  class CImpl; // forward declaration
  CImpl* pimpl_;
};
class C::CImpl { /*...*/ };

C::C() : pimpl_( new CImpl ) { }
C::~C() { delete pimpl_; }    
\end{verbatim}

New way: less code (no need to call delete in destructor)
\begin{lstlisting}
class C
    {
    public:
      C();
      /*...*/
    private:
      class CImpl; // forward declaration
      std::auto_ptr<CImpl> pimpl_;
    };

    // file c.cpp
    //
    class C::CImpl { /*...*/ };

    C::C() : pimpl_( new CImpl ) { }
\end{lstlisting}

  \item sole ownership to a raw pointer:

\textcolor{red}{\bf const std::auto\_ptr}: never loose ownership
\begin{lstlisting}
const std::auto_ptr<T> pt1 (new T());
   // making pt1 const guarantees that
   // the pointer owned by pt1 never be
   // transferred to another auto_ptr
   // and also copy statement, assignment statement 
   //     are illegal (i.e. never work) 

   auto_ptr<T> pt2( pt1 ); // illegal
   auto_ptr<T> pt3;
   pt3 = pt1;              // illegal
   pt1.release();          // illegal
   pt1.reset( new T );     // illegal   
\end{lstlisting}


  \item exception-safe code: with the cost of extra dynamic memory allocation
  
A function dynamically returns an \verb!auto_ptr!  
\begin{lstlisting}
   auto_ptr<String> f()
    {
      auto_ptr<String> result = new String;
      *result = "some value";
      cout << "some output";
      return result;  // rely on transfer of ownership;
                      // this can't throw
    }
    
  auto_ptr<String> theName;
  theName = f();    
\end{lstlisting}
Here, \verb!result! is an \verb!auto_ptr! so there is no memory copy, it just
takes the ownership of the existing memory. Also, once we call it, the ownership
of the memory is again transferred to \verb!theName!, and of course, there is no
memory copy here.

\textcolor{red}{Problem with other codes}:
{\bf Fail Approach 1}: static String object
\begin{verbatim}
    String f()
    {
      String result;
      result = "some value";
      cout << "some output";
      return result;
    }
    String theName;
    theName = f();    
\end{verbatim}
There are two pitfalls: the copy assignment inside the function, and the copy
constructor is invoked when we call the function (as the result is returned by
value). If either copy fails, \verb!f()! already completed all its work while
the result is irretrievably lost.

{\bf Fail Approach 2}: avoid copy, by passing a non-\verb!const! String
reference parameter 
\begin{verbatim}
 void f( String& result )
    {
      cout << "some output";
      result = "some value";
    }
\end{verbatim}
There is still one assignment here which might still fail.


\end{enumerate}

References:
\begin{enumerate}
  \item \url{http://ootips.org/yonat/4dev/smart-pointers.html}
  \item \url{http://www.gotw.ca/gotw/025.htm}
  \item
  \url{http://stackoverflow.com/questions/94227/smart-pointers-or-who-owns-you-baby}
  \item 
  \url{http://www.gotw.ca/publications/using_auto_ptr_effectively.htm}
\end{enumerate}


\subsection{unique\_ptr<type> template (move-only objects) C++11}
\label{sec:unique_ptr}

If you want to hold pointers data in a container
(Chap.\ref{chap:C++_containers}), you should use this smart pointer.
They work properly when containers are resized or moved, and will still be
destroyed when the container is destroyed.

Unlike \verb!std::auto_ptr<T>!, \verb!std::unique_ptr! explicitly prevents
copying of its contained pointer, i.e. compiling error.
You can not do copy or assignmet, as its copy constructor and assignment
operators are explicitly deleted.
 
If you really want to transfer the ownership of the raw pointer, then programmer
has to explicitly call \verb!std::move()! function. This prevents no mistake.


\verb!std::unique_ptr<T>! is a class template, and that you are declaring that
the given pointer is uniquely owned by the given \verb!unique_ptr!.
There is never any doubt about who owns it, so in any copy assignment or copy
constructor, unless you use {\bf move semantic} to tell it to change the
ownership using \verb!std::move()!.

For C++14, we should also add usage of \verb!std::make_unique! where we use
\verb!std::unique_ptr! (Sect.\ref{sec:make_unique}).


Example:

\begin{lstlisting}
std::unique_ptr<Foo>  p (new Foo(41));

std::vector<*Foo> v;

v.push_back( p ); // this is illegal

v.push_back( std::move(p) ); // you need to explicitly telling that
         // the 'p' is giving up the ownership and 
         // you are moving the pointer to the contaienr 'v'
         // and 'p' is reset to null state
 
 // when passing to a function, we pass by reference        
void inc_baz( std::unique_ptr<foo> &p )
{
    p->baz++;
}         
\end{lstlisting}
Like any other smart pointer, $p$ can be used pretty much like a raw pointer,
i.e. using \verb!*! and \verb!->! operator work exactly like you would expect,
and are very efficient (i.e. usually generating nearly the same assembly code as
raw pointer access). 

C++11 provide a way to define a unique object called {\bf move-only objects},
i.e. the object cannot be copied, but can only be moved 

% is an
% excellent feature of C++11 that allows the programmer to manage dynamically created objects with simple ownership
% semantics. 

{\bf WEAKNESS}: The only location it can be is on the stack, and making it a
member of a class is IMPOSSIBLE (or giving undefined behavior).

{\bf APPLICATION}:
\begin{itemize}
  \item the cost of using \verb!unique_ptr! is minimal, i.e. almost the
  same assembly code as using raw pointer!
  
%   automatic destruction of the contained
%   pointer when the poitner goes out of scope (function exit normally or an exception occurs.)
%   
%   It means no need to track them and call \verb!delete! 
  
  \item We only use move-only objects for thoses with short-term life, i.e. the
  pointer will be eventually being hold by something else, not the
  \verb!unique_ptr!
  
  \item  Move semantics will be used automatically any place you create an rvalue
reference. 

\begin{verbatim}
 # returning a unique_ptr from a function
 return p;
 
 
 # passing a newly constructed unique_ptr object
 process (std::unique_ptr<Foo>(new Foo(41));
\end{verbatim}

  \item \verb!const unqiue_ptr<T>!
  
We use \verb!const unique_ptr!
template.

\begin{verbatim}
const std::unique_ptr<T> myFunc();
\end{verbatim}
As we cannot call \verb!std::move! on a \verb!const! object, we need to use
either \verb!const&! or \verb!const&&!.

\begin{verbatim}
const std::unique_ptr<T> &ptr = myFunc();
\end{verbatim}
  
\end{itemize}


Example: The usual case for one to have a \verb!unique_ptr! in a class is to be able to
use inheritance (otherwise a plain object would often do as well, see RAII)

\begin{verbatim}
struct Base
{
    //some stuff
};

struct Derived : public Base
{
    //some stuff
};

struct Foo {
    std::unique_ptr<Base> ptr;  //points to Derived or some other derived class
};
\end{verbatim}
NOW: the goal is, as said, to make Foo copiable. $\rightarrow$ For this, one
needs to do a deep copy of the contained pointer to ensure the derived class is
copied correctly.


Example: actual code
\begin{verbatim}
  StructType* st = lc->sim->getStructType("DimensionStruct");
std::unique_ptr<Struct> dims;
st->getStruct(dims);
dims->initialize(dimList);
StructDataItem* dimsDI = new StructDataItem(dims);

\end{verbatim}
which causes the following error
\begin{verbatim}
error: use of deleted function ‘std::unique_ptr<_Tp, _Dp>::unique_ptr(const std::unique_ptr<_Tp, _Dp>&) 
    [with _Tp = Functor; _Dp = std::default_delete<Functor>]’
           rvalue.reset(new FunctorDataItem(transfer));
           
rvalue.reset(new FunctorDataItem(transfer));

\end{verbatim}

The reason is that you're using
\begin{verbatim}
StructDataItem(std::unique_ptr<Struct> data);
\end{verbatim}
which pass a \verb!unique_ptr! by value, which is wrong as there is no copy-constructor semantics for a \verb!unique_ptr!.
You cannot copy a \verb!unique_ptr!. You can only move it.
A solution is to pass by reference
\begin{verbatim}
StructDataItem(std::unique_ptr<Struct>& data);
\end{verbatim}


\subsection{make\_unique (C++14)}
\label{sec:make_unique}

In C++11, to create a \verb!unique_ptr!
\begin{verbatim}
std::unique_ptr<SomeObject> a = new SomeObject(...)
\end{verbatim}

Since C++14, we can do, without using \verb!new! operator, but using
\begin{verbatim}
std::unique_ptr<SomeObject> a = std::make_unique(SomeObject(...))
\end{verbatim}

WHY THIS FEATURE
\begin{enumerate}
  \item teach uers not to use 'new/delete' or 'new[]/delete[]'
  
  \item can write shorter code using \verb!auto! keyword
  
   \verb!unique_ptr<LongTypeName> up(new LongTypeName(args))! must mention LongTypeName twice
 
 BETTER with
 \begin{verbatim}
 auto up = make_unique<LongTypeName>(args);
 \end{verbatim}
  
  \item carefully implemented for exception safety and is recommended over directly calling \verb!unique_ptr! constructors
\end{enumerate}

WHEN NOT TO USE: Don't use \verb!make_unique! if  you
\begin{enumerate}
  \item  need a custom deleter or 
  
  \item are adopting a raw pointer from elsewhere.
  
\end{enumerate}
\url{https://stackoverflow.com/questions/37514509/advantages-of-using-stdmake-unique-over-new-operator}

Constructs an object of type T and wraps it in a \verb!std::unique_ptr!.
\url{https://en.cppreference.com/w/cpp/memory/unique_ptr/make_unique}

\begin{lstlisting}
struct Vec3
{
    int x, y, z;
    Vec3() : x(0), y(0), z(0) { }
    Vec3(int x, int y, int z) :x(x), y(y), z(z) { }
    friend std::ostream& operator<<(std::ostream& os, Vec3& v) {
        return os << '{' << "x:" << v.x << " y:" << v.y << " z:" << v.z  << '}';
    }
};


{
    // Use the default constructor.
    std::unique_ptr<Vec3> v1 = std::make_unique<Vec3>();
    
    // Use the constructor that matches these arguments
    std::unique_ptr<Vec3> v2 = std::make_unique<Vec3>(0, 1, 2);
    // Create a unique_ptr to an array of 5 elements
    std::unique_ptr<Vec3[]> v3 = std::make_unique<Vec3[]>(5);
}
\end{lstlisting}

\subsection{make\_shared (C++14)}
\label{sec:make_shared}


NOTE: \verb!std::make_shared! (which has \verb!std::allocate_shared!), 


\subsection{shared\_ptr (using a counter) C++11}
\label{sec:shared_ptr_C++11}

\verb!std::shared_ptr! allow more than one owner to manage the
lifetime of the object in memory.

\textcolor{red}{Initialize the pointer}: Prior to C++17, \verb!shared_ptr! could
not be used to manage dynamically allocated arrays.
\begin{enumerate}
  
  \item  Whenever possible, use the \verb!make_shared! function to create a
  \verb!shared_ptr! when the memory resource is created for the first time.
  
  WHY: it is  exception-safe. It uses the same call to allocate the memory for
  the control block and the resource, which reduces the construction overhead.
  
  IF NOT: you have to use an explicit new expression to create the object before
  you pass it to the \verb!shared_ptr! constructor.
  
\end{enumerate}
\begin{verbatim}
// Use make_shared function when possible.
auto sp1 = make_shared<Song>(L"The Beatles", L"Im Happy Just to Dance With You");

// Ok, but slightly less efficient. 
// Note: Using new expression as constructor argument
// creates no named variable for other code to access.
shared_ptr<Song> sp2(new Song(L"Lady Gaga", L"Just Dance"));

// When initialization must be separate from declaration, e.g. class members, 
// initialize with nullptr to make your programming intent explicit.
shared_ptr<Song> sp5(nullptr);
//Equivalent to: shared_ptr<Song> sp5;
//...
sp5 = make_shared<Song>(L"Elton John", L"I'm Still Standing");
\end{verbatim}
\url{https://docs.microsoft.com/en-us/cpp/cpp/how-to-create-and-use-shared-ptr-instances?view=vs-2019}

\textcolor{red}{Usage}: you can copy it, pass it by value in function arguments,
and assign it to other \verb!shared_ptr! instances.

All the instances point to the same object, and share access to one "control
block" that increments and decrements the reference count whenever a new
\verb!shared_ptr! is added, goes out of scope, or is reset.

\textcolor{red}{Destruct}: The pointer is automatically deallocated only when
the counter reaches zero. When the reference count reaches zero, the control
block deletes the memory resource and itself. By default, \verb!shared_ptr! will
call delete on the managed object when no more references remain to it. From
C++17, Sect.\ref{sec:shared_ptr_C++17}.




\subsection{std::default\_delete (C++11)}
\label{sec:std::defalut_delete}

This is the default destruction policy used by std::unique_ptr when no deleter is specified.

1) The non-specialized default_delete uses delete to deallocate memory for a single object.

2) A partial specialization for array types that uses delete[] is also provided.

\begin{lstlisting}
    {
        std::unique_ptr<int> ptr(new int(5));
    } // unique_ptr<int> uses default_delete<int>
 
    {
        std::unique_ptr<int[]> ptr(new int[10]);
    } // unique_ptr<int[]> uses default_delete<int[]>
    
 
     {
        std::shared_ptr<int> shared_good(new int[10], std::default_delete<int[]>
());
    } // the destructor calls delete[], ok
    
\end{lstlisting}

IMPORTANT:
\begin{lstlisting}
   // default_delete can be used anywhere a delete functor is needed
   std::vector<int*> v;
   for(int n = 0; n < 100; ++n)
      v.push_back(new int(n));
   std::for_each(v.begin(), v.end(), std::default_delete<int>());
\end{lstlisting}

\subsection{shared\_ptr (using a counter) C++17}
\label{sec:shared_ptr_C++17}

Since C++17, we can use \verb!std::shared_ptr! to manage a dynamically allocated array, i.e.
when you allocate using new[] you need to call delete[], and not delete, to free the resource.

In order to correctly use \verb!shared_ptr! with an array, you must supply a custom deleter.
STL C++17 provides this default deleter \verb!std::default_delete!
\url{https://en.cppreference.com/w/cpp/memory/default_delete}

\begin{lstlisting}

\end{lstlisting}

\begin{lstlisting}
 // argument is 
 //    T[N]
 // or T[]
std::shared_ptr< int[] > sp(new int[10]);


std::shared_ptr<int> sp(new int[10], std::default_delete<int[]>());
\end{lstlisting}

\begin{lstlisting}
template< typename T >
struct array_deleter
{
  void operator ()( T const * p)
  { 
    delete[] p; 
  }
};

std::shared_ptr<int> sp(new int[10], array_deleter<int>());
\end{lstlisting}


\url{https://stackoverflow.com/questions/13061979/shared-ptr-to-an-array-should-it-be-used}




\subsection{operator: reset(), .release()}	
\label{sec:reset()_smart_pointer}
\label{sec:release()_smart_pointer}

EXAMPLE 01: memory leak
\begin{verbatim}
auto v =  make_unique<int>(12);
v.release();     // is this possible?
\end{verbatim}
NOTE: \verb!.release()! is used to release ownership of the managed object without deleting it. 

This is how we use \verb!.release()!, must tracked the returned value [by a raw pointer, or passing it into another smart pointer]

\begin{lstlisting}
auto v = make_unique<int>(12);  // manages the object
int * raw = v.release();        // pointer to no-longer-managed object
delete raw;                     // needs manual deletion
\end{lstlisting}

We use \verb!reset()! 
\begin{itemize}
  \item to delete the internal raw pointer being hold by the smart pointer, and 
  
\begin{lstlisting}

auto v = make_unique<int>(12);  // manages the object
v.reset();                      // delete the object, leaving v empty

\end{lstlisting}


  \item potentially ask it to track a new one (if we pass the new one via argument)

\begin{lstlisting}

auto v = make_unique<int>(12);  // manages the object
v.reset(new int(10));      // delete the object, 
                     // and then ask v's internal raw pointer to track the new onw 
\end{lstlisting}

\url{https://stackoverflow.com/questions/25609457/does-unique-ptrrelease-call-the-destructor}

\end{itemize}
