\chapter{XML C++}


\section{LibXML++ (C++)}
\label{sec:LibXML++}

Feature-wise it is very complete, including XPath, charset conversions (by
Glibmm) and everything that you'd expect in an XML library. It uses traditional
DOM and SAX APIs, which counts as a pro or a con depending on whom you ask from.
One possible issue is that the dependencies of the library are extremely heavy
(due to the use of Glibmm). Still, it appears to be the only decent XML library
for C++.

\section{TinyXML}
\label{sec:TinyXML}

TinyXML does not parse XML according to the specification, so I would recommend against it, even though it works for simple documents.


\section{XERCES (C++)}
\label{sec:XERCES}

	
Xerces-C++ is a validating XML parser written in a portable subset of C++.
Xerces-C++ makes it easy to give your application the ability to read and write
XML data.
\url{http://xerces.apache.org/xerces-c/}

\url{http://stackoverflow.com/questions/2126541/xerces-c-load-read-and-save-alternatives}

Example on how to use Xerces-C++:
\url{http://libprf1.tigris.org/files/documents/1338/13256/libprf1-0.1R3.tar.gz}

\begin{lstlisting}
// Mandatory for using any feature of Xerces.
#include <xercesc/util/PlatformUtils.hpp>

// Use the Document Object Model (DOM) API
#include <xercesc/dom/DOM.hpp>

// Required for outputing a Xerces DOMDocument
// to a standard output stream (Also see: XMLFormatTarget)
#include <xercesc/framework/StdOutFormatTarget.hpp>

// Required for outputing a Xerces DOMDocument
// to the file system (Also see: XMLFormatTarget)
#include <xercesc/framework/LocalFileFormatTarget.hpp>
\end{lstlisting}

\url{http://www.codeproject.com/Articles/32762/Xerces-for-C-Using-Visual-C-Part}