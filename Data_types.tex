\chapter{Data types}

When you use data, it's important to know about data alignment
(Sect.\ref{sec:data-alignm-gran} and Sect.\ref{sec:memory_alignment}),
especially in struct or class. It's important to know about (1) declaration
specifiers and (2) declarator-id.
\begin{verbatim}
static unsigned long int *x[N];
\______________________/  \___/
     /                     \
    /                       \
 declaration specifiers   declarator-id   
\end{verbatim}

\begin{itemize}
  \item  A {\bf declarator-id} (id) is the name being declared, here is \verb!x!, possibly
surrounded by {\it operators} such as *, [], (), and (in the case of C++)
\verb!&!.  A declarator may not contain any operator at all, e.g.
\begin{verbatim}
int x;
\end{verbatim}

   \item In a {\bf declaration specifiers}, there are two groups: (1) type specifiers which
contribute to the type of the declarator-id, (2) non-type information that
applies directly to the declarator-id. Here, \verb!unsigned long int! is the
type-specifier, and \verb!static! is the non-type information that tell \verb!x!
has statically allocated storage (Sect.\ref{sec:static_C}).
\end{itemize}


\section{Introduction}

\subsection{aggregate (a special type of class): C++98/03}
\label{sec:aggregate-class}
\label{sec:aggregate-class-C++98-C++03}

Definition in C++98/03 for \textcolor{red}{aggregate}: An aggregate is an array
or a class (clause 9) with no user-declared constructors (12.1), no private or
protected non-static data members (clause 11), no base classes (clause 10), and
no virtual functions (10.3).

First of all, any array is an aggregate. An array is an aggregate even if it is
an array of non-aggregate class type.

In C++, the term \verb!class! refers to all classes, structs, and unions A class
(or struct, or union) is an aggregate if and only if it satisfies the criteria
from the above definitions.
\begin{enumerate}
  \item This does not mean an aggregate class cannot have constructors, in
  fact it can have a default constructor and/or a copy constructor as long as
  they are implicitly declared by the compiler, and not explicitly by the user  
  
  \item No private or protected non-static data members. 
  
  You can have as many private and protected member functions (but not
  constructors) as you like and not violate the rules for aggregate classes
  
  You can have as many private or protected static data members and member
  functions as you like and not violate the rules for aggregate classes
  
  \item An aggregate class CAN have a user-declared/user-defined copy-assignment
  operator and/or destructor
  
  \item 
\end{enumerate}

Below are example of VIOLATION, i.e. NOT an aggregate.
\begin{lstlisting}
class NotAggregate1
{// No virtual function
  virtual void f() {} //remember? no virtual functions
};

class NotAggregate2
{// data MUST be public 
  int x; //x is private by default and non-static 
};

class NotAggregate3
{// No user-defined constructor
public:
  NotAggregate3(int) {} //oops, user-defined constructor
};

class Aggregate1
{
public:
  NotAggregate1 member1;   //ok, public member
  Aggregate1& operator=(Aggregate1 const & rhs) {/* */} //ok, copy-assignment  
private:
  void f() {} // ok, just a private function
};
\end{lstlisting}


\subsection{aggregate in C++11}
\label{sec:aggregate-class-C++11}


The standard definition of an aggregate has changed slightly, but it's still
pretty much the same: 
{\it An aggregate is an array or a class (Clause 9) with no {\bf user-provided
constructors} (12.1), {\bf no brace-or-equal-initializers for non-static data
members (9.2)}, no private or protected non-static data members (Clause 11), no
base classes (Clause 10), and no virtual functions (10.3).  
}  (the bold text is the new part).
\begin{enumerate}
  \item  Previously, an aggregate could have no user-declared constructors, but
  now it can't have user-provided constructors. Is there a difference? Yes,
  there is, because now you can declare constructors and default them:  
  
\begin{verbatim}
struct Aggregate {
    Aggregate() = default; // though declared by user, it asks the compiler to
                          // generate the default implementation
};
\end{verbatim}  

With C++11, this is still an aggregate because a constructor (or any special
member function) that is defaulted on the first declaration is not
user-provided.
  
  \item  with this new standard, we can initialize members directly in the class
  like this:
  
\begin{verbatim}
struct NotAggregate {
    int x = 5; // valid in C++11
    std::vector<int> s{1,2,3}; // also valid
};
\end{verbatim}
Using this feature makes the class no longer an aggregate because it's basically
equivalent to providing your own default constructor.

\end{enumerate}

 

\subsection{Why we use aggregate data type}
\label{sec:aggregate-class-wht-we-use-it}

Unlike non-aggregate classes, can be initialized with curly braces
\verb!{}!. This initialization syntax is commonly known for arrays,

\begin{verbatim}
// notice the different name 'm' and 'n'
// which mean we can initialize an array of size 'n'
//   with less values

Type array_name[n] = {a1, a2, ..., am};
\end{verbatim}
It means that
\begin{itemize}
  \item if 'm < n': he first m elements of the array are initialized with a1,
  a2, ..., am and the other $(n - m)$ elements are, if possible,
  value-initialized (Sect.\ref{sec:value-initialized}). 
  
  \item if 'm = n': ith element of the array is initialized with ai
  
  \item if 'm > n':  compiler will issue an error
\end{itemize}

Example:
\begin{lstlisting}
class A
{
public:
  A(int) {} //no default constructor
};
class B
{
public:
  B() {} //default constructor available
};


int main()
{
  A a1[3] = {A(2), A(1), A(14)}; //OK n == m
  A a2[3] = {A(2)}; //ERROR A has no default constructor. Unable to value-initialize a2[1] and a2[2]
  B b1[3] = {B()}; //OK b1[1] and b1[2] are value initialized, in this case with the default-ctor
  int Array1[1000] = {0}; //All elements are initialized with 0;
  int Array2[1000] = {1}; //Attention: only the first element is 1, the rest are 0;
  bool Array3[1000] = {}; //the braces can be empty too. All elements initialized with false
  int Array4[1000]; //no initializer. This is different from an empty {} initializer in that
  //the elements in this case are not value-initialized, but have indeterminate values 
  //(unless, of course, Array4 is a global array)
  int array[2] = {1, 2, 3, 4}; //ERROR, too many initializers
}
\end{lstlisting}

\subsection{Initialize data}

In C++1998 there are 2 types of initialization: zero and default.
\begin{verbatim}
new A - indeterminate value
new A() - zero-initialize

new B - default construct (B::m is uninitialized)

new B() - default construct (B::m is uninitialized)

new C - default construct (C::m is zero-initialized)

new C() - default construct (C::m is zero-initialized)
\end{verbatim}

In C++2003 a 3rd type of initialization, value initialization was added.
\begin{verbatim}
new A - indeterminate value

new A() - value-initialize A, 
          which is zero-initialization if A is a POD.

new B - default-initializes (leaves B::m uninitialized)

new B() - value-initializes B which zero-initializes all fields since its default ctor is compiler generated as opposed to user-defined.

new C - default-initializes C, which calls the default ctor.

new C() - value-initializes C, which calls the default ctor.
\end{verbatim}

\subsection{zero-initialized}

\subsection{default-initialized}

\subsection{value-initialized (since C++2003)}
\label{sec:value-initialized}	


The definition below is imprecise and a bit incorrect but it should give you the
basic idea.
\begin{itemize}
  \item When an object of scalar type (bool, int, char, double, pointers, etc.) is
{\bf value-initialized} it means it is initialized with 0 for that type (false
for bool, 0.0 for double, etc.). When an object of class type with a user-declared
default constructor is value-initialized its default constructor is called.

  \item If the default constructor (of a class) is implicitly defined (i.e. by
  the compiler, NOT by the user) then all nonstatic members are recursively
value-initialized.
\end{itemize}

A reference cannot be value-initialized. Value-initialization for a
non-aggregate class can fail if, for example, the class has no appropriate
default constructor. 




\subsection{POD type (Plain-Old Data type)}
\label{sec:POD-plain-old-datatype}

POD stands for Plain Old Data - that is, a class (whether defined with the
keyword \verb!struct! or the keyword \verb!class!) without constructors,
destructors and \verb!virtual! members functions
(Sect.\ref{sec:virtual_member-function}).

\textcolor{red}{Why POD}: Compiling a POD in C++ gives you the same memory
layout as a struct compiled in C.


POD is all built-in data types (e.g. int, char, float, long, unsigned char,
double, etc.) and all aggregation (Sect.\ref{sec:aggregate-class}) of POD data. 

How to check if something is a POD? Well, there is a struct for that called
\verb!std::is_pod!
\begin{verbatim}
namespace std {
// Could use is_standard_layout && is_trivial instead of the builtin.
template<typename _Tp>
  struct is_pod
  : public integral_constant<bool, __is_pod(_Tp)>
  { };
}
\end{verbatim}

In C++ 98/03, a Plain Old Data Structure in C++ is an aggregate class that
contains only PODS as members, has no user-defined destructor, no user-defined copy
assignment operator, and no nonstatic members of pointer-to-member type.
\url{https://stackoverflow.com/questions/4178175/what-are-aggregates-and-pods-and-how-why-are-they-special/4178176#4178176}

In C++11, 


\subsection{non-POD data types}
\label{sec:non-POD-class}


A POD type is a C++ type that has an equivalent in C, and that uses the same
rules as C uses for initialization, copying, layout, and addressing.


A non-POD class is a class that contains a user-defined constructor, or
destructor, or virtual functions, or derived from a base class, or contains
members that are one of the ones I mentioned.

In other words, if the type is not 'C' compatible, it is non-POD.

\url{http://www.parashift.com/c++-faq-lite/pod-types.html}



\subsection{Name lookup}

C++ classifies names in different ways, i.e. different terminologies. 

\begin{itemize}
  \item identifier : a name consisting of an uninterupted sequence of letters,
  underscotre (\verb!_!) and digits. CONSTRAINTS: it cannot start with a digit,
  and some are reserved as keywords. NOTE: The concept of letters should be
  taken broadly and includes universal character names (UCN)
  
  \item operator-function-id: 
\begin{verbatim}
operator new
operator []
operator &
\end{verbatim}

   \item conversion-function-id: denote a user-defined implicit conversion
   operator
\begin{verbatim}
operator int&
\end{verbatim}

   \item template-id: 
\begin{verbatim}
List<T, int, 0>
\end{verbatim}

   \item unqualified-id: it can be any of the (identifier,
   operator-function-id, conversion-function-id, template-id) or a destructor
   name 
\begin{verbatim}
~Data
~List<T, T, N>
\end{verbatim}

   \item qualified-id: an unqualified-id becomes a qualified-id when it comes
   with namespace or class information, or just the global scope resolution
   operator
\begin{verbatim}
::X			// NOTE: ::X refer to the type
S::x
Array<T>::y
::N::A<T>::z
\end{verbatim}
\end{itemize}

There are two non-standard terms:
\begin{itemize}
  
   \item \verb!qualified name!: a name that undergoes a qualified lookup, in
   which the scope to which the name belongs is explicitly mentioned using a
   scope operator (::) or a member access operator (\verb!->! or \verb!.!).
   
\begin{verbatim}
this->count 	;// qualified name

count 			// unqualified name
\end{verbatim}
   
  
%   \item \verb!unqualified name!:

  \item a {\bf dependent name} : a name that depends in some way to a
  template parameter
\begin{verbatim}
std::vector<T>::iterator      // an dependent name if T is a template parameter

std::vector<int>::iterator      // a nondependent name

ident(x,y,z)      // 'ident' is a dependent name if and only if
                  // any of its argument has a type that depends on 
                  // a template parameter
\end{verbatim}
      
\end{itemize}

% However, there are two major naming concepts:
% \begin{enumerate}
%   \item a {\bf qualified name} : 
% \end{enumerate}

Name lookup attempts to find all names available at the point of usage. 
\begin{enumerate}
  \item  If the
qualified name has the scope is a class, then the base class is also searched,
before any other enclosing scopes, i.e. wrapped by \verb!{ ... }!
\begin{verbatim}
int x;
class B{ int i;};
class D::public B{};
void foo(D* pd) {
  pd->i = 3; // find B::i
  D::x = 1 ; // error, do not find ::x in the enclosing scope
}
\end{verbatim}
  
   \item The unqualified name go through (1) ordinary lookup, i.e. is looked up
   in a successively manner from the inner nost scope to the outer scope via
   enclosing scopes, (2) or sometimes undergoes {\it argument-dependent lookup}
   (ADL or Koenig lookup, extended Koenig lookup - named after Andrew Koenig).
   
\begin{verbatim}
extern int count ;; // (1)

int lookup_(int count) // (2)
{
   if (count < 0) {
       int count = 1;  // (3)
       lookup_(count)  // use 'count' in (3)
   }
   
   return count +  ::count; 
      // the first and unqualified 'count' is from (2)
      // the second and qualified 'count' is from (1)
}
\end{verbatim}
\end{enumerate}

\subsection{Qualified and unqualified types}

In computer programming, a fully qualified name is an unambiguous name that
specifies which object, function or variable the name refer to without regarding
to the context of using it. 
\begin{itemize}
  \item In Unix pathname, the full pathname should start with \verb!/!, e.g.
  /home/user/this/folder/file.txt
  
  \item In URL: wikipedia.com/thisfile
  \item In C++, Tcl, Perl, Ruby: So, the namespace should be explicit, and the two
colons (::) is used to distinguish a fully qualified name from a regular name.

\begin{verbatim}
 // Perl
$package2::scalar
 
  // C++
std::vector  
\end{verbatim}
  
\end{itemize}




A {\bf qualified} name is the one that has some sort of indication of where it
belong, e.g. class specification, namespace specification. 


 A {\bf qualified} data type is a built-in data type that has information
about the storage size, range of values, or precision of the type.  


From C99 standard,
the intrinsic data type are {\it unqualified}
\begin{verbatim}
Any type so far mentioned is an unqualified type. Each unqualified type has
several qualified versions of its type, corresponding to the combinations of
one, two, or   all three of the const, volatile, and restrict qualifiers.

The qualified or unqualified versions of a type are distinct types that belong
to the same type category and have the same representation and alignment
requirements.   
\end{verbatim}

To make a type qualified, we use a type qualifier which was introduced in ANSI
C. NOTE: The qualifier not part of the ANSI C89/90 has two leading underscores,
as the extension, and become normal in C99
\begin{enumerate}
  \item \verb!const!: Sect.\ref{sec:const_C} and Sect.\ref{sec:const_C++}.
  
prohibit writing access, i.e. eliminate potential side-effects across function
calls involving the single object. Notice the difference in {\bf down
qualification} between C and C++. C++ being more type restrictive, will not
allow implicit down qualification. However, C++ allows implicit up
qualification.

  \item \verb!volatile!: Sect.\ref{sec:volatile_C}
  
disable optimization on object's storage, i.e. allowing an object to reside on
the global memory that is accessible by other processes or hardware.
   
  \item \verb!__unaligned! :
  
Typically, the memory address of an allocator-id is the multiple of the size of
the type. This is known as {\it aligned memory}. However, when using packed
structures, unaligned memory can occur. A pointer pointing to non-aligned memory
can cause alignment error. To tell the compiler to generate additional code, we
declare the pointer as \verb!__unaligned!
  
  \item \verb!__restrict! (apply to pointer type only),
  or \verb!restrict! (from C99):  Sect.\ref{sec:pointer_restrict}
  
\end{enumerate}
Using type qualifier, it gives greater control over compiler's optimizations.




\section{Macro values}


The maximum value for a floating-point is represented as
\begin{itemize}
  \item in C: use \verb!FLT_MAX! (defined in \verb!<float.h>!)
  \item in C++: use \verb!FLT_MAX! (defined in \verb!<cfloat>!)
  \item in pure C++: use \verb!std::numeric_limits<float>::max! (defined in
  \verb!<climits>!)
\end{itemize}
NOTE: \verb!MAXFLOAT! is a non-standard macro name.

Many other macros are defined in header files 
\begin{itemize}
  \item  \verb!<climits>! (limits.h):  defines constants with the limits of
  fundamental integral types
  
Example: \verb!LONG_MIN!, \verb!INT_MIN!  
  
  \item \verb!<cfloat>! (float.h): define the limits for fundamental
  floating-point type
  
  \item \verb!<cstdint! (stdint.h): define the  limits for width-specific
  integral types and other typedef types
\end{itemize}
\url{http://www.cplusplus.com/reference/climits/}


\section{Header files}

\subsection{stddef.h (cstddef)}
\label{sec:stddef.h}


The header file \verb!stddef.h! defines certain types.  This file must be
included in all kinds of programs as it defines essential macros: NULL (which
is can also be found in other header files: stdio.h, stdlib.h, string.h,
time.h, wchar.h)

\begin{verbatim}
ptrdiff_t            a signed integer type that represents 
                     the result of pointer subtraction
                      
size_t               unsigned integral type that represents
                     the result of sizeof() operations
                     (found in <stddef.h>, stdlib.h, wchar.h, 
                     stdio.h, string.h) 
                     
max_align_t  [C11/C++11] (type with widest scalar alignment)
nullptr_t    [C11/C++11] (null-pointer type)

wchar_t              wide-character type
                     (found in <stddef.h>, <stdlib.h>, <wchar.h>
                      <cwchar>(C++), 
\end{verbatim}

In C++, their namespace is \verb!std!, e.g. \verb!std::ptrdiff_t!. 
The macros representing the range of values for the above types are defined in
the header file \verb!stdint.h!
\begin{verbatim}
  TYPE        MIN/MAX
ptrdiff_t    PTRDIFF_T_MIN    PTRDIFF_T_MAX
             -65535           +65535

size_t       N/A                SIZE_MAX
                                65535
                                
wchar_t         WCHAR_MIN       WCHAR_MAX

\end{verbatim}


\begin{enumerate}
  \item \verb!ptrdiff_t! : an alias of signed integer type, to represent the
  result of any valid pointer-subtraction operation. This is only used in the
  case of pointers to element of an array.
  
  \item \verb!size_t! : an alias of the unsigned integer type, which is used to
  represent the size of any objects in bytes; i.e. this is the type of the value
  returned by \verb!sizeof()! operator.
  
NOTE: \verb!size_t! is used as the type of the parameters (typically to
represent the length (count in bytes) of a string or a specific buffer) in some
functions from \verb!<cstdlib>! header-file, e.g \verb!bsearch, qsort, calloc!,
\verb!malloc!, \verb!realloc!, \verb!mblen, mbtowc, mbstowc, wcstombs!.

  \item \verb!max_align_t! :  Sect.\ref{sec:memory_alignment}
  
\end{enumerate}

The file \verb!stddef.h! (or cstddef) also defines some macros
\begin{verbatim}
NULL                   (null-pointer)
offsetof(...)          (return the member offset)
\end{verbatim}

\section{Type check}


\subsection{typeof()}
\label{sec:typeof}

In C, GCC provides as an extension, while in C++, we need to use Boost library. 
However, it only gives the type of any expression at compile time.

Then we can use
\begin{lstlisting}
  #define max(a,b) \
       ({ typeof (a) _a = (a); \
           typeof (b) _b = (b); \
         _a > _b ? _a : _b; })
\end{lstlisting}

C++ Boost: use \verb!demangle()!
\begin{lstlisting}
#include <boost/units/detail/utility.hpp>

To_main_msg_evt ev("Failed to initialize cards in " + boost::units::detail::demangle(typeid(*_IO_card.get()).name()) + ".\n", true, this);
\end{lstlisting}
\url{http://stackoverflow.com/questions/1986418/typeid-versus-typeof-in-c}

Since C++11, the standard is \verb!decltype()! (Sect.\ref{sec:decltype}).
Also, for runtime type detection, use \verb!typeid! (Sect.\ref{sec:typeid})

\subsection{name demangle}
\label{sec:name_mangling}

Transforming C++ ABI identifiers (like RTTI symbols) into the original C++
source identifiers is called 'demangling.'

\begin{lstlisting}
// https://gcc.gnu.org/onlinedocs/libstdc++/manual/ext_demangling.html


\end{lstlisting}


\subsection{decltype (C++11)}
\label{sec:decltype}

C++11 introduces \verb!decltype()! as a way to return the type of the argument
which can be an object or a function at {\it compile-time} 

\begin{lstlisting}
vector<int> v;
decltype(v)::value_type i = 0; // int i = 0;

const int&& foo();
const int bar();
int i;
struct A { double x; };
const A* a = new A();
decltype(foo()) x1; // type is const int&&
decltype(bar()) x2; // type is int
decltype(i) x3; // type is int
decltype(a->x) x4; // type is double
decltype((a->x)) x5; // type is const double&
\end{lstlisting} 

\subsection{typeid: <typeinfo> (C++11) - RTTI}
\label{sec:type_info}
\label{sec:typeid}
\label{sec:RTTI}

C++11 provides a feature called RTTI ({\bf Runtime Type Identification}) that
allows the code to detect the type (e.g. class-name) of an object at runtime.
We use header file \verb!<typeinfo>! and the function \verb!typeid()!.

Example: inside the class method
\begin{verbatim}
std::cout << typeinfo(*this).name() ;


// if we already know the class-name
// we can use macro

#define quote(x) #x
std::cout << quote(classA);

\end{verbatim}

Example: on an object
\begin{verbatim}
ClassA obj;

std::cout << typeinfo(obj).name();
\end{verbatim}


\verb!typeid()! operator returns an object of the class \verb!type_info! operator (as a const-qualified lvalue). 
Then we can use \verb!names()! function on that object.
The syntax is: \verb!typeid(x).name()! with \verb!x! is  any type or any
expression (array or pointer) that has a type, to obtain the string
representation of the type.

The compilers have much freedom in implementing \verb!std::typeinfo! functions'
behavior. 


Example: \url{http://www.panix.com/~elflord/cpp/gotchas/index.shtml}
\begin{verbatim}
#include <typeinfo>
#include <iostream>

void foo(int a[])
{
	int *c;
	a++; // legal: type of a is int*

	std::cout << "sizeof int[] passed to function: "  <<
		sizeof(a) << std::endl; // same as sizeof(int*)
	std::cout << "typeid of int*: " << typeid(c).name() << std::endl;
	std::cout << "typeid of int[] passed to function: " << typeid(a).name() 
		<< std::endl; // pointer
}

void bar ( char a[][5] )
{
	char (*c)[5];
	std::cout << "sizeof char a[][5] passed to function: "  <<
		sizeof(a) << std::endl; // same as sizeof(char(*)[5])
	std::cout << "sizeof char a[][5] passed to function: "  <<
		sizeof(a[0]) << std::endl; // same as sizeof( char[5])

	std::cout << "typeid of char(*)[5]: " << typeid(c).name() << std::endl;
	std::cout << "typeid of char[][5] passed to function: " << typeid(a).name() 
		<< std::endl; // pointer
	std::cout << "sizeof a[0]: "<< sizeof(a[0]) << std::endl;


}

int main()
{
	int a[] = {2,3,4};
	std::cout << "Sizeof int[3]: " << sizeof(a) << std::endl; // 3*sizeof(int)
	std::cout << "typeid(int[]): " << typeid(a).name() << std::endl; 
	foo(a);
	std::cout << "---------------------------------" << std::endl;

	char b[3][5]; 
	std::cout << "size: char b[3][5]: " << sizeof(b) << std::endl; // 3*5*sizeof(char)
	std::cout << "sizeof b[0]:  " << sizeof(b[0]) << std::endl; // 5*sizeof(char)
	bar (b);
}
\end{verbatim}


Result
\begin{verbatim}
Sizeof int[3]: 12
typeid(int[]): A3_i
sizeof int[] passed to function: 4
typeid of int*: Pi
typeid of int[] passed to function: Pi
---------------------------------
size: char b[3][5]: 15
sizeof b[0]:  5
sizeof char a[][5] passed to function: 4
sizeof char a[][5] passed to function: 5
typeid of char(*)[5]: PA5_c
typeid of char[][5] passed to function: PA5_c
sizeof a[0]: 5
\end{verbatim}

\subsection{<type\_traits> (C++11)}
\label{sec:type_trait}
\label{sec:type_traits-header}

Many intrinsic data types are \textcolor{red}{fundamental}
\textcolor{blue}{arithmetic} types, which can be groupped into
\textcolor{red}{integral} types or \textcolor{red}{floating point} types.
\begin{verbatim}
// integral types
bool
char, char16_t, char32_t, wchar_t, signed char
short int, int, ...

// floating-point types
float
double
long double
\end{verbatim}

Since C++11, a new header is provided \verb!<type_traits>! that allows us to
check of a variable is of arithmetic type, we use \verb!std::is_arithmetic!
Other functions: \verb!is_floating_point()!, \verb!is_fundamental()!,
\verb!is_signed()!, \verb!is_unsigned!.

Example: \url{http://www.cplusplus.com/reference/type_traits/is_arithmetic/}
\begin{verbatim}
#include <iostream>
#include <type_traits>
#include <complex>

int main() {
  std::cout << std::boolalpha;
  std::cout << "is_arithmetic:" << std::endl;
  std::cout << "char: " << std::is_arithmetic<char>::value << std::endl;
  std::cout << "float: " << std::is_arithmetic<float>::value << std::endl;
  std::cout << "float*: " << std::is_arithmetic<float*>::value << std::endl;
  std::cout << "complex<double>: " << std::is_arithmetic<std::complex<double>>::value << std::endl;
  return 0;
}
\end{verbatim}


\section{Integer: int vs. size\_t vs ptrdiff\_t}
\label{sec:size_t}

There is also \verb!size_type! - Sect.\ref{sec:size_type}.

You should not forget that replacing \verb!int! (or POD type) to \verb!size_t!
or \verb!ptrdiff_t!, though increase portability, the memory size needed for the
program will greatly increase as well. And increase of the necessary memory size
will slow down the application's work for cache will store fewer objects being
dealt with.

"64-bit machine" can mean many things, but typically means that the CPU has
registers that big. The sizeof a type is determined by the compiler, which
doesn't have to have anything to do with the actual hardware (though it
typically does); in fact, different compilers on the same machine can have
different values for these. Size of a pointer should be 8 byte on any 64-bit
C/C++ compiler, but not necessarily size of int.


It had been assumed for a long time that the size of \verb!int! coincides with the size
of a computer word (microprocessor's capacity).  \verb!int! is also used to indicate the size of pointers.
Correspondingly, address arithmetic was built with the use of \verb!int! and \verb!unsigned! types as well.
The following table list type's size as defined in different operating system and microprocessors, \ref{tab:type-size-C++}.

\begin{table}[hbt]
\begin{center}
\caption{Type's sizes} 
\begin{tabular}{ccccccc} 
  \hline
short & int & long & ptr & long long & Label & Examples \\
  \hline\hline
 & 16 &  & 16 & IP16 & PDP-11 UNIX (1973) \\
 16 & 16 & 32 & 16 & & IP16L32 & PDP-11 UNIX (1977), multiple instructions for long type being used \\
 16 & 16 & 32 & 32 & & I16LP32 & MC6800 processor (1980); Apply MacIntosh 68K; Microsoft O/S (plus extras for x86 segments) \\
 16 & 32 & 32 & 32 & & ILP32  & IBM 370; VAX UNIX; many workstations \\
 16 & 32 & 32 & 32 & 64 & ILP32LL, ILP32LL64 & Win32; Amdahl; Convex; 1990 UNIX systems (IP16L32); multiple instructions for long long type being used \\
 16 & 32 & 32 & 64 & 64 & LLP64, IL32LLP64, P64 & 64-bit systems: Win64 (X64, IA64) \\
 16 & 32 & 64 & 64 & 64 & LP64, I32LP64 & 64-bit systems: Most UNIX systems (Linux, Solaris, DEC OSF/1 Alpha, SGI Irix, HP UX11) \\
 16 & 64 & 64 & 64 & 64 & ILP64  & 64-bit systems: HAL, logical analog of ILP32 \\
 64 & 32 & 32 & 64 & 64 & SILP64 & 64-bit systems: UNICOS 
\end{tabular}
\end{center}
\label{tab:type-size-C++}
\end{table}

However, with the evolution of computer architecture, their capacity increased,
it became irrational to further increase \verb!int! type's sizes. To avoid such
change, new types have been added to C/C++ languages:
\begin{enumerate}
  \item \verb!ptrdiff_t!: the base signed integer type for C/C++ language
  
  The type's size is chosen so that it could store the maximum size of a theoretically possible array of any type
  
\begin{lstlisting}
for (ptrdiff_t i = 0; i < n; i++)
  a[i] = 0;
\end{lstlisting}

The code provide safety, portability and good performance. 
 
Like in \verb!size_t!, \verb!ptrdiff_t! can safely store a pointer except for a pointer to a class function. 
\verb!ptrdiff_t! type has its synonym \verb!intptr_t! whose name indicates more clearly that it can store a pointer.


\verb!ptrdiff_t! type is usually used for loop counters, array indexing, size storage and address arithmetic

 
  \item \verb!size_t!: the base unsigned integer type of C/C++ language
  
  the type of the result returned by sizeof operator. A
  variable of \verb!size_t! type can safely store a pointer.
  The exception is pointers to class functions but this is a special case.
   
The type's size is chosen so that it could store the maximum size of a
theoretically possible array of any type On a 32-bit system \verb!size_t! will
take 32 bits, on a 64-bit one 64 bits.

The types \verb!size_t! and \verb!uintptr_t! are synonyms. \verb!size_t! type is
usually used for loop counters, array indexing and address arithmetic.

The maximum possible value of \verb!size_t! type is constant \verb!SIZE_MAX!.

\end{enumerate}



\section{Portable primitive integer data types}
\label{sec:integer}

The ISO/IEC 9899:1990 standard specified that C should have four signed and
unsigned integer types: \verb!char! (yes, char is an integer type), short, int,
and long. In 1999, \verb!_Bool! was added as a single bit integer type
(Sect.\ref{sec:Boolean}). However, the standard doesn't specify the size of the
integer types, other than saying \verb!int! and \verb!short! should be at least 16-bits;
and \verb!long! at least the size of \verb!int! and not smaller than 32-bits;
and \verb!long long! at least the size of \verb!long!.
\footnote{\url{http://code.jasonbhill.com/c/portable-integer-types/}}

The main difference as we see below is because Visual C++ use LLP64 model, while
most UNIX system use LP64, and UNICOS use
SILP64.\footnote{\url{http://en.wikipedia.org/wiki/64-bit}}
\begin{itemize}
  \item On 32-bit or 64-bit using Visual C++ on Windows machines: 
\begin{verbatim}
short                    2 bytes
int                      4 bytes
long                     4 bytes
long long                8 bytes
\end{verbatim}
  \item On 64-bit machines Xubuntu O/S
\begin{verbatim}
short                    2 bytes
int                      4 bytes
long                     8 bytes
long long                8 bytes
\end{verbatim}

   \item On 64-bit UNICOS-based systems (CRAY T3E)
\begin{verbatim}
short                    8 bytes
int                      8 bytes
long                     8 bytes
long long                8 bytes
\end{verbatim}
\end{itemize}
With LP64 for 64-bit systems (i.e. long and pointer are 64-bit), 'int' are
32-bit, while long and pointers are 64-bit. With ILP64 for 64-bit systems, the
three  types \verb!int!, \verb!long!,  and pointers are all 64-bit. With SILP64
for 64-bit system, even \verb!short!  is also 64-bit. \textcolor{red}{Most
64-bit platforms use LP64; while Windows still use LLP64}

Unlike Java or C\#, the primitive integer data types in C++ can vary in size
(bytes) depending on the platforms (32-bit or 64-bit, Windows or Linux). An
example is that \verb!int! is not always a 32-bit integer. Thus, different names
were given, by different compilers, to indicate 32-bit integer, e.g.
\verb!uint32! or \verb!dword!. Microsoft uses \verb!__int16, __int32, __int64!,
instead of \verb!short, int, long long!.

The extended integer types is not provided by the compiler, 
\verb!<stdint.h>! (from GCC 4.5), which defines the following primitive types.

\begin{verbatim}
  TYPE          RANGES of VALUES                 MAX-unsigned-value
int8_t          [-2^7, 2^7-1]                       UINT8_MAX
uint8_t         [  0, 2^8-1]

int16_t         [-2^{15}, 2^{15}-1]
uint16_t        [   0,  2^{16}-1]

int32_t         [-2^{31}, 2^{31}-1]
uint32_t        [   0, 2^{32}-1]

int64_t         [-2^{63}, 2^{63}-1]
uint64_t        [   0, 2^{64}-1]

int_least8_t    [INT_LEAST8_MIN, INT_LEAST8_MAX]  UINT_LEAST8_MAX
int_least16_t
int_least32_t
uint_least8_t
uint_least16_t
uint_least32_t

int_fast8_t    [INT_FAST8_MIN, INT_FAST8_MAX]
int_fast16_t
int_fast32_t
uint_fast8_t
uint_fast16_t
uint_fast32_t

int_fast64_t
uint_fast64_t

intmax_t
uintmax_t

intptr_t
\end{verbatim}
Explain 'LEAST' and 'FAST': we use \verb!LEAST! to indicate a type that has the
number of bits at least equal to the preceding number; we use \verb!FAST! to get
the fastest integer type with the width at least equal to the preceding number
value.

We can convert a constant number to one of the above type using the given macros
by including \verb!stdint.h! (which defines \verb!__STDC_CONSTANT_MACROS!)
\begin{enumerate}
  \item \verb!int_least!N\verb!_t! : two function-like macros which accept
  constants in decimal, octal or hexadecimal
\begin{verbatim}
INTN_C(value)
UINTN_C(value)
\end{verbatim}
Example: \verb!UINT32_C(123)! expands to \verb!123UL!.

   \item \verb!intmax_t, uintmax_t! :
\begin{verbatim}
INTMAX_C(value)
UINTMAX_C(value)
\end{verbatim} 
\end{enumerate}


Other options to use fixed-size integer type for portable code
\begin{enumerate}
  \item On Microsoft Visual Studio: Use
  \url{https://code.google.com/p/msinttypes/}
  
  \item \verb!#include <boost/cstdint.h>! from Boost library, which use
  \verb!stdint.h! (C99)
  
\begin{verbatim}
#include <boost/cstdint.h>  /* C language */
boost::int16_t
boost::int32_t
boost::int64_t

#include <boost/cstdint.hpp> /* C++ */
\end{verbatim}  
NOTE: It's recommended to use \verb!inttypes.h!. 
\url{http://www.boost.org/doc/libs/1_46_1/libs/integer/doc/html/boost_integer/cstdint.html}

NO use of macro \verb!BOOST_NO_INT64_T!
\begin{verbatim}
#include <boost/cstdint.hpp> 
int_least64_t
uint_least64_t
\end{verbatim}

  \item Use C99 standard (from ISO/IEC 9899:1999) with \verb!#include!
  \verb!<stdint.h>!, and use types ending with \verb!_t!, e.g. \verb!uint32_t!
  (or use C++0x with \verb!<cstdint>!).
  Make sure the compiler supports this newer standards.
  
NOTE: We use \verb!stdint.h! for minimal set of definitions. If we want to
extend code portability in functions like \verb!scanf, printf!, then we need to
use \verb!inttypes.h! (see next)
  
  \item Use \verb!#include <inttypes.h>! which
  comprises \verb!stdint.h!, and facilities for handling \verb!printf, scanf!

\begin{verbatim}
int8_t                    1 bytes
int16_t                   2 bytes
int32_t                   4 bytes
int64_t                   8 bytes

uint8_t
uint16_t
uint32_t
uint64_t
               32-bit machines         64-bit machines
intptr_t     32-bit signed integer          64-bit
uintptr_t    32-bit unsigned integer        64-bit
\end{verbatim}
To deal with printing out the value of the corresponding type, the file also
provide the macros for \verb!printf! and \verb!scanf!
\url{http://pubs.opengroup.org/onlinepubs/009695399/basedefs/inttypes.h.html}

We need to replace the left with the right
\begin{verbatim}
 DON'T USE               USE
short                 int16_t
unsigned short        uint16_t
int                   int32_t
unsigned int          uint32_t
long                  int32_t or int64_t
unsigned long         uint32_t or unint64_t
long long             int64_t
unsigned long long    uint64_t
\end{verbatim}

\url{http://stackoverflow.com/questions/7597025/difference-between-stdint-h-and-inttypes-h}

Example: instead of using \verb!%ld! or \verb!%lld! we use PRIx8 to indicate 
8-bit, and PRId32 to indicate 32-bit.
\begin{Verbatim}
#include <inttypes.h>

uint8_t smallval;
int32_t longval;
...
printf("The hexadecimal value of smallval is %" PRIx8
       ", the decimal value of longval is %" PRId32 ".\n",
       smallval, longval);

    // Only requires stdint.h to compile:
    uint16_t myvar = 65535;

    // Requires inttypes.h to compile:
    printf("myvar=%" PRIu16 "\n", myvar);  
\end{Verbatim}


\begin{verbatim}
                 printf         scanf
int8_t          PRId8          SCNd8
uint8_t         PRIu8          SCNu8

int16_t         PRId16         SCNd16
uint16_t        PRIu16         SCNu16

int32_t         PRId32         SCNd32
uint32_t        PRIu32         SCNu32

int64_t         PRId64         SCNd64
uint64_t        PRIu64         SCNu64

\end{verbatim}

  \item Use \url{http://www.azillionmonkeys.com/qed/pstdint.h}
  
  \item Use \url{http://sourceforge.net/projects/ptypes/} (C++ portable types)
\end{enumerate}
\url{http://www.qnx.com/developers/docs/6.4.1/dinkum_en/c99/inttypes.html#PRId8}

Macro conventions: 
\begin{enumerate}
  \item first three letter: PRI (use with printf), and SCN (use with scanf)
  \item fourth letter: 'x' = hexadecimal, 'u' = unsigned, 'o' = octal, 'i' =
  integer, 'd' = decimal
  \item extra letters: '8' = 8-bit, '16' = 16-bit, '32' = 32-bit, '64' = 64-bit,
  'FAST8' = fast eight bit, (FAST16, FAST32, FAST64), 'LEAST8' = least eight-bit
  (LEAST16, LEAST32, LEAST64), PTR = pointer, MAX = maximum supported bit size
\end{enumerate}

\url{http://h21007.www2.hp.com/portal/site/dspp/menuitem.863c3e4cbcdc3f3515b49c108973a801/?jumpid=reg_R1002_USEN&ciid=a868e67018695110VgnVCM100000275d6e10RCRD}

\subsection{unsigned long long}


\verb!unsigned long long! is guaranteed to be at least 64 bits, regardless of the platform.
\begin{verbatim}
// the same
unsigned long long
unsigned long long int
\end{verbatim}

\verb!unsigned long! is guaranteed to be at least 32 bits.
\begin{verbatim}
unsigned long int
unsigned long 
\end{verbatim}

\verb!init! is NOT guarantee to be 32 bits. On some systems, int is 16 bits. 
\url{https://stackoverflow.com/questions/2550345/whats-the-difference-between-unsigned-long-long-int-in-c-c}


With regards to the formatting, with printf, you should use \verb!"%llu"! (since
it is unsigned); \verb!"%lld"! is for 
\verb!signed long long!.



\url{https://stackoverflow.com/questions/10631600/unsigned-long-long-vs-unsigned-longportability-point-of-view}



\subsection{Avoid using pre-defined data type}

\begin{Verbatim}
// Instead of:
   long int time;
   short int mouseX;
   char* menuName;
   
   // Use (for example):
   typedef long int TimeStamp;
   typedef short int Coordinate;
   class String { /* ... */ };
   
   // and:
   TimeStamp time;
   Coordinate mouseX;
   String menuName;
\end{Verbatim}
\url{https://www.doc.ic.ac.uk/lab/cplus/c++.rules/chap18.html}

\subsection{Avoid using unsigned type}

As given in Sect.\ref{sec:math_comparison}, a potential bug can occur if we
compared a negative signed integer vs. an unsigned integer.

It's recommended to avoid unnecessary complexity by using \verb!int! if we can. 

Only use \verb!unsigned! for bitfields and binary masks. Use explicit cast
operation at any comparison with signed or unsigned.

\subsection{std::vector<int>::size\_type vs. std::size\_t}
\label{sec:size_type}

\verb!std::vector<int>::size_type! is {\bf unsigned} as in the C++ standard
\begin{verbatim}
    unsigned integral type  |  
    a type that can represent the size of the largest
         object in the allocation model
\end{verbatim}
while \verb!int! it NOT necessarily unsigned. 

The primary time to use \verb!size_type! is in a template
(Chap.\ref{chap:containers}). Although \verb!std::vector<T>::size-type! is
usually \verb!size_t!, \verb!some_other_container<T>::size_type! might be some
other type instead.

Using \verb!std::vector<int>::size_type! as the datatype in indexing is
recommended as it can represent the largest index (as it is unsigned).
Only the type
\begin{verbatim}
vector<int>::size_type
\end{verbatim}
guarantees that its type can be used to index all vector elements.

Using \verb!size_t! is discussed in Sect.\ref{sec:size_t}.


\subsection{short, int, long, long long}

There is nothing in the C++ standard that dictates that a long should be bigger
than int (it certainly isn't on most 32-bit systems). All the standard says is
that the size of \verb!short <= int <= long! - and that \verb!short! is at least
16 bits, if memory serves [not necessarily expressed as "should be at least 16
bits"].

In Windows, the reason that MS choose to make \verb!long! 32 bits even on a
64-bit system is that the existing Windows API, for historical reasons use a mixture of
\verb!int! and \verb!long! for similar things, and the expectation is that this
is a 32-bit value (some of this goes back to times when Windows was a 16-bit
system). So to make the conversion of old code to the new 64-bit architecture, they choose to
keep \verb!long! at 32 bits, so that applications mixing int and long in various
places would still compile.

NOTE:
\begin{verbatim}
sizeof(char) <= sizeof(short) <= sizeof(int) <= sizeof(long) <= sizeof(long long)
\end{verbatim}

On 16 bit platforms, it is usual for both short and int to be 16 bits.

On 32 bit platforms (and the 36 and 48 bit platforms that still exist), int and
long are almost always the same size. 

On modern 64 bit platforms (with byte addressing), the rational solution would
be to make all four types have different sizes (although one could argue that
according to the standard, int should be 64 bits, which would mean that int,
long and long long all had the same size.

\url{http://stackoverflow.com/questions/18353168/why-long-int-has-same-size-as-int-are-this-modifier-works-at-all}

\section{Data of fixed size (C++11)}

\begin{verbatim}
sizeof(char) <= sizeof(short) <= sizeof(int) <= sizeof(long) <= sizeof(long long)

sizeof(char) == 1 (note: a char could still be more than 8 bits!)

SHRT_MIN <= -32767 and SHRT_MAX >= 32767 (implies short is at least 16 bits)

INT_MIN <= -32767 and INT_MAX >= 32767 (implies int is at least 16 bits)

LONG_MIN <= -2147483647 and LONG_MAX >= 2147483647 (implies long is at least 32 bits)

LLONG_MIN <= -9223372036854775807 and LLONG_MAX >= 9223372036854775807 (implies long long is at least 64 bits)
\end{verbatim}

Header file: \verb!<cstdin>! 
\url{http://en.cppreference.com/w/cpp/types/integer}

\subsection{8-bit type}

\begin{verbatim}
int8_t
\end{verbatim}


\subsection{16-bit type}

\begin{verbatim}
int16_t
\end{verbatim}
\subsection{32-bit type}


\section{Boolean (BOOL, BOOLEAN, VARIANT\_BOOL)}
\label{sec:Boolean}

The C programming language uses "int" as its boolean type (Sect.\ref{sec:integer}).
\url{http://blogs.msdn.com/b/oldnewthing/archive/2004/12/22/329884.aspx}

In C89, \verb!bool! is the common macro (with \verb!true, false!) to be used as boolean
type. However, using it needs to be carefully in logical comparison, and passing
to function.

In C99, \verb!_Bool! is the new intrinsic data type as single-bit integer type.
A standard header file \verb!<stdbool.h>!

\subsection{in Windows (Visual C++, C++/CLI)}

Since Windows 1.0, \verb!BOOL! is defined of type \verb!int!.
\begin{verbatim}
typedef int BOOL;
\end{verbatim}

Since OS/2 NT, BOOLEAN was created, when they decided to write a new operating
system from scratch
\begin{verbatim}
typedef BYTE  BOOLEAN;
\end{verbatim}

Visual Basic folks came up with \verb!VARIANT_BOOL!.  Basic uses -1 to represent
"true" and 0 to represent "false", and \verb!VARIANT_BOOL! was designed to
preserve this behavior.
\begin{verbatim}
typedef short VARIANT_BOOL;
#define VARIANT_TRUE ((VARIANT_BOOL)-1)
#define VARIANT_FALSE ((VARIANT_BOOL)0)
\end{verbatim}


\verb!bool! is a C++ data type that has value \verb!true! and \verb!false!. You
won't see this used much (if at all) in Win32 because Win32 tries to remain
C-compatible.
\url{http://blogs.msdn.com/b/oldnewthing/archive/2004/12/22/329884.aspx}
\begin{verbatim}
bool drinkingUnderAge = true;
\end{verbatim}

To retrieve a value typed in by user (True, False) into a bool value, we use
\verb!bool::Parse()! static method.
\begin{verbatim}
bool drivingUnderAge = false;

Console::WriteLine(L"Were you driving under age?");
Console::Write(L"If Yes, enter True. Otherwise enter False: ");
drivingUnderAge = bool::Parse(Console::ReadLine());
\end{verbatim}


C++/CLI can use System.Boolean managed data type.  You can pass an argument of
type bool or Boolean to a function. Such an argument would be treating as
holding a \verb!true! or \verb!false! value.
\begin{verbatim}
protected: void Dispose(bool disposing);
\end{verbatim} 
\url{http://www.functionx.com/cppcli/conditions/Lesson16.htm}


For more information on using Visual C++, read Sect.\ref{sec:VC++_Windows}.


\section{Union}
\label{sec:union_C}

Defining a union is similar to defining a struct, except the keyword
\verb!union! is used, instead of \verb!struct!.
\begin{verbatim}
union [optional_tag] {
   type_1 id_1;
   type_2 id_2;
   ...
} [optional_variable_definitions]; 
\end{verbatim}

The MAIN difference between \verb!union! and \verb!struct! is that only one
field in \verb!union! should be used at a time. All data members of a
\verb!union! type start from the same memory address. It's mainly used in C, not
in C++; and are almost exclusively used for type-safe polymorphism.
\textcolor{red}{'union' only works with POD (plain-old data) for the members}.
In C++, boost library feature a union-like class \verb!boost::variant!.
\begin{verbatim}
boost::variant<std::string, Foo, Bar>
\end{verbatim}
which allows (1) use non-POD data types (Sect.\ref{sec:non-POD-class}), (2) static type safety.

Using \verb!union! is important where memory
is more scarce, e.g. embedded system.
\footnote{\url{http://stackoverflow.com/questions/4788965/c-c-when-would-anyone-use-a-union-is-it-basically-a-remnant-from-the-c-only}}

Another usage is to access individual byte fields, by putting a \verb!struct!
inside a \verb!union!.
\begin{verbatim}
union bits32_tag {
  int whole;
  struct {char c0, c1, c2, c3;} byte; /* four 8-bit bytes */
} ;

union bits32_tag value;
\end{verbatim}


 This is mainly used: one single data type that can handle
all event types without using unnecessary memory. 

Example:
\url{http://hg.libsdl.org/SDL/file/b4b0e9f1669c/include/SDL_events.h#l420}
\begin{Verbatim}
struct my_variant_t {
    int type;
    union {
        char char_value;
        short short_value;
        int int_value;
        long long_value;
        float float_value;
        double double_value;
        void* ptr_value;
    };
};

/* construct a new float variant instance */
void init_float(struct my_variant_t* v, float initial_value) {
    v->type = VAR_FLOAT;
    v->float_value = initial_value;
}

/* Increments the value of the variant by the given int */
void inc_variant_by_int(struct my_variant_t* v, int n) {
    switch (v->type) {
    case VAR_FLOAT:
        v->float_value += n;
        break;

    case VAR_INT:
        v->int_value += n;
        break;
    ...
    }
}
\end{Verbatim}

Example:
\begin{lstlisting}
union foo {
  int a;   // can't use both a and b at once
  char b;
} foo;

typedef union
{
  unsigned char color[4];
  int       new_color;
}       u_color;
\end{lstlisting}
i.e. 'a' and 'b' starts at the same memory address
\begin{verbatim}
+----+----+----+----+----+----+
|    |    |    |    |    |    |
+----+----+----+----+----+----+
\__________________/
       |
       a
\___/
  |
  b       
\end{verbatim}

Example: in C++, this allows us to access the coordinate as an array or the
elements.
\begin{Verbatim}
class Vector
{
        union 
        {
            double _coord[3];
            struct 
            {
                double _x;
                double _y; 
                double _z;
            };

        };
...
}
\end{Verbatim}

Example: in image processing
\begin{Verbatim}
   union {   // dimension from left to right   // union for the left to right dimension
        uint32_t            m_width;
        uint32_t            m_sizeX;
        uint32_t            m_columns;
    };

    union {   // dimension from top to bottom   // union for the top to bottom dimension
        uint32_t            m_height;
        uint32_t            m_sizeY;
        uint32_t            m_rows;
    };
\end{Verbatim}

\subsection{Initialize data}

To initialize a union data, in C89, we have to do it explicitly
\begin{lstlisting}
typedef union {
  int x;
  float y;
  void *z;
} thing_t;

thing_t foo;
foo.x = 2;
\end{lstlisting}

In C99 we can use
\begin{lstlisting}
union {
      char birthday[9];
      int age;
      float weight;
      } people = { .age = 14 };
\end{lstlisting}

In C++, unions can have constructors.
\begin{lstlisting}
union U 
{
   A a;
   B b;

   U() { memset( this, 0, sizeof( U ) ); }
};
\end{lstlisting}


\subsection{Usage}
Assigning values to more than one data members may give unexpected results,
\begin{verbatim}
union foo;
foo.a = 5;
foo.b = 'c';   //DON'T DO THIS
\end{verbatim}
i.e. the goal is to use only one member, depending on some situation. When
assigning data to one member, and print out the value from the other member,
technically, the result is undefined behavior in C standard; yet in most
compiler implementation, the result is well-defined.

USAGE 1: Unions are particularly useful in {\it Embedded Programming} or direct
memory access is needed.
\begin{lstlisting}
typedef union
{
    struct {
        unsigned char byte1;
        unsigned char byte2;
        unsigned char byte3;
        unsigned char byte4;
    } bytes;
    unsigned int dword;
} HW_Register;
HW_Register reg;
\end{lstlisting}
then we do
\begin{lstlisting}
reg.dword = 0x12345678;
reg.bytes.byte3 = 4;
\end{lstlisting}

Example: To be able to a single bit in the register/memory address
\begin{lstlisting}
typedef union
{
    struct {
        unsigned char b1:1;
        unsigned char b2:1;
        unsigned char b3:1;
        unsigned char b4:1;
        unsigned char reserved:4;
    } bits;
    unsigned char byte;
} HW_RegisterB;
HW_RegisterB reg;

x = reg.bits.b2;
\end{lstlisting}

Example:
\begin{lstlisting}
typedef union {
    unsigned char control_byte;
    struct {
        unsigned int nibble  : 4;
        unsigned int nmi     : 1;
        unsigned int enabled : 1;
        unsigned int fired   : 1;
        unsigned int control : 1;
    }
} ControlRegister;
\end{lstlisting}

Example: C int that is 16-bit long, so that we can retrieve the higher 8-bits or
lower 8-bits easily
\begin{lstlisting}
union data {
    int data;
    struct {
        unsigned char higher;
        unsigned char lower;
    } parts;
};
\end{lstlisting}

USAGE 2: As a replacement for object-oriented inheritance, where we don't know
for example which connection to use
\begin{lstlisting}
        Connection
     /       |       \
  Network   USB     VirtualConnection
\end{lstlisting}
then we declare
\begin{lstlisting}
struct Connection
{
    int type;
    union
    {
        struct Network network;
        struct USB usb;
        struct Virtual virtual;
    }
};
\end{lstlisting}



% USAGE: It's used to convert between the binary operations of integers and floats
% \begin{lstlisting}
% union
% {
%   int i;
%   float f;
% } u;
% 
% // Convert floating-point bits to integer:
% u.f = 3.14159f;
% printf("As integer: %08x\n", u.i);
% \end{lstlisting}


References:
\begin{enumerate}
  \item \url{http://stackoverflow.com/questions/252552/why-do-we-need-c-unions}
  \item
  \url{http://www.edn.com/design/integrated-circuit-design/4394915/Managing-the-8--to-32-bit-processor-migration}
\end{enumerate}

\section{Struct}
\label{sec:struct}

\verb!struct! is a way to create a new data type, which is composed from one or
many components, each components has its own type (simple, or more complicated).
We can use in C (Sect.\ref{sec:struct-C}) and C++ (Sect.\ref{sec:struct-C++}).

It is important to know the memory layout, i.e. how to best order the data components so that least memory footprint.

\subsection{C}
\label{sec:struct-C}

Data structure are stored sequentially in memory, so depending on the order of
declaration, \verb!id_1! always precede \verb!id_1! in memory. An important
aspect that will be covered is data alignment (Sect.\ref{sec:struct_padding}). 

SYNTAX:
\begin{verbatim}
struct [optional_tag] {
   type_1 id_1;
   type_2 id_2;
   ...
} [optional_variable_definitions]; 
\end{verbatim}

SUGGESTIONS:
\begin{enumerate}
  \item Add the tag to the struct definition so that we can use it to define new
  variable of that struct easily
\begin{Verbatim}
struct date_tag { 
   short mm;
   short dd;
   short yy;
}  var_1;

struct data_tag var_2, var_3;

// another option is to use typedef
typedef struct  { 
   short mm;
   short dd;
   short yy;
} data_type;

data_type var_1, var_2, var_3; 
\end{Verbatim}

  \item Don't mix struct definition with variable definition, i.e. put the
  variable definition on a separate line, not on the same line with struct
  definition. Example: put \verb!var_1! on the same line with \verb!var_2!,
  \verb!var_3!.
  
  \item Variables of identical types can be put on the same definition
\begin{verbatim}
struct data_tag {
   short mm,dd,yy;
}
\end{verbatim}
\end{enumerate}

IMPORTANT: A struct can have bitfields, unnamed fields, and word-aligned fields.
These are obtained by following the field declaration with a colon (:), and a
number representing a field length in bit (Sect.\ref{sec:struct_bitfield}). 
\begin{verbatim}
/* keep Process ID information */

struct pid_tag {
  unsigned int inactive : 1;  /*  bitfield, with 1 bit length */
  unsigned int          : 1; /*  unnamed field, with 1 bit length */
  unsigned int refcount : 6; 
  unsigned int          : 0; /* padding */
  short pid_id;
  struct pid_tag *link; /* to child processes */
}

struct CELL {   // Declare CELL bitfield
    unsigned short character  : 8;  // 00000000 ????????
    unsigned short foreground : 3;  // 00000??? 00000000
    unsigned short intensity  : 1;  // 0000?000 00000000
    unsigned short background : 3;  // 0???0000 00000000
    unsigned short blink      : 1;  // ?0000000 00000000
} screen[25][80];       // Array of bitfields 
\end{verbatim}


A struct can wrap an entire array to do entire array copy using a single
struct assignment (Sect.\ref{sec:copy_C-array}).

A struct can be used as a single node in different dynamic data structures, e.g.
lists, trees.
\begin{verbatim}
struct node_tag;

struct node_tag {
  int datum;
  struct node_tag * next;
}

struct node_tag a,b;
a.next = &b;
a.next.next = NULL;
\end{verbatim}

\subsection{C++}
\label{sec:struct-C++}

In C, to be able to use a struct as a type, we need to use \verb!typedef!. In
C++, we don't need to to \verb!typedef!. Another important feature of a
\verb!struct! in C++ is that we can have member function, just like we can do
for class. However, a struct is considered as a POD (plain-old-data), i.e. it
lacks many features of an object-oriented paradigm.
Example: there is no inheritance; there is no privacy setting, i.e. all member
functions and data members are in \verb!public! mode
(Sect.\ref{sec:struct_class}).


Example: If the data member is \verb!std:string!
(Sect.\ref{sec:string_C++}), the internal data of \verb!std:string! hold the
reference to the memory where the actual data stored, but not the data itself.
So, a \verb!std::string! object always have the same size, not matter how long
the string is for the object. So using \verb!new! operator, the size of the
struct object is always known as compile-time. For the data member, initially, it only set the char
pointer to 0. 
\begin{verbatim}
struct Airline {
    string Name;
    int diameter;
    int weight;
};

Airline* myPlane = new Airline;
\end{verbatim}

\subsection{access specifier(s): public, private, protected}
\label{sec:access-specifier}

\begin{mdframed}

In CUDA, we have the following access specifiers:
\begin{verbatim}
__global__ 

__device__
\end{verbatim}
\end{mdframed}

\subsection{std::tuple}
\label{sec:tuple}

\subsection{std::pair<T, U>}
\label{sec:pair}
\label{sec:std::pair}

\begin{verbatim}
template <class T, class U>
struct pair
{
  T first;
  U second;
}
\end{verbatim}

So to make a pair, we have two options
\begin{enumerate}
  \item explicit writing the type with \verb!pair!
  
\begin{verbatim}
// if 'y' and 'z' are int
auto x = std::pair<int, int>(y, z);
\end{verbatim}
  
  \item just use \verb!std::make_pair!
  
\begin{verbatim}
auto x = std::make_pair(y, z);
\end{verbatim}
\end{enumerate}


\section{memory layout in struct in C++ and class }
\label{sec:memory-layout-data-members-in-struct-C++}
\label{sec:memory-layout-data-members-in-class-C++}

Note that the class-keys \verb!struct! and \verb!class! have no difference
regarding layout whatsoever: Their only difference are access-rights which only
exist at compile-time. All members before the first access specifier (Sect.\ref{sec:access-specifier}) in a class are considered private,
while in a struct they are public.
 

It's important to know if the class is an inherited class or not, as it may have data members from the parent class(s)



\begin{verbatim}
struct A
{
    int i, j;
    std::string str;

private:

    float f;

protected:

    double d;
};
\end{verbatim}

C++11 standard ([class.mem]/14)
\begin{verbatim}

Nonstatic data members of a (non-union) class with the same access control
(Clause 11) are allocated so that later members have higher addresses within a
class object. The order of allocation of non-static data members with different
access control is unspecified (11).

\end{verbatim}
It guarantess
\begin{verbatim}
i has a smaller address than j and

j has a smaller address than str

\end{verbatim}

It only says the order, but not that the first variable actually start at the
"first address"? Lets assume a class without inheritance.

\url{https://stackoverflow.com/questions/26939609/how-is-the-memory-layout-of-a-class-vs-a-struct}

\subsection{standard layout class (C++)}
\label{sec:memory-layout-standard-layout-class}

C++11 [class]/7: A standard-layout class is a class that:

\begin{verbatim}

has no non-static data members of type non-standard-layout class (or array of such types) or reference,

has no virtual functions (10.3) and no virtual base classes (10.1),

has the same access control (Clause 11) for all non-static data members,

has no non-standard-layout base classes,
either has no non-static data members in the most derived class and at most one
base class with non-static data members, or has no base classes with non-static
data members, and

has no base classes of the same type as the first non-static data member.[110]

110) This ensures that two subobjects that have the same class type and that
belong to the same most derived object are not allocated at the same address
(5.10).
\end{verbatim}

\section{Struct vs. Class}
\label{sec:struct_class}

In C++, compare between {\it struct} and {\it class}, they are capable of doing
the same things, except in struct, everything is public by default, while in
class, everything is default to private and can be made public. In C++, for
C-compatibility, {\it struct} is still being used.

Some people use {\it struct} to indicate that the class is POD (plain old data)
object. POD doesn't mean no methods; it means no inheritance (i.e. no virtual
functions, no constructors, destructors), or \verb!operator =!.


\textcolor{red}{If you want to write a library, then you may want to use {\it
class} so that users cannot mess with the data.}

Example: The two usages below
are the same \footnote{\url{http://www.cplusplus.com/forum/beginner/5980/}}
{\small
\begin{verbatim}
struct null_stream: std::ostream       class null_stream: public std::ostream
  {                                      {
  struct nullbuf: std::streambuf         public:
    {                                      class nullbuf: public std::streambuf
    int overflow( int c )                    {
      {                                      public:
      return traits_type::not_eof( c );        int overflow( int c )
      }                                          {
    }                                            return traits_type::not_eof( c );
    m_sbuf;                                      }
                                             }
  null_stream():                             m_sbuf;
    std::ios(     &m_sbuf ),
    std::ostream( &m_sbuf )                null_stream():
    { }                                      std::ios(     &m_sbuf ),
  };                                         std::ostream( &m_sbuf )
                                             { }
                                         };
\end{verbatim}
}
% \begin{minipage}[t]{0.5\textwidth}
% 
% \end{minipage}
% \begin{minipage}[t]{0.5\textwidth}	
% \end{minipage}

It's important to know how the compiler layout data elements in a struct
(Sect.\ref{sec:memory_alignment}). Also, \verb!sizeof()! doesn't return the
value equal to the total size of the struct's members. The reason is that the
compiler automatically insert the unused memory into a structure so that data
members are optimally aligned for better performance. In other words: (1) the
struct's starting address align with the widest data elements in the struct, (2)
the data member's starting address is aligned with the multiple of the data
member's size. 

\textcolor{red}{Both C and C++ standard say that structure alignment is
implementation defined}. For this reason, when dealing with libraries that will
be used by different compilers, it is important to understand how the compilers
align data. Some compilers have command-line settings and/or special
\verb!#pragma! statements to change the structure alignment settings
(Sect.\ref{sec:pragma}). Most RISC architectures (e.g. ARM, MIPS, PowerPC,
Cell), in particular, will throw an exception and stop the execution of
a program on an unaligned access. However, the x86 chips are rather unique as it
allows unaligned access. Unaligned access is a feature often found in CISC architectures.

SUGGESTION: \textcolor{red}{One can minimize the size of structures by putting
the largest data types at the beginning of the structure and the smallest data types
at the end of the structure.}
\begin{Verbatim}
struct X  /* align 4-byte */
{
    short s; /* 2 bytes */
             /* 2 padding bytes */
    int   i; /* 4 bytes */
    char  c; /* 1 byte */
             /* 3 padding bytes */
};

struct Z
{
    int   i; /* 4 bytes */
    short s; /* 2 bytes */
    char  c; /* 1 byte */
             /* 1 padding byte */
};
\end{Verbatim}

To know the offset (number of bytes) between the beginning of a given structure
\verb!structure_type! and the data member \verb!member!, we use the function
defined in \verb!<stddef.h>!
\begin{verbatim}
#include <stddef.h>

size_t offsetof ( structure_type, member)
\end{verbatim}

\section{Bit shift}
\label{sec:bit-shift}


To create a value that form by bit shift of two different values, e.g.
\begin{itemize}
  \item val1 is stored at lower 16-bits
  \item val2 is stored at higher 16-bits 
\end{itemize}
of bitcombined variable.
\begin{verbatim}
 ...................  = bitcombined
\________/\_______/
   val2      val1
\end{verbatim}
So we do

\begin{lstlisting}
short val1 = 43;
short val2 = 6;

int bitcombined = val2 << 16; //shift-left 16-bits
bitcombined |= val1;        //or-operator the lower 16-bit value
\end{lstlisting}
\textcolor{red}{or}
\begin{verbatim}
bitcombined=low | (high<<8)

bitcombined=low | (high<<16)

\end{verbatim}


Now, to get 
\begin{verbatim}
 //the lower-byte
low=input & 0xff ;

 // the lower-2-bytes
low=input & 0xffff ;
 
\end{verbatim}

and to get the 
\begin{verbatim}
 //the upper-byte (suppose 2-byte value)
low= (input >> 8) & 0xff ;

 // the lower-2-bytes (suppose 4-byte value)
low= (input >> 16) & 0xffff ;
 

\end{verbatim}

\section{Bitfields}
\label{sec:bitfields}

\subsection{C}
\label{sec:struct_bitfield}

\subsection{-- define a bitfield using struct}

The traditional bitfields use \verb!struct! to define bitfields, and the colon
(:) to define how many bits per each field. 

\textcolor{red}{\bf IMPORTANT}: when using bitfields, the type must be either
\verb!int, unsigned int!, or \verb!signed int! (or qualified version of one of
these). It is implementation-dependent whether \verb!int! bitfields can be
negative or not.

Example: 3 fields, each with 10-bit is organized into a single 32-bit value
\begin{verbatim}
high                                              low
0|0|................|.................|..............|
      10 bits           10 bits           10 bits
      writers          waitToRead        readers
\end{verbatim}
REMEMBER: the order above (low bit to hight bit) is just an example, it is
compiler implementation-dependent and O/S dependent
(Sect.\ref{sec:bitfield-not-portable}).

\begin{lstlisting}
struct {
   // the field using lower bits first
   uint32_t  readers:10;
   uint32_t  waitToRead:10;
   uint32_t  writers:10;
}
\end{lstlisting}

\subsection{-- define a bitfield using macros}

This is better than using \verb!struct! as it provides a safe bitfield, i.e.
it performs safety checks to ensure that every operation on the bitfield fits
within the available number of bits.

\begin{lstlisting}
BEGIN_BITFIELD_TYPE(Status, uint32_t)           // type name, storage size
    ADD_BITFIELD_MEMBER(readers, 0, 10)         // member name, offset, number of bits
    ADD_BITFIELD_MEMBER(waitToRead, 10, 10)
    ADD_BITFIELD_MEMBER(writers, 20, 10)
END_BITFIELD_TYPE()
\end{lstlisting}
\url{http://preshing.com/20150324/safe-bitfields-in-cpp/}

The above set of macros defines a new bitfield type \verb!Status! with three
members. The second argument to \verb!BEGIN_BITFIELD_TYPE()! must be an unsigned
integer type. The second argument to \verb!ADD_BITFIELD_MEMBER()! specifies each
member's offset, while the third argument specifies the number of bits.

\subsection{-- bitfield not portable}
\label{sec:bitfield-not-portable}

Example:
\begin{verbatim}
struct ParsedInt {
    unsigned int f1:1;
    unsigned int f2:3;
    unsigned int f3:4;
};
uint8_t i;
struct ParsedInt *d = &i;
\end{verbatim}

{\bf Code with bitfields it not portable}: C doesn't specify the layout of
bitfields which can be little-endian or big-endian (Sect.\ref{sec:byte-order}).
In the above declaration, different scenarios can happen at memory level, the
padding byte (Sect.\ref{sec:bitfield-padding}) can be added any where (the
beginning, the end or in the middle).

\begin{minipage}[t]{0.5\textwidth}
\begin{verbatim}
PADDING : 8
f1 : 1
f2 : 3
f3 : 4

or

PADDING : 8
f3 : 4
f2 : 3
f1 : 1

or
\end{verbatim}
\end{minipage}
\begin{minipage}[t]{0.5\textwidth}
\begin{verbatim}
f1 : 1
f2 : 3
f3 : 4
PADDING : 8

or

f3 : 4
f2 : 3
f1 : 1
PADDING : 8
\end{verbatim}
\end{minipage}

The specific implementation is compiler-dependent: (1) whether a bitfield can
straddle a storage-unit boundary, (2) the order of the bitfields within a unit
(6.7.2.1); and the alignment of the addressable storage unit allocated to hold
bitfield is unspecified. So, we don't assume the exact location of the bitfield.
As a matter of fact, it is not a good idea to use bitfield when you want to
control the resulting layout.
{\small 
\begin{verbatim}
ISO/IEC 9899: 6.7.2.1/10

An implementation may allocate any addressable storage unit large enough to hold
a bit-field. If enough space remains, a bit-field that immediately follows
another bit-field in a structure shall be packed into adjacent bits of the same
unit. If insufficient space remains, whether a bit-field that does not fit is
put into the next unit or overlaps adjacent units is implementation-defined. The
order of allocation of bit-fields within a unit (high-order to low-order or
low-order to high-order) is implementation-defined. The alignment of the
addressable storage unit is unspecified.     
\end{verbatim}
}
\begin{enumerate}
  \item In Windows: The ordering of data declared as bit fields is from low to
  high bit.
  
\end{enumerate}

A better approach is using \verb!#pragma! which is compiler-specific means.

The only portable solution is to use bitwise operators. The generated code
machine is exactly the same, but it is deterministic when using bitwise
operators. The only thing with bitfields that can be trusted is when using
byte-packed, i.e. a chunk of boolean values (not bit values). In C++, use
\verb!std::bitset<N>!


\subsection{-- bitfield padding (alignment): unnamed bitfield}
\label{sec:bitfield-padding} 
\label{sec:bitfield-unnamed-bitfield}

\textcolor{red}{Each bitfield must fit within the boundary of the declared
type}.

In the example below, the declared type is \verb!unsigned short!, i.e. the
boundary is every 16-bits. As the fourth bitfield with 8-bit, instead of
starting at bit 14, it starts at bit 16, i.e. there are two bits skipped.

Example:
\begin{lstlisting}
// bit_fields1.cpp
// compile with: /LD
struct Date {
   unsigned short nWeekDay  : 3;    // 0..7   (3 bits)
   unsigned short nMonthDay : 6;    // 0..31  (6 bits)
   unsigned short nMonth    : 5;    // 0..12  (5 bits)
   unsigned short nYear     : 8;    // 0..100 (8 bits)
};
\end{lstlisting}
explains
\begin{verbatim}
high                                                  low
x|x|x|x|x|x|x|x|...........|x|x|........|...........|....|
                  8 bits         5 bits   6 bits    3bits
                  nYear          nMonth   nMonthDay  nWeekDay
\end{verbatim}

Example: here, the declared type is \verb!unsigned!, i.e. the boundary is every
32-bit. Also, using {\bf unnamed bitfield of zero-length} force the alignment to
the next boundary
\begin{lstlisting}
struct Date {
   unsigned nWeekDay  : 3;    // 0..7   (3 bits)
   unsigned nMonthDay : 6;    // 0..31  (6 bits)
   unsigned           : 0;    // Force alignment to next boundary.
   unsigned nMonth    : 5;    // 0..12  (5 bits)
   unsigned nYear     : 8;    // 0..100 (8 bits)
};
\end{lstlisting}
explains (here in Windows, little-endian byte-order)
\begin{verbatim}
Windows-specific:
  lower byte                |     higher-byte
xxxxxx|............|........|xxxxxxxxxx|.........|.....|
           nYear    nMonth         nMonthDay nWeekDay


Linux-specific:
  lower byte                |     higher-byte
 xxxxxxxxxxx|.........|.....|xxxxxx|............|........|
             nMonthDay nWeekDay        nYear      nMonth
             
xx = skipped bits           
\end{verbatim}
\url{https://msdn.microsoft.com/en-us/library/ewwyfdbe.aspx}



\subsection{-- bit twiddling: combine bitfields}

\textcolor{red}{\bf HOW TO GENERATE THE VALUES}: How to combine the bitfields
into a single value? - we use bit twiddling. Consider the first example above
with 3 bitfields, each of 10 bits:

\begin{lstlisting}
  // the bitfield with lowest bit no need to shift
  // the next bitfield here shift 10 bits
  // the third bitfield here shift 20 bits
uint32_t status = readers | (waitToRead << 10) | (writers << 20);
\end{lstlisting}

\subsection{-- extract bitfield}

To extract single-byte, we do
\begin{lstlisting}
unsigned char bytes[4];
int x = 0x0000101D;
bytes[0] = (x >> 24) & 0xFF;
bytes[1] = (x >> 16) & 0xFF;
bytes[2] = (x >>  8) & 0xFF;
bytes[3] =  x        & 0xFF;
\end{lstlisting}


\subsection{C++}
\label{sec:bitset}

Classes and structures can contain members that occupy less storage than an
integral type. These members are specified as bit fields


 In C++, we use \verb!std::bitset<N>!.

\section{Enum}
\label{sec:enum}

The goal is to create a set of descriptive names that can be used
as values for integer-typed variable. 

From the early days of C language, it provides the \verb!#define! approach
(Sect.\ref{sec:macros-define}) which has a major limitation.
\begin{verbatim}
#define mon 0
#deine tue 1
#define wed 2
\end{verbatim}

A better choice is using \verb!enum!, which allows us to access these names via a type.

An \verb!enum! type can be named or unnnamed. If named, then we can use that name at different place.
\begin{verbatim}
enum { mon, tue, wed } my_day;

  // named enum type 'day_t'
enum day_t { mon, tue, wed } my_day;

enum day_t work_day;  //either this

day_t work_day; // or we can obmit 'enum'
\end{verbatim}

IMPORTANT: The size of an element of \verb!enum! type is not fixed. 

The C language standard does not say what size it should be; only that it should
be big enough to fit any enumerator. However, we can force the size using \verb!: type!
\begin{verbatim}
enum day_t : char { mon, tue, wed };

  // 
enum day_t : char { mon, tue, wed } my_day;
\end{verbatim}

IMPORTANT: We can use the value of an \verb!enum! type in a \verb!case! statement
\begin{verbatim}
day_t some_var;
....//do some stuff with some_var
switch (some_var)
{
case mon:
break;
case tue:
break;
}
\end{verbatim}


Since C++11, \verb!enum class! or \verb!enum struct! keywords are added 
(Sect.\ref{sec:enum_class-C++11}).


\subsection{using \#define}
\label{sec:macro-define}

\verb!#define! enables programmers to create a set of descriptive
names that can be used as values for integer-typed variable.

\begin{verbatim}
#define NORTH_WIND        0
#define SOUTH_WIND        1
#define EAST_WIND         2
#define WEST_WIND         3     
#define NO_WIND           4       

int wind_direction = NO_WIND;
\end{verbatim}

The problem with this approach is that it doesn't prevent users/programmers
accidently assign non-sense value to the variable \verb!wind_direction! 

Example:
\begin{verbatim}
wind_direction = 10 ; // unexpected, but accept by the compiler
\end{verbatim}
as the variable is an \verb!int! type and thus any assignment with an integer
value cannot be detected during compilation. 

\verb!enum! was added to solve this problem (Sect.\ref{sec:enum-C}).

\subsection{old-style enum (in C)}
\label{sec:enum-C}

\verb!enum! keyword defines a list of descriptive names, and variables defined
as of this enum-type can only get values from the list

\begin{verbatim}
// SYNTAX: enum optional_tag { value1, value2, value3 } var_1, var_2;

enum { MON, TUE, WED, THUR, FRI, SAT, SUN} day;
  // MON = 0, TUE = 1, ...

//By defining the optional tag: WEEKDAYS, we can reuse the enum
//    without having to redefine the full list
enum WEEKDAYS { MON, TUE, WED, THUR, FRI, SAT, SUN} day;

enum WEEKDAYS day2; 
\end{verbatim}
IMPORTANT: in C, we need to have \verb!enum! before the optional tag. Another
way is to combine it with \verb!typedef! (Sect.\ref{sec:enum-typedef-C}).

The idea behind \verb!enum! is to design a user-defined type that only accept a
restricted few number of values
\footnote{\url{http://www.cprogramming.com/tutorial/enum.html}}.
\textcolor{red}{By default}, the first descriptive name, i.e. \verb!value1! is a
descriptive name representing the value 0, and \verb!value2! is a descriptive
name representing the value one unit increment to the previous value, i.e. 1.
An optional equal sign (=) can be used to break this default value assignment
Also, multiple enum members may share the same associated value.

\begin{verbatim}
enum WEEKDAYS { MON = 2, TUE, WED = 5, THUR, FRI, SAT, SUN} day;
  // TUE = 3
  // THUR = 6, FRI = 7, SAT = 8, SUN=9

day = MON; // okay
day = 2 ; // okay (as mapping to 2
day = 10; // compile-time error, as no name mapping to 10
\end{verbatim}

\verb!enum! is designed as a type-safe to pass a value in a set to a function,
without using \verb!int!, and to be more descriptive as an enum of one type is
not implicitly convertible to another type, and integers are not implicitly
convertible to enum types.

\begin{verbatim}
enum day_tag { mon, tue=2, wed, thu=7, fri} ;
  // so   mon <- 0,
  //      wed <- 3,
  //      fri <- 8

enum day_tag { mon, tue=0, wed, thu, fri=0} ;
  // so   mon <- 0,
  //      tue <- 0
  //      web <- 1
  //      thu <- 2
  //      fri <- 0
\end{verbatim}

\textcolor{red}{Be aware} if  your enum has multiple descriptive names mapping
to the same integer value, you may get unexpected results when you use
\verb!Enum.Parse()! in C\# which will arbitrarily return the first element that
has the requested value.


Example:
\begin{verbatim}
enum Color { Red,Green,Blue };
enum Size { Big,Little };

void f( Color c ) {
}

void f( Size s ) {
}

int main() {
    f( Red );
    f( Big );
}
\end{verbatim}
% Another reason to use \verb!enum! is that we can use the descriptive name in
% checking the value (rather than using the associated integer values) in
% debugging. 

Example: we can use the descriptive name in checking the value (rather than
using the associated integer values) in debugging. 
\begin{verbatim}
  // the optional tag 'wind_directions_t' 
  //   serve as the the type name
enum wind_directions_t {NO_WIND, NORTH_WIND, SOUTH_WIND, EAST_WIND, WEST_WIND};

// in C++
wind_directions_t wind_direction = NO_WIND;
// in C (add 'enum')
enum wind_directions_t wind_direction = NO_WIND;

wind_direction = 453; // doesn't work, we get a compiler error!
\end{verbatim}



\textcolor{red}{Be aware} when using sentinal value as a descriptive name
\begin{verbatim}
enum Color 
{
    Red,
    Green,
    Blue,
    Max = Blue
}  
\end{verbatim}
Max is an example of a sentinal value. This is not recommended per the MSDN
design guidelines for enumerations because it violates the separation of
concerns between the raw value and it's state.

\subsection{--- major limitation of enum in C}
\label{sec:enum-limitations}

\textcolor{red}{LIMITATION}: One major limitation is that enum values were
unscoped--in other words, you couldn't have two enumerations that shared the
same name:
\begin{verbatim}
enum E1 { C1=45, C2=453, C3=77};

//Error - this second enum is at the same scope as E1
//and contains a value of the same name C3
enum E2 { C3=36, C4=73, C5=384 };
\end{verbatim}
\url{http://www.cprogramming.com/c++11/c++11-nullptr-strongly-typed-enum-class.html}

There are two solutions:
\begin{enumerate}
  \item put them into different classes - Sect.\ref{sec:enum-inside-class}
  
  \item put them into different namespaces - Sect.\ref{sec:enum-inside-namespace}
  
  \item (C++11) using strongly typed enum, i.e. \verb!enum class!
  (Sect.\ref{sec:enum_class-C++11})
\end{enumerate}
\url{http://www.cplusplus.com/forum/general/21893/}

\subsection{ * enum inside namespace}
\label{sec:enum-inside-namespace}

To limit the scope of enum, we can put into a namespace.

Example: put in different namespaces
\begin{verbatim}
namespace E1
{
          enum E {C1=45, C2=453, C3=77};
}

namespace E2
{
          enum E {C1=36, C2=73, C3=384};
}

    E1::E e1=E1::C3;
    E2::E e2=E2::C3;
\end{verbatim}

Example:
\begin{verbatim}
enum MyType { Type1 = 1, Type2};
 
enum YourType {Type1, Type2};  // error: Type1 has already been declared in the
                           //current scope
\end{verbatim} 
The solution is to put them into 2 different namespaces
\begin{verbatim}
namespace MyType 
{
  type { Type1 = 1, Type2};
}

namespace YourType 
{
  type { Type1, Type2};
}

// then we can use
MyType::type  a;
YourType::type b;

a = MyType::Type1;
b = YourType::Type1;
\end{verbatim}

\subsection{ * enum inside class}
\label{sec:enum-inside-class}

A better control of the scope when using enums than putting in the namespace is
putting in a class.

Example: put in different classes
\begin{verbatim}
class MyClass
{
    //This enum is in a different  scope from E1 - so C3 here is ok
public:
    enum E3 {C3=36, C4=73, C5=384};
};

    MyClass::E3 enum_var2 = C3;//Error- these are from different enums 

    MyClass::E3 enum_var3 = MyClass::C3; //OK
\end{verbatim}

Here, we access the enum value directly using the class name, i.e.
ClassName::EnumValue. So, still, we cannot have two enums of the same
EnumValue in one class. C++11 provides a more intuitive access using
ClassName::EnumName::EnumValue.

\subsection{typdef enum (in C)}
\label{sec:enum-typedef-C}
\label{sec:typedef-enum-C}

We can use \verb!typedef! (Sect.\ref{sec:typedef}) to help defining a new
type name that map to an \verb!enum!. This is considered as using a data type
more intuitive in C.

If you don't use \verb!typedef!, in C (not C++) you were required to use
\verb!enum EnumName! to refer to a data element of the enumerated type.

\textcolor{red}{C code}
\begin{lstlisting}
// two ways
typedef enum e_TokenType
{
    blah1   = 0x00000000,
    blah2   = 0X01000000,
    blah3   = 0X02000000
} TokenType;

TokenType variable;
void f( TokenType x );

// otherwise
enum TokenType
{
    blah1 = 0x00000000,
    blah2=0x01000000,
    blah3=0x02000000
};

enum TokenType my_type;
void f( enum TokenType x )
\end{lstlisting}

\subsection{old-style enum in C++}
\label{sec:enum-C++}

There are a few main different of \verb!enum! in C++ vs. C language
\begin{enumerate}
  \item enumeration constants are of the enumerated type, not int
  
  An example of enumerated type is 
\begin{verbatim}
enum day_t {mon, tue, wed};
 // then 
 //   mon
 // is of type 'enum day_t' or 'day_t'   
 // (which is compatible with some integer type, not necessarily int).
\end{verbatim}
  
  \item we don't have to re-use \verb!enum! keyword
\end{enumerate}

SYNTAX:
\begin{verbatim}
enum [identifier] [: type] { enum-list}; *

// identifier = optional tag name
\end{verbatim}

Using \verb!enum! in C++ is similar to that in C, except in \textcolor{red}{C++
code}: we don't have to re-use \verb!enum! keyword on the tag name
\begin{lstlisting}
enum TokenType
{
    blah1 = 0x00000000,
    blah2=0x01000000,
    blah3=0x02000000
};

void f( TokenType x ); // Correct C++, Error in C
\end{lstlisting}

Of course, we can also use \verb!typedef! (Sect.\ref{sec:typedef-enum-C}).
Based on the limitations as described in Sect.\ref{sec:enum-limitations}, we
should considered using \verb!enum class! - a new feature from C++11
(Sect.\ref{sec:enum_class-C++11}).

\subsection{enum class (C++11)}
\label{sec:enum_class-C++11}

C++11 provides the new keyword \verb!enum class! and \verb!enum struct! to help
defining enum that helps to avoid naming clash
(Sect.\ref{sec:enum-limitations}).
This is called {\it scoped enum}, while the original C/C++ enum is called {\it
unscoped enum}.  SYNTAX:
\begin{verbatim}
enum [class|struct] [identifier] [: type] { enum-list}; *
\end{verbatim}

Their default underlying type is \verb!int! but can be changed to other integral types
to manage the sizes of enums memory efficiently. They cannot be type casted
implicitly but explicitly.  {\it Old style C++ enums are still available--if you
want them--largely for backward compatibility with existing code bases. }
(Sect.\ref{sec:enum-C++}).

\textcolor{red}{LIMITATION: No class method is allowed in an enum class|struct.}



Example: using enum class
\begin{verbatim}
// this code will compile (if your compiler supports C++11 strongly typed enums)
enum class Color {RED,  // = 0 (by default) 
                  GREEN, BLUE}; 
                  
enum class Feelings {EXCITED, MOODY, BLUE};

Color color = Color::GREEN;
if ( Color::RED == color )
{
    // the color is red
}

enum class Enumeration {
    Val1, Val2, Val3 = 100, Val4 // = 101
};
\end{verbatim}
Each enum value is scoped within the name of the enum class.
The new style requires putting the class name in front of the value, e.g.
\verb!Color::GREEN!.

\textcolor{red}{NOTE}: \verb!enum class! values are not implicitly converted to
integers; therefore, they cannot be compared to integers either (the expression
Enumeration::Val4 == 101 gives a compiler error).

\textcolor{red}{Advantage}: Enum classes have another advantages over old-style
enums. You can have a forward declaration to a strongly typed enum, meaning that
you can write code like:
\begin{verbatim}
enum class Mood; // forward declaration
 
void assessMood (Mood m);
  
// the class can be defined later on:
enum class Mood { EXCITED, MOODY, BLUE };
\end{verbatim}

\textcolor{red}{Advantage}: You can explicitly set the number of bytes to store
each element in your enum--you can use any signed or unsigned integer type.
The default type is \verb!int!, but you can also use char, unsigned long, \ldots
\begin{verbatim}
// we only have three colors, so no need for ints!
enum class Colors : char { RED = 1, GREEN = 2, BLUE = 3 };
\end{verbatim}

\textcolor{red}{Advantage}: In C++11, we can use data types in \verb!<cstdint>!
header to specify the exact size. For example, sometimes you want to have a
32-bit integer, not just an int that might have different sizes on different
architectures.
\begin{verbatim}
#include <cstdint>
enum class Colors : std::int8_t { RED = 1, GREEN = 2, BLUE = 3 };
\end{verbatim} 
The cstdint header includes types such as 
\begin{verbatim}
std::int8_t
std::int16_t
std::int32_t
std::int64_t 
\end{verbatim}
(as well as unsigned versions that begin with u: \verb!std::uint8_t!).


\textcolor{red}{\bf Example:}
The data is declared in .cpp file, e.g. state.cpp
\begin{lstlisting}
#include "state.h"

const int NUM_PEOPLE=300;
State Person[NUM_PEOPLE];
\end{lstlisting}

To use this data, we can put the name as \verb!extern! in the header file, e.g. state.h, 
\begin{lstlisting}
#ifndef STATES_H
#define STATES_H

enum State { happy,sad,mad };

extern const int NUM_PEOPLE;
extern State Person[]; // do not provide the size here

#endif
\end{lstlisting}
and include that header file everywhere
\begin{lstlisting}
#include "state.h"

int main()
{
    for(int i=0;i<NUM_PEOPLE;++i) Person=happy;

    return 0;
}
\end{lstlisting}
\url{https://www.gamedev.net/forums/topic/413799-using-extern-with-enums-in-c/}

\subsection{enum struct (C++11)}
\label{sec:enum_struct-C++11}

\verb!enum struct! is accepted as a synonym to \verb!enum class!
(Sect.\ref{sec:enum_class-C++11})

\subsection{Good practice}

\subsection{- Default as a descriptive name}

It is good to have a descriptive name \verb!Default! and assign to the default
descriptive name in the enumeration.
\begin{verbatim}
public enum Scope
{
    Transient,
    Singleton,
    Default=Transient
}
\end{verbatim}

\subsection{- map user-input (an integer) to a variable of a given enum}

A user-input can be an integer value, the question is how to assign this value 
to a variable of a given enum type. A safe option is using \verb!static_cast!.

NOTE: The associated integer values in the enum may be discontinuous

\begin{lstlisting}
enum Test
{
    A, B
}

int a = 1;

  // not a good option
Test castEnum = (Test)a;

  // a safer option
Test castEnum = static_cast<Test>(a);

wind_directions_t wind_direction = NO_WIND;

std::cin >> static_cast( wind_direction );

\end{lstlisting}


To prevent users from entering a crazy value, this can be a good option
{\small 
\begin{verbatim}
std::cout << "Please enter NORTH, SOUTH, EAST, WEST, or NONE for our wind direction";
std::cout << std::endl;

string input_wind_dir;
cin >>

wind_directions_t wind_dir;

if ( user_wind_dir == "NORTH" )
{
        wind_dir = NORTH_WIND;
}
else if ( user_wind_dir == "SOUTH" )
{
        wind_dir = SOUTH_WIND;
}
else if ( user_wind_dir == "EAST" )
{
        wind_dir = EAST_WIND;
}
else if ( user_wind_dir == "WEST" )
{
        wind_dir = WEST_WIND;
}
else if ( user_wind_dir == "NONE" )
{
        wind_dir = NO_WIND;
}
else
{
        std::cout << "That's not a valid direction!" << std::endl;
}
\end{verbatim}
}

% Here are some examples:
% \begin{verbatim}
% enum enumType 
% { 
%     A = 0,
%     B,  //default is +1 of the previous element, i.e. A+1
%     C,
%     enumTypeEnd
% };
% 
% enum Foo {
%   One, //default is 0
%   Two,
%   Three,
%   Last
% };
% 
% enum Foo {
%   One = 1,
%   Two = 9,
%   Three = 4,
%   Last
% };
% \end{verbatim}


\subsection{- index an enum}

Question: can we use indexing to access an element in the enum, e.g. Test[0] to
get 'hi'? 
\begin{lstlisting}
enum Test{hi, bye};
\end{lstlisting}

ANSWER: NO, you cannot use array-indexing style, but you can use the 
integer value associated with the descriptive name.
  
If you have
\begin{verbatim}
enum Test{hi = 0, bye};
\end{verbatim}
then you can simply refer to 'hi' with the number 0 and 'bye'
with the number 1. NOTE: this really defeats the whole purpose of using an
enumeration in the first place.
  

% REMEMBER: Unless specified otherwise, enums start numbering at 0, and increment
% by 1 each entry.
% \begin{verbatim}
% enum Test
% {
%     hi, //0
%     bye, //1
%     count //2
% }
% \end{verbatim}



\subsection{- traverse through an enum?}

\verb!enum! is not designed for looping (iterate) through its element.
However, if 2 adjacent elements are only 1 value difference, then we can use
the loop.
\begin{verbatim}
enum Foo {
  One = 1,
  Two,
  Three,
  Last
};

for ( int fooInt = One; fooInt != Last; fooInt++ )
{
   Foo foo = static_cast<Foo>(fooInt);
   // ...
}

enum Sounds {
  explosionSnd = 101,
  attackSnd = 102,
  ... // each constant increases by 1
  victorySnd = 185
}

Sounds firstSnd = explosionSnd;
Sounds lastSnd = victorySnd;

for ( int theSound = firstSnd;
theSound <= lastSnd; ++theSound )
  CheckSoundIsPresent(theSound); 
\end{verbatim}

IMPORTANT: This, however, doesn't work for this case where the elements are not
contiguous
\begin{verbatim}
enum Foo {
  One = 1,
  Two = 9,
  Three = 4,
  Last
};
\end{verbatim}

A better way to deal with \verb!enum! elements is to use \verb!switch! statement
\begin{verbatim}
switch ( foo )
{
    case One:
        // ..
        break;
    case Two:  // intentional fall-through
    case Three:
        // ..
        break;
    case Four:
        // ..
        break;
     default:
        assert( ! "Invalid Foo enum value" );
        break;
}
\end{verbatim}

\section{Type alias}
\label{sec:type-alias}

\subsection{typedef}
\label{sec:typedef-C}

We can use \verb!typedef! to define a new name, i.e. an alias, to a given type.

\subsection{'using' specifier to do define a template class alias}


We can define an alias to a template class by using {\bf using} specifier
\begin{verbatim}
//using specifier
using identifier attr(optional) = type-id ;	


template < template-parameter-list >
using identifier attr(optional) = type-id ;


template<class T>
struct Alloc { };

template<class T>
using Vec = vector<T, Alloc<T>>; // type-id is vector<T, Alloc<T>>


Vec<int> v; // Vec<int> is the same as vector<int, Alloc<int>>
\end{verbatim}

\section{float vs. double}
\label{sec:float}
\label{sec:double}

This type represents single-precision floating-point numbers.

NOTE:
\begin{verbatim}
error: short, signed or unsigned invalid for 
\end{verbatim}
it means there is no discrimination between 'signed' or 'unsigned' for a
floating-point number.

A floating-point constant must have a fractional part, which have decimal points
(.) and can contain exponents.
\begin{verbatim}
// default is to type 'double'
8.
8.0
8.0e0    //use 'e' or 'E' the same
8.e0     

// if suffix 'f' or 'F': then type is 'float'
8.F

// if suffix 'l' or 'L': then type is 'long double' (128-bit)
\end{verbatim}
\url{http://msdn.microsoft.com/en-us/library/tfh6f0w2.aspx}

This is syntax error
\begin{verbatim}
8F    // error in C++; but okay in C#
\end{verbatim}

\section{128-bit floating-point (long double)}

It's supported in intrinsic operations since C89. But until C99, it can be used
in STL. Quadruple precision with 128-bit can be supported using libquadmath
library and header file \verb!<quadmath.h>! since GCC 4.6
\footnote{\url{http://stackoverflow.com/questions/13780219/link-libquadmath-with-c-on-linux}}.
\begin{lstlisting}
extern "C" {
    #include <quadmath.h>
}

int main()
{
    __float128 foo=123;
    cosq(foo);
    return 0;
}
\end{lstlisting}
To compile, with linking to external library \verb!libquadmath.so!
(Sect.\ref{sec:linking})
The library is 
\begin{verbatim}
 /usr/lib64/libquadmath.so.0
\end{verbatim}
Example:
\begin{verbatim}
g++ f128.cpp -lquadmath
\end{verbatim}



\section{Complex}
\label{sec:type-complex}
\label{sec:complex}

\subsection{C99}
\label{sec:complex-C}

In C99, a new header file \verb!<complex.h>! was added which provides 
data type \verb!complex!.

a complex and imaginary specifiers \verb!_Complex! and \verb!_Imaginary!
were added. They need to be used with either float, double, or long double.
\begin{lstlisting}

\end{lstlisting}

\subsection{C++}
\label{sec:complex-C++}

C++ is newer than C and use the header \verb!<complex>! which define a template
class \verb!std::complex<>! that can be instantiated for float, double, or long
double.

\begin{lstlisting}
template<class T>
class complex
{
  T real;
  T img;
}
\end{lstlisting}
So, C++ uses the templated types \verb!std::complex<float>!,
\verb!std::complex<double>!, and \verb!std::complex<long double>! for dealing
with complex numbers. 

Those classes function very similarly to the C99 types but are
non-interchangeable.

Reference:
\url{https://en.wikipedia.org/wiki/C_mathematical_functions#complex.h}

\url{http://stackoverflow.com/questions/6363247/what-is-a-complex-data-type-and-an-imaginary-data-type-in-c}

\section{Software-emulated floating-point}

References:
\begin{enumerate}
  \item \url{http://gcc.gnu.org/wiki/Software_floating_point}
  \item \url{http://koala.cs.pub.ro/lxr/#glibc+2.9/soft-fp/subtf3.c#L35}
\end{enumerate}

\section{sig\_atomic\_t}
\label{sec:sig_atomic_t}

The type \verb!sig_atomic_t! is defined in \verb!signal.h! header file
\begin{verbatim}
   TYPE         MIN                MAX
sig_atomic_t   SIG_ATOMIC_MIN    SIG_ATOMIC_MAX
            (signed: -127       127
            unsigned: 0         255
            
\end{verbatim}

\url{https://www.securecoding.cert.org/confluence/display/seccode/SIG31-C.+Do+not+access+shared+objects+in+signal+handlers}


\section{'const'}
\label{sec:const_keywords}

\subsection{'const' with function definition}
\label{sec:const_function}

\verb!const! can be used in two cases with a function: (1) in a returned value,
(2) at the end of function definition.

CASE 1: The meaning of \verb!const! last in a C++ method definition means that
the method is NOT allowed to change any class member (except those that are marked
\verb!mutable! - Sect.\ref{sec:mutable_keyword}) 
\begin{verbatim}
lass Foo 
{
public:
    int Bar(int random_arg) const
    {
        // code
    }
};
\end{verbatim}

A function 
\begin{verbatim}
int Foo::Bar(int random_arg)

// function call
Foo f; f.Bar(4);
\end{verbatim}
is mapped to
\begin{verbatim}
int Foo_Bar(Foo* this, int random_arg)

// becomes
Foo f; Foo_Bar(&f, 4)
\end{verbatim}

Adding \verb!const! at the end, then the example above becomes
\begin{verbatim}
int Foo::Bar(int random_arg) const

// equivalent
int Foo_Bar(const Foo* this, int random_arg)
\end{verbatim}


CASE 2: A function that return a const value is not allowed to be modified. It
means that it cannot be used as a Lvalue (Sect.\ref{sec:rvalue_lvalue}). This
applies to the case when we want it to return a temporary object.
\textcolor{red}{Since C++11, it is strongly advised to return a value of
non-const object so that we can take full advantage of Rvalue references}
(Sect.\ref{sec:xvalue_glvalue_prvalue}). So, using CASE 2 is considered
obsolete.

\begin{verbatim}
const Object myFunc(){
    // do something
    
    return myObject;
}
\end{verbatim}

Example: the \verb!operator +! return a non-const object
\begin{verbatim}
(a+b).expensive()
\end{verbatim}

Example:
\begin{verbatim}

class Bar {
   SomeType& foo() {
      //Actual implementation 
   }
   const SomeType& foo() const {
        return const_cast<Bar*>(this)->foo();
   }
};

\end{verbatim}


\subsection{Positions of 'const' in relation to start (*) pointers and data
type}
\label{sec:const_its-position}

The 'const' keyword was added to C in C89 (Sect.\ref{sec:const_C}) and to C++
in C93 (Sect.\ref{sec:const_C++}). The meaning of 'const' is different between C
and C++. 

Difference \# 1:
\begin{enumerate}
  \item In C, the address for that object cannot be changed, but the value can
  be changed. So, we don't have to initialize the value.
  \item In C++, the value or the content of the object cannot be changed. So, we
  MUST initialize the value or the content of the object at the declaration.
\end{enumerate}
\textcolor{red}{const applies to the thing left of it. If there is nothing on
the left then it applies to the thing right of it.} Here is how it means in C++
\begin{verbatim}
const Object* obj; // can't change data
Object const * obj; // can't change data

Object* const obj; // can't change pointer
const Object* const obj; // can't change data or pointer
Object const* const obj; // a constant pointer to a constant object
\end{verbatim}
If you allow typedef's pointers (which is NOT recommended), then it's better to
put \verb!const! on the right side of the target to avoid confusing (see
Sect.\ref{sec:typedef_pointers}). 

A similar scenario with function pointer where \textcolor{red}{start (*) applies
to the thing left of it. If there is nothing on the left then it applies to the
thing right of it}
% Typically, you put 'const' on the left
% side of the data type to specify the data is not modifiable.
\begin{verbatim}
void * function1(void) declares a function which returns void *,
void (* function2)(void) declares a function pointer to a function which returns void.
\end{verbatim}

\begin{framed}
{\bf Why read right-to-left?} This is probably how the compiler is designed when
parsing the code
\end{framed}

In C++, as the content of the object CANNOT be changed,  the 'const' object
requires the class of the object need to have default constructor.

Difference \# 2:
\begin{enumerate}
  \item In C, it has default external linkage
  \item In C++, it has default internal linkage
\end{enumerate}

NOTE: Using 'const' with char (Sect.\ref{sec:const_char})


\url{http://stackoverflow.com/questions/5503352/const-before-or-const-after}

\section{Programming skils}

\subsection{Mixing 'char' and 'int' NOT RECOMMENDED}

\begin{minipage}[t]{0.5\textwidth}
\begin{lstlisting}
#include <stdio.h>

int main()
{

unsigned char  a;
unsigned char  b;
unsigned int  c;

a = 0;
b = -5;

c =  (a + b); 

printf("c hex: %x\n", c);
printf("c dec: %d\n",c);

}
\end{lstlisting}
Output: 
\begin{verbatim}
c hex: fb
c dec: 251
\end{verbatim}

\end{minipage}
\begin{minipage}[t]{0.5\textwidth}
\begin{lstlisting}









a = 0;
b = 5;

c = (a-b);




}
\end{lstlisting}
Output:
\begin{verbatim}
c hex: fffffffb
c dec: -5
\end{verbatim}
\end{minipage}

\textcolor{red}{WHY}: We need to understand at the assembly level. \verb!char!
are promoted to \verb!int! to do arithmetic. There is no problem with a=0x00,
which is extended to 0x00000000. For the case of b, even though we assign b=-5,
but b is unsigned, it's breaking the rule as a negative value is interpreted as
a very large positive number. So indeed 251 is assigned (which is the sum of the
max value (\verb!UCHAR_MAX!=255) with the reduced value module one (-5+1=-4)) to
be as there is no sign extended, and thus b=0xfb extended to 0x000000fb.
\begin{verbatim}
# First case
0x00000000 + 0x000000fb = 0x000000fb

# Second case
0x00000000 - 0x00000005 = 0xfffffffb
\end{verbatim}


\section{Windows Data Types}
\label{sec:WindowsDataTypes}


Windows Data Types are widely used in Windows to define function return values,
function and message parameters, and structure members. Most of the pointer-type
names begin with a prefix of P or LP (P=32-bit pointer, LP=long-pointer=64-bit
pointer). Handles refer to a resource that has been loaded into memory.
\footnote{\url{http://msdn.microsoft.com/en-us/library/windows/desktop/aa383751(v=vs.85).aspx}}


\begin{Verbatim}
PCWSTR         pointer to NULL-terminated string of 16-bit Unicode-characters
               defined in WinNT.h  
               typedef CONST WCHAR *PCWSTR;
               
\end{Verbatim}