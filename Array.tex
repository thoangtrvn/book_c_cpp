\chapter{Array}
%\section{Array}
\label{sec:array}

To see the difference between a pointer and an array, read
Sect.\ref{sec:pointer_array}.

There are different ways to create an array, each with its purpose of use
\begin{enumerate}
  \item fixed (compile-time known size) array whose elements of intrinsic type -
  Sect.\ref{sec:array-fixed-size-intrinsic}
  
  \item \textcolor{red}{std::array} (Sect.\ref{sec:std::array}) - fixed (compile-time known size)
  container whose elements can be of any type
  
  \item \textcolor{red}{std::valarray} (Sect.\ref{sec:std::valarray}) - dynamic (run-time known size)
   
  It is intended to bring some of the speed of Fortran to C++, as you can't make
  a valarray of pointers, (this avoid pointer aliasing -
  Sect.\ref{sec:pointer-aliasing}), so the compiler can make assumptions about
  the code and optimise it better.
   
   
  \item \textcolor{red}{std::vector} (Sect.\ref{sec:std::vector})
  
  
  \item \textcolor{red}{std::list} (Sect.\ref{sec:std::list})
\end{enumerate}
\begin{verbatim}
//create on stack/ data segment
// x point to a locate on stack/data segment
// x cannot point to anywhere else
int x[10];

//C++
// x a pointer pointing to a memory 
//   allocated on heap (free store)
// x can point to somewhere else
int* x = new int[10];
  
   //need to delete
delete []x;

/////////////////////////////
const int num_rows = 7;
const int num_columns = 5;


int box[num_rows][num_columns];
\end{verbatim}

\section{Type of indexing: int vs size\_t vs ptrdiff\_t ? }

In C++, the default size for array indices is \verb!size_t!
(Sect.\ref{sec:size_t}) which is a 64 bits unsigned 64-bits integer on most
x86-64 platforms. However, the motivation for using an unsigned type as index or
size in the standard is based on constraints only relevant to 16 bit machines.

The design behind \verb!size_t! goes all the way back to the 1970s in C, when the
\verb!unsigned! type was added as a way to convert some abuses of pointer arithmetic
over to integer arithmetic.

The fact \verb!unsigned! can’t be negative also means it can’t represent error
conditions easily in the same type, leading to hacks like \verb!ssize_t!, which is
essentially '\verb!size_t!, but signed'.
Unsigned types also work out poorly for down-counting loops.
\begin{verbatim}
int a[N];
for (unsigned i = N - 1; i >= 0; --i) { // an unfinite loop
   // do something with a[i]
}
\end{verbatim}


The natural type for any integral type in C++ is \verb!int!. If you're worried
about the sizes being so big that they don't fit into an int, \verb!ptrdiff_t!
would be appropriate. This is, after all, the type of the results of subtraction
of pointers or iterators. (The fact that v.size() has a different type than
\verb!v.end() - v.begin()! is really a design flaw in the standard library.)


A number of different sources suggest that \verb!unsigned! \verb!size_t! is
largely considered a design mistake in C; and C++ has inherited that mistake and
propagated it into many library interfaces.

A proposed solution: gsl::index
\url{https://github.com/isocpp/CppCoreGuidelines/blob/master/CppCoreGuidelines.md#es107-dont-use-unsigned-for-subscripts-prefer-gslindex}

\url{https://www.quora.com/Why-do-some-C-programs-use-size_t-instead-of-int-What-are-the-advantages}

\url{http://soundsoftware.ac.uk/c-pitfall-unsigned.html}

The types \verb!size_t! and \verb!ptrdiff_t! enable you to write well-portable code.
\begin{itemize}
  \item  for size : use \verb!size_t! 
  \item for any arrithmetic operation: use \verb!ptrdiff_t!
  
  there exists a type \verb!ptrdiff_t!, which is a signed integer that can hold the difference of two array indices
  
\end{itemize}

As an alternative to these two types, Windows-developers can use types 
\verb!DWORD_PTR!, \verb!SIZE_T!, \verb!SSIZE_T!

\url{https://www.viva64.com/en/a/0050/}




\section{array with compile-time size}
\label{sec:array-fixed-size-intrinsic}

Example:
\begin{verbatim}
string myinput;
string mylist[]={"a", "b", "c"};
\end{verbatim}


Example: find if a string is in the array of strings
\begin{lstlisting}
  std::string *begin = mylist;
  std::string *end = mylist + 3;

  if (std::find(begin, end, "b") != end)
  {
    std::cout << "find" << std::endl;
  }

if (std::find_if(std::begin(mylist), std::end(mylist),
     [](const std::string& s){ return s == "b";}) != std::end(mylist))
\end{lstlisting}

\section{(a pointer to an array) or (an array whose elements are
pointer to\ldots) }
\label{sec:pointer-to-an-array}
\label{sec:array-whose-elements-are-pointers}

One question is the precedence rule between * and [] operator.
Here, \verb!x! is a declarator-id, with the symbol \verb!*! means 'a pointer to'
and \verb![]! means 'array of'. So, the question is x is 'an array of pointer'
or 'a pointer pointing to an array'. 

The nearest precedence chart for C and C++ states that [] operator has higher
precedence than \verb!*!. 

\begin{verbatim}
  // declare and allocate an array 
  // whose elements contains a value of type <type>
<type> x[N];

 // declare and allocate an array of pointers
 // [] has higher precedence than *, so 'x' is an array  
 // each element must be dereference individually, i.e. *x[i]
<type> * x[N];

 // unless it is wrapped by parentheses, then 
 //  'x' is a pointer
 // declare a pointer (without allocating) to an array of ints
<type> (* x)[N];
(<type> * x)[N];

\end{verbatim}

Thus, the declaration *x[N] means that \verb!x! is an
array before it's a pointer. This can be changed using parentheses which has the
highest precedence of all


% \begin{verbatim}
%   // x is a pointer to
% (int* x) [N];
% \end{verbatim}

\section{copy C-array}
\label{sec:copy_C-array}

There is no simple fast way to do array copy. In C, we have to traverse each
element
\begin{verbatim}
// copy array a to b
for (int i = 0; i < size; i++)
   b[i] = a[i];
\end{verbatim}

However, if we wrap the array inside a struct, we can copy the entire array
with  a single assignment statement (Sect.\ref{sec:struct})
\begin{verbatim}
struct s_tag { int a[100];}

struct s_tag a,b;

b = a; // easily
\end{verbatim}

Another option is to treat the array as a contiguous memory region, and we use
functions in Sect.\ref{sec:memcpy()}.

\section{2D C-array (higher dimension)}
\label{sec:C_array_2D}

C doesns't have true multi-dimensional array.
A multidimensional array is basically just a normal, flattened, array (in
memory) with some extra syntatic sugar for accessing.

\begin{verbatim}
int arr[slow][fast];

// In 2D:
// arr[0][0], arr[0][1],...,arr[0][fast-1],arr[1][0]...
//  ...   arr[slow-1][0],...,arr[slow-1][fast-1]


// In 3D:
int arr[slower][slow][fast];
\end{verbatim}

So, to index an elemenent, we cannot use  \verb!a[i,j]!, but it has to be
\begin{verbatim}
a[i][j]
\end{verbatim}
In C, a two-dimensional array \verb!a[X][Y]! is indeed a 1D array of 'X'
elements, each element is a 1D array of 'Y' elements. 

Example:
\begin{verbatim}
int x[10] = {0,1,2,3,4,5,6,7,8,9};
int y[2][5] = {{0,1,2,3,4},{5,6,7,8,9}};
\end{verbatim}


%Instead, it's an array of 'array'
RULE:
\begin{verbatim}
type name[size1][size2]...[sizeN];
// with size1 is the slowest changing
//      sizeN is the fastest changing
\end{verbatim}


\textcolor{red}{Passing to a function}: The knowledge about the size of the
'array elements' is important. That's why we will see that when passing a 2D
array to a function, we need to tell, minimum, the value of the faster changing
dimension (Sect.\ref{sec:function_pass-array}). With more than two dimensions, all but the
first dimension are required.
\begin{verbatim}
func(int a[][7], int b[][5][6])
{
 //	...
}
\end{verbatim}

As the fastest changing dimension is on the right-side, so if the convention is
that left-side refers to row, and right-side refers to column, then C-array is
row-based, i.e. we need to provide data row-by-row
\begin{verbatim}
  // [row][col]
int a[3][4] = {  
 {0, 1, 2, 3} ,   /*  initializers for row indexed by 0 */
 {4, 5, 6, 7} ,   /*  initializers for row indexed by 1 */
 {8, 9, 10, 11}   /*  initializers for row indexed by 2 */
};

 // or just
int a[3][4] = {0,1,2,3,4,5,6,7,8,9,10,11};

 // we don't need to know how big the array 'x'
 // the minimum is it's an array pointing to 'WHAT'
 // here 'WHAT' is an array of 7 elements
int x[][7];

\end{verbatim}

When we loop through elements, loop through column 
\begin{verbatim}
int i, j;
for(i = 0; i < 5; i++)
{
    for(j = 0; j < 7; j++)
       a2[i][j] = 0;
}
    
\end{verbatim}

We will talk about how to pass multi-dimensional array to function
(Sect.\ref{sec:function_pass-array}), or how to address it using pointers
(Sect.\ref{sec:pointer-2Darrays}).


\section{maximum number of dimension in array}

GCC allows arrays of up to 29 dimensions although actually using an array of
more than three dimensions is very rare.

To see how to work with high-dimensional array, read Sect.\ref{sec:C_array_2D}.

\url{http://www.nongnu.org/c-prog-book/online/x697.html}


\section{std::vector (C++, dynamics size)}

Sect.\ref{sec:std::vector}


\section{VLA (size at compile-time)}
\label{sec:VLA}

Variable-length array is a C99 feature (GCC accepts this as an extension in
C90). IMPORTANT: This is NOT a feature of C++ (C++ standard states that array
size must be a constant expression (8.3.4.1 -
Sect.\ref{sec:constant-expression})). Even though the size of the array is
unknown until runtime, the size then CANNOT be changed, i.e. a crucial feature
known as {\it immutable size}. The advantage of using VLA is the ability to
create small-size array with unknown size at compile time, without wasting space
or calling instructions.

The potential of creating a large array on stack (which has less space
available), rather than on the heap (Sect.\ref{sec:heap_stack}) is a problem as
C99 VLA has no way to indicate a failure. One solution is to use \verb!alloca()!
function, which dynamically allocate memory on stack, not on heap
(Sect.\ref{sec:alloca}).

In C++, you are suggested to use \verb!std::vector! (Sect.\ref{sec:stl_vector})
as an alternative to VLA. Neverthelss, vector is not the same as VLA, as
vector still use dynamic memory (heap). 

Example: VLA in C99
\begin{verbatim}
void foo(int n) {
    int values[n]; //Declare a variable length array
}
void foo(int x, int y, int z) {
    int values[x][y][z]; //Declare a variable length array
}


 FILE *
     concat_fopen (char *s1, char *s2, char *mode)
     {
       char str[strlen (s1) + strlen (s2) + 1];
       strcpy (str, s1);
       strcat (str, s2);
       return fopen (str, mode);
     }
\end{verbatim}

\begin{framed}
The advantage of using variable length array (VLA) is that data on heap, so it
stays close to each other. Data is also on the cache.
\end{framed}

GCC has an extension to accept VLA as a member of a \verb!struct! or
\verb!union!
\begin{Verbatim}
   void
     foo (int n)
     {
       struct S { int x[n]; };
     }
\end{Verbatim}

\section{array with alloca()}
\label{sec:alloca}


In C language, dynamically allocated array with malloc/calloc resides on heap,
which requires an explicit call to free memory.

However, for small size arrays, we can try to dynamically allocated it on stack.
Allocate data on stack is much faster (as the program just adjust the value
of the stack pointer). Even though it's fast (especially useful for small-sized
array), there are certain number of important factors
\begin{enumerate}
  \item data will be automatically freed when the stack unwinds
  \item the implementation of \verb!alloca()! is machine and compiler-dependent.
  So, using it is discouraged for portable code.
  \item it was expected to be part of C++/1x but then was
  dropped.\url{http://www.stroustrup.com/C++11FAQ.html#C99}
\end{enumerate}
\textcolor{red}{In C++, variable-length array is preferred} (Sect.\ref{sec:VLA}).

Stack sizes: Most compilers including Visual Studio let you specify the stack size
\begin{itemize}

  \item  In Visual Studio the default stack size is 1 MB i think, so with a
  recursion depth of 10,000 each stack frame can be at most ~100 bytes which
  should be sufficient for a DFS algorithm.

  \item On some (all?) linux flavours the stack size isn't part of the
  executable but an environment variable in the OS
  
  You can then check the stack size with ulimit -s and set it to a new value
  with for example ulimit -s 16384.
  
  \item In Linux, stacks for threads are often smaller.
  
  You can change the default at link time, or change at run time also. 
\begin{verbatim}
glibc i386, x86_64 7.4 MB
Tru64 5.1 5.2 MB
Cygwin 1.8 MB
Solaris 7..10 1 MB
MacOS X 10.5 460 KB
AIX 5 98 KB
OpenBSD 4.0 64 KB
HP-UX 11 16 KB
\end{verbatim}
  
\end{itemize}

\verb!alloca()! function dynamically allocate memory on stack. Thus, there is no
need to free the memory with pointer pointed to data allocated using
\verb!alloca()!.


{\bf Example}: using \verb!alloca()!
\begin{Verbatim}
#include <malloc.h>

void foo (int n)
{
    int *values = (int *)alloca(sizeof(int) * n);
}

int OverflowMyStack(int start) {
    if (start == 0)
        return 0;

    char * p = (char *)_alloca(4096);
    *p = '0';
    return OverflowMyStack(start - 1);
}

int main () {
    return OverflowMyStack(512);
} 
\end{Verbatim}

\section{std::array (size known at compile-time): C++11}
\label{sec:std::array}


In \verb!<array>! header file, \verb!std::array! is a container holding an array
of data elements of any type (i.e. templated). The pointer pointing to the
internal data structure referencing to this array of data elements can be directly accessed via using
\verb!data()! member function. The limitation is that it only supports static
initialization, i.e. the size must be specify at compile-time.

Check Sect.\ref{sec:Allocator} for knowing about Allocator.
\begin{verbatim}
#include <array>
#include <iostream>
int main()
{
 //with known-size of 10 elements, 
 //...data are allocated on the stack
std::array<int, 10> vals{};
 
for (int count=0; count < vals.size(); ++count)
  vals[count]=count+1;
  
std::cout<<"the first value is: "    
  <<vals.front()<<std::endl;  
std::cout<<"the last value is: "<<vals[(vals.size()-1)]  
  <<std::endl;
  
std::cout<<"the total number of elements is: " <<vals.size()<<std::endl;

// directly access to its internal array structure
int* alias=vals.data(); 
std::cout<<"the third element is: "<<alias[2]<<std::endl;
}
\end{verbatim}

Example: check if a string present in an array of strings
\begin{lstlisting}
std::array<std::string, 3> mylist = { "a", "b", "c" };

if (std::find(std::begin(mylist), std::end(mylist), "b") != std::end(mylist))
{
  cout << "find" << endl;
}
\end{lstlisting}

\section{std::dynarray (C++14)}
\label{sec:dynarray-C++14}

C++14 provides two new mechanisms with array-like objects and arrays whose size
known at runtime
\begin{enumerate}
  \item a class template to simulate dynamics array \verb!std::dynarray<T>! with
  elements of type T that are stored contiguously, i.e. \verb!&a[n] = @a[0]+n!.
  
  The objects are created on free-store or on the stack, and there is no way
  which one is being used.
  
  
  \item a core language features: runtime-sized arrays with automatic storage
  duration
\end{enumerate}
The two proposals were approved in Apr-2013 (Bristol meeting) and the new
standard header file must be included \verb!<dynarray>!.

\verb!std::dynarray! constructor takes a parameter indicating the number of
elements, and \textcolor{red}{the size must be known at compile-time and thus
cannot be changed}. There is no default constructor as the is no accepted
default size for a dynarray object. The size zero is allowed.

\verb!std::dynarray! provides random access and reverse iterators. 

\section{std::valarray (C++98)}
\label{sec:std::valarray}

\verb!std::valarray! is the class for representing and manipulating arrays of values.
It supports element-wise mathematical operations and various forms of
generalized subscript operators, slicing and indirect access.


valarray is kind of an orphan that was born in the wrong place at the wrong
time. During the standardization of C++98, valarray was designed to allow some
sort of fast mathematical computations. However, around that time Todd
Veldhuizen invented expression templates and created blitz++, and similar
template-meta techniques were invented, which made valarrays pretty much
obsolete before the standard was even released

\begin{mdframed}

  It is intended to bring some of the speed of Fortran to C++ (for the code
  running on vector-processor, of course), i.e. let some FORTRAN
  vector-processing goodness rub off on C++.
  However, this requires compiler implementation to take advantage of this.
  However, the necessary compiler support never really happened.
  Until recently:
  \begin{itemize}
  \item  Intel has implemented an optimized version of valarray for their compiler. 
  
  It uses the Intel Integrated Performance Primitives (Intel IPP) to improve performance
  
  \item 
  \end{itemize}

  Why std::valarray is faster (if the compiler knows how to optimize the code, on the vector-processor)? 
  
  as you can't make a valarray of pointers, (this avoid pointer aliasing -
  Sect.\ref{sec:pointer-aliasing}), so the compiler can make assumptions about
  the code and optimise it better.
  
It's an attempt at optimization, fairly specifically for the machines that
were used for heavy-duty math when it was written -- specifically, vector
processors like the Crays. For a vector processor, what you generally wanted to
do was apply a single operation to an entire array, then apply the next
operation to the entire array, and so on until you'd done everything you needed
to do. As such, valarrays (value arrays) are intended to bring some
of the speed of Fortran to C++. You wouldn't make a valarray of pointers so the
compiler can make assumptions about the code and optimise it better. (The main
reason that Fortran is so fast is that there is no pointer type so there can be
no pointer aliasing.)
\url{https://stackoverflow.com/questions/1602451/c-valarray-vs-vector}

  It is intended to bring some of the speed of Fortran to C++, as you can't make
  a valarray of pointers, (this avoid pointer aliasing -
  Sect.\ref{sec:pointer-aliasing}), so the compiler can make assumptions about
  the code and optimise it better.
\end{mdframed}

Both std::vector and std::valarray store the data in a contiguous block.
However, they access that data using different patterns, and more importantly,
the API for std::valarray encourages different access patterns than the API for
std::vector.

For more convenience, use \verb!std::vector! (Sect.\ref{sec:std::vector}).

How to construct/declare a
valarray variable: \url{http://stdcxx.apache.org/doc/stdlibug/22-2.html}

Example:
\begin{lstlisting}
#include <valarray>
#include <algorithm>
#include <iterator>
#include <iostream>

int main()
{
    std::valarray<int> a { 1, 2, 3, 4, 5};
    std::valarray<int> b = a;
    std::valarray<int> c = a + b;
    std::copy(begin(c), end(c),
        std::ostream_iterator<int>(std::cout, " "));
}
// output: 2, 4, 6, 8, 10
\end{lstlisting}

Reference : \url{http://en.cppreference.com/w/cpp/numeric/valarray}

Valarrays also have classes which allow you to slice them up in a reasonably
easy way although that part of the standard could use a bit more work. You get
\verb!slice!, \verb!slice_array!, \verb!gslice! and \verb!gslice_array! to play
with pieces of a valarray, and make it act like a multi-dimensional array. You
also get \verb!mask_array! to "mask" an operation (e.g. add items in x to y, but
only at the positions where z is non-zero). To make more than trivial use of
valarray, you have to learn a lot about these ancillary classes, some of which
are pretty complex and none of which seems (at least to me) very well
documented.

Resizing them is destructive and they lack iterators until C++11.
Since C++11, iterators is added to valarray
\url{http://www.cplusplus.com/reference/valarray/valarray/begin/}

\url{http://stackoverflow.com/questions/1602451/c-valarray-vs-vector}

\section{Blitz++}
\label{sec:Blitz++}

Blitz++ is a C++ class library for scientific computing which provides
performance on par with Fortran 77/90. It uses template techniques to achieve
high performance. Blitz++ provides dense arrays and vectors, random number
generators, and small vectors (useful for representing multicomponent or vector
fields).

\url{https://sourceforge.net/projects/blitz/}


\section{Loop throught high-dimensional array quickly}
\label{sec:loop-through-array}

\url{http://nadeausoftware.com/articles/2012/06/c_c_tip_how_loop_through_multi_dimensional_arrays_quickly}


\section{Element-wise operations}

We have a number of options
\begin{enumerate}
  \item std::valarray - Sect.\ref{sec:std::valarray}
  
  \item std::transform on vectors - Sect.\ref{sec:std::transform}
  
  \item Blitz++ - Sect.\ref{sec:Blitz++}
\end{enumerate}