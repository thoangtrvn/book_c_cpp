
\chapter{Source code documentation (C++)}
\label{chap:source-code-docum}


A documentation part of the code can be inlined with the code by putting them
inside the comments. C++ support 2 forms of comments
\begin{itemize}
  \item single-line comment: start with \verb!//!
  \item multi-line comment: start with \verb!/*! and end with \verb!*/!
  
\end{itemize}

There are different styles for writing documentation and there are different
tools to extract the information written in these styles. Here are some styles
\begin{enumerate}
  \item XML style: each line starts with three forward slases (\verb!///!), and
  then there comes a number of tags, e.g. \verb!<summary>...</summary>!. More
  information: Sect.\ref{sec:Documentation_XMLstyle}.
    
  \item JavaDoc-style
  \item C/C++-style
\end{enumerate}

A quite comprehensive list of tools for document generation:
\url{http://engtech.wordpress.com/2006/07/05/inline-source-code-documentation-language-independent/}

You can automate the task of template generation by using some add-ins depending
on the IDE and the style you use
\begin{enumerate}
  \item Visual Studio IDE and XML comment style:
  \begin{itemize}
    \item GhostDoc: \url{http://submain.com/products/ghostdoc.aspx}
    (Ctrl-Shift-D)
    \item Atominer Pro Documentation:
    \url{https://visualstudiogallery.msdn.microsoft.com/7912CCF4-60B8-4132-BACE-5ACACEB7233B/}
    (10\$)
  \end{itemize} 
  \item 
\end{enumerate}
\url{http://www.hlrnet.com/frprdocu.htm}


\section{Visual Studio}

Visual Studio has a built-in tool to extract the documentation written in
XML-comment style. You compile the code with \verb!/doc! option, the compiler
then search and generate an XML documentation file. Then, using this file, we
can create a final document (PDF, HTML, \ldots) using a custom tool

\url{http://msdn.microsoft.com/en-us/library/b2s063f7.aspx}
\begin{enumerate}
  \item Sandcastle
  \item Doxygen
\end{enumerate}

\section{ROBOdoc}
\label{sec:robodoc}


ROBOdoc\footnote{\url{http://www.xs4all.nl/~rfsber/Robo/stabledownload.html}}
allows us to document the source code from any programming
language. To install RoboDOC, we can either download the source code, compile
ourself or using a binary package
\begin{verbatim}
// Ubuntu
apt-get install robodoc
\end{verbatim}


Each source file (.c, .f95, etc.) needs to follow some rules:
\begin{itemize}
\item headers (Sect.\ref{sec:headers}): At each subroutine/function, the
documentation is called a header
\item items (Sect.\ref{sec:items}): inside each header
\item sections (Sect.\ref{sec:sections})
\end{itemize}

\subsection{Headers and header types}
\label{sec:headers}

Each header contains 3 different elements (a beginner marker, a number
of items (Sect.\ref{sec:items}), an end marker). 

\begin{verbatim}
/****f* financial.library/StealMoney
  *  NAME
  *    StealMoney -- Steal money from the Federal Reserve Bank. (V77)
  *  SYNOPSIS
  *    error = StealMoney( userName, amount, destAccount, falseTrail )
  *  FUNCTION
  *    Transfer money from the Federal Reserve Bank into the
  *    specified interest-earning checking account.  No records of
  *    the transaction will be retained.
  *  INPUTS
  *    userName    - name to make the transaction under.  Popular
  *                  favorites include "Ronald Reagan" and
  *                  "Mohamar Quadaffi".
  *    amount      - Number of dollars to transfer (in thousands).
  *    destAccount - A filled-in AccountSpec structure detailing the
  *                  destination account (see financial/accounts.h).
  *                  If NULL, a second Great Depression will be
  *                  triggered.
  *    falseTrail  - If the DA_FALSETRAIL bit is set in the
  *                  destAccount, a falseTrail structure must be
  *                  provided.
  *  RESULT
  *    error - zero for success, else an error code is returned
  *           (see financial/errors.h).
  *  EXAMPLE
  *    Federal regulations prohibit a demonstration of this function.
  *  NOTES
  *    Do not run on Tuesdays!
  *  BUGS
  *    Before V88, this function would occasionally print the
  *    address and home phone number of the caller on local police
  *    976 terminals.  We are confident that this problem has been
  *    resolved.
  *  SEE ALSO
  *    CreateAccountSpec(),security.device/SCMD_DESTROY_EVIDENCE,
  *    financial/misc.h
  ******
  * You can use this space for remarks that should not be included
  * in the documentation.
  */
\end{verbatim}

The beginner marker is in the
first line which tells
\begin{itemize}
\item the header type or the element (f = function, c=class, d=constant (from
define), h=module, m=method, s=structure, t=types, u=unit-test, v=variable,
*=generic header for everything else).

Header types:
\begin{verbatim}
c	Header for a class
d	Header for a constant (from define)
f	Header for a function
h	Header for a module in a project
m	Header for a method
s	Header for a structure
t	Header for a types
u	Header for a unit test
v	Header for a variable
*	Generic header for everything else
\end{verbatim}

You can use two-character header where the first one is 'i' and the second one
is any of the above character. i=internal. An internal header type is special to
document internal function that you don't want client to see (i.e. a
programmer's manual rather than a user's manual). So, if you want to include
them in the manual, you need to run robodoc with \verb!--internal! option, or
\verb!--internalonly!.

You can also define your own header-type by modifying \verb!robodoc.rc! file.
If you want to define a new header type, use the configuration file,
and add three new information (for each header type) 
\begin{enumerate}
\item the single character
\item title for this type as used in the master index
\item name of the file in which the index of this type is stored. 
\item (optional) sorting priority - which decides whether order of the
  header to appear in the generated file
\end{enumerate}
For example:
\begin{verbatim}
headertypes:
    J  "Projects"          robo_projects    2
\end{verbatim}

\item the name of the element (e.g. StealMoney)
\item the module where the element belong to (e.g. financial.library)
\end{itemize}
\begin{verbatim}
****f* financial.library/StealMoney  
\end{verbatim}
ROBOdoc use ``/'' as the separator between the module name and element
name. If multiple elements have the similar documentation, we can
group them using comma (``,'') and can span multiple lines
\begin{verbatim}
****f* financial.library/StealMoney, Steal_Money
\end{verbatim}

The default header marker, line marker, and end line marker are
\begin{verbatim}

/****       C, C++
 *
 ***/

//****      C++
//
//***
\end{verbatim}

The header marker block and the end marker can be changed to fit to
your programming language. For example:
\begin{verbatim}
header markers:
    !>****
end markers:
    !_****
\end{verbatim}



\subsection{Items}
\label{sec:items}

Inside a header (Sect.~\ref{sec:headers}), there is a number of items.
Each item begins with an item name, e.g. NAME, SYNOPSIS, followed by
an item body (at the new line). Each line of an item (name/body) starts with a
remark marker, e.g. \verb!*! for C/C++ (Sect.~\ref{sec:remarks-comments}).
\begin{verbatim}
  *  FUNCTION
  *    Transfer money from the Federal Reserve Bank into the
  *    specified interest-earning checking account.  No records of
  *    the transaction will be retained.
\end{verbatim}


\begin{verbatim}
/****f* financial.library/StealMoney
  *  NAME
  *
  *  SYNOPSIS
  *
  *  FUNCTION
  *
  *  INPUTS
  *
  *  RESULT
  *
  *  EXAMPLE
  *
  *  NOTES
  *
  *  BUGS
  *
  *
  *  SEE ALSO
  *
  ******
\end{verbatim}

Some items you may want not to be extracted by default. So, you need
to add them to the list of ``ignore items'' in the configuration file
(Sect.~\ref{sec:conf-file-robod}).


\subsection{Sections}
\label{sec:sections}

\subsection{Remarks (comments)}
\label{sec:remarks-comments}

Inside the code, you may have comment (remarks) that you don't want it
to be added to the documentation. You can tell ROBOdoc to recognize
such lines using ``remark markers'' block
\begin{verbatim}
remark markers:
  !!$
\end{verbatim}
The remark marker is also used to start the item name and item body. 


If you want to intervene the code with the 

If you want block comment, you define ``remark begin markers'' and
``remark end markers''. However, it's worthy to know that Fortran have
no support for block comment. 

\subsection{Configuration file (robodoc.rc)}
\label{sec:conf-file-robod}

ROBODoc can be configured with a configuration file, with the default
name is \verb!robodoc.rc!. ROBODoc searches the your current directory
for the configuration file. If it can't find it there it will search
in \$HOME/, then \$HOMEDRIVE\$HOMEPATH/, and finally in
/usr/share/robodoc/.  You can also tell ROBOdoc to use a configuration
file with different name
\begin{verbatim}
robodoc --rc  htmlsingle.rc
robodoc --rc  rtfsingle.rc
robodoc --rc  htmlmulti.rc
\end{verbatim}

\begin{framed}
  This file lists out the names of items, ignore items, source items,
  frequently used options, headertypes, ignore files, accept files,
  header markers, remark markers, end markers, remark begin markers,
  remark end markers, and translations for English terms.

  Each name, followed by semicolon (``:''), marks the start of a
  block. Inside each block, you can define a number of values, which
  must be preceded by at least one space. 

\end{framed}

\begin{verbatim}
items:
    NAME
    FUNCTION
    SYNOPSIS
    INPUTS
    OUTPUTS
    SIDE EFFECTS
    HISTORY
    BUGS
    EXAMPLE
ignore items:
    HISTORY
    BUGS
item order:
    FUNCTION
    SYNOPSIS
    INPUTS
    OUTPUTS
source items:
    SYNOPSIS
preformatted items:
    INPUTS
    OUTPUTS
format items:
    FUNCTION
    SIDE EFFECTS
options:
    --src ./source
    --doc ./doc
    --html
    --multidoc
    --index
    --tabsize 8
headertypes:
    J  "Projects"          robo_projects    2
    F  "Files"             robo_files       1
    e  "Makefile Entries"  robo_mk_entries
    x  "System Tests"      robo_syst_tests
    q  Queries             robo_queries
ignore files:
    README
    CVS
    *.bak
    *~
    "a test_*"
accept files:
    *.c
    *.h
    *.pl
header markers:
    /****
    #****
remark markers:
    *
    #
end markers:
    ****
    #****
header separate characters:
    ,
header ignore characters:
    [
remark begin markers:
    /*
remark end markers:
    */
source line comments:
    //
keywords:
    if
    do
    while
    for
\end{verbatim}


For example, the default setting for Fortran is
\begin{verbatim}
__****f* ModuleName/Foo
__ FUNCTION
__   Foo computes the foo factor
__   using a fudge factor.
__ SYNOPSIS
int Foo( int fudge )
__ INPUTS
__   fudge -- the fudge factor
__ SOURCE

 more source code..

__****
\end{verbatim}



\subsection{A complete sample in C/C++}
\label{sec:complete-sample-cc++}

\begin{verbatim}
/****f* financial.library/StealMoney
  *  NAME
  *    StealMoney -- Steal money from the Federal Reserve Bank. (V77)
  *  SYNOPSIS
  *    error = StealMoney( userName, amount, destAccount, falseTrail )
  *  FUNCTION
  *    Transfer money from the Federal Reserve Bank into the
  *    specified interest-earning checking account.  No records of
  *    the transaction will be retained.
  *  INPUTS
  *    userName    - name to make the transaction under.  Popular
  *                  favorites include "Ronald Reagan" and
  *                  "Mohamar Quadaffi".
  *    amount      - Number of dollars to transfer (in thousands).
  *    destAccount - A filled-in AccountSpec structure detailing the
  *                  destination account (see financial/accounts.h).
  *                  If NULL, a second Great Depression will be
  *                  triggered.
  *    falseTrail  - If the DA_FALSETRAIL bit is set in the
  *                  destAccount, a falseTrail structure must be
  *                  provided.
  *  RESULT
  *    error - zero for success, else an error code is returned
  *           (see financial/errors.h).
  *  EXAMPLE
  *    Federal regulations prohibit a demonstration of this function.
  *  NOTES
  *    Do not run on Tuesdays!
  *  BUGS
  *    Before V88, this function would occasionally print the
  *    address and home phone number of the caller on local police
  *    976 terminals.  We are confident that this problem has been
  *    resolved.
  *  SEE ALSO
  *    CreateAccountSpec(),security.device/SCMD_DESTROY_EVIDENCE,
  *    financial/misc.h
  ******
  * You can use this space for remarks that should not be included
  * in the documentation.
  */
\end{verbatim}

References:
\begin{itemize}
\item
  \url{http://www.sesp.cse.clrc.ac.uk/Publications/tools_report_2005/node16.html}
\end{itemize}


\section{Doxygen (Doxymacs)}
\label{sec:doxygen-doxymacs}

Doxygen is a mature tool, equivalent to Javadoc. It can be used from its
graphical wizard, from the command line or as part of a make process.
With C++, doxygen supports 2 style of comments.
\begin{verbatim}
/**/ multi-line comment
// until the end of line comment
\end{verbatim}
However, you needs to add more details for Doxygen to recognize, which can be in
different forms
\footnote{\url{http://www.stack.nl/~dimitri/doxygen/manual/docblocks.html\#cppblock}}

\begin{enumerate}
  \item Use two \verb!*! (JavaDoc style) and the intermediate \verb!*! for
  each new line is required as well
\begin{verbatim}
/**
 * ... text ...
 */
\end{verbatim}

  \item Use \verb/*!/ (Qt style)
\begin{verbatim}
/*!
 * ... text ...
 */
 
 /*!
 ... text ...
*/
\end{verbatim}

  \item Use block of at least two \verb!///! or  \verb.//!. We need a blank line
  to ends the documentation block
\begin{verbatim}
///
/// ... text ...
///

//!
//!... text ...
//!
\end{verbatim}
  \item More visible way (with two slashes to the end of )
\begin{verbatim}
/********************************************//**
 *  ... text
 ***********************************************/
 
/////////////////////////////////////////////////
/// ... text ...
///////////////////////////////////////////////// 
\end{verbatim}
  \item 
\end{enumerate}

If you use multi-line commment, to disable code, you cannot use /* and */. So
you need to use
\begin{verbatim}
#if 0 / #endif
\end{verbatim} 


Remember that
\begin{enumerate}
  \item Entities that are members of classes are only documented if their class
  is documented. 
  
  \item Entities declared at namespace scope are only documented if their
  namespace is documented. 
  
  \item Entities declared at file scope are only documented if their file is
  documented. 
  
  So to document functions at file scope, we need this below line in the header
  file in which it is declared
  \begin{verbatim}
  /** @file */
  \end{verbatim}
  or
  \begin{verbatim}
  /*! \file */
  \end{verbatim}
\end{enumerate}



\url{http://www.stack.nl/~dimitri/doxygen/history.html}

\subsection{Tips}
Avoid the temptation to comment too much. Describe what you need, and no more.
Doxygen gives you lots of tags, but you don't have to use them all.
\begin{verbatim}
You don't always need a @brief and a detailed description.
You don't need to put the function name in the comments.
You don't need to comment the function prototype AND implementation.
You don't need the file name at the top of every file.
You don't need a version history in the comments. (You're using a version control tool, right?)
You don't need a "last modified date" or similar.

You should tell the return code, and any exceptions that the function can raise. 
\end{verbatim}

Doxygen work well with C-style code. However, it also support
different programming language (Fortran, ...). Doxygen uses a configuration file
to control its behavior. This file contains entities called tags
\begin{enumerate}
  \item \verb!PROJECT_NAME!:
  \item \verb!INPUT=./src!
  \item \verb!OUTPUT_DIRECTORY!:
  \item \verb!STRIP_FROM_PATH!:
  \item \verb!OPTIMIZE_FOR_FORTRAN=YES!:
  \item \verb!FILE_PATTERNS = *.f95!
  \item \verb!IMAGE_PATH=./images!: location where we put images to be included
  in the source code using \verb!\image! command
  \item \verb!GENERATE_HTML=YES! : generate navigable pages using a browser
  
  We can also export the documentation in LaTeX, Doc, HTML, etc.
\end{enumerate}



Documentation should be added into the header files, using the style
\begin{verbatim}
!> @file
!! This is the information about this file

!> @brief a test module
!------------------------------------------------------------------------------
! NASA/GSFC, Software Integration & Visualization Office, Code 610.3
!------------------------------------------------------------------------------
!
! MODULE: Module Name
!
!> @author
!> Module Author Name and Affiliation
!
! DESCRIPTION: 
!> Brief description of module.
!
! REVISION HISTORY:
! DD Mmm YYYY - Initial Version
! TODO_dd_mmm_yyyy - TODO_describe_appropriate_changes - TODO_name
!------------------------------------------------------------------------------

module MyModule_mod
   
   use AnotherModule_mod
   
   implicit none
   
   public MyModule_type            ! TODO_brief_description
 
   public someFunction             ! TODO_brief_description
   
   ! NOTE_avoid_public_variables_if_possible
   
contains
 
   !---------------------------------------------------------------------------  
   !> @author 
   !> Routine Author Name and Affiliation.
   !
   ! DESCRIPTION: 
   !> Brief description of routine. 
   !> @brief
   !> Flow method (rate of change of position) used by integrator.
   !> Compute \f$ \frac{d\lambda}{dt} , \frac{d\phi}{dt},  \frac{dz}{dt} \f$
   !
   ! REVISION HISTORY:
   ! TODO_dd_mmm_yyyy - TODO_describe_appropriate_changes - TODO_name
   !
   !> @param[in] inParam      
   !> @param[out] outParam      
   !> @return returnValue
   !---------------------------------------------------------------------------  
   
   function someFunction
      use AnotherModule  
      
      real, intent(in) :: inParam        
      real, intent(out) :: outParam       
      real, intent(inout) :: inOutParam   !TODO_description
      real :: returnValue                 
      
      real :: someVariable                !> @var Variable description
 
      ! TODO_insert_code_here
 
   end function someFunction
   
end module MyModule_mod
\end{verbatim}

To run doxygen
\begin{verbatim}
doxygen doxyfile.conf
\end{verbatim}

RULES:
\begin{enumerate}
  \item Comments relevant for Doxygen start with \verb.!>.
  \item The documentation for subroutine/function should be placed above it.
  Paragraphs are separated by a blank line. The first paragraph is used as a
  short description in the overview tables.
  
  \item The tags start with either \verb!\! or \verb!@!. Here are examples
\begin{verbatim}
@author
@authors
@version
@warning = some warning related to the code
@param   = documentation on parameters. The documentation for parameters can be
placed right next to them
   subroutine ...(...)
      logical intent(IN):: test !> what 'test' means here

@bug 
@deprecated
@details
@return
@see   = refers to another entity
@todo      
\end{verbatim}

   \item To add LaTeX code, put them between \verb!\f$! and \verb!\f$!.
   
\begin{verbatim}
\f$ \frac{d\lambda}{dt} , \frac{d\phi}{dt},  \frac{dz}{dt} \f$
\end{verbatim}

\end{enumerate}

IMPORTANT THINGS TO AVOID: 'in-body' comments (comment starting with \verb.!>.
in the body of the subroutine/function) don't seem to work for Fortran.


\begin{enumerate}
  \item Multi-line comment:
  \begin{verbatim}
  /*
   * This is the description
   * for the following function
   */
   void do_this(int argc, char **argv)
  \end{verbatim}
  \item Single-line comment:
\end{enumerate}


References:
\begin{itemize}
\item \url{http://www.stack.nl/~dimitri/doxygen/install.html}
\end{itemize}

\subsection{Doxygen with IDE}

\begin{itemize}
  \item Eclipse IDE: Eclipse CDT has doxygen support (even if awkward) and will
  create empty doxygen comments for you if you request it to
  
  \item Visual Studio: you need to use 

  \item 
\end{itemize}
\subsection{Doxygen in Windows}

We can use \verb!DoxyWizard! tool to create the configuration
file called \verb!doxyfile!. 

There are reports that Doxygen doesn't generate documentation for C function or
C++ global functions (i.e. those not belonging to a class).
\footnote{\url{http://linux.m2osw.com/doxygen-does-not-generate-documentation-my-c-functions-or-any-global-function}}
Doxygen ignores the documentation of global entities (like functions) if 
the file itself which contains them is not documented. This behavior 
comes as a surprise to many. A quote from 
\url{http://www.stack.nl/~dimitri/doxygen/docblocks.html\#structuralcommands}

Question 3: \url{http://www.star.bnl.gov/public/comp/sofi/doxygen/faq.html}
\begin{verbatim}
To document a member of a C++ class, you must also document the class 
itself. The same holds for namespaces. To document a global C function, 
typedef, enum or preprocessor definition you must first document the 
file that contains it (usually this will be a header file, because that 
file contains the information that is exported to other source files). 
\end{verbatim} 

Let's repeat that, because it is often overlooked: to document global 
objects (functions, typedefs, enum, macros, etc), you must document the 
file in which they are defined. In other words, there must at least be a 

\begin{verbatim}
/*! \file */
\end{verbatim} 

or a 

\begin{verbatim}
/** @file */ 
\end{verbatim}

line in this file. 



%%% Local Variables: 
%%% mode: latex
%%% TeX-master: "fortran_manual"
%%% End: 
