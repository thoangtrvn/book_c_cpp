\chapter{Lex + Yacc}

Lex and Yacc was the two of the most popular tools in the early days of Unix
hacker's kit. They are used to construct the tool that can recognize input from
a given grammar, and help to build grammar parser, and semantic analyzer.
Nowadays, they should be replaced with Flex and Bison (Chap.\ref{chap:Bison}).

They were developped since 1970s and thus during the time, many global
variables and magic macros were added and hanging out all over the place.

It becomes a particular problem if you want to run multiple instances of your
generated parser (or, heaven forfend, multiple parsers with different grammars)
in the same binary without having them interfere with each other.

\section{Compiler overview}
\label{sec:compiler-overview}

A compiler has 3 parts, with the flow of information as given:
\begin{enumerate}
  \item front-end

\begin{verbatim}
Lexer/Scanner  ---> Parser   --> Semantic Analyzer
\end{verbatim}
  
  \item middle-end

\begin{verbatim}
Semantic Analyzer ---> 
     Optimizers
\end{verbatim}  
  
  \item back-end
  
\begin{verbatim}
Optimizers --->
      code generator
\end{verbatim}
\end{enumerate}

{\bf Lexer/Scanner}: is the process of converting a sequence of characters into
a sequence of tokens. \textcolor{red}{Tools that help to generate a
lexer/scanner:} Lex (Sect.\ref{sec:Lex}) or Flex (Sect.\ref{sec:Flex}).

Example: input
\begin{verbatim}
foo = 1 - 3**2
\end{verbatim}
converted into
\begin{verbatim}
Lexeme              Token type
-------------------------------
foo                 Variable
=                   Assignment Operator
1                   Number
-                   Subtraction Operator
3                   Number
**                  Power Operator
2                   Number
\end{verbatim}


{\bf Parser }: The parser will generate a tree of grammar structure; 
identify syntax error during this stage. To help generate a parser, we can use
tools like Yacc (Sect.\ref{sec:Yacc}) or Bison (Sect.\ref{sec:bison}).

{\bf Semantic Analyzer}:  performing semantic checks, e.g. type checking, object
binding, etc. 

{\bf Optimizer}: generate intermediate codes, make program run fasters


{\bf Code generator}: convert intermediate code into machine-dependent code

\section{Lex}
\label{sec:Lex}

Lex was written by Mike Lesk and Eric Schmidt
(the Google guy). It is the tool that enables you to develop/generate a lexical
analyzer. 

It is no longer used, and should be replaced by Flex (Sect.\ref{sec:Flex}).


\chapter{Flex + Bison}
\label{chap:Bison}
\label{chap:Flex}

\url{http://www.cs.uic.edu/~spopuri/cparser.html}

Flex and bison are code-generating tools designed to aid in compiler development
(Sect.\ref{sec:compiler-overview}. Flex is a generator-purpose lexer generator
Bison is a general purpose parser generator. Both virually guaranteed to be
faster than anything you could write manually in a reasonable amount of time.
Also, updating and fixing Flex and Bison source files is a lot easier than
updating and fixing custom parser code. Flex and Bison have mechanisms for error
handling and recovery, which is something you definitely don't want to try to
bolt onto a custom parser. Flex and Bison have been around for a long time, so
they far freer from bugs than newer code.

User needs to define an annotated context-free grammar
(Sect.\ref{sec:bison-grammar-file}), then \verb!bison! utility can help
generating a parser: a GLR parser or LALR(1) parser, that can parse an text
input written in that language.

The generated parser is in C or C++ code, that can be compiled easily into a
binary program or library.

\section{Flex}
\label{sec:Flex}

Flex is {\bf f}ast lexical analyzer generator (Sect.\ref{sec:Lex}). 

Stage 1 done by {\bf Flex} (fast lexical analyzer): we use Flex to
define the grammar.

\begin{verbatim}
*.l ---[flex]---> produce a file containing C/C++ source code 
                       (lex.yy.c or lex.yy.C)
\end{verbatim}

The file \verb!lex.yy.c! contains \verb!yylex()! created by flex (Sect.\ref{sec:yylex}).
The syntax of \verb!.l! specification file is discussed in
Sect.\ref{sec:flex_grammar-file}.
  
NOTE: user-code can also be added here too.

\subsection{Flex version change from 2.5.4 to above}

\verb!FlexLexer! class since Flex 2.5.5 has a few incompatible changes
\begin{verbatim}
FlexLexer.h defines 2 classes: FlexLexer and yyFlexLexer

// the abstract class that provide external (C++ interfaces) provided
//  to flexx C++ lexer objects
class FlexLexer{

  virtual void yyrestart(std::istream* s) = 0; //existing - using pointer
  virtual void yyrestart(std::istream& s) = 0; // new - using reference
  
  // new
  // support using reference, besides the traditional using pointer
  int yylex(std::istream& new_in, std::ostream& new_out)
  {
     switch_streams(new_in, new_out);
     return yylex();
   }
}


class yyFlexLexer: public FlexLexer {
  // a specific lexer class that we can use
  //  or define a subclass of this class
  
  // new 
  // support reference arguments
  yyFlexLexer(std::istream& arg_yyin, std::ostream& arg_yyout);
  
  // new
  yy_buffer_state* yy_create_buffer(std::istream& s, int size);
  
  void yyrestart(std::istream& s);
  
  // new (using stack-based structure)
  //    push and pop
  void yypush_buffer_state(yy_buffer_state* new_buffer);
  void yypop_buffer_state();
  
  // new
  virtual void switch_streams(std::istream* new_in = 0,
                                std::ostream* new_out = 0);
  virtual int yywrap();
  
  
  private:
    void ctor_common();
  
  // removed (obsolete) and is replacec by a stack structure
  protected:
    //struct yy_buffer_state* yy_current_buffer;   
    
  //instead
  //  a stack structure is used
   size_t yy_buffer_stack_top;        /**< index of top of stack. */
   size_t yy_buffer_stack_max;        /**< capacity of stack. */
   yy_buffer_state** yy_buffer_stack; /**< Stack as an array. */
   void yyensure_buffer_stack(void);
    
}

\end{verbatim}

NOTE:
\begin{verbatim}
// to create multiple lexer classes, we use -P flag
//    to rename each yyFlexLexer class to some other name xxFlexLexer
//    see below, e.g. MdlLexer.h file

In this header file, it is important to include the FlexLexer.h only once

 #if !defined(yyFlexLexerOnce)
    #include <FlexLexer.h>
 #endif
\end{verbatim}


\subsection{-- changed required for a user-defined xxxLexer class}


MdlLexer.h: if we write a user-defined class that derive from \verb!yyFlexLexer! class (Sect.\ref{sec:yyFlexLexer})
\begin{verbatim}
 #if !defined(yyFlexLexerOnce)
    #include <FlexLexer.h>
 #endif

class MdlLexer : public yyFlexLexer
{

}
\end{verbatim}

LensLexer.h: if we write a user-defined class that derive from \verb!yyFlexLexer! class (Sect.\ref{sec:yyFlexLexer})
\begin{verbatim}
 #if !defined(yyFlexLexerOnce)
    #include <FlexLexer.h>
 #endif

class LensLexer : public yyFlexLexer
{

}
\end{verbatim}

also, we need to use option 
\begin{verbatim}
%option c++
%option yyclass='LensLexer'
%option noyywrap
\end{verbatim}
inside 

\section{Use C++ with Flex and Bison}
\label{sec:example-Flex-Bison-C++}

Bison and Flex were written in C, but more important their output is C code.
However, what if we want to generate C++ with Flex and Bison!

Example: we want it to parse a source file with 4 portions
\begin{itemize}
  \item a header with just say 'sNaZZle' and version number
  \item a section with a series of statements declaring type, e.g. type foo
  \item a section with a series of statements containing actual data
  \item a section with only a single statement (as a footer), \verb!end!
\end{itemize}
, e.g. the example
code
\begin{verbatim}
sNaZZle 1.3
type foo
type bar
type bas
0 0 10 5        foo
20 10 30 20     foo
7 8 12 13       bas
78 124 100 256  bar
end
\end{verbatim}
check \url{http://aquamentus.com/flex_bison.html}

As a first stage, you need to define the grammar and save in a \verb!.l! file -
this file is readable by Flex - Sect.\ref{sec:flex_grammar-file}. The generated
output file is \verb!lex.yy.c! (Sect.\ref{sec:lex.yy.c})

\section{Flex grammar file or scanner specification (*.l): output is tokens}
\label{sec:flex_grammar-file}

The input file for flex is called a "scanner specification" and it is
conventional for it to have an ``.l'' suffix, e.g. \verb!snazzle.l! file for the
language described in Sect.\ref{sec:example-Flex-Bison-C++}

The grammer file need to be converted into tokens for use by Bison
(Sect.\ref{sec:connect-scanner-parser}).
\verb!flex! is the tool to convert the below file. \verb!flex! is called 
lexer, tokenizer, scanner. This phase is called lexical analysis.

Once having tokens, the parser (created by Bison) can analyze the correctness of
any input file written in the given grammar.

The input file has 3 sections, separated by \verb!%%!, i.e. we have two
\verb!%%! lines: first section is 'definitions', second section is 'rules', and
the third seciton is verbatim code, i.e. user-code that get copied directly to
the generated C/C++ source output.

\begin{verbatim}
definitions
%%
rules
%%
user code
\end{verbatim}

\subsection{definition section}
\label{sec:flex_grammar-definitions-section}

{\bf IMPORTANT}: In the definitions section (but not in the rules section -
Sect.\ref{sec:flex_grammar-rules-section}), an unindented comment (i.e., a line
beginning with "/*") is also copied verbatim to the output up to the next "*/".

\begin{verbatim}
DIGIT    [0-9]
ID       [a-z][a-z0-9]*
\end{verbatim}

In the definitions and rules sections, any indented text or text enclosed in
\verb!%{! and \verb!%}! is copied verbatim to the output (with the \verb!%{}!'s
removed). The \verb!%{}!'s must appear unindented on lines by themselves.

\url{http://dinosaur.compilertools.net/flex/flex_6.html}

\subsection{rules section}
\label{sec:flex_grammar-rules-section}

 The rules section of the flex input contains a series of rules of the form:
\begin{verbatim}
pattern   action
\end{verbatim}
with \verb!pattern! must be unindented, and the action must be on the same line.

Example: \verb!*.l! file with the rules section
\begin{verbatim}
%%
L : L ';' E
  | E
;
E : E ',' P
  | P
;
P : 'a'
  | (M)
;
M : /* nothing */
  | L
;
%%
\end{verbatim}

In the definitions and rules sections, any indented text or text enclosed in
\verb!%{! and \verb!%}! is copied verbatim to the output (with the \verb!%{}!'s
removed). The \verb!%{}!'s must appear unindented on lines by themselves.


In the rules section, any indented or \verb!%{}! text appearing before the first
rule may be used to declare variables which are local to the scanning routine
and (after the declarations) code which is to be executed whenever the scanning
routine is entered

\subsection{user-code section}
\label{sec:flex_grammar-user-code-section}

The user code section is simply being copied to `lex.yy.c' generated output
file (Sect.\ref{sec:lex.yy.c}).

Here, you put any function that you want to be put into the generated
\verb!lex.yy.c! (or .cc) file, that can be used in the Bison *.y file.




\section{Bison}
\label{sec:bison}

Stage 2 done by {\bf Bison}: we use Bison to develop language parsers -
  Sect.\ref{sec:bison-grammar-file}


\begin{verbatim}
*.y ---[bison]--> produce a file containing C/C++ source code
\end{verbatim}  
which contains \verb!yyparse()! function created by bison.

This function will in turn call yylex() created by flex from your scanner
specification. It groups the words into sentences, the sentences into paragraphs
and the paragraphs into sections; and enables users to define specific behavior,
i.e. what function to called; upon reading a particular 
sentence/paragraph/section.


\subsection{yytname: token tables}

There are three ways of writing terminal symbols in the grammar
\begin{enumerate}
  \item a named token, using ALL UPPERCASE
  
\begin{verbatim}
Each such name must be defined with a Bison declaration such as %token
\end{verbatim}
  
  \item A character token type (or literal character token) is written in the
  grammar using the same syntax used in C for character constants; for example,
  '+' is a character token type.
  
  Thus, the token type '+' is used to represent the character `+' as a token
  
  \url{http://dinosaur.compilertools.net/bison/bison_6.html}
  
  \item A literal string token is written like a C string constant; for example, "<=" is a literal string token.
  
\end{enumerate}

Since Bison 2.6, Bison was  changed to use \verb!YY_NULL! instead of NULL to end
its yytname list


\begin{verbatim}
--token-table
	      This switch causes the name.tab.c output to include
	      a list of token names in order by their token  num-
	      bers;   this is defined in the array yytname.  Also
	      generated are #defines for YYNTOKENS, YYNNTS,  YYN-
	      RULES, and YYNSTATES.
\end{verbatim}

\subsection{Bison version change}

\subsection{-- bison 2.4,1}
\label{sec:copyright-bison-2.4}

The generated file, e.g. speclang.tab.C has copyright
\begin{verbatim}
A Bison parser, made by GNU Bison 2.4.1
Copyright (C) 1984, 1989, 1990, 2000, 2001, 2002, 2003, 2004, 2005, 2006
\end{verbatim}

\subsection{-- bison 2.7}
\label{sec:copyright-bison-2.7}

The generated file, e.g. speclang.tab.C has copyright
\begin{verbatim}
Bison implementation for Yacc-like parsers in C

Copyright (C) 1984, 1989-1990, 2000-2012
\end{verbatim}

\url{https://compilers.iecc.com/comparch/article/12-12-006}

\subsection{-- bison 2.6}

Bison was recently changed
to use \verb!YY_NULL! instead of NULL (or 0) to end its yytname list


\subsection{-- bison 2.5.x}

\url{https://lists.gnu.org/archive/html/info-gnu/2012-06/msg00004.html}

Fix bugs:
\begin{enumerate}
  \item  YYBACKUP works as expected.
  
  \item when -std is passed to GCC, i.e. \verb!__STRICT_ANSI__! is defined, then 
  \verb!__attribute__! is no longer disabled (previous Bison versions disable this)
  
  
  \item support C++11: 
  
\begin{verbatim}
C and C++ parsers use "nullptr" instead of "0" when __cplusplus is 201103L
  or higher.
\end{verbatim}
\end{enumerate}

\subsection{-- bison 3.x}

This major release  drop support for generating parsers in K\&R C. Also
\begin{enumerate}
  \item no longer use \verb!%pure_parser! but use Sect.\ref{sec:reentrant-parser}
  
 NOTE: Use \verb!%pure-parser! to be compatible with Bison 2.0 and 3.0; though for future-proof, it should be something else. 
  
  \item no longer use \verb!YYPARSE_PARAM! or \verb!YYLEX_PARAM! and \verb!YYLEX! macro names;
  
   as \verb!yylex! function is called directly, if we need to pass extra
   argument, use \verb!%lex-param! and \verb!%parse-param! option in speclang.y
   file. This may requires modifying yyerror as well.
   
\end{enumerate}

Bison will start using some of the C99 features for its own code, especially the
definition of variables after statements. The generated C parsers still aim at C90.
  

\subsection{-- bison 3.0.2 (2013)}

\url{https://gist.github.com/dajobe/6115089}

\subsection{-- bison 3.0.3 }

\subsection{-- bison 3.0.4 (2015)}

This is a long overdue bug fix release
\begin{enumerate}
  \item support named \verb!%union!

\begin{verbatim}
%union foo {int ival;};

\end{verbatim}
This is for Yacc-compability, but its usefullness is not clear. This is however, not required for POSIX YACC.

  \item Fix bug (broken C parser) when
  
 \verb!%defines! is used together with \verb!%define api.value.type union!
 
 \begin{verbatim}
 %defines
 
 %define api.value.type union
 \end{verbatim}
 
\end{enumerate}

\subsection{-- bison 3.0.5 (2018)}

Documents: \url{https://www.gnu.org/software/bison/manual/html_node/index.html}

\section{Bison grammar file or parser specification (*.y): output is behavior}
\label{sec:bison-grammar-file}

Bison is a general-purpose parser generator that converts a grammar, e.g. an
annotated context-free grammar into a deterministic LR or generalized LR (GLR)
parser employing LALR(1) parser tables. 

NOTE: Bison can also generate IELR(1) or canonical LR(1) parser tables; but it
is at experimental stages. It is recommended to read Sect.\ref{sec:compiler-overview}.

Suppose you want to compile C++ code (i.e. input file whose content follows C++
specification language), then you need to write a \verb!speclang.y! file (or any file with \verb!.y! extension)

\begin{verbatim}
%{
  Prologue
%}

Bison declarations

%%
Grammar rules
%%

Epilogue
\end{verbatim}

This Bison specification file has three sections (separated by lines containing
\verb!%%!). So, there are two \verb!%%! in the file
\begin{enumerate}
  
  \item A Prologue (enclosed within \verb!%{! and \verb!%}!) precedes a Bison
  declaration section
  
  \item Grammar rule section
  
  \item Epilogue (or Additional C code) section
\end{enumerate}


\subsection{-- first section}
\label{sec:bison-grammar-file-first-section}

\textcolor{red}{The first section} of a bison file (from the start of the file
to the first \verb!%%!) is called the Definitions Section.
It contains two parts: 
\begin{enumerate}
  \item a Prologue, i.e. the C code  put between \verb!%{! and
\verb!%}! which is copied directly to the output file uninterpreted. Such code
forms the "C declarations" section of the input file.

  \item Bison declarations
\end{enumerate}
You may have more than one Prologue section, intermixed with the Bison declarations. 
When in doubt, it is usually safer to put prologue code before all Bison declarations, rather than after. 

\url{https://www.gnu.org/software/bison/manual/html_node/Grammar-Outline.html}


\subsection{---- Prologue}

This section provides information that copied (without no modification) to the
C++ generated file, and is placed at the top of the resulting source file or
which is needed by bison to create the parser.

{\tiny
\begin{verbatim}
%{
 // any thing in between %{ %}
 // will be added to the output C/C++ generated source file
 
 #include "some-header.h"
 
 // other choices (to add an additional argument to yyparse function)
 // Bison 2.x only
 #define YYPARSE_PARAM parm
 
 // other choices (to add an additional argument to yylex function)
 // Bison 2.x only
 #define YYLEX_PARAM parm
 
 // if we want to write a custom yyparse function, map the call to yyparse 
 //  to this function
 #define yyparse lensparse
%}
\end{verbatim}  
}

If you have more than one Prologue, i.e. intermixed with Bison declaration, one
way to explicitly tell which block should reside at the top of the generated
file, is by usingg \verb!%code top! directive (for the block to reside at the
beginning of the file) and \verb!%code! unqualified directive (for the other
blocks)
\begin{verbatim}
%code top {
  // C / C++ codes
  
}
\end{verbatim}

IMPORTANT RULES:
\begin{itemize}
  \item lines begining with white-spaces will be copied verbatim into the
  resulting generated C/C++ source file near the top.
  
  you should restrict yourself to comments on such lines because there is no
  guarantee about their exact position in the resulting source file, e.g.
  speclang.tab.C if Bison input file is speclang.y file
  
  \item lines contained between a \verb!%{! and a \verb!%}! are also copied
  verbatim into the generated C/C++ source file.   
\end{itemize}

\subsection{---- Bison declaration}

This contains options that tell Bison how to generate code in speclang.tab.C file

{\tiny
\begin{verbatim}
%pure_parser
%locations
%lex-param parm
%parse-param {void* parm}

%union {
  double V_double;
  int V_int;
  std::string * P_string;
}

%token <V_double> DOUBLE_CONSTANT
%token POST
%token THREADS

 /* types for non-terminal */
%type <P_phase> sharedPhase

%left OR AND EQUAL

%start mdlFile
  /* mdlFile is the root or the first non-terminal rule
     to be considered
  */
%%
\end{verbatim}  
}

IMPORTANT RULES:
\begin{itemize}
  \item lines begining with white-spaces will be copied verbatim into the
  resulting generated C/C++ source file near the top.
  
  you should restrict yourself to comments on such lines because there is no
  guarantee about their exact position in the resulting source file, e.g.
  speclang.tab.C if Bison input file is speclang.y file
  
  \item The other lines beginning with a \% are information that bison needs to
   construct the parser.
   
{\tiny
\begin{verbatim}
%token SOMETOKEN  = list the tokens that the scanner returns to the parser
  conventional to name these with all uppercase letters (I add the T_ prefix as
  well)
  the scanner function yylex(), should use these tokens as return values

%start prog_name  = tells the parser that prog_name is the first rule 
                    to look at (the start-start non-terminal)
   Non-terminals are conventionally written in lower case
   
\end{verbatim}   
}
   
   \item you CANNOT use  automatic storage for C++ \verb!std::string! in
   \verb!%union!  (or any other string class with non-trivial constructor), you
   need to use dynamic (heap)  \verb!std::string *! (string pointer).
\begin{verbatim}
%union {
    std::string *str;
    double V_double;
    C_edgeset_extension* P_edgeset_extension;
}
\end{verbatim}   

The \verb!%union! section declares all data types that you need to use for
semantic values. The data types are given with names, e.g. \verb!str! and
\verb!V_double!; and these names are used in \verb!%type! and
\verb!%token! for the terminal and non-terminal symbols.
\begin{verbatim}
%type <P_edgeset_extension> edgeset_extension
\end{verbatim}

Example:
\begin{verbatim}
edgeset_extension: DOT TYPE '(' declarator ')' {
  SyntaxError* localError = 
       new SyntaxError(CURRENTFILE,
                       @1.first_line,
                       "EdgeSet Extension",
                       ". Type ( Declarator )");
  $$ = new C_edgeset_extension($4, localError);
}
\end{verbatim}
 
You will need to change all of the uses of \verb!yylval->str! or \verb!$$!,
\verb!$1!, etc.where \verb!$N %type! is to use dynamically allocated strings.
So, instead of \verb!$$ = "success";!, we use
\begin{verbatim}
some_terminal: RULE {
   $$ = new std::string("success");
}
\end{verbatim}  
\url{http://stackoverflow.com/questions/24925776/whats-wrong-with-these-yacc-and-lex-file-using-c-string}

   \item \verb!%locations!: 
   
   
   \item \verb!%lex-param! : the value to it is the name of the argument passed to yylex when being called inside yyparse
   
   
   \item \verb!%parse-param!: the list of values to it are pasted exactly to
   arguments (from second argument) to yyparse() and yyerror()
   
\end{itemize}

\subsection{-- second section}
\label{sec:bison-grammar-file-second-section}

\textcolor{red}{The second section} of a bison file (from the start of
the first \verb!%%! to the second \verb!%%!) is called the Rules Section.

This section describes the non-terminals in the grammar (those things which are
built from tokens and other non-terminals). The rule has the form 

{\tiny
\begin{verbatim}
a non-terminal ":"  sequence-of-patterns-separated-by-vertical-bars(|) 
   ;  /* end with a semi-colon */
\end{verbatim}
}

After each pattern, there is an optional to insert the block of C/C++ code by
enclosing them between \verb!{! and \verb!}!.

Example:
\begin{verbatim}
%%

mdlFile:  parserLineList {

}
;

parserLineList: parserLine {

}
| parserLineList parserLine {

}
;

parserLine: struct {
   HIGH_LEVEL_EXECUTE(parm, $1);
}
| interface {
   HIGH_LEVEL_EXECUTE(parm, $1);
}
| edge {
   HIGH_LEVEL_EXECUTE(parm, $1);
}
| node {
   HIGH_LEVEL_EXECUTE(parm, $1);
}
| variable {
   HIGH_LEVEL_EXECUTE(parm, $1);
}
| constant {
   HIGH_LEVEL_EXECUTE(parm, $1);
}
| functor {
   HIGH_LEVEL_EXECUTE(parm, $1);
}
%%
\end{verbatim}
  
\subsection{-- third section}
\label{sec:bison-grammar-file-third-section}

\textcolor{red}{The third section}

\subsection{tokens recognizable by BISON}

Bison 
\url{http://dinosaur.compilertools.net/bison/bison_13.html}


\subsection{running bison}

\begin{verbatim}
bison -v -d lang.y
\end{verbatim}

\url{http://www.di-mgt.com.au/converting_from_lex_and_yacc.html}

\subsection{FlexLexer.h}

There are many changes in FlexLexer.h file at different versions of flex.
So, make sure you use the right one.

The generated scanner uses the line:
\begin{verbatim}
#include <FlexLexer.h>

\end{verbatim}
which means that the FlexLexer.h file will be searched for in system include directories. 
\url{http://stackoverflow.com/questions/25758776/fatal-error-flexlexer-h-no-such-file-or-directory}

\url{https://github.com/bingmann/flex-bison-cpp-example/blob/master/src/FlexLexer.h}

\url{https://panthema.net/2007/flex-bison-cpp-example/flex-bison-cpp-example-0.1.4/doxygen-html/classExampleFlexLexer.html}

\section{Scanner}

Here, we discusses different scanner functions as generated, e.g. in speclang.tab.C from speclang.y file (Sect.\ref{sec:bison-grammar-file})
\begin{itemize}
  \item yylex - Sect.\ref{sec:yylex}
  
  \item yyparse - Sect.\ref{sec:yyparse}
\end{itemize}

The generated file, speclang.tab.C has 
\begin{enumerate}
  \item  The first part is copyright information (Sect.\ref{sec:copyright-bison-2.4}, Sect.\ref{sec:copyright-bison-2.7})

  \item The second part: macros names
  
\begin{verbatim}
#define YYBISON  1 


#define YYBISON_VERSION "2.4.1"

#define YYSKELETON_NAME "yacc.c"

// using %pure-parser or not?
#define YYPURE 1

// push parser?
#define YYPUSH 0

// pull parser?
#define YYPULL 1


// until bison 2.4.1
// ... using location?
#define YYLSP_NEEDED 1

// until bison 2.4.1
//   enabling token table?
#ifndef YYTOKEN_TABLE
 #define YYTOKEN_TABLE 0
#endif
\end{verbatim}

Since Bison 2.6, \verb!YY_NULL! is used to end the yytname table 
\begin{verbatim}
#ifndef YY_NULL
#  if defined __cplusplus && 201103L <= __cplusplus  
#   define YY_NULL nullptr 
#else
# define YY_NULL 0
#endif
#endif
\end{verbatim}


From Bison 2.6 to 2.9
\begin{verbatim}
// this section will be replaced by 
//  #include "speclang.tab.h"
#ifndef YY_YY_SPECLANG_TAB_H_INCLUDED
  # define YY_YY_SPECLANG_TAB_H_INCLUDED
  /* Enabling traces.  */
  #ifndef YYDEBUG
  # define YYDEBUG 0
#endif
\end{verbatim}

   \item YYDEFACT[STATE-NAME]:  
   
The array \verb!yydefact[]! define default reduction number in state STATE-NUM, i.e.
what to perform when YYTABLE doesn't specify something else to do


   \item \verb!YYTABLE[YYPACT[STATE-NUM]]!:
   
The array \verb!yytable[]! define what to do in state STATE-NUM. 
\begin{itemize}
  \item if positive value: shift that token
  
  \item if negative value: reduce the rule which number is opposite
  
  \item (until Bison 2.4.1): if zero, do what YYDEFACT says

Since Bison 2.6, it uses two macros
\begin{verbatim}
#define yypact+value_is_default(Yystate) \
   (!!((Yystate) == (-656)))
   
#define yytable_value_is_error(Yytable_value) \
   YYID (0)
\end{verbatim}
  
  \item if \verb!YYTABLE_NINF!, error
  
\begin{verbatim}
#define YYTABLE_NINF -226
\end{verbatim}
\end{itemize}

Example: bison 2.4.1
\begin{verbatim}
// in Bison 2.4.1
yyn = yypact[yystate];
if (yyn == YYPACT_NINF)
  goto yydefault;


yyn = yytable[yyn];
if (yyn <= 0)
{
  if (yyn == 0 || yyn == YYTABLE_NINF)
     goto yyerrlab;
  yyn = -yyn;
  goto yyreduce;
}
\end{verbatim}

Example: bison 2.7
\begin{verbatim}
if (yypact_value_is_default (yyn))
  goto yydefault;
  
yyn = yytable[yyn];
if (yyn <= 0)
{
  if (yytable_value_is_error (yyn))
     goto yydefault;
  yyn = -yyn;
  goto yyreduce;
}
\end{verbatim}
\end{enumerate}




\subsection{yylex(): non-reentrant vs reentrant}
\label{sec:yylex}

The lexical analyzer function, \verb!yylex!, recognizes tokens from the input stream
and returns them to the parser. 

Bison does not create this function automatically; you must write it so that
yyparse can call it - Sect.\ref{sec:yyparse}. The function \verb!yylex()! is
sometimes referred to as a lexical scanner. To write it, we just need to map the
name \verb!yylex! to the real lexical scanner we write, e.g. \verb!lenslex!


By default, \verb!yylex! has minimal 2 arguments, to use a third argument, we define the name for the macro 
\verb!YYLEX_PARAM!, as this function is called inside \verb!yyparse! the name passed to \verb!YYLEX_PARAM! should be the name of the 
variable inside \verb!yyparse!.

\begin{verbatim}
// in Bison 2.4.1 and earlier, the call to yylex is encapsulated via the macro name YYLEX
#define YYLEX_PARAM parm
#ifdef YYLEX_PARAM
  # define YYLEX yylex (&yylval, &yylloc, YYLEX_PARAM)
#else
  # define YYLEX yylex (&yylval, &yylloc)
#endif


// in Bison 2.7 and earlier, the call to yylex is encapsulated via the macro name YYLEX
// if inside speclang.y we have
//  %lex-param parm
//    then a third argument with name 'parm' is always present
#ifdef YYLEX_PARAM
  # define YYLEX yylex (&yylval, &yylloc, YYLEX_PARAM)
#else
  # define YYLEX yylex (&yylval, &yylloc, parm)
#endif

\end{verbatim}

\subsection{-- example 1 - user-defined lenslex}


Example: in the first section of bison specification file - Sect.\ref{sec:bison-grammar-file-first-section}
\begin{verbatim}
#define yylex lenslex
#define YYLEX_PARAM parm

int lenslex(YYSTYPE *lvalp, YYLTYPE *locp, void *context);

inline int lenslex(YYSTYPE *lvalp, YYLTYPE *locp, void *context)
{
  // must return the numeric code for the type of token 
  //   it has just found, or 0 for end-of-input
  //
  return ((LensContext *) context)->
         lexer->lex(lvalp, locp, (LensContext *) context);
}

\end{verbatim}


\subsection{-- example 2 - simple}

Example:
If you use flex to generate C code, then it uses
\begin{verbatim}
// no argument
int yylex();
\end{verbatim}
 a function without specifying anything about its parameter list.
NOTE: (int yylex(void) would declare it to have no parameters.

Each time it is called, it accesses to the input stream of data using
global communication variables \verb!yylval! and \verb!yylloc!
\begin{verbatim}
yylex ()
{
  ...
  if (c == EOF)     /* Detect end of file. */
    return 0;
  ...
  if (c == '+' || c == '-')
    return c;      /* Assume token type for `+' is '+'. */
  ...
  return INT;      /* Return the type of the token. */
  ...
}

\end{verbatim}


The function \verb!yylex()! takes input (a file name or a sequence of
characters) and break them apart into a series of tokens (words); tokens are
passed to the parser - \verb!yyparse()! which is another function generated by
Bison (Sect.\ref{sec:bison}).

\begin{mdframed}
 
The lexical analyzer function, \verb!yylex!, recognizes tokens from the input
stream and returns them to the parser.
Bison does not create this function automatically; you must write it so that
yyparse can call it. So, \verb!yylex! can be the result of using Flex
(Sect.\ref{sec:Flex}) to create a lexical scanner.
\begin{itemize}

  \item  In simple programs, yylex is often defined at the end of the Bison grammar file. 

  \item If yylex is defined in a separate source file, you need to arrange for the
token-type macro definitions to be available there. To do this, use the `-d'
option when you run Bison, so that it will write these macro definitions into a
separate header file `name.tab.h' (or speclang.tab.h) which you can include in
the other source files that need it.

\url{http://www.iro.umontreal.ca/~fourniep/bison/bison_7.html}

\end{itemize}

\end{mdframed}

\subsection{yyparse(): non-reentrant vs reentrant}

When calling bison and passing the bison specification file, e.g. speclang.y -
Sect.\ref{sec:bison-grammar-file}, it generates the file, with the same name,
and extension \verb!.tab.c!, e.g. \verb!speclang.tab.c!.

\begin{verbatim}
bison -v -d speclang.y
\end{verbatim}

Inside \verb!speclang.tab.c!, it generates \verb!yyparse! function. By default,
this is a non-argument function. However, user can define a \verb!void*!
argument to \verb!yyparse! by defining the macro \verb!YYPARSE_PARAM! in the
first section - Sect.\ref{sec:bison-grammar-file-first-section}

\begin{verbatim}
// in C++
int yyparse (void);

// in C
int yyparse();
\end{verbatim}

Example: the generated code (speclang.tab.c)  has a section like this
\begin{verbatim}
    /* Prevent warnings from -Wmissing-prototypes.  */
    
#ifdef YYPARSE_PARAM
   #if defined __STDC__ || defined __cplusplus
       int yyparse (void *YYPARSE_PARAM);
   #else
       int yyparse ();
   #endif
#else /* ! YYPARSE_PARAM */
       #if defined __STDC__ || defined __cplusplus
       int yyparse (void);
       #else
       int yyparse ();
   #endif
#endif /* ! YYPARSE_PARAM */
\end{verbatim}
and a section for the body of the \verb!yyparse! function
\begin{verbatim}
// in Bison 2.4.1 and earlier, the call to yylex is encapsulated via the macro name YYLEX

#ifdef YYPARSE_PARAM
   #if defined __STDC__ || defined __cplusplus
       int yyparse (void *YYPARSE_PARAM)
   #else
       int yyparse ()
   #endif
#else /* ! YYPARSE_PARAM */
       #if defined __STDC__ || defined __cplusplus
       int yyparse (void)
       #else
       int yyparse ()
   #endif
#endif /* ! YYPARSE_PARAM */
{

  //... 
  
  if (yychar == YYEMPTY)
  {
    YYDPRINTF((stderr, "Reading a token: "));
    
    yychar = YYLEX;
  }
  
  // ...
}

\end{verbatim}

In Bison 2.x, yyparse() call yylex() via the macro name \verb!YYLEX! (Sect.\ref{sec:yylex}).
However, in Bison 3.0, it calls directly \verb!yylex!. This can be a problem, if you define a yylex with 3 arguments. 
\begin{verbatim}
  if (yychar == YYEMPTY)
  {
    YYDPRINTF((stderr, "Reading a token: "));
    
    yychar = yylex(&yylval, &yylloc);
  }
\end{verbatim}

To resolve that issue, enforce passing a third argument with a given name by defining \verb!%lex-param! in speclang.y
\begin{verbatim}
%lex-param parm
\end{verbatim}
then the generated speclang.tab.C will has
\begin{verbatim}
  if (yychar == YYEMPTY)
  {
    YYDPRINTF((stderr, "Reading a token: "));
    
     yychar = yylex(&yylval, &yylloc, parm);
  }
\end{verbatim}


\begin{mdframed}

In C, a function with no argument means 'unknown' parameter list 
\begin{verbatim}
void myfunc(); // unknown parameter list
void myfunc(void); // no parameter
\end{verbatim}

C++ language grammar does not allow function with unknown parameter list, i.e.
you need to provide the precise parameter list.
\end{mdframed}


\subsection{yyerror}
\label{sec:yyerror}

The Bison parser detects a parse error or syntax error whenever it reads a token which cannot satisfy any syntax rule. 

The macro YYERROR 

The Bison parser expects to report the error by calling an error reporting function named yyerror, which you must supply.
It is called by yyparse whenever a syntax error is found, and it receives one argument, which is \verb!char*! argument.

\begin{verbatim}
yyerror (s)
     char *s;
{
  fprintf (stderr, "%s\n", s);
}

\end{verbatim}

However, if speclang.y has
\begin{verbatim}
'%locations %define api.pure' 
\end{verbatim}
then the generated yyerror
\begin{verbatim}
 void yyerror (char const *msg);                 /* Yacc parsers.  */
 
 void yyerror (YYLTYPE *locp, char const *msg);  /* GLR parsers.   */
\end{verbatim}


NOTE: If you need to track line error, you can access to the line using yylloc (YYLTYPE struct):

\begin{verbatim}
#define YY_USER_ACTION {yylloc.first_line = yylineno; \
        yylloc.first_column = colnum;                 \
        colnum=colnum+yyleng;                         \
        yylloc.last_column=colnum;                    \
        yylloc.last_line = yylineno;}


void yyerror (const char *s)
{

        fprintf(stderr, "%s line:%d", s,yylloc.first_line);
}

\end{verbatim}
\url{https://lists.gnu.org/archive/html/help-bison/2010-01/msg00022.html}


If we also use \verb!%parse-param! (Sect.\ref{sec:parse-param-option-bison})

\subsection{non-reentrant parser: yylex(), yyparse()}
\label{sec:yyparse}

For reentrant parser, check Sect.\ref{sec:yyparse-C++}.

For a non-reentrant parser, the Bison parser is actually a C function named
\verb!yyparse!. This functions also uses many C identifiers starting with
\verb!yy!, e.g. yylex.



\begin{mdframed}

We call the function yyparse to cause parsing to occur.
This function reads tokens, executes actions, and ultimately returns when it
encounters end-of-input or an unrecoverable syntax error.
The return value is 0 if parsing was successful; otherwise a non-zero value
indicating the type of error, e.g. 1.

You can also write an action which directs yyparse to return immediately without reading further, by calling either
\begin{verbatim}
YYACCEPT
    Return immediately with value 0 (to report success).
YYABORT
    Return immediately with value 1 (to report failure).
\end{verbatim}

\end{mdframed}

\subsection{C++ : yylex, yyparse}
\label{sec:yyparse-C++}
\label{sec:yylex-C++}

For C non-reentrant parser, check Sect.\ref{sec:yyparse-C++}.
Here, we discuss using C++ reentrant parser (i.e. thread-safe) where global
communication variables \verb!yylval! and \verb!yylloc! cannot be used, as it
does not maintain the property of a reentrant function
(Sect.\ref{sec:reentrant-parser}).


You can add the parameters to yylex(), and yyparse() based on the following
order
\begin{enumerate}
  
  \item YYSTYPE* first (if \verb!%define api.pure full! is used)
  
  \item YYLTYPE* second (if \verb!%locations! is used)
  
  \item Since Bison 3.0: (for yylex()) any number of arguments next (based on the argument of
  \verb!%lex-param!) and (for yyparse()) any number of arguments next (based on
  the argument of \verb!%parse-param!)
  
  In Bison 2.0, we just need to define macros \verb!YYLEX_PARAM! or
  \verb!YYPARSE_PARAM! for the names of the third arguments if needed.
  
  \item finally, arguments defined from \verb!%param! option
\end{enumerate}
\url{http://sourceforge.net/p/urjtag/mailman/message/31850904/}

Example:
\begin{enumerate}
  \item To have
\begin{verbatim}
int yylex   (scanner_mode *mode, environment_type *env);
int yyparse (parser_mode *mode, environment_type *env);
\end{verbatim}

in the *.l file, add
\begin{verbatim}
%lex-param   {scanner_mode *mode}
%parse-param {parser_mode *mode}
%param       {environment_type *env}
\end{verbatim}

IMPORTANT: For C++ reentrant parser, we can use these options to define extra arguments to yylex, yyparse, and yyerror.
\begin{itemize}
  \item If you wish to pass additional arguments to yylex, use \verb!%lex-param! 
  
  \item If you wish to add additional arguments to yyparse and yyerror, use
  \verb!%parse-param!

One or more extra arguments to yyparse
\begin{verbatim}
%parse-param {int *nastiness} {int *randomness}
\end{verbatim}

Example:
\begin{verbatim}
%parse-param {int *randomness}
\end{verbatim}
then speclang.tab.C 
\begin{verbatim}
void yyerror (int *randomness, const char *msg);
int  yyparse (int *randomness);
\end{verbatim}

If if both \verb!%define api.pure full! (or just \verb!%define api.pure!) and \verb!%locations! are used; or (since Bison 3.0) if
\verb!%parse-param! and \verb!%locations! is used:
\begin{verbatim}
void yyerror (YYLTYPE *llocp, int *randomness, const char *msg);
int  yyparse (int *randomness);
\end{verbatim}

So we need to use modify the interface of yyerror, if \verb!%parse-param! is used (since Bison 3.0).

  \item To pass additional arguments to both yylex(\ldots) and yyparse(\ldots), use
  \verb!%param!.
\end{itemize}

  \item To have
\begin{verbatim}
int yylex   (YYSTYPE *lvalp, scanner_mode *mode, environment_type *env);
int yyparse (parser_mode *mode, environment_type *env);
\end{verbatim}  
we need to use 
in the *.l file, add
\begin{verbatim}
%lex-param   {scanner_mode *mode}
%parse-param {parser_mode *mode}
%param       {environment_type *env}
%define api.pure full
\end{verbatim}

  \item To have
\begin{verbatim}
int yylex   (YYSTYPE *lvalp, YYLTYPE *llocp,
             scanner_mode *mode, environment_type *env);
int yyparse (parser_mode *mode, environment_type *env);
\end{verbatim}
\end{enumerate}
in the *.l file, add
\begin{verbatim}
%lex-param   {scanner_mode *mode}
%parse-param {parser_mode *mode}
%param       {environment_type *env}
%define api.pure full
%locations
\end{verbatim}

% Unfortunately, the code block containing the definition of yylex is inserted
% into the generated code before YYLTYPE is defined.

You may have the error
\begin{verbatim}
error: too few arguments to function 'int mdllex(YYSTYPE*,YYLTYPE*, void*)'
\end{verbatim}
as you have \verb!#define yylex mdllex!, 
and somewhere, you use
\begin{verbatim}
yychar = yylex(&yyval, &yylloc);
\end{verbatim}
To fix this, add \verb!%lex-param {context}! to the *.l file.

\url{http://www.gnu.org/software/bison/manual/html_node/Pure-Calling.html}

\url{http://www.gnu.org/software/bison/manual/html_node/Parser-Function.html}

\subsection{C++ : yyerror}

In early version of bison, you only have
\begin{verbatim}
void yyerror (const char *msg);
\end{verbatim}
to call when you want to indicate an error.

However, this is different in newer version of bison, if you add
\begin{verbatim}
%parse-param {int *randomness}
\end{verbatim}
then you will have the generated one
\begin{verbatim}
void yyerror (int *randomness, const char *msg);
int  yyparse (int *randomness);
\end{verbatim}

if you have \verb!%define api.pure full! and \verb!%locations!, then you will
have
\begin{verbatim}
void yyerror (YYLTYPE *llocp, int *randomness, const char *msg);
int  yyparse (int *randomness);
\end{verbatim}

To resolve the missing of the first version, you need to add this to \verb!*.y!
file inside the code block
\begin{verbatim}
%code provides {
void yyerror (char const*);
%}
\end{verbatim}
% int  yylex (YYSTYPE*, YYLTYPE*);
% extern "C" int yyparse ();

\subsection{generate C code or C++ code}
\label{sec:lex.yy.c}
\label{sec:lex.yy.cc}

NOTE: The generated file (from the grammar file \verb!*.l!) is \verb!lex.yy.c!
(C-code) or \verb!lex.yy.cc! (C++-code) depending on the options being used.


Flex provides two different ways to generate scanners for use with C++.

\begin{enumerate}

  \item simply compile the C code generated by flex, i.e. \verb!lex.yy.c!
  file, using a C++ compiler.
  
  You can then use C++ code in your rule actions instead of C code.
  
  IMPORTANT: you remains using \verb!FILE*! variables instead of C++
  \verb!std::istream! or \verb!std::ostream!. Also, default input source for
  your scanner remains \verb!yyin!, and default echoing is still done to \verb!yyout!.
  
  \item generate C++ code with \verb!flex -+! or \verb!flex --c++! (or the
  equivalent \verb!%option c++! in the *.l file) with the generated file is
  \verb!lex.yy.cc!.
  
  The generated scanner includes the header file `FlexLexer.h', which defines
  the interface to two C++ classes. 
  \begin{itemize}
    \item \verb!FlexLexer! class: a base class defines general scanner class
    interface
    
    \item \verb!yyFlexLexer! class: derived from FlexLexer
    
    Note that a yyFlexLexer object contains its entire scanning state. Thus you
    can use such objects to create reentrant scanners
    (Sect.\ref{sec:reentrant-parser}).
    You can instantiate multiple instances of the same yyFlexLexer class, and
    you can also combine multiple C++ scanner classes together in the same program
    using \verb!-P! option
  \end{itemize}

\end{enumerate}
\url{http://flex.sourceforge.net/manual/Code_002dLevel-And-API-Options.html}

\url{http://stackoverflow.com/questions/22239614/bison-how-to-fix-this-too-many-arguments-to-function-int-yylex-error}

\subsection{C++ : yywrap}
\label{sec:yywrap_C++}

If you get the error
\begin{verbatim}
 lex.yy.cc:(.text+0x402): undefined reference to yywrap'
\end{verbatim}

{\bf What yywrap() is for?} (when you work with multiple input files) - The
scanner calls this function on end of file, so you can point it to another file
and continue scanning its contents. If someone calls the compiler with multiple
source files, I store them in a list or array, and then yywrap() is called at
the end of each file to give you a chance to continue with a new file.

\begin{enumerate}
  \item You can use the default \verb!yywrap()! by linking agains \verb!-lfl!
(libfl.so, libfl.a)
\begin{verbatim}
sudo apt-get install libfl-dev
\end{verbatim}

The package is Multi-Arch enabled, so the files are under the
\verb!/usr/lib/i386-linux-gnu/! or \verb!/usr/lib/x86_64-linux-gnu/! directory,
depending on whether you are on 32 or 64 bit. Or possibly under
\verb!/usr/lib/armhf-linux-gnu/! if you are on ARM.
\begin{verbatim}
locate libfl
\end{verbatim}
\url{http://askubuntu.com/questions/289547/where-are-flex-libraries-located}

  \item  If you don't need this, modify \verb!*.l! file
  
add
\begin{verbatim}
%option noyywrap
\end{verbatim}
in *.l file (make sure it is outside \verb!%{! \verb!%}! region
(Sect.\ref{sec:flex_grammar-file}).

\url{http://solaris2linuxport.blogspot.com/2010/12/porting-isuues-for-yacc-and-lex-to.html}

  \item Implement your own yywrap(), and put it into the third section
  in the *.l file (after the second \verb!%%!)
  
\begin{lstlisting}
int yywrap() {
   // open next reference or source file and start scanning
   if((yyin = compiler->getNextFile()) != NULL) {
      line = 0; // reset line counter for next source file
      return 0;
   }
   return 1;
}
\end{lstlisting}
\url{http://stackoverflow.com/questions/1811125/undefined-reference-to-yywrap}

\end{enumerate}


\subsection{C++ : use a custom parser (e.g. MdlLexer class or LensLexer class)}

Options: \url{http://flex.sourceforge.net/manual/Scanner-Options.html}

First, your parser need to be derived from \verb!yyFlexLexer! class.


To use your parser, either
\begin{itemize}
  \item use flex with \verb!--yyclasss=NAME!
  
  \item put into *.l file: \verb!%option yyclass=NAME!

\begin{verbatim}
%option c++
%option yyclass="MdlLexer"
\end{verbatim} 

  \item if your custom parser is only a change in the prefix (as flex use
  \verb!yy! as the default prefix), you have a few option
  
  \begin{enumerate}
    \item put this code in your *.l file,
  
  \begin{verbatim}
#undef yyFlexLexer
#define yyFlexLexer xxFlexLexer
#include <FlexLexer.h>


#undef yyFlexLexer
#define yyFlexLexer zzFlexLexer
#include <FlexLexer.h>
  \end{verbatim}
  
    \item change the prefix dynamically, replace PREFIX with the prefix you want
    to use
\begin{verbatim}
 // compile with 
flex -PPREFIX
flex --prefix=PREFIX

 // or put in the *.l file
%option prefix="PREFIX"
\end{verbatim}
    \end{enumerate}

Example: \verb!%option prefix="zz"! for the zzFlexLexer class scanner  
\end{itemize}

So flex will place your actions in the member function \verb!foo::yylex()!
instead of \verb!yyFlexLexer::yylex()!. It also generates a
\verb!yyFlexLexer::yylex()! member function that emits a run-time error (by
invoking \verb!yyFlexLexer::LexerError()!) if called.

    

\section{Parser: not reentrant vs. reentrant}

How to have multiple instances of your generated parser (or, heaven forfend,
multiple parsers with different grammars) in the same binary without having them
interfere with each other? This requires defining a reentrant parser
(Sect.\ref{sec:reentrant-parser}).
In other words, 
the \verb!yyparse()! and \verb!yylex()! driver functions in the generated code
have to be reentrant. 

Example: Bison/Flex grammar that parses CVS master files  -  often thousands of
them in a single run. The goal is to (a) to be able to spread the job across
multiple processors so the work gets done faster, and (b) not to allow I/O waits
for some masters being parsed to block compute-intensive operations on others. 



\subsection{not reentrant}
\label{sec:not-reentrant-parser}
\label{sec:parser-not-reentrant}

Normally, Bison generates a parser which is not reentrant. This is suitable for
most uses, and it maintains compatibility with Yacc.
So, they use statically allocated variables for communication with yylex(),
including \verb!yylval! and \verb!yylloc!.
\begin{verbatim}
int yylex();
\end{verbatim}

However, a non-reentrant program is not safe to be used in asynchronous
scenario, e.g. called from a signal handler. As it uses static global variable, 
in a multiple threads of control, a nonreentrant program must be called only
within interlocks. 

\subsection{reentrant (pure-parser)}
\label{sec:reentrant-parser}
\label{sec:pure_parser-bison}

In a reentrant function, no globals allowed; they have to keep their parsing and
lexing state in purely stack- and thread-local storage, and deliver their
results back the way ordinary reentrant C functions would do it (that is,
through structures referenced by pointer arguments).

NOTE: Yacc and Lex couldn't do this, as they use a bunch of global variables.

A reentrant program is one which does not alter in the course of execution, i.e.
it maintains itself as pure read-only code. Reentrancy is important whenever
asynchronous execution is possible;

\textcolor{red}{In Bison, to indicate a reentrant parser (for yyparse())
function), we add one of these lines} to the \verb!*.y! file; (and also to
\verb!.l! file - as described shortly)
\begin{verbatim}

  // compatible with Bison 3.x
  // same effect as using pure-parser, except that Bison is 
  // more careful to warn about unreasonable usage in the latter case.
%define api.pure full

  // deprecated since Bison 2.6.2 (still supported in Bison 3.0)
%pure-parser

  // the original keyword (no longer supported in Bison 3.1)  
%pure_parser
\end{verbatim}
\url{https://code.google.com/p/lilypond/issues/detail?id=3488}

\url{https://www.gnu.org/software/bison/manual/html_node/Pure-Decl.html}

NOTE: \verb!yyparse()! is the parser function (Sect.\ref{sec:yyparse-C++}),
and the convention for calling yyparse() itself is unchanged. But if we use 3
lines \url{http://esr.ibiblio.org/?p=6341}
\begin{verbatim}
%define api.pure full
%lex-param {yyscan_t scanner}
%parse-param {yyscan_t scanner}
\end{verbatim}
\verb!yyscan_t! is (in effect) a special private structure used to hold your
scanner state. 
\url{http://www.gnu.org/software/bison/manual/html_node/Pure-Decl.html}
\begin{verbatim}
yyparse(yyscan_t *scanner)
\end{verbatim}


\textcolor{red}{In Flex, to indicate a reentrant scanner (for yylex()
function), you need to update the grammar file}
(Sect.\ref{sec:flex_grammar-file})
\begin{verbatim}
%option reentrant bison-bridge
\end{verbatim}

IMPORTANT: The result is that the communication variables \verb!yylval! and
\verb!yylloc! become local variables in \verb!yyparse()! - Sect.\ref{sec:yyparse}, and a different
calling convention is used for the lexical analyzer function \verb!yylex()!.
\textcolor{red}{Notice the different in the interfaces of yyparse() and yylex()
functions, from no arguments, to}

\verb!lvalp! (or previously yylval) is where yylex() will put its token value
when it's called by yyparse().
\begin{itemize}
  \item either 
\begin{verbatim}
int
yylex (YYSTYPE *lvalp)
{
  ...
  *lvalp = value;  /* Put value onto Bison stack.  */
  return INT;      /* Return the type of the token.  */
  ...
}
\end{verbatim}

  \item or this (if \verb!%locations! is used, i.e. we can get access to
  textual line number via \verb!yylineno! in the parser code)
\begin{verbatim}
int
yylex (YYSTYPE *lvalp, YYLTYPE *llocp)
{
  ...
  *lvalp = value;  /* Put value onto Bison stack.  */
  return INT;      /* Return the type of the token.  */
  ...
}
\end{verbatim}
or
\begin{verbatim}
yylex(YYSTYPE *yylval_param, yyscan_t yyscanner)
\end{verbatim}

\end{itemize}
and the two global variable \verb!yylval! and \verb!yylloc! are replaced by two
pointers passed by arguments to yylex(lvalp, llocp)



\subsection{example}

\url{http://fossies.org/linux/collectd/src/liboconfig/parser.c}

\section{Connect Scanner output to Parser input}
\label{sec:connect-scanner-parser}

The output of a scanner is a token as any tokens defined in \verb!%token!
option. In the *.l file, we define the list of tokens using \verb!%token!
option, and the grammar rule to recognize these token from the input
(Sect.\ref{sec:flex_grammar-file}).

The parser (*.y file) uses the \verb!yyparse()! which returns 1 if the input
satisfies the grammar rule, and then it decides the action to take for each
token detected (Sect.\ref{sec:yyparse}).

The *.y file only has access to tokens names (in uppercase), but not others
(e.g. \verb!YY_CURRENT_BUFFER!, \verb!YY_BUFFER_STATE!) defined by flex which
can only be used in the generated \verb!lex.yy.c! (or \verb!lex.yy.cc!) file.
\textcolor{red}{If you want to access them in a bison grammar}, the easiest way
is to encapsulate the use of them in a small function you define in the 3rd
section of the .l file. Then you call that function from the .y file or any
other source file.



\section{Your compiler}

Your own code will probably call the function \verb!yyparse()! created by bison
from your parser specification, which in turns call \verb!yyflex()! created by
flex.

You can use yylex() independent of yyparse() and you can supply your own yylex()
to be used with a bison generated yyparse(). 


When you call yyparse() from your code, the parser will repeatedly call yylex()
to obtain tokens (words and punctuation) from the input stream. With each new
token, the parser will decide whether to push the token onto its internal stack
or to reduce a sequence of tokens on the stack to create {\bf a new language
part} (called {\bf non-terminal}) to push onto the stack. 
One of the non-terminals is the "start-state". When this non-terminal is
matched, yyparse() will return.
When the parser returns, its internal stack vanishes and so you are responsible
for finding some method of returning the information you need from that stack

Now, you can to execute a block of C/C++ code that you provide each time bison
matches a pattern.

EXPLAINS:
\url{http://www.mactech.com/articles/mactech/Vol.16/16.07/UsingFlexandBison/index.html}
	
\section{Troubleshoort}

