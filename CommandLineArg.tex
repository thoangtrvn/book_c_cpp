\chapter{Command-line arguments}
\label{chap:command-line-argument}

\section{History}

Unix tradition encourages the use of command-line switches to control programs, so that options can be specified from scripts.
We can pass the input via the command-line arguments
\begin{verbatim}
python <source.py>  arg1 arg2
\end{verbatim}
\url{http://www.faqs.org/docs/artu/ch10s05.html}

A program can accepts commands (tcp, serial), and keyword arguments
\begin{verbatim}
my_prog tcp <host> <port>  [--timeout=<second>]
my_prog serial <port> [--baud=9600] [--timeout=<second>]
my_prog (-h | --help | --version)
\end{verbatim}
A command can accepts as a series of positional arguments (\verb!<host>, <host>!) 
and/or keyword arguments (\verb!--timeout=<second>!).


There are 3 types of command-line arguments:
\begin{itemize}
  \item positional argument
  \item optional argument
  \item keyword argument (named argument)
  \item flags
\end{itemize}

\begin{mdframed}

David Goodger pointed out that the term \verb!keyword argument! - on Unix
platforms - traditionally called \verb!options!;  what he calls \verb!values!
are traditionally called \verb!option arguments!; and what he calls
\verb!positional arguments!, the Open Group calls \verb!operands!.
\begin{itemize}
  \item First of all, the term option tends to be Unix-specific; on Windows the term parameter is more frequently used
  \item Second, the investigation began with command-line parsers, and in the context of a discussion of parsers and parsing, keyword argument seems a more traditional and appropriate term than option. 
  \item Third, the usual definition of option is not very useful.
  \item And finally, the term option implies optionality.  Whether an argument is optional or required is a semantic issue rather than a syntactical issue. 
\end{itemize}

\end{mdframed}

\url{https://pythonconquerstheuniverse.wordpress.com/2010/07/25/command-line-syntax-some-basic-concepts/}


\subsection{-- positional argument}
\label{sec:positional-argument}

{\bf positional arguments}: a bare value and its position in a list of
  arguments identifies it.
\begin{verbatim}
rename file_a.txt  file_b.txt
\end{verbatim}  
The order is important in positional arguments.

\subsection{-- optional argument}
\label{sec:optional-argument}

({\bf optional argument}) or the term \verb!keyword argument! is used to
  cover both {\bf named arguments} and {\bf flags}.
  
A \verb!keyword argument! requires a sigil (a special character of string of
characters that indicates the beginning of a key).

\begin{itemize}
  \item In Windows, a sigil is a forward slash (/)
  \item In Unix-like O/S, a sigil is a dash (-) for single-character name, and
  two dash (\verb!--!) for longer name

Some applications, especially on Unix, make a distinction between
single-character keys and multi-character keys ("long options"), with a
single-dash sigil "-" indicating the beginning of a single-character key, and a
double dash "\verb!--!" sigil indicating the beginning of a multi-character key.

  \item Some applications use multiple sigils.  With the plus sign '+' as a
  sigil, for instance, it is possible to use flags to turn options on and off.
    
   \item EXCEPTION: some programs do not use sigil in front of the optional
   argument

\begin{verbatim}
tar xvf <filename.tar>
\end{verbatim}
This is not recommended.
   
   \item  Some programs allow several single-letter options to be preceded by a
   single dash e.g. "ps -elf" instead of "ps -e -l -f". 
   
   This is also not recommended.
   \url{http://courses.cms.caltech.edu/cs11/material/c/mike/misc/cmdline_args.html}
\end{itemize}

\subsection{-- named argument}
\label{sec:named-argument}
{\bf named arguments}: a (keyword, value) pair, where the key
  identifiers the value
  
  It is  identified by placing a dash (-) before the optional argument's name
\begin{verbatim}
my_prog -a value
my_prog -a=value

my_prog --afullname value
my_prog --afullname=value
\end{verbatim}  
% 

A \verb!named argument! needs a mechanism to separate keys from argument values
\begin{itemize}
  \item use whitespace 
\begin{verbatim}
rename  -o file_a.txt  -n file_b.txt
\end{verbatim}  

  \item use equal sign (=) in Unix, and colon (:) in Windows
\begin{verbatim}
 # Unix
rename  -o=file_a.txt  -n=file_b.txt
 # Windows
rename  /o:file_a.txt  /n:file_b.txt
\end{verbatim}

  \item use the known-length (1-character)
  
\begin{verbatim}
rename  -ofile_a.txt  -nfile_b.txt
\end{verbatim}
\end{itemize}

Several fixed-length \verb!flags! can be concatenated
\begin{verbatim}
tar -x -v -f  some_filename.tar

tar -xvf some_filename.tar
\end{verbatim}
NOTE: "--xvf" (note the double dash) indicates a single multi-character flag: "xvf".

\subsection{-- flags}
\label{sec:flags-in-argument}

{\bf flags}: a stand-alone keyword, whose presence or absence provides
  information to the application
\begin{verbatim}
my_prog -a -b

my_prog --afullanme --bfullname
\end{verbatim}  

Although it is possible to think of flags as degenerate named arguments (named
arguments that have a key but no value), it is easier to think of flags as a
distinct type of argument, different from named arguments.


\section{Parsing command-line arguments}

We need to loop through the number of argument \verb!argc!, including the name
of the excutable file as the 0-th argument. 

\subsection{Simple}


{\small 
\begin{verbatim}
// When passing char arrays as parameters they must be pointers
int main(int argc, char* argv[]) {
    if (argc < 5) { // Check the value of argc. If not enough parameters have been passed, inform user and exit.
        //print help
        std::cerr << "Usage is -in <infile> -out <outdir>\n"; 
        std::cin.get();
        exit(0);
    } else { // if we got enough parameters...
        char* myFile, myPath, myOutPath;
        std::cout << argv[0];
        for (int i = 1; i < argc; i++) { /* We will iterate over argv[] to get the parameters stored inside.
                                          * Note that we're starting on 1 because we don't need to know the 
                                          * path of the program, which is stored in argv[0] */
            if (i + 1 != argc) // Check that we haven't finished parsing already
                if (argv[i] == "-f") {
                    // We know the next argument *should* be the filename:
                    myFile = argv[i + 1];
                } else if (argv[i] == "-p") {
                    myPath = argv[i + 1];
                } else if (argv[i] == "-o") {
                    myOutPath = argv[i + 1];
                } else {
                    std::cout << "Not enough or invalid arguments, please try again.\n";
                    Sleep(2000); 
                    exit(0);
            }
            std::cout << argv[i] << " ";
        }
        //... some more code
        std::cin.get();
        return 0;
    }
}
\end{verbatim}
}

\subsection{Using std::string::find()}

\begin{lstlisting}
#include <algorithm>

char* getCmdOption(char ** begin, char ** end, const std::string & option)
{
    char ** itr = std::find(begin, end, option);
    if (itr != end && ++itr != end)
    {
        return *itr;
    }
    return 0;
}

bool cmdOptionExists(char** begin, char** end, const std::string& option)
{
    return std::find(begin, end, option) != end;
}

int main(int argc, char * argv[])
{
    if(cmdOptionExists(argv, argv+argc, "-h"))
    {
        // Do stuff
    }

    char * filename = getCmdOption(argv, argv + argc, "-f");

    if (filename)
    {
        // Do interesting things
        // ...
    }

    return 0;
}
\end{lstlisting}
\url{http://stackoverflow.com/questions/865668/parse-command-line-arguments}

\subsection{Using GetOpt class - C++}

You can use GNU GetOpt (LGPL) or one of the various C++ ports, such as getoptpp
(GPL).

\url{http://www.chemie.fu-berlin.de/chemnet/use/info/libgpp/libgpp_39.html}

\subsection{Using getopt() - C language}
\label{sec:getopt()}

Here, we use \verb!getopt()! and \verb!optarg! variable. IMPORTANT: We need to
end the list of argument with semicolon (:)
\begin{verbatim}
#include <getopt.h>

std::string val1;
int val2;

//ERROR:you will get std::logic_error
//while ((c = getopt(argc,argv,"i:v:e:f")) != -1)

//CORRECT
while ((c = getopt(argc,argv,"i:v:e:f:")) != -1)
{
  switch(c) {
  case 'i':
  	val1 = optarg;
	break;
  case 'v':
    val2 = atoi(optarg);
    break;
  case 'e':
  	break;
  case 'f':
    break;
    
   default: /* '?' */
            fprintf(stderr, "Usage: %s [-t nsecs] [-n] name\n",
                    argv[0]);
            exit(EXIT_FAILURE);
 }
}
\end{verbatim}

\subsection{OptionParser}

\url{http://stackoverflow.com/questions/865668/parse-command-line-arguments}

\url{http://sourceforge.net/projects/optionparser/}

\subsection{CLPP: C++ Command-line parser (use Boost)}

\url{http://sourceforge.net/projects/clp-parser/}

\subsection{Using Boost.Program\_Options}

{\small \begin{verbatim}
#include <boost/program_options.hpp>
namespace po = boost::program_options;

int main(int argc, char** argv){
  try{
    int opt;
    po::options_description desc(``Allowed options'');
    desc.add_options() // list all options
       (``help,h'', ``produce help message''),

       //the second part after ',' is short, i.e. --file or -f
       (``file,f'', po::value<std::string>(), ``the file to read in''),
       (``task'', po::value<std::vector<int> >(), ``accept multiple values'')

       //we can set default value here (10) and the address (&opt) to store the
       //value
       // so if user don't pass the argument, it use 10 
       ("optimization", po::value<int>(&opt)->default_value(10), 
          "optimization level")

       //so if user pass the argument with no value, it use 10
       //, e.g. ./a.out --width   #use 10
       //       ./a.out --witdh=80 # use 80
       //NOTE: for short option, the value must be immediately after the option
       //./a.out -w80   # implicit_value not used
       //./a.out -w 80  # wrong: 80 parsed as extra arg if implicit_value is
       //               defined 
      ("witdh,w", po::value<int>(&opt)->implicit_value(10), 
          "width")

       //it can accept multiple values
       //or use the option several times
       //either way, the result will be a vector
       ("include-path,I", po::value< vector<string> >(), 
             "include path")
       ("input-file", po::value< vector<string> >(), "input file")
       
       // you don't want a vector but a single value and that 
       // a single value can span multiple tokens
       // e.g. --email beadle@mars beadle2@mars
       ("email", value<string>()->multitoken(), "email to send to")
       
       // we can add semantic
       // which need to use boost::program_options::value_semantic class
       ("email", value< vector<string> >()
        ->composing()->notifier(&your_function), "email")          
    ;	
    
    po::variables_map vm; //store values of options (of arbitrary types)
    po::store(po::parse_command_line(argc, argv, desc), vm);
    po::notify(vm);
  
    if (argc == 1 || vm.cout(``help'')){
       std::cout << desc << ``\n'';
       return 0;
    }

//IMPORTANT: Once you use address (e.g. &opt)
// you don't have to use if for getting option's value    
    if (vm.cout(``file'')) //check the using of one argument option
    {
       myfile = std::string(vm[``part''].as<std::string>());
    }
    if (vm.cout(``type'')) //integer
    {
       mytype = vm[``type''].as<int>();
    }
    if (vm.cout(``task'')) //multiple values
    {
      std::vector<int> list_task = vm[``task''].as< std::vector<int> >();
      //loop
      for (std::vector<int>::iterator iter=list_task.begin(); iter <
          list_task.end(); iter++) {
          //do something
      } 
    }
  }catch (std::exception &e){
    std::cerr << ``error: `` << e.what() << ``\n'';
    return 1;
  }catch (...){
    std::cerr << ``Exception unknown type \n'';
  }
}
\end{verbatim}}

There are case that we want the option-name to be hidden, e.g.
\begin{verbatim}
compiler a.cpp
compiler --input-file=a.cpp
\end{verbatim}
then we need to tell \verb!input-file! is the ``positional options''
\begin{verbatim}
po::positional_options_description p;
p.add("input-file", -1); // (name, how-many-positional-options)
   // value=1 : exactly one, first, positional option is assigned to input-file
   // value=2 : the first two positional option is assigned to input-file
   // value=-1: any others 
  // all positional options should be translated into 
  // values for 'input-file' option

po::variables_map vm;
  //and we use different function
  // command_line_parser(...)
po::store(po::command_line_parser(ac, av).
          options(desc).positional(p).run(), vm);
po::notify(vm);
\end{verbatim}

Now you want to have options organized into different groups: some are read from
the config file, so that frequently used options can be passed easily, or we
split options into different groups (some can only be used  in the config file,
some should be hidden from user, etc.), we then use more complicated approach
like the following. NOTE: 
The value passed to the command line should override the value from the config file.
\begin{verbatim}
// Declare a group of options that will be 
// allowed only on command line
po::options_description generic("Generic options");
generic.add_options()
    ("version,v", "print version string")
    ("help", "produce help message")    
    ;
    
// Declare a group of options that will be 
// allowed both on command line and in
// config file
po::options_description config("Configuration");
config.add_options()
    ("optimization", po::value<int>(&opt)->default_value(10), 
          "optimization level")
    ("include-path,I", 
         po::value< vector<string> >()->composing(), 
         "include path")
    ;

// Hidden options, will be allowed both on command line and
// in config file, but will not be shown to the user.
po::options_description hidden("Hidden options");
hidden.add_options()
    ("input-file", po::value< vector<string> >(), "input file")
    ;     
\end{verbatim}

then we put them into groups we want
\begin{verbatim}
po::options_description cmdline_options;
cmdline_options.add(generic).add(config).add(hidden);

po::options_description config_file_options;
config_file_options.add(config).add(hidden);

po::options_description visible("Allowed options");
visible.add(generic).add(config);
\end{verbatim}


\url{http://www.boost.org/doc/libs/1_54_0/doc/html/program_options/overview.html}

% \section{File locking}
% \label{sec:file-locking}

\subsection{Using popt library}
\label{sec:popt_lib}

The popt library exists essentially for parsing command line options. Some
specific advantages of popt are no global variables (allowing multiple passes in
parsing argv), parsing an arbitrary array of argv-style elements (allowing
parsing of command-line-strings from any source), a standard method of option
aliasing, ability to exec external option filters, and automatic generation of
help and usage messages.    

\url{http://directory.fsf.org/wiki/Popt}

\subsection{TCLAP}
\label{sec:TCLAP}

\url{http://tclap.sourceforge.net/}

